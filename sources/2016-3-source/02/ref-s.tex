

{\footnotesize

\vskip 5mm
%\newpage

\noindent {\small\textbf{References} }

\vskip 3mm



1. Kiendl~J., Schmidt~R., Wuchner~R., Bletzinger~K.-U.
Isogeometric shape optimization of shells using semi-analytical
sensitivity analysis and sensitivity weighting. \textit{Comput.
Methods Appl. Mech. Eng.}, 2014, vol.~274, pp.~148--167.

2. Makarov~A.~A. Matricy dobavleniya i udaleniya uzlov dlya
nepolinomial'nyh splajnov [Knot insertion and knot removal
matrices for nonpolynomial splines]. {\it  Vychislitel'nye metody
i programmirovanie} [{\it Numerical methods and programming}],
2012, vol.~13, pp.~74--86. (In Russian)

3. Makarov~A.~A. Raschet prochnosti i ustojchivosti podkreplennyh
obolochek i ras\-pa\-ral\-le\-li\-va\-nie [Calculation of the
strength and stability of reinforced membranes and
parallelization]. {\it Trudy molodyh uchenyh: Sb. tez. dokl.
konferencii molodyh uchenyh}  [{\it Abstracts of the conference of
young scientists}], vol.~1. Saint Petersburg, ITMO University
Publ., 2011, pp.~272. (In Russian)

4. Golovanov~N.~N., Il'yutko~D.~P., Nosovskij~G.~V., Fomenko~A.~T.
\textit{Komp'yuternaya geometriya} [{\it Computational geometry}].
Moscow, Academia Publ., 2006, 511~p. (In Russian)

5. Golovanov~N.~N. \textit{Geometricheskoe modelirovanie} [{\it
Geometric modeling}]. Moscow, Academia Publ., 2011, 272~p. (In
Russian)

6. Voloshinov~D.~V. Ispol'zovanie metodov geometricheskogo
modelirovaniya dlya av\-to\-ma\-ti\-zi\-ro\-van\-no\-go
proektirovaniya i issledovaniya slozhnyh tekhnicheskih
poverhnostej [Using of geometric modeling for computer-aided
design and research of complex technical surfaces]. {\it
Nauch.-tekhn. vedomosti S.-Peterb. politekh. un-ta} [{\it St.
Petersburg State Polytechnical University Journal}], 2006, no.~44,
pp. 152--157. (In Russian)

7. Filippov~V.~A. \textit{Osnovy geometrii poverhnostej obolochek
prostranstvennyh konstrukcij} [{\it Basics of geometry of surfaces
of shells of spatial structures}]. Moscow, Fizmatlit Publ., 2009,
191 p. (In Russian)

8. Belyaeva~Z.~V., Mityushov~E.~A. Geometricheskoe modelirovanie
prostranstvennyh konstrukcij [Geometric modeling of spatial
structures]. {\it Vestn. Tomsk. gos. arhitek.-stroit. un-ta} [{\it
Vestnik TSUAB}], 2010, no.~1~(26), pp.~53--63. (In Russian)

9. Aseev~A.~V., Makarov~A.~A. O vizualizacii ehlementov
podkreplennyh tonkostennyh obolochek [On visualisation of
reinforced thin shell elements]. {\it Komp'yuternye instrumenty v
obrazovanii} [{\it Computer Tools in Education}], 2014, no.~2,
pp.~35--45. (In Russian)

10. \textit{Nguyen-Thanh~N., Valizadeh~N., Nguyen~M.~N.,
Nguyen-Xuan~H., Zhuang~X., Are\-ias~P., Zi~G., Bazilevs~Y., De
Lorenzis~L., Rabczuk~T.} An extended isogeometric thin shell
analysis based on Kirchhoff�Love theory. {\it  Comput. Methods
Appl. Mech. Eng.},  2015, vol.~284, pp.~265--291.

11. Karpov~V.~V. \textit{Prochnost' i ustojchivost' podkreplennyh
obolochek vrashcheniya. Part~1: Modeli i algoritmy issledovaniya
prochnosti i ustojchivosti podkreplennyh obolochek
vra\-shche\-niya} [{\it The strength and stability of reinforced
shells of revolution. Part~1:  Models and algorithms for research
of strength and stability of reinforced shells of revolution}].
Moscow, Fizmatlit Publ., 2010, 288~p. (In Russian)

12. Aseev~A.~V., Makarov~A.~A., Semenov~A.~A. Vizualizaciya
napryazhenno-de\-for\-mi\-ro\-van\-no\-go sostoyaniya tonkostennyh
rebristyh obolochek [Visualization of stress-strain state of
thin-walled ribbed shells]. {\it Vestn. grazhdanskih inzhenerov}
[{\it Bulletin of Civil Engineers}], 2013, no.~3~(38),
pp.~226--232. (In Russian)

13. Moskalenko~L.~P. Effektivnost' podkrepleniya pologih obolochek
rebrami peremennoj vysoty [Efficiency of the reinforcement of flat
shells edges with ribs of variable height]. {\it Vestn.
grazhdanskih inzhenerov} [{\it Bulletin of Civil Engineers}],
2011, no.~3~(28), pp.~46--50. (In Russian)\newpage

14. Preobrazhenskij~I.~N., Grishchak~V.~Z. \textit{Ustojchivost' i
kolebaniya konicheskih obolochek} [{\it Stability and oscillations
of conical shells}]. Moscow, Mashinostroenie Publ., 1986, 240~p.
(In Russian)

15. Palij~O.~M., Spiro~V.~E. \textit{Anizotropnye obolochki v
sudostroenii. Teorija i raschet} [{\it Anisotropic membranes in
shipbuilding. Theory and evaluation}]. Leningrad, Sudostroenie
Publ., 1977, 386~p. (In Russian)

16. Bezuhov~N.~I. \textit{Osnovy teorii uprugosti, plastichnosti i
polzuchesti} [{\it Fundamentals of the theory of elasticity,
plasticity and creep}]. Moscow, Vysshaya shkola  Publ., 1968,
512~p. (In Russian)

\vskip 2mm

{\bf For citation:} Aseev A.~V.,~Makarov A.~A. On visualization of
some thin shells and their stress-strain state. {\it Vestnik of
Saint Petersburg University. Series~10. Applied mathematics.
Computer science. Control processes}, \issueyear, issue~\issuenum,
pp.~\pageref{p2}--\pageref{p2e}.
\doivyp/spbu10.\issueyear.\issuenum02

}
