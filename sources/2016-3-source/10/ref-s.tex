

{\footnotesize

\vskip 3mm
%\newpage

\noindent {\small\textbf{References} }

\vskip 2mm



1. Karelin V. V. {Penalty functions in a control problem}.
\textit{Automation and Remote Control}, 2004, no.~3, pp.~483--492.

%\bibitem{demgiakar}
2. Demyanov V. F., Giannessi F., Karelin V. V. Optimal control
problems via exact penalty functions. \textit{Journal of Global
Optimization}, 1998, vol.~12, pp.~215--223.

%\bibitem{BaranovKazarinovHomenuk}
3. {Baranov A. Y., Khazarinov Y. F., Khomenuk V. V.} Gradient
optimization methods of nonlinear systems of automatic control.
\textit{Proc. ``Applied Tasks of Technical Cybernetics''}, 1966,
pp.~307--316.

%\bibitem{Kelley1}
4. {Kelly H. J.} Gradient theory of optimal flight paths.
\textit{ARS Journal}, 1960, vol.~30, no.~10, pp.~947--954.

%\bibitem{Kelley2}
5. {Kelly H. J.} Method of gradients. \textit{Optimiz. techn.
applic. aerospace syst.} New York, London, Acad. Press, 1962,
pp.~205--254.

%\bibitem{Moiseev}
6. {Moiseev N. N.} {\it Jelementy teorii optimal'nyh sistem} [{\it
Elements of the theory of optimal systems}]. Moscow, Nauka Publ.,
1975. 528~p. (In Russian)

%\bibitem{egorov}
7. Egorov A. I. {\it Osnovy teorii upravlenija} [{\it Basics of
the control theory}]. Moscow, Fizmatlit Publ., 2004, 504~p. (In
Russian)

%\bibitem{antipin}
8. {Antipin A. S., Khoroshilova E. V.} Linear programming and
dynamics. \textit {Proc. of the Institute of Mathematics and
Mechanic UrO RAS}, 2013, vol.~19, no.~2, pp.~7--25.

%\bibitem{demrub2}
9. {Demyanov V. F., Rubinov A. M.} {\it Priblizhjonnye metody
reshenija jekstremal'nyh zadach} [{\it Approximate methods of
solving extremal problems}]. Leningrad, Leningrad University
Publ., 1968, 178~p. (In Russian)

%\bibitem{tamasyan}
10. Tamasyan G. Sh. {Numerical methods in problems of calculus of
variations for functionals depending on higher order derivatives}.
\textit{Journal of Mathematical Sciences}, 2013, vol.~188, no.~3,
pp.~299--321.

%\bibitem{demrub}
11. Demyanov V. F., Rubinov A. M. {\it Osnovy negladkogo analiza i
kvazidifferencial'noe ischislenie} [{\it Basics of nonsmooth
analysis and quasidifferentiable optimization}]. Moscow, Nauka
Publ., 1990, 432~p. (In Russian)

%\bibitem{daugavet}
12. Daugavet V. A. {\it Chislennye metody kvadratichnogo
programmirovanija} [{\it Numerical methods of quadratic
programming}]. Saint Petersburg, Saint Petersburg State University
Publ. House, 2001, 128~p. (In Russian)

%\cite{zubov}
%\textbf{����� �. �.} ������ �� ������ ����������. �.: �����, 1969. 497~c.
%
%\cite{dragstykwiaszcz}
%P. Drag, K. Styczen, M. Kwiatkowska, A. Szczurek, ``A Review on the Direct and Indirect Methods for Solving Optimal Control Problems with Differential-Algebraic Constraints,'' \textit{Studies in Computational Intelligence}, vol.~610, pp.~91--105, 2015.
%
%\cite{wuteochao}
%C.Z. Wu, K.L. Teo, Y. Zhao, ``Numerical Method for a Class of Optimal Control Problems subject to Nonsmooth Functional Constraints,'' \textit{Journal of Computational and Applied Mathematics}, vol.~217, iss.~2, pp. 311--325, 2008.
%
%\cite{haipinh}
%M. Haider, M.R. De Pinho, ``A Maximum Principle for Optimal Control Problems with State and Mixed Constraints,'' \textit{ESAIM -- Control, Optimization and Calculus of Variations}, vol.~21, iss.~4, pp. 939--957, 2015.
%
%\cite{clarpinh}
%F. Clarke. M.R. De Pinho, ``Optimal Control Problems with Mixed Constraints,'' \textit{SIAM Journal on Control and Optimization}, vol.~48, iss.~7, pp. 4500--4524, 2010.
%
%\cite{liyuteodu}
%B.~Li, C.J. Ju, K.L.Teo, G.R. Duan, ``An Exact Penalty Function Method for Continuous Inequality Constrained Optimal Control Problem,'' \textit{Journal of Optimization Theory and Applications}, vol.~151, iss.~2, pp. 260--291, 2012.
%
%\cite{gaozhangwang}
%X. Gao, X. Zhang, Y. Wang, ``A Simple Exact Penalty Function Method for Optimal Control
%Problem with Continuous Inequality Constraints,'' \textit{Abstract and Applied Analysis}, vol.~2014, pp.~1--12, 2014.
%
%\cite{jialinyuteodu}
%C. Jiang, Q. Lin, C. Yu, K. L. Teo, ``An Exact Penalty Method for Free Terminal Time Optimal Control Problem with Continuous Inequality Constraints,'' \textit{Journal of Optimization Theory and Applications}, vol.~154, iss.~1, pp. 30--53, 2012.

%\cite{fominyh}
%Fominyh A. V., \textit{Numerical methods in the problem of constructing a program control} // Automation and Remote Control. (in Russ.) (in print)

%\cite{demyanov}
%Demyanov V.F. \textit{Extremum conditions and variation calculus.}  Moscow, Vysshaya shkola, 2005. 335 p. (in Russ.)
%
%\bibitem{kras}
13. Krasovskij N. N. {\it Teorija upravlenija dvizheniem} [{\it
Motion control theory}]. Moscow, Nauka Publ., 1968, 476~p. (In
Russian)

%\bibitem{kumar}
14. Kumar V. A control averaging technique for solving a class of
singular optimal control problems. \textit{Intern. J. Control},
1976, vol.~23, no.~3. pp.~361--380.

%\bibitem{srochko}
15. Srochko V. A., Khamidulin R. G. The method of successive
approximations in optimal control problems with boundary
conditions. \textit{USSR Computational Mathematics and
Mathematical Physics}, 1986, vol.~188, issue~2, pp.~113--122.

%\bibitem{evtushenko}
16. Evtushenko Yu. G. {\it Metody reshenija jekstremal'nyh zadach
i ih prilozhenie k sistemam optimizacii} [{\it Methods for solving
extreme problems and their application to systems of
optimization}]. Moscow, Nauka Publ., 1982, 432~p. (In Russian)
    % \end{thebibliography}



\vskip 2mm

{\bf For citation:} Fominyh A. V. The hypodifferential descent
method in the problem of constructing an optimal control. {\it
Vestnik of Saint Petersburg University. Series~10. Applied
mathematics. Computer science. Control processes}, \issueyear,
issue~\issuenum, pp.~\pageref{p10}--\pageref{p10e}.
\doivyp/spbu10.\issueyear.\issuenum10




}
