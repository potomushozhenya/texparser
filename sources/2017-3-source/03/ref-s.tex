

{\footnotesize

\vskip 3mm
%\newpage

\noindent{\small \textbf{References} }

\vskip 2mm

1. Nilson~Nils~J. {\it  Problem-solving methods in artificial
intelligence.} New York, McGRAW-HILL BOOK COMPANY Press, 1971, 280
p. (Russ. ed.: Nilson~N. {\it  Iskusstvennyi intellekt. Metody
poisra reshenii}.  Moscow, Mir Publ., 1973, 270 p.)\newpage

2.  Garey~M. R., Johnson~D. S. {\it Computers and Intractability:
A Guide to the Theory of NP-Completeness.} San Francisco, Freeman
Press, 1979, 340~p. (Russ. ed.: Garey~M. R., Johnson~D. S. {\it
Vychislitelnye mashiny i trudnoreshaemye zadachi}. Moscow, Mir
Publ., 1982, 416 p.)

3. Kosovskaya~T. M. Nekotorye zadachi iskusstvennogo intellekta,
dopuskaiushchie formalizatsiiu na~iazyke ischisleniia predikatov,
i~otsenki chisla shagov ikh resheniia [Some artificial
intelligence problems permitting formalization by means of
predicate calculus language and upper bounds of their solution
steps]. {\it  SPIIRAS Proceedings}, 2010, no.~14, pp.~58--75. (In
Russian)

4. Kossovskaya~T. M. Raspoznavanie ob"ektov iz~klassov, zamknutykh
otnositel'no gruppy preobrazovanii [Partial deduction of predicate
formula as an instrument for recognition of an object with
incomplete description]. {\it Vestnik of Saint-Petersburg
University. Series~10. Applied Mathematics. Computer Science.
Control Processes}, 2009, iss.~3, pp.~45--55. (In Russian)

5. Zhuravlev~Yu.~I. Ob algoritmah raspoznavaniya s
predstavitelnymi naborami (o logichskih algoritmah) [Recognition
algorithms  with representative sets (logical algorithms)]. {\it
Zhurnal vychislitelnoi matamatiki i matematicheskoi fiziki
$[$Computational Mathematics and Mathematical Physics}], 2002,
vol.~2, no.~9, pp.~1425--1435. (In Russian)

6. Kossovskaya~T. M. Dokazatel'stva otsenok chisla shagov
resheniia nekotorykh zadach raspoznavaniia obrazov, imeiushchikh
logicheskie opisaniia [Proofs of the number of steps bounds for
solving of some pattern recognition problems with logical
description]. {\it Vestnik of Saint Petersburg University.
Series~1. Mathematics. Mechanics. Astronomy}, 2007, no.~4,
pp.~82--90. (In Russian)

7. Russel~S. J., Norvig~P. {\it Artificial Intelligence. A Modern
Approach.} Pearson Education, Inc., 2003, 1412 p. (Russ. ed.:
Russel~S. J., Norvig~P. {\it Iskusstvennyi intellekt: sovremennyi
podhod}.  Moscow, ``Viliams'' Publ., 2006, 1408~p.)

8. Kossovskaya~T. M. Mnogourovnevye opisaniia klassov dlia
umen'sheniia chisla shagov resheniia zadach raspoznavaniia
obrazov, opisyvaemykh formulami ischisleniia predikatov [Level
descriptions of classes for decreasing step number of pattern
recognition problem solving described by predicate calculus
formulas]. {\it Vestnik of Saint Petersburg University. Series~10.
Applied Mathematics. Computer Science. Control Processes}, 2008,
iss.~1, pp.~64--72. (In Russian)

9. Kosovskaya~T. M. Podkhod k~resheniiu zadachi postroeniia
mnogourovnevogo opisaniia klassov na~iazyke ischisleniia
predikatov [An approach to the construction of a level description
of classes by means of a predicate calculus language]. {\it
SPIIRAS Proceedings}, 2014, no.~3(34), pp.~204--217. (In Russian)

10. Kleene~S. C. {\it Mathematical logic.} New York, Wiley, Dover
Publ.,  1967. 398~p. (Russ. ed.: Kleene~S. {\it Matematicheskaia
logika}.  Moscow, Mir Publ., 1973, 480~p.)



%12.{\it Kossovskaya~T.M.} Proofs of the number of steps bounds for solving of some pattern recognition problems with logical description~// Vestnik SanktPeterburgskogo Universiteta. Seriya 1. 2007, No. 4, pp.~82--90. (In Russian)

11. Kossovskaya~T. M. Chastichnaia vyvodimost' predikatnykh formul
kak sredstvo raspoznavaniia ob"ektov s~nepolnoi informatsiei
[Partial deduction of predicate formula as an instrument for
recognition of  an object with incomplete description]. {\it
Vestnik of Saint Petersburg University. Series~10. Applied
Mathematics. Computer Science. Control Processes}, 2009, iss.~1,
pp.~74--84. (In Russian)

12. Rybalov~A. N. Ob odnom genericheskom otnoshenii rekursivno
perechislimykh mnozhestv [A Generic relation on Recursively
Enumerable Sets]. {\it Algebra and Logic}, 2016, vol.~55, no.~5,
pp.~587--596. (In Russian)

13. Kosovskaya~T. Distance between objects described by predicate formulas. {\it International Book Series. Information Science and Computing. Book 25.  Mathematics of Distances and Applications}. %(Michel Deza, Michel Petitjean, Krasimir Markov (eds)),
 Sofia, Bulgaria,  ITHEA  Publ., 2012, pp.~153--159.

14. Kosovskaya~T. M. Samoobuchaiushchaiasia set' s~iacheikami,
realizuiushchimi predikatnye formuly [Self-training network with
the nells implementing predicate formulas]. {\it SPIIRAS
Proceedings}, 2015, no.~6(43), pp.~94--113. (In Russian)






\vskip 2mm

{\bf For citation:}   Kosovskaya T. M., Petrov D. A. Extraction of
a maximal common sub-formula of predicate formulas for the solving
of some Artificial Intelligence problems. {\it Vestnik of Saint
Petersburg University. Applied Mathematics. Computer Science.
Control Processes}, \issueyear, vol.~13, iss.~\issuenum,
pp.~\pageref{p3}--\pageref{p3e}.
\doivyp/spbu10.\issueyear.\issuenum03



}
