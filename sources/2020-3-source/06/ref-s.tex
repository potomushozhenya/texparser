
{\normalsize

\vskip 4.5mm%6mm

\noindent{\bf Process control issues of fine grinding in a
planetary mill%$^{*}$ }

}

\vskip 2mm

{\small

\noindent{\it S. K. Atanov$\,^{1}$%
, A. Z. Bigaliyeva$\,^{1,2}$%
, N. K. Apachidy$\,^2$%
, A. V. Rusak$\,^3$%
%, I.~�. Famylia%$\,^2$%

 }

\vskip 2mm

%%%%%%%%%%%%%%%%%%%%%%%%%%%%%%%%%%%%%%%%%%%%%%%%%%%%%%%%%%%%%%%%%%

%\efootnote{
%%
%\vspace{-3mm}\parindent=7mm
%%
%\vskip 0.1mm $^{*}$ This work was supported by the Russian Science Foundation (project 19-72-00178).\par
%%
%%\vskip 2.0mm
%%
%%\indent{\copyright} �����-������������� ���������������
%%�����������, \issueyear%
%%
%}

%%%%%%%%%%%%%%%%%%%%%%%%%%%%%%%%%%%%%%%%%%%%%%%%%%%%%%%%%%%%%%%%%%

{\footnotesize

\noindent%
$^1$~%
L. N. Gumilyov Eurasian National University, 2, Satpayev ul.,
Nursultan,


\noindent%
\hskip2.45mm%
010000, Kazakhstan


\noindent%
$^2$~%
Karaganda Technical University,   56, Nursultan Nazarbayev pr.,
Karaganda,

\noindent%
\hskip2.45mm%
 100000, Kazakhstan

%\newpage
\noindent%
$^3$~%
St. Petersburg National Research University of Information
Technologies, Mechanics and Optics,

\noindent%
\hskip2.45mm%
 49,  Kronverksky pr., St. Petersburg, 197101, Russian
Federation


%\noindent%
%$^1$~%
%St.\,Petersburg State University, 7--9, Universitetskaya nab.,
%
%\noindent%
%\hskip2.45mm%
%St.\,Petersburg, 199034, Russian Federation

}

%%%%%%%%%%%%%%%%%%%%%%%%%%%%%%%%%%%%%%%%%%%%%%%%%%%%%%%%%%%%%%%%%%

\vskip1.8mm%3mm


\noindent \textbf{For citation:}  Atanov S. K., Bigaliyeva A. Z.,
Apachidy N. K., Rusak A. V. Process control issues of fine
grinding in a planetary mill. {\it Vest\-nik of Saint
Pe\-ters\-burg University. Applied Mathematics. Computer
Science. Control Pro\-cesses}, %\,
\issueyear,
vol.~16, iss.~\issuenum,~pp.~\pageref{p6}--\pageref{p6e}. \\
\doivyp/\enskip%
\!\!\!spbu10.\issueyear.\issuenum06  (In Russian)

\vskip3mm

{\leftskip=7mm\noindent The article is devoted to the study of
controlled grinding processes in a planetary mill. The possibility
of continuous monitoring of the fineness of grinding by using the
laws of automatic control, synthesized on the basis of the
optimization approach, is considered. A mathematical model of the
control object in the linear stationary approximation is formed.
The dynamics of the obtained model is compared with data from
field experiments according to the main quality criteria of
dynamic processes. For this model, a linear-quadratic controller
is synthesized, which is a combination of a Kalman filter that
generates an optimal RMS estimate of the object's state vector
with a linear-quadratic optimal state controller. The article
provides details of the implementation of the adopted approach and
presents the results of modeling dynamic processes for the
considered closed-loop control system.\\[1mm]
\textit{Keywords}: linear quadratic gaussian control,
linear-quadratic controller, Kalman filter, estimation, planetary
mill.\par}

\vskip4mm%5mm

\noindent \textbf{References} }

\vskip 1.8mm%2mm

{\footnotesize

1. Sibirceva N. B., Potapenko A. N., Semernin A. N. Osobennosti
avtomatizacii zagruzki syr'evoj mel'nicy v sostave ASDU [Features
of automation of loading of a raw mill as a part of ADMS].
\textit{Izvestiya Samarskogo nauchnogo centra Rossijskoj akademii
nauk} [{\it Proceedings of Samarsk scientific centre of Russian
Academy of Sciences}], 2011, vol.~13, no.~1(3), pp.~641--645. (In
Russian)

2. Lazareva O. V., Podkamennyj Yu. A. Avtomatizirovannyj sposob
upravleniya kompleksom izmel'cheniya i klassifikacii
almazosoderzhashchih rud [Automated method for controlling the
complex of grinding and classification of diamond-containing
ores]. \textit{Vestnik Iskutskogo gosudarstvennogo tehnicheskogo
universiteta} [{\it Vestnik of Irkutsk State Technical
University}], 2014, no.~4(87), pp.~128--132. (In Russian)

3. Veremej E. I. {\it Srednekvadratichnaya mnogocelevaya
optimizaciya}. Ucheb. posobie [{\it Root-mean-squa\-re
multi-purpose optimization}. Textbook]. St. Petersburg, Izd-vo
Sankt-Peterburgskogo Go\-su\-dar\-st\-ven\-no\-go Universiteta
[Saint Petersburg State University Press], 2016, 407~p. (In
Russian)

4. Veremej E. I. \textit{Linejnye sistemy s obratnoj svyaz'yu}.
Ucheb. posobie [{\it Linear feedback systems}. Textbook]. St.
Petersburg, Lan' Publ., 2013, 448~p. (In Russian)

5. Veremey E. I.  Efficient spectral approach to SISO problems of
H2-Optimal Synthesis. \textit{Applied Mathematical Sciences},
2015, vol.~9, no.~79, pp.~3897--3909.

6. Veremey E. I. Dynamical correction of positioning control laws.
\textit{IFAC Proceedings} (IFAC Papers Online), 2013, vol.~9,
iss.~1, pp.~31--36.

7. Chrif L., Kadda Z. M. Aircraft control system using LQG and LQR
controller with optimal Estimation---Kalman filter design.
\textit{Procedia Engineering}, 2014, vol.~80, pp.~245--257.

8. \textit{Razrabotka tekhnologii i kompleksa tekhnicheskih
sredstv izvlecheniya metalla iz tekhnogennogo syr'ya.} Otchyot o
NIR (zaklyuch.) [{\it Development of technology and complex of
technical means for ex\-tracting metal from man-made raw
materials}. Report on research work (conclusion)]. Ruk.
A.~M.~Ga\-zaliev; ispoln. N.~K.~Apachidi i dr.  Karaganda,
Karagandinskii Technicheskii Universitet, 2015, 71~p., no.~GR~0113
RK00155 [Manag. A.~M.~Ga\-zaliev; exec.: N. K. Apachidi et al.
Karaganda, Karaganda Technical University Publ., 2015, 71 p.,
N~GR~0113 RK00155]. (In Russian)

9. Alekseev A. A. \textit{Identifikaciya i diagnostika sistem}.
Ucheb. posobie [\textit{Identification and diagnostics of
systems}. Textbook]. Moskow, Akademia Publ., 2009, 351~p. (In
Russian)


\vskip1.5mm Received:  January 14, 2020.

Accepted: August 13, 2020.

\vskip4.5%6
mm A~u~t~h~o~r'~s \ i~n~f~o~r~m~a~t~i~o~n:%

\vskip1.5mm \textit{Sabirzhan K. Atanov} --- Dr. Sci. in Technics,
Professor; atanov5@mail.ru

\vskip1.5mm \textit{Al'fia Z. Bigalieva} --- Senior Lecturer;
alfija84@mail.ru

\vskip1.5mm \textit{Nikolay K. Apachidy} --- Senior Lecturer;
apachidi@bk.ru

\vskip1.5mm \textit{Alena V. Rusak} --- PhD in Technics, Associate
Professor; alenrusak@yandex.ru

}
