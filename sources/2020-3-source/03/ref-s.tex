
{\normalsize

\vskip 4.2mm%6mm

\noindent{\bf Predicting the dynamics of the coronavirus (COVID-19)\\
 epidemic based on the case-based reasoning approach%$\,^*$%
}

}

\vskip 1.5mm%2mm

{\small

\noindent{\it V. V. Zakharov, Yu. E. Balykina%$\,^1$%
%, I.~�. Famylia%$\,^2$%

 }

\vskip 1.5mm%2mm

%%%%%%%%%%%%%%%%%%%%%%%%%%%%%%%%%%%%%%%%%%%%%%%%%%%%%%%%%%%%%%%%%%

%\efootnote{
%%
%\vspace{-3mm}\parindent=7mm
%%
%\vskip 0.1mm $^{*}$ This work is supported by Russian Science
%Foundation (project N~18-71-00006).%\par
%%
%%\vskip 2.0mm
%%
%\indent{\copyright} �����-������������� ���������������
%�����������, \issueyear%
%%
%}

%%%%%%%%%%%%%%%%%%%%%%%%%%%%%%%%%%%%%%%%%%%%%%%%%%%%%%%%%%%%%%%%%%

{\footnotesize

\noindent%
%$^1$~%
St.\,Petersburg State University, 7--9, Universitetskaya nab.,
St.\,Petersburg,

\noindent%
%\hskip2.45mm%
199034, Russian Federation

}

%%%%%%%%%%%%%%%%%%%%%%%%%%%%%%%%%%%%%%%%%%%%%%%%%%%%%%%%%%%%%%%%%%

\vskip2.0mm%3mm

\noindent \textbf{For citation:}  Zakharov V. V., Balykina Yu. E.
Predicting the dynamics of the coronavirus (COVID-19) epidemic
based on the case-based reasoning approach. {\it Vestnik of
Saint~Pe\-ters\-burg University. Applied Mathematics. Computer
Science. Control Processes}, \issueyear, vol.~16, iss.~\issuenum,
pp.~\pageref{p3}--\pageref{p3e}. %\\
\doivyp/\enskip%
\!\!\!spbu10.\issueyear.\issuenum03  (In Russian)

\vskip2.0mm%3mm

{\leftskip=7mm\noindent The case-based rate reasoning (CBRR)
method is presented for predicting future values of the
coronavirus epidemic's main parameters in Russia, which makes it
possible to build short-term forecasts based on analogues of the
percentage growth dynamics in other countries. A new heuristic
method for estimating the duration of the transition process of
the percentage increase between specified levels is described,
taking into account information about the dynamics of
epidemiological processes in countries of the spreading chain. The
CBRR software module has been developed in the MATLAB environment,
which implements the proposed approach and intelligent proprietary
algorithms for constructing trajectories of predicted epidemic
indicators.\\[1mm]
\textit{Keywords}: modeling, forecasting, COVID-19 epidemic,
percentage rate of increase, case-based reasoning, heuristic.
\par}

\vskip 5mm

\noindent \textbf{References} }

\vskip 2mm

{\footnotesize

1. Novel Coronavirus Global Research and Innovation Forum: Towards
a Research Roadmap. \textit{WHO}. Available at:
www.who.int/emergencies/diseases/novel-coronavirus-2019\slash
global-research-on-novel-coronavirus-2019-ncov (accessed: June 15,
2020).

2. Layne S.\,P., Hyman J.\,M., Morens D.\,M., Taubenberger J.\,K.
New coronavirus outbreak: Framing questions for pandemic
prevention. \textit{Sci. Transl. Med.}, 2020, vol.~12, iss.~534,
no.~eabb1469. \\https://doi.org/10.1126/scitranslmed.abb1469

3. Wu J.\,T., Leung K., Leung G.\,M. Nowcasting and forecasting
the potential domestic and international spread of the 2019-nCoV
outbreak originating in Wuhan, China: a modelling study.
\textit{Lancet}, 2020, vol.~395, iss.~10225, pp.~689--697.
https://doi.org/10.1016/S0140-6736(20)30260-9

4. Models of Infectious Disease Agent Study. \textit{MIDAS
Coordination Center}. Available at: https://midasnetwork.us/
(accessed: June 15, 2020).

5. Mandal M., Jana S., Nandi S.\,K., Khatua A., Adak S., Kar
T.\,K. A model based study on the dynamics of COVID-19: Prediction
and Control. \textit{Chaos, Solitons and Fractals}, 2020,
vol.~136, no.~109889. https://doi.org/10.1016/j.chaos.2020.109889

6. Fanelli D., Piazza F. Analysis and forecast of COVID-19
spreading in China, Italy and France. \textit{Chaos, Solitons and
Fractals}, 2020, vol.~134, no.~109761.
https://doi.org/10.1016/j.chaos.2020.109761

7. Bekirosab S., Kouloumpou D. SBDiEM: A new mathematical model of
infectious disease dynamics. \textit{Chaos, Solitons and
Fractals}, 2020, vol.~136, no.~109828.
https://doi.org/10.1016/j.chaos.2020.109828

8. Barmparis G.\,D., Tsironis G.\,P.~Estimating the infection
horizon of COVID-19 in eight countries with a data-driven
approach. \textit{Chaos, Solitons and Fractals}, 2020, vol.~135,
no.~109842. \\https://doi.org/10.1016/j.chaos.2020.109842

9. Kondratyev M.\,A. Forecasting methods and models of disease
spread. \textit{Computer Research and Modeling}, 2013, vol.~5,
iss.~5, pp.~863--882.

10. Schmidt R., Waligora T. Influenza forecast: Case-Based
Reasoning or statistics? \textit{Proceedings of the 11th
International conference on knowledge-based intelligent
information and engineering systems. Pt~I. Series Lecture Notes in
Computer Science}, 2007, vol.~4692, pp.~287--294.

11. Viboud C., Boelle P.\,Y., Carrat F., Valleron A.\,J., Flahault
A. Prediction of the spread of influenza epidemics by the method
of analogues. \textit{American Journal of Epidemiology}, 2003,
vol.~158, iss.~10, pp.~996--1006.

12. \textit{Johns Hopkins Coronavirus Resource Center}. Available
at: https://coronavirus.jhu.edu/data (accessed: June 29, 2020).


\vskip 1.5mm

Received:  July 20, 2020.

Accepted: August 13, 2020.
%
%
\vskip4.2mm A~u~t~h~o~r'~s \ i~n~f~o~r~m~a~t~i~o~n:
%
%
\vskip2mm \textit{Victor V. Zakharov} ---  Dr. Sci. in Physics and
Mathematics, Professor; v.zaharov@spbu.ru
%
\vskip2mm \textit{Yulia E. Balykina} --- PhD in Physics and
Mathematics, Associate Professor; j.balykina@spbu.ru\par
%
}
