
{\normalsize

\vskip 4.5%6
mm

\noindent{\bf On the theory of constructive construction of a
linear controller%$^{*}$ }

}

\vskip 2.5%3
mm

{\small

\noindent{\it A.~M.~Kamachkin, N.~A.~Stepenko, G.~M.~Chitrov%$\,^1$%
%, V.~M.~Bure$\,^{1,2}$%
%, O.~A.~Mitrofanova$\,^{1,2}$%
%, E.~P.~Mitrofanov$\,^{1,2}$%
%, I.~�. Famylia%$\,^2$%

 }

\vskip 2.5%3
mm%2mm

%%%%%%%%%%%%%%%%%%%%%%%%%%%%%%%%%%%%%%%%%%%%%%%%%%%%%%%%%%%%%%%%%%

%\efootnote{
%%
%\vspace{-3mm}\parindent=7mm
%%
%\vskip 0.1mm $^{*}$ This work was supported by the Russian Foundation for Basic Research (grant N~19-29-05184 mk).\par
%%
%%\vskip 2.0mm
%%
%%\indent{\copyright} �����-������������� ���������������
%%�����������, \issueyear%
%%
%}

%%%%%%%%%%%%%%%%%%%%%%%%%%%%%%%%%%%%%%%%%%%%%%%%%%%%%%%%%%%%%%%%%%

{\footnotesize

\noindent%
%$^2$~%
St.\,Petersburg State University, 7--9, Universitetskaya nab., St.\,Petersburg,

\noindent%
%\hskip2.45mm%
199034, Russian Federation

}

%%%%%%%%%%%%%%%%%%%%%%%%%%%%%%%%%%%%%%%%%%%%%%%%%%%%%%%%%%%%%%%%%%

\vskip3mm


\noindent \textbf{For citation:}  Kamachkin A.~M., Stepenko N.~A.,
Chitrov G.~M. On the theory of constructive construction of a
linear controller. {\it Vest\-nik of Saint~Petersburg University.
Applied Mathematics. Computer
Science. Control Pro\-cesses}, %\,
\issueyear,
vol.~16,~iss.~\issuenum,~pp.~\pageref{p9}--\pageref{p9e}. \\
\doivyp/\enskip%
\!\!\!spbu10.\issueyear.\issuenum09  (In Russian)

\vskip3mm
%\newpage

{\leftskip=7mm\noindent The classical problem of stationary
stabilization with respect to the state of a linear
sta\-tio\-na\-ry control system is investigated. Efficient, easily
algorithmic methods for constructing controllers of controlled
systems are considered: the method of V.\:I.~Zubov and the method
of P.~Brunovsky. The most successful modifications are indicated
to facilitate the construction of a linear controller. A new
modification of the construction of a linear regulator is proposed
using the transformation of the matrix of the original system into
a block-diagonal form. This modification contains all the
advantages of both V.\:I.~Zubov's method and P.~Brunovsky's
method, and allows one to reduce the problem with multidimensional
control to the problem of stabilizing a set of independent
subsystems with scalar control for each subsystem.\\[1mm]
\textit{Keywords}: stabilization of movements, linear regulator,
controllable canonical forms.
\par}

\vskip6mm%5mm

\noindent \textbf{References} }

\vskip 3mm%2mm

{\footnotesize

1. Zubov~V.~I. \textit{Teoriya optimal'nogo upravleniya sudnom i
drugimi podvizhnymi ob"ek\-ta\-mi} [\textit{Theory of optimal
control of a ship and other mobile objects}]. Leningrad,
Sudostroenie Publ., 1966, 352~p. (In Russian)

% {Smir}
2. Smirnov~E.~Ya. \textit{Stabilizaciya programmnyh dvizhenij}
[\textit{Stabilization of program move\-ments}].
St.~Pe\-ters\-burg, Saint Petersburg University Press, 1997,
307~p. (In Russian)

% {Klmn}
3. {Kalman~R.~E., Falb~P.~L., Arbib~M.~A.} \emph{Topics in
mathematical system theory}. Second ed. New York, McGraw-Hill Book
Company Press, 1969, 358~p.

% {Bru}
4. {Brunovsky~P.~A.} {A classification of linear controllable
systems}. \textit{Kybernetika}, 1970, vol.~6, no.~3, pp.~173--188.

% {LeoShum}
5. {Leonov~G.~A., Shumafov~M.~M.} \textit{Metody stabilizacii
linejnyh upravlyaemyh sistem} [\textit{Methods of stabilization of
linear controlled systems}]. St. Petersburg, Saint Petersburg
University Press, 2005, 419~p. (In Russian)


% {Tam}
6. {Smirnov~N.~V., Smirnova~T.~E., Tamasyan~G.~Sh.}
\textit{Stabilizaciya programmnyh dvizhenij pri polnoj i nepolnoj
obratnoj svyazi} [\textit{Stabilization of program movements with
full and incomplete feedback}]. St.~Pe\-ters\-burg, SOLO Publ.,
2013, 131~p. (In Russian)

% {Lue}
7. {Luenberger~D.} {Canonical forms for linear multivariable
systems}. \textit{IEEE Transactions on Automatic Control}, 1967,
vol.~12, iss.~3, pp.~290--293.
https://doi.org/10.1109/TAC.1967.1098584

% {KKh}
8. {Kamachkin~A.~M., Shamberov~V.~N., Chitrov~G.~M.} {Normal
matrix forms to decom\-position and control problems for
manydimentional systems}. \textit{Vest\-nik of Saint~Petersburg
University. Applied Mat\-he\-matics. Computer Science. Control
Pro\-ces\-ses}, 2017, vol.~13,
iss.~4, pp.~417--430.\\
https://doi.org/10.21638/11701/spbu10.2017.408

\vskip1.5mm Received:  June 24, 2020.

Accepted: August 13, 2020.

\vskip6mm A\,u\,t\,h\,o\,r'\,s \ i\,n\,f\,o\,r\,m\,a\,t\,i\,o\,n:%

\vskip2mm \textit{Alexander M. Kamachkin}~--- Dr. Sci. in Physics and Mathematics, Professor; %\\
 a.kamachkin@spbu.ru

\vskip2mm \textit{Nikolai A. Stepenko}~--- PhD in Physics and
Mathematics, Associate
Professor; %\\
n.stepenko@spbu.ru

\vskip2mm \textit{Gennady M. Chitrov}~--- PhD in Physics and
Mathematics, Associate
Professor; %\\
chitrow@gmail.com

}
