
\newpage

{\normalsize

%\vskip 6mm

\noindent{\bf Model study of the influence of multi-joint muscles on the frequency\\ characteristics of the human body%$\,^*$%
}

}

\vskip 2mm

{\small

\noindent{\it V. P.~Tregubov, N. K. Egorova%$\,^1$%
%, I.~�. Famylia%$\,^2$%

 }

\vskip 2mm

%%%%%%%%%%%%%%%%%%%%%%%%%%%%%%%%%%%%%%%%%%%%%%%%%%%%%%%%%%%%%%%%%%

%\efootnote{
%%
%\vspace{-3mm}\parindent=7mm
%%
%\vskip 0.1mm $^{*}$ This work is supported by Russian Science
%Foundation (project N~18-71-00006).%\par
%%
%%\vskip 2.0mm
%%
%\indent{\copyright} �����-������������� ���������������
%�����������, \issueyear%
%%
%}

%%%%%%%%%%%%%%%%%%%%%%%%%%%%%%%%%%%%%%%%%%%%%%%%%%%%%%%%%%%%%%%%%%

{\footnotesize

\noindent%
%$^1$~%
St.\,Petersburg State University, 7--9, Universitetskaya nab.,
St.\,Petersburg,

\noindent%
%\hskip2.45mm%
199034, Russian Federation

}

%%%%%%%%%%%%%%%%%%%%%%%%%%%%%%%%%%%%%%%%%%%%%%%%%%%%%%%%%%%%%%%%%%

\vskip2.0mm%3mm

\noindent \textbf{For citation:}  Tregubov V. P., Egorova N. K.
Model study of the influence of multi-joint muscles on the
frequency characteristics of the human body. {\it Vestnik of
Saint~Pe\-ters\-burg University. Applied Mathematics. Computer
Science. Control Processes}, \issueyear, vol.~16, iss.~\issuenum,
pp.~\pageref{p5}--\pageref{p5e}. \\
\doivyp/\enskip%
\!\!\!spbu10.\issueyear.\issuenum05  (In Russian)

\vskip2.0mm%3mm

{\leftskip=7mm\noindent It is noted that the Kelvin---Voigt model
is unsuitable for describing some polymers and biological tissues.
In these cases, a three-component combination of elements, which
consists of a spring and damper, connected in parallel and a
spring sequentially attached to them, is used. The force
characteristic of such a combination includes not only the strain,
strain rate, and force, but also the rate of force change.
Examples of such systems are the blood vessel wall and the
intervertebral disc, which have been given special attention.
Since the motion of such systems is described by an ordinary
third-order differential equation, they are classified as systems
with a non-integer number of degrees of freedom. For a single-mass
oscillating system with one and a half degrees of freedom, a
transfer function was constructed using the Laplace transform. In
addition, the amplitude-frequency response (AFR) was also
constructed. Analysis of this characteristic showed that
increasing the damping coefficient from zero to infinity first
leads to a decrease in its maximum to a certain non-zero value,
and then to an increase and reaching infinity with an infinite
value of the damping coefficient. The same feature is demonstrated
on a two-mass system of chain structure, each link of which has
one and a half degrees of freedom. A sequential combination of
seven such links was used to model the lumbar spine in the
structure of a General body model of a sitting person subject to
vertical vibration. Multi-link elastic-viscous joints were used to
model the multi-articular muscles of the lumbar spine. Additional
experimental studies are needed to determine the numerical values
of the parameters of the proposed model.\\[1mm]
\textit{Keywords}: mechanical system, amplitude-frequency
response, non-integer number of degrees of freedom, resonant
frequency.
\par}

\vskip 5 mm

\noindent \textbf{References} }

\vskip 2 mm

{\footnotesize

1. {Toth R.}  Multiple degree-of-freedom nonlinear spinal model.
{\it Proceeding of 19th~Annual Conference on Engineering in
Medicine and Biology}.  San Francisco, California, 1967,
pp.~126--132.

2. { Bai Xian-Xu, Xu Shi-Xu, Cheng Wei, Qian Li-Jun.} On
4-degree-of-freedom bio\-dynamic models of seated occupants:
Lumped-parameter modeling. {\it Journal of Sound and Vibration},
2017, vol.~402, no.~18, pp.~122--141.

3. {Thompson G.~T.} In vivo determination of mechnical properties
of the human ulna by mechanical impedance tests: Experimental
results and improved mathematical model. {\it Medical and
Biological Engineering}, 1976, vol.~14, pp.~253--262.

4. {Orne D.} The in vivo driving-point impedance of the human ulna
--� a viscoelastic beam modal. {\it Journal of Biomechanics},
1974, vol.~7, pp.~ 249--257.

5. {Kizilova N.~N.} Presser wave propagation in liquid field. {\it
Fluid dynamics}, 2006, vol.~41, no.~3, pp.~434--446.

6. {Andronov A.~A., Haikin S.~E.} {\it Teoriya kolebanij} [{\it
Theory of oscillations}]. Moscow, Leningrad, ONTI NKTP~USSR Publ.,
1937, 519~p. (In Russian)

7. {Andronov A.~A., Witt A.~A., Haikin S.~E.} {\it Teoriya
kolebanij} [{\it Theory of oscillations}]. Moscow, Fizmatgiz
Publ., 1959, 916~p. (In Russian)

8. {Panovko Ya.~G., Gubanova I.~I.} {\it Ustojchivost' i
kolebaniya uprugih sistem}  [{\it Stability and vibrations of
elastic systems}].  Moscow, Nauka Publ., 1987, 352~p. (In Russian)

9. {Orne D.,~King L. Y.} A mathematical model of spinal response
to impact. {\it Journal of Biomechanics}, 1971, vol.~4, no.~1,
pp.~49--71.

10.{ Zharnov A.~M., Zharnova O.~A.} Biomekhanicheskie processy v
mezhpozvonkovom diske shejnogo otdela pozvonochnika pri ego
dvizhenii [Biomechanical processes in the intervertebral disc of
the cervical spine during its movement]. {\it  Russian Journal of
Biome\-cha\-nics}, 2013, vol.~17, no.~1, pp.~32�-40. (In Russian)

11. {Tregubov V.~P., Egorova N.~K.} Model'noe izuchenie vliyaniya
mnogosustavnyh myshc  na chas\-totnye harakteristiki tela
cheloveka [Model study of the influence of multi-joint muscles on
the fre\-quen\-cy characteristics of the human body]. {\it Vestnik
of Saint Peters\-burg University. Applied \,Mathematics.
Com\-pu\-ter Science. Control\, Processes}, 2020, vol.~16,
iss.~2, pp.~150--164.\\
https://doi.org/10.21638/11701/spbu10.2020.207 (In Russian)

12. {Ciach M., Awrejcewicz J., Maciejczak A., Radek M.}
Experimental and numerical investigations of C5--C6 cervical
spinal segment before and after discectomy using the Cloward
operation technique. {\it Acta of Bioengineering and
Biomechanics}, 1999, vol.~1, no.~1, pp.~101--105.

\vskip 1.5mm

Received:  June 24, 2020.

Accepted: August 13, 2020.
%
%
\vskip4.2mm A~u~t~h~o~r'~s \ i~n~f~o~r~m~a~t~i~o~n:
%
%
\vskip2mm \textit{Vladimir P.~Tregubov} --- Dr. Sci. in Physics and Mathematics, Professor; %\\
v.tregubov@spbu.ru
%
\vskip2mm \textit{Nadezhda K. Egorova} --- Postgraduate Student;
nadezhda\_ego@mail.ru\par
%
}
