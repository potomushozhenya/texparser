

{\footnotesize

\vskip 4mm
%\newpage

\noindent {\small\textbf{References} }

\vskip 3mm

1. Sklyarov~A.~A., Sklyarov~S.~A. Sinergeticheskij podkhod k
upravleniyu bespilotnym letatel'nym apparatom v srede s vneshnimi
vozmushheniyami [Synergistic approach to the quadrocopter control
in environment with external perturbations].  \emph{News of SFedU.
Technical science}, 2012, no.~8, pp.~159--170. (In Russian)

2.  Bresciani~T. \emph{Modeling,  identification  and  control of
a  quadrotor  helicopter}.  Master thesis. Sweden, Lund, Lund
University, 2008, 184~p.

3.  Garcia~Carrillo~L.~R., Dzul~A., Lozano~R., Pegard~C.
\emph{Quad Rotorcraft Control: Vision-Based Hovering and
Navigation.} London, Heidelberg, New York, Dordrecht, Springer,
2012, 179~p.

4. Popkov~A.~S., Smirnov~N.~V., Baranov~O.~V. Real-time
quadrocopter optimal sta\-bi\-li\-za\-tion. \emph{Intern.
Conference ``Stability and Control Processes'' in Memory of
V.~I.~Zubov}. Saint Petersburg, October 5--9, 2015, pp.~123--125.

5. Popov~N.~I., Emel'yanova~O.~V. Dinamicheskie osobennosti
monitoringa vozdushnykh linij e'lektroperedachi s pomoshh'yu
kvadrokoptera [Dynamic features of overhead high-voltage lines
monitoring by means of quadrocopter].  \emph{Sovremennye problemy
nauki i obrazovaniya} [\emph{Modern problems of science and
education}], 2014, no.~2. Available at: http://cyberleninka.ru/
(accessed: 04.02.2016). (In Russian)\newpage

6. Popov~N.~I., Emel'yanova~O.~V., Yacun~S.~F., Savin~A.~I.
Issledovanie kolebanij kvadrokoptera pri vneshnikh periodicheskikh
vozdejstviyakh [Research of ocsillations of quadrocopter under
influence of external periodic disturbance]. \emph{Fundamental'nye
issledovaniya} [\emph{Fundamental research}], 2014, no.~1,
pp.~28--32. (In Russian)

7. Baranov~O.~V. Modelirovanie processa upravleniya bespilotnym
letatel'nym apparatom --- kvadrokopterom [Modeling of the control
process of quadrocopter UAV]. \emph{Processy upravleniya i
ustojchivost'} [\emph{Control Processes and Stability}], 2015,
vol.~2, no.~1. pp.~23--28. (In Russian)

8. {\it Polyotnye kontrollery} [{\it Flight controllers}].
Available at: http://multicopterwiki.ru/index.php/
Polyotnye\_kontrollery (accessed: 22.01.2016). (In Russian)

9. Efimov~B. \emph{Programmiruem kvadrokopter na Arduino (chast'
1)} [\emph{On programming of Arduino-quadrocopter}]. Available at:
http://habrahabr.ru/post/227425/ (accessed: 22.01.2016). (In
Russian)

10. Fyorsman~P., Kashvix~S., Kryuger~T., Shnetter~P., Vilkens~S.
Integrirovannaya navigacionnaya sistema na osnove ME'MS dlya
adaptivnogo upravleniya poletom bespilotnogo apparata [MEMS-based
integrated navigation system for adaptive flight control of
unmanned aircraft]. \emph{Giroskopiya i navigaciya}
[\emph{Gyroscopy and Navigation}], 2013, no.~1, pp.~3--18. (In
Russian)

11. Moskalenko~A. \emph{Ispol'zovanie inercial'noj navigacionnoj
sistemy $($INS$)$ s neskol'kimi datchikami na primere zadachi
stabilizacii vysoty kvadrokoptera} [\emph{Using an IMU with
multiple sensors on the problem of stabilization of quadrocopter
height}]. Available at:  http://geektimes.ru/post/255736/
(accessed: 22.01.2016). (In Russian)

12. Popkov~A.~S., Baranov~O.~V., Smirnov~N.~V. Application of
adaptive method of linear programming for technical objects
control. \emph{Intern. Conference on Computer Technologies in
Physical and Engineering Applications $($ICCTPEA$)$}. Ed.
E.~I.~Veremey. Saint Petersburg, 2014, pp.~141--142.

13. Kovalenko~V.~V. \emph{Malogabaritnaya inercial'naya sistema}:
uchebnoe posobie. [\emph{Compact inertial system}: a guide].
Chelyabinsk, SUSU Publ., 2010, 53~p. (In Russian)

14. Branec~V.~N., Shmyglevskij~I.~P. \emph{Vvedenie v teoriyu
besplatformennykh inercial'nykh naviga-\linebreak cionnykh sistem}
[\emph{Introduction to the theory of IMU}]. Moscow, Nauka Publ.,
1992, 280~p. (In Russian)

15. Xajang~Chao, Kelvin~Kupmans, Long~Di, Yang~Kvan~Chen.
\emph{Sravnitel'naya ocenka byudzhet-\linebreak nykh inercial'nykh
izmeritel'nykh blokov dlya bespilotnykh letatel'nykh apparatov}
[\emph{Comparative analysis\linebreak of budget inertial
measurement unit for unmanned aerial vehicles}]. Available at:
http://blaskor.ru/
ru/sravnitelnaya-otsenka-byudzhetnykh-inertsialnykh-izmeritelnykh-blokov-dlya-bespilotnykh-letatelnykh- 
apparatov.html (accessed: 22.01.2016). (In Russian)

16. Klyuenkov~A.~L. Realizaciya adaptivnogo metoda v odnoj zadache
optimal'nogo upravleniya [Implementation of an adaptive method for
optimal control problem].  \emph{Processy upravleniya i
ustojchivost'} [\emph{Control Processes and Stability}], 2015,
vol.~2, no~1, pp.~53--58. (In Russian)

17. Smirnov~N.~V., Solov'eva~I.~V. Primenenie metoda pozicionnoj
optimizacii dlya mnogo-\linebreak programmnoj stabilizacii
bilinejnykh sistem [Application of the positional optimization
method for multi-\linebreak program stabilization of bilinear
systems]. \emph{Vestnik of Saint Petersburg University. Series 10.
Applied mathematics. Computer science. Control processes}, 2009,
issue~3, pp.~253--261. (In Russian)





}
