

{\footnotesize

\vskip 2mm
%\newpage

\noindent {\small\textbf{References} }

\vskip 1.5mm



1. Vasil'ev F. P. {\it Metody optimizacii} [{\it Optimization
methods}]. Moscow, Publ. House ``Faktorial Press'', 2002, 824~p.
(In Russian)

2. Zubov V. I. {\it Obshhij metod mnozhitelej Lagranzha i
optimizacija processov v sploshnyh sredah} [{\it The general
method of Lagrange multipliers and optimization of processes in
continuous media}]. Doct. dis. Moscow, RAS, 2002, 250~p. (In
Russian)

3. Albu A. F., Zubov V. I. Issledovanie zadachi optimal'nogo
upravlenija processom kristallizacii veshhestva v novoj postanovke
dlja ob'ekta slozhnoj geometricheskoj formy [Research objectives
of optimum control the process of crystallization agent at the new
formulation for object complex geometric shapes]. {\it Zh.
vychisl. matem. i matem. fiz.} [\textit{J. of Calcul. mathematics
and mathem. physics}], 2014, vol.~54, no.~12, pp.~1879--1893. (In
Russian)

4. Gukasov A. K., Gukasova E. V. Chislennoe reshenie zadachi
optimal'nogo upravlenija granicej fazovogo perehoda [The numerical
solution of optimal control problem of the phase transition
boundary]. {\it Fundamental'nye  issledovanija} [\textit{Basic
Research}], 2014, no.~12--11, pp.~2325--2329. (In Russian)

5. Krektuleva R. A., Batranin A. V. Sovmestnoe reshenie obratnoj
zadachi teploprovodnosti i zadachi optimal'nogo proektirovanija v
tehnologii svarki neplavjashhimsja jelektrodom [The joint solution
inverse problem of heat conduction and the problem of optimal
design at technology of welding with non-consumable electrode].
{\it Izvestija Tomskogo politehnicheskogo universiteta}
[\textit{Bulletin of the Tomsk Polytechnic University}], 2012,
vol.~320, no.~2, pp.~104--109. (In Russian)

6. Mel'nikova Ju. S. Matematicheskoe modelirovanie upravlenija
nestacionarnym temperaturnym polem v dvuhfaznyh sredah
[Mathematical modeling of time-dependent temperature field in
two-phase media]. {\it Nauka i obrazovanie} [\textit{Science and
education}], 2012, no.~2. The electronic edition. Avaiable at:
http://technomag.edu.ru/doc/330390.html (accessed: 12.02.2016) (In
Russian)

7. Buchko N.\,A. Jental'pijnyj metod chislennogo reshenija zadach
teploprovodnosti v promer-\linebreak zajushhih ili protaivajushhih
gruntah [Enthalpy method of numerical solutions problems of heat
con-\linebreak duction at freeze or thawing of soil]. {\it
SPbGUNTiPT}. Avaiable at: http://refportal.com/upload/files/
entalpiiny\_metod\_chislennogo\_resheniya.pdf (accessed:
12.02.2016)
    (In Russian)

8. Vasil'ev V. I., Maksimov A. M., Petrov E. E., Tsypkin G. G.
Mathematical model of the freezing-thawing of saline frozen soil.
\textit{Journal of Applied Mechanics and Technical Physics}, 1995,
vol.~36, issue~5, pp.~689--696.

9. Nekrasov S. A. {\it Interval'nye i dvustoronnie metody dlja
rascheta s garantirovannoj tochnost'ju jelektricheskih i magnitnyh
sistem} [{\it Interval and bilateral methods of calculation with
guaranteed accuracy of electric and magnetic systems}]. Doct. dis.
Novocherkassk, South-Russian State Politechnical University, 2002,
310~p. (In Russian)

10. Nekrasov S. A. Modelirovanie fazovyh perehodov pervogo roda
metodom integral'nyh uravnenij v sluchae stacionarnogo
peremeshhajushhegosja poverhnostnogo istochnika [Modeling of phase
transitions of the first kind by the method of integral equations
in the case of a stationary moving surface source]. {\it
Inzh.-fiz. zhurn.} [\textit{Engineering and Physical Journal}],
1994, vol.~66, no.~6, pp.~754--757. (In Russian)

11. Nekrasov S. A. Zadacha Stefana [Stefan problem]. Pt I. {\it
Differencial'nye uravnenija} [\textit{Differential Equations}],
1996, vol.~32, no.~8, pp.~1114--1121. (In Russian)

12. Nekrasov S. A. Zadacha Stefana [Stefan problem]. Pt II. {\it
Differencial'nye uravnenija} [\textit{Differential Equations}],
1996, vol.~32, no.~9, pp.~1254--1258. (in Russian)






}
