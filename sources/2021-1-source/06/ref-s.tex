
{\normalsize

\vskip 6mm

\noindent{\bf Conceptual model of a data transmission network for a polymodal\\ control system in critical state infrastructure%$^{*}$%
}

}

\vskip 3%2
mm

{\small

\noindent{\it S. I. Saitov$^1$, V. Yu. Budkov$^2$, D. K.  Levonevsky$^2$, A. V. Denisov$^2$%
%, I.~�. Famylia%$\,^2$%

}

\vskip 3%2
mm

%%%%%%%%%%%%%%%%%%%%%%%%%%%%%%%%%%%%%%%%%%%%%%%%%%%%%%%%%%%%%%%%%%

%\efootnote{
%%
%\vspace{-3mm}\parindent=7mm
%%
%\vskip 0.1mm $^{*}$ This work was supported by the Russian Foundation for Basic Research (grant N~20-07-01086).%\par
%%
%%\vskip 2.0mm
%%
%%\indent{\copyright} �����-������������� ���������������
%%�����������, \issueyear%
%%
%}

%%%%%%%%%%%%%%%%%%%%%%%%%%%%%%%%%%%%%%%%%%%%%%%%%%%%%%%%%%%%%%%%%%

{\footnotesize

\noindent%
$^1$~%
Information Technologies, Mechanics and Optics University,


\noindent%
\hskip2.45mm%
49, Kronverkskiy pr.,
St.\,Petersburg, 197101, Russian Federation



\noindent%
$^2$~%
St. Petersburg Federal Research Center of the Russian Academy of
Sciences,


%\noindent%
%\hskip2.45mm%
%St. Petersburg Institute for informatics and automation of the Russian Academy of Sciences,

\noindent%
\hskip2.45mm%
39, 14-ia linia V. O., St.\,Petersburg, 199178, Russian Federation

%\noindent%
%$^1$~%
%St.\,Petersburg State University, 7--9, Universitetskaya nab.,
%St.\,Petersburg,
%
%\noindent%
%\hskip2.45mm%
%199034, Russian Federation


}

%%%%%%%%%%%%%%%%%%%%%%%%%%%%%%%%%%%%%%%%%%%%%%%%%%%%%%%%%%%%%%%%%%

\vskip3mm%3mm


\noindent \textbf{For citation:}  Saitov S. I., Budkov V. Yu., Levonevsky D. K., Denisov A. V. Conceptual model of a data transmission network for a polymodal control system in critical state infrastructure. {\it Vestnik of Saint~Petersburg Uni\-ver\-si\-ty.
Ap\-plied Mat\-he\-matics. Computer Science. Control
Processes},\,\issueyear,
vol.~17,~iss.~\issuenum,~pp.~\pageref{p6}--\pageref{p6e}. %\\
\doivyp/\enskip%
\!\!\!spbu10.\issueyear.\issuenum06  (In Russian)

\vskip3mm

{\leftskip=7mm\noindent Approaches to modeling a data transmission network in a remotely controlled system for distributed critical state facilities are considered in the article. To solve the problem of increasing the efficiency of channel resource costs, a new analytical model of the information distribution system from the standpoint of teletraffic theory is proposed. The model describes the service of extraordinary calls with absolute priorities, failures and gradient reservation of the channel resource for messages of modalities with the characteristics of operators in the data transmission network of critical facilities.\\[1mm]
\textit{Keywords}: channel resource costs, data transmission network, multimodal information presentation, system of priority service of calls with obvious losses, probability call loss.
\par}

\vskip6mm

\noindent \textbf{References} }

\vskip 2mm

{\footnotesize

1. Shneps-Shneppe M. A. \textit {Telekommunikacii Pentagona:
cifrovaja transformacija i kiberzashhita} [\textit{Pentagon
telecommunications: digital transformation and cyber defense}].
Moscow, Gorjachaja linija--Telekom Publ., 2017, 272~p. (In
Russian)

2. \textit{Ob odobrenii Kontseptsii federal'noi sistemy
monitoringa kriticheski vazhnykh ob"ektov i (ili)
po\-ten\-tsial'no opasnykh ob"ektov infrastruktury Rossiiskoi
Federatsii i opasnykh gruzov $[$On approval of the Concept of the
federal monitoring system for critical facilities and (or)
potentially hazardous infrastructure facilities of the Russian
Federation and dangerous goods$]$}. Rasporjazhenie Pravitel'stva
RF ot 27 avgusta 2005 g. no. 1314-r [Guidelines of Russian
Federation Government  N 1314-r at August 27, 2005].   Moscow,
Sobranie zakonodatel'stva Russian Federation Publ., 2015. (In
Russian)

3. Saitov I. A., Tregubov R. B. \textit {Teoreticheskie osnovy analiza i optimizacii ierarhicheskih mnogourovnevyh marshrutizirujushhih sistem} [\textit{Theoretical foundations of analysis and optimization of hierarchical multilevel routing systems}]. Orel, Akademy of FSO of the Russian Federation Publ., 2017, 587~p. (In Russian)

4. Goncharov S. M., Borshevnikov A. E. Ispolzovanie tekhnologij vysokonadezhnoj biometricheskoj autentifikacii v kriticheski vazhnyh objektah [Use of highly reliable biometric authentication technologies in critical facilities]. \textit {Informacionnaja bezopasnost' regionov} [\textit{Information security of regions}], 2015, vol.~4, iss.~21, pp.~18--23. (In Russian)

5. Saitov S. I. Mnogomodalnaya dinamicheskaya autentifikaciya tekhnicheskogo personala kriticheski vazhnyh objektov [Multimodal dynamic authentication of technical personnel of critical objects]. \textit {Sovremennye materialy, tehnika i tehnologii} [\textit{Modern materials, equipment and technology}], 2017, vol.~4, iss.~12, pp.~36--39 (In Russian)

6. Gnidko K. O., Lomako A. G. Kontrol potencialnogo opasnogo vzaimodejstviya na individualnoe i gruppovoe soznanie potrebitelej multimedijnogo kontenta [Control of potential dangerous interactions on individual and group consciousness of multimedia content consumers]. \textit {Trudy SPIIRAN} [\textit{Proceedings of SPIIRAS}], 2015, vol.~1, pp.~9--33. https://doi.org/10.15622/sp.38.2  (In Russian)

7. Basov O. O. \textit {Modeli i metod sinteza polimodal'nyh infokommunikacionnyh sistem} [\textit{Models and method of synthesis of polymodal information and communication systems}]. Dis. Dr. Sci. in Engineering. Orel, Akademy of FSO of the Russian Federation Publ., 2016, 292~p. (In Russian)


8. Basov O. O., Karpov A. A., Saitov I. A. \textit
{Metodologicheskie osnovy sinteza polimodal'nyh
info\-kom\-muni\-kacionnyh sistem gosudarstvennogo upravlenija}
[\textit {Methodological basis for the synthesis of polymodal
infocommunication systems of state administration}]. Orel, Akademy
of FSO of the Russian Federation Publ., 2015, 263~p. (In Russian)

9. Saitov S. I., Igol'nikov V. K., Basov O. O., Saitov I. A. \textit {Sposob mul'tipleksirovanija cifrovyh signalov pri mnogomodal'nom predstavlenii informacii} [\textit{Method for multiplexing digital signals with multi-modal presentation of information}]. Patent RF no. 2674463. Declare Yanuary 31, 2018; publ. December 11, 2018, bull. no.~35, 14~p. (In Russian)

10. Iversen V. �. \textit{Teletraffic engineering and network planning}. Kgs. Lyngby, Technical University of Denmark Publ., 2015, 382~p.

11. Ross K. \textit{Multiservice loss models for broadband telecommunication networks}. London, Springer Publ., 1995, 343~p. https://doi.org/10.1007/978-1-4471-2126-8


12. Grimm C., Schluchtermann G. \textit{IP traffic theory and
performance}. Berlin, Heidelberg, Sprin\-ger-Verlag Publ.,
2008,
487~p. (Springer Series on Signals and Communication Technology). \\ https://doi.org/10.1007/978-3-540-70605-2

13. Stepanov S. N. \textit{Osnovy teletrafika mul'tiservisnyh setej} [\textit{The basics of teletraffic multiservice networks}]. Moscow, Jeko-Trendz Publ., 2010, 392~p. (In Russian)

14. Stepanov S. N. \textit{Teorija teletrafika: koncepcii, modeli, prilozhenija} [\textit{Teletraffic theory: concepts, models, applications}]. Moscow, Gorjachaja linija--Telekom Publ., 2015, 868~p. (In Russian)

15. ITU-T Recommendation I.350. \textit{General aspects of quality of service and network performance in digital networks, including ISDNs}. Geneva, ITU-T Publications Press, 1993.


16. Kornilov S. A. Model' zvena mul'tiservisnoj seti
sleduyushchego pokoleniya s prioritetnoj dis\-ciplinoj
obsluzhivaniya [Next-generation multiservice network link model
with priority service dis\-cip\-li\-ne]. \textit
{Telekommunikacii} [\textit{Telecommunications}], 2017, no.~10,
pp.~35--42. (In Russian)


\vskip1.5mm Received:  October 01, 2020.

Accepted: January 15, 2021.


\vskip6mm A~u~t~h~o~r~s' \ i~n~f~o~r~m~a~t~i~o~n:%

\vskip2mm \textit{Sergey I. Saitov} --- Postgraduate Student;
sami.stv@mail.ru

\vskip2mm \textit{Viktor Yu. Budkov} --- PhD in Engineering,
Senior Research Collaborator; visharmail@gmail.com

\vskip2mm \textit{Dmitrii K. Levonevsky} --- PhD in Engineering, Research Collaborator; DLewonewski.8781@gmail.com

\vskip2mm \textit{Alexandr V. Denisov} --- Junior Research Collaborator; sdenisov93@mail.ru

}
