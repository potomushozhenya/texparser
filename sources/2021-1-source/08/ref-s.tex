
{\normalsize

\vskip 6mm

\noindent{\bf Asymptotic properties and stabilization of  a neutral type system\\ with constant delay%$^{*}$%
}

}

\vskip 3%2
mm

{\small

\noindent{\it B. G. Grebenshchikov%$\,^{2,3}$%
%, I.~�. Famylia%$\,^2$%

}

\vskip 3%2
mm

%%%%%%%%%%%%%%%%%%%%%%%%%%%%%%%%%%%%%%%%%%%%%%%%%%%%%%%%%%%%%%%%%%

%\efootnote{
%%
%\vspace{-3mm}\parindent=7mm
%%
%\vskip 0.1mm $^{*}$ This work was supported by the Russian Foundation for Basic Research (grant N~20-07-01086).%\par
%%
%%\vskip 2.0mm
%%
%%\indent{\copyright} �����-������������� ���������������
%%�����������, \issueyear%
%%
%}

%%%%%%%%%%%%%%%%%%%%%%%%%%%%%%%%%%%%%%%%%%%%%%%%%%%%%%%%%%%%%%%%%%

{\footnotesize

%\noindent%
%$^1$~%
%St.\,Petersburg State University, 7--9, Universitetskaya nab.,
%St.\,Petersburg,
%
%\noindent%
%\hskip2.45mm%
%199034, Russian Federation

\noindent%
%$^2$~%
Ural Federal University named after the first President of Russia
B. N. Yeltsin,

\noindent%
%\hskip2.45mm%
19, ul. Mira,
Yekaterinburg, 620002, Russian Federation
}

%%%%%%%%%%%%%%%%%%%%%%%%%%%%%%%%%%%%%%%%%%%%%%%%%%%%%%%%%%%%%%%%%%

\vskip3mm%3mm


\noindent \textbf{For citation:} Grebenshchikov B. G.
 Asymptotic properties and stabilization of  a neutral type system with constant delay. {\it Vestnik of Saint~Petersburg Uni\-ver\-si\-ty.
Ap\-plied Mat\-he\-matics. Computer Science. Control
Processes},\,\issueyear,
vol.~17,~iss.~\issuenum,~pp.~\pageref{p8}--\pageref{p8e}. \\
\doivyp/\enskip%
\!\!\!spbu10.\issueyear.\issuenum08  (In Russian)

\vskip3mm

{\leftskip=7mm\noindent The problem of obtaining sufficient
conditions for the asymptotic stability for a certain class of
linear systems of a neutral type with constant delay is analyzed
in the article. Some coefficients of these systems in the right
side have an exponential factor. As a consequence, the study of
the stability of such systems with the help of the
Lyapunov---Krasovskii functionals is not possible; methods of
receiving asymptotic appreciations lead to extremely rough
results. By applying the apparatus of difference systems and the
properties of simpler systems, which the author examined previous,
sufficient conditions for the exponential stability of such
systems are obtained. As an example, a second-order system is
considered. The graphs of the solutions of the corresponding
system, both without neutral members and with the original system
where the right-hand side contains neutral terms, are provided. On
the basis of theory difference systems, the author proposes an
algorithm of stabilization for some systems of a similar type.\\[1mm]
\textit{Keywords}: delay, exponential stability, difference systems, stabilization, control.
\par}

\vskip6mm

\noindent \textbf{References} }

\vskip 3%2
mm

{\footnotesize

1. Bellman R., Cooke K.  {\it Differential-difference equations}.
New  York, Academic Press, 1963, 480~p. (Rus. ed.: Bellman R.,
Cooke K.  {\it Differencial'no-raznostnyje uravnenija}.  Moscow,
Mir Publ., 1967, 548~p.)

2. Elsholz L. E., Norckin C. B. {\it Vvedenie v teoriju
differencialnykh  uravnenij s otcklonja\-jushchimcja argumentom}
[{\it Introduction to theory of differential  equations with
deviat argu\-ment}]. Moscow, Nauka Publ., 1971, 296~p. (In Russian)


3. Vlasov V. V., Ivanov S. A. Ocenky resheniy neodnorodnykh
differential'no-raznostnykh uravneniy neytralnogo typa [Estimates
of solutions  of non ordinary differential-difference equations of
neutral type]. {\it Izvestiya vuzov, Matematica} [{\it Russian
Mathematics}], 2006, no.~3, pp.~24--30. (In Russian)

4. Grebenshchikov B. G., Rozhkov V. I. Asymptoticheskoe povedenie
odnoy stacionarnoy systemy  s zapazdyvaniem [Asymptotical behavior
once system with delay]. {\it Differenz. uravneniya} [{\it
Differential Equations}],  1993, vol.~29, no.~5, pp.~751--758. (In Russian)

5.  Kolmanovskiy V. B., Nosov V. R. {\it Ustoichivost' i
periodicheskie rezhimy reguliruemyh system s posledeystviem} [{\it
Stability of periodical regimes of regular systems delay}].
Moscow,  Nauka Publ., 1984, 448~p. (In Russian)

6.  Ockendon J. R., Tayler A. B. The dynamics of a current
collection system for an electric locomotive.  {\it Proc. Soc.
London. Ser.~A.}, 2002, vol.~322, pp.~447--468.
https://doi.org/10.1098/rspa.1971.0078


7. Markvardt N. G., Vlasov I. I. {\it Contactnaia set'} [{\it
Contact net}]. Moscow, Transport Publ., 1978, 325~p. (In Russian)

8. Grebenshchikov B. G., Novikov S. I. O neustojchivosti
nekotoroj sistemy s linejnym zapazdyvaniem [Instability of systems
with linear  delay reduceble to singularty pertubed ones]. {\it Izvestija vuzov. Matematika} [{\it Russian Mathematics}], 2010, no.~2, pp.~1--10.
 https://doi.org/10.3103/S1066369x10020015 (In Russian)


9. Zhabko A. P., Tihomirov O. G., Chizhova O. N. O
stabilizacii klassa sistem s  proporcional'nym zapazdyvaniem [About
stabilization same class of system with  proportional delay]. {\it
Vestnik of Saint Petersburg University. Applied Mathematics.
Computer Sciences. Control Processes},  2018, vol.~14, iss.~2,
pp.~165--172. (In Russian)

10.  Sesekin A. N., Shlyakov A. S. On the stability of
discontinuous solutions of bilinear   systems with impulse action,
constant and linear delays. {\it AIP Conference
Proceedings}, 2019, vol.~2172, no.~030009,  pp.~1--5.

11. Barbashin E. A. {\it Vvedenije v teoriju ustoichivosty
dvigenija} [{\it Introduction to theory of  stability of motion}].
Moscow, Nauka Publ., 1967, 225~p. (In Russian)


12. Grebenshchikov B. G. Ob ustoichivosti  lineinyh   system s
postoiannymi zapazdyvaniem i  exponential'nymi coefficientami
[About stability some linear systems with constant delay and
exponential coefficients]. {\it Mathematicheskiy analiz. Voprosy
theoryi, historyi i methodiki prepodovaniya} [{\it Mathematical
analis. Questions of theory, history and methodology of
educations}]. Eds by N.~M.~Matveev. Leningrad, 1990, pp.~38--48.
(In Russian)


13. Grebenshchikov B. G., Loznikov A. B.  Asymptoticheskie
svoistva necotoryh system s li\-neinym zapazdyvaniem.
Functional'no-differentzial'nye uravneniya:  theoryia i
prilozhe\-niya [Asymptotical properties of some systems with
linear delay]. {\it Materialy conferentsii posvyashchennoy
95-letiyu so dnya rozh\-de\-niya professora N.~V.~Azbeleva} [{\it
Proceedings of conference with memory of year born prof.
N.~V.~Az\-be\-lev}.] Perm, Perm Research Polytechnic University
Press, 2018, pp.~51�59. (In Russian)

14. Halanai A., Wexler D. {\it Kachestvennaia teoryia impulsnykh
system}  [{\it Qualitaive theory of puls systems}]. Moscow, Mir
Publ., 1971, 312~p. (In Russian)


15.  Fihtengolz G. M.  {\it Differenzialnoe i integralnoe
ischislenie.} V~3~t. [{\it Differential and integral calculus.}
In~3~vol.].  Moscow,  Nauka Publ., 2002, vol.~3, 655~p. (In
Russian)\pagebreak


16. Grebenshchikov B. G.   Ob ustoichivosti  nestazionarnyx system
s bol'shim  zapaz\-dyvaniem. {\it Ustoichivost' i nelineinye
kolebaniya} [{\it Stability and non linear oscilations}]. Eds by
S.~N.~Shimanov. Sverdlovsk, Sverdlovsk State University Press, 1984, pp.~18--19. (In Russian)


17. Grebenshchikov B. G. Ob ustojchivosti po pervomu priblizheniju
odnoj   nestacionarnoj sistemy s zapazdyvaniem [About stability by
first approximation some unstationar  system  with delay]. {\it
Izvestija vuzov. Matematika} [{\it Russian Mathematics}], 2012,
no.~2, pp.~34--42.\\
https://doi.org/10.3103/3103/S1066369x12020041 (In Russian)


18. Grebenshchikov B. G.  Ob ustoichivosti  stazionarnykh  system
lineinykh  differencialnykh uravnenij neutralnogo typa s
zapazdyvaniem, line'no zaviciashchim ot vremeni [About stability
stationary  systems of linear differential delay equations, whose
delay linear depends of time]. {\it Izvestiya vuzov, Matematica}
[{\it Russian mathematics}], 1991, no.~7, pp.~69--71. (In Russian)

19. Grebenshchikov B. G. O stabilizacii stationarnykh lineinykh
system s zapazdyvaniem, lineino zaviciashchim ot vremeny [About
stabilization stationary linear systems with delay, linear depends
of time]. {\it Izvestiya vuzov, Matematica} [{\it Russian
Mathematics}],  1991, Dep. VINITY, no.~4384--92. (In Russian)


\vskip1.5mm Received:  October 02, 2020.

Accepted: January 15, 2021.


\vskip6mm A~u~t~h~o~r~s' \ i~n~f~o~r~m~a~t~i~o~n:%

\vskip2mm \textit{Boris G. Grebenshchikov} --- PhD in Physics and Mathematics;
b.g.grebenshchikov@urfu.ru

}
