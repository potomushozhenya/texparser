
{\normalsize

\vskip 6mm

\noindent{\bf Markov moment for the agglomerative method of
clustering\\ in Euclidean space}

}

\vskip 2mm

{\small

\noindent{\it A.\ V.\ Orekhov%$\,^1$%
%, I.~�. Famylia%$\,^2$%
%, I.~�. Famylia%$\,^2$%

 }

\vskip 2mm

%%%%%%%%%%%%%%%%%%%%%%%%%%%%%%%%%%%%%%%%%%%%%%%%%%%%%%%%%%%%%%%%%%

{\footnotesize

\noindent%
%$^1$~%
St.\,Petersburg State University, 7--9, Universitetskaya nab.,
St.\,Petersburg,

\noindent%
%\hskip2.45mm%
199034, Russian Federation

%\noindent%
%$^2$~%
%St.\,Petersburg State University, 7--9, Universitetskaya nab.,
%St.\,Petersburg,

%\noindent%
%\hskip2.45mm%
%199034, Russian Federation

}

%%%%%%%%%%%%%%%%%%%%%%%%%%%%%%%%%%%%%%%%%%%%%%%%%%%%%%%%%%%%%%%%%%

\vskip3mm


\noindent \textbf{For citation:}  Orekhov A. V. Markov moment for
the agglomerative method of clustering in Euclidean space. {\it
Vestnik of Saint~Petersburg University. Applied Mathematics.
Computer Science. Control Processes},\,\issueyear,
vol.~15,~iss.~\issuenum,~pp.~\pageref{p6}--\pageref{p6e}.
\doivyp/\enskip%
\!\!\!spbu10.\issueyear.\issuenum06  (In Russian)

\vskip3mm

{\leftskip=7mm\noindent When processing large arrays of empirical data
or large-scale data, cluster analysis remains one of the primary methods
of preliminary typology, which makes it necessary to obtain formal rules
for calculating the number of clusters. The most common method for
determining the preferred number of clusters is the visual analysis
of dendrograms, but this approach is purely heuristic. The number of
clusters and the end moment of the clustering algorithm depend on each other.
Cluster analysis of data from $n$-dimensional Euclidean space using
the ``single linkage'' method can consider as a discrete random process.
Sequences of ``minimum distances'' define the trajectories of this
process. The ``approximation-estimating test'' allows us to establish
the Markov moment when the growth rate of such a sequence changes from
linear to parabolic, which, in turn, may be a sign of the completion
of the agglomerative clustering process. The calculation of the number
of clusters is the critical problem in many cases of the automatic
typology of empirical data. For example, in medicine with cytometric
analysis of blood, automated analysis of texts and in other instances
when the number of clusters not known in advance.\\[1mm]
\textit{Keywords}: cluster analysis, least squares method, Markov
moment.
\par}

\vskip5mm

\noindent \textbf{References} }

\vskip 2mm

{\footnotesize

1.\, Everitt B. S.\; {\it Cluster analysis.} Chichester, West
Sussex, UK, John Wiley \& Sons Ltd. Press, 2011, 330~p.

2.\, Duda R. O., Hart P. E., Stork D. G.\; {\it Pattern
classification.} 2nd ed. New York, Chichester, Wiley Press, 2001.
654~p.

3.\, Calirnski T., Harabasz J.\; A dendrite method for cluster
analysis. {\it Communications in\linebreak Statistics}, 1974,
no.~3, pp.~1--27.

4.\, Baxter M. J.\; {\it Exploratory multivariate analysis in
archaeology}. Edinburgh, Edinburgh University Press, 1994, 307~p.

5.\, Sugar C. A., James G. M.\; Finding the number of clusters in
a dataset. {\it Journal of the American Statistical Association},
2003, vol.~98, no.~463, pp.~750--763.

6.\, Granichin O. N., Shalymov D. S., Avros R., Volkovich Z.\;
{Randomizirovannyy algoritm nakhozhdeniya kolichestva klasterov}
[A randomized algorithm for estimating the number of clusters].
{\it Avtomatika i telemekhanika} [{\it Automation and Remote
Control}], 2011, no.~4, pp.~86--98. (In Russian)

7.\, Shalymov D. S.\; Randomizirovannyy metod opredeleniya
kolichestva klasterov na mnozhestve dannykh. [Randomized method
for determining the number of clusters on a data set]. {\it
Nauch\-no-tekh\-ni\-ches\-kiy vestnik Sankt-Peterburgskogo
gosudarstvennogo universiteta informatsionnykh tekhnologiy,
mekhaniki i optiki} [{\it Scientific and Technical Gazette of
Saint Petersburg State University of Information Technologies,
Mechanics and Optics}], 2009, no.~5~(63), pp.~111--116. (In
Russian)

8.\, Zhang G., Zhang C., Zhang H.\; Improved $k$-means algorithm
based on density Canopy. {\it Knowledge-Based Systems}, 2018,
vol.~145, pp.~1--14.

9.\, Jiali W., Yue Z., Xv L.\; Automatic cluster number selection
by finding density peaks.  {\it 2016 2nd IEEE International
Conference on Computer and Communications (ICCC)}. {\it IEEE
Proceedings}. Chengdu, China, 2016, no.~7924655, pp.~13--18. doi:
10.1109 / CompComm.2016.7924655

10.\, Cordeiro de Amorim R., Hennig C.\; Recovering the number of
clusters in data sets with noise features using feature rescaling
factors.  {\it Information Sciences}, 2015, vol.~324,
pp.~126--145.

11.\, Lozkins A., Bure V. M.\; Veroyatnostnyy podkhod k
opredeleniyu lokal'no-optimal'nogo chisla klasterov [A
probabilistic approach to determining the locally optimal number
of clusters]. {\it Vestnik of Saint Petersburg University. Applied
Mathematics. Computer Science. Control Processes}, 2016, vol.~13,
iss.~1, pp.~28--37. (In Russian)

12.\, Shalymov D. S.\; Algoritmy ustoychivoy klasterizatsii na
osnove indeksnykh funktsiy i funktsiy ustoychivosti  [Algorithms
for stable clustering based on index functions and stability
functions]. {\it Stokhasticheskaya optimizatsiya v informatike}
[{\it Stochastic optimization in computer science}], 2008, vol.~4,
no.~1-1, pp.~236--248. (In Russian)

13.\, Steinhaus H.\; Sur la division des corps materiels en
parties. {\it Bull. Acad. Polon. Sci. C1. III}, 1956, vol.~IV,
pp.~801--804.

14.\, Lloyd S.\;  Least squares quantization in PCM. {\it IEEE
Transactions on Information Theory}, 1982, vol.~28, iss.~2,
pp.~129--137. doi: 10.1109/TIT.1982.1056489

15.\, Hartigan J. A.\; {\it Clustering algorithms}. New York,
London, Sydney, Toronto, John Wiley \& Sons Inc. Press, 1975,
351~p.

16.\, Aldenderfer M. S., Blashfield R. K.\; {\it Cluster
analysis.} Newburg Park, Sage Publications Inc. Press, 1984, 88~p.

17.\, Wald A.\; {\it Sequential analysis.} New York, John Wiley \&
Sons Inc. Press, 1947, 212~p.

18.\, Sirjaev A. N.\; {\it Statistical sequential analysis:
Optimal stopping rules.} New York, American Mathematical Society
Publ., 1973, 174~p.

19.\, Shiryaev A. N.\; {\it Optimal stopping rules.}  Berlin,
Heidelberg, Springer Press, 2009, 220~p.

20.\, Orekhov A. V.\; Criterion for estimation of stress-deformed
state of SD-materials. {\it AIP Conference Proceedings}, 2018,
vol.~1959, pp.~070028. doi: 10.1063/1.5034703

21.\, Orekhov A. V.\; Approksimatsionno-otsenochnyye kriterii
napryazhenno-deformiruyemogo sostoyaniya tverdogo tela
[Approximation-evaluation tests for a stress-strain state of
deformable solids]. {\it Vestnik of Saint Petersburg University.
Applied Mathematics. Computer Science. Control Processes}, 2018,
vol.~14, iss.~3, pp.~230--242.
doi.org/10.21638/11702/spbu10.2018.304 (In Russian)

22.\, McCaffrey J.\; Test run --- $k$-means++ data clustering.
{\it MSDN Magazine}, 2015,  vol.~30, no.~8, pp.~62--68.

23.\, Zurochka A. V., Khaydukov S. V., Kudryavtsev I. V.,
Chereshnev V. A.\; {\it Protochnaya tsitometriya v meditsine i
biologii}. 2-e izd. [{\it Flow cytometry in medicine and biology}.
2nd ed.]. Yekaterinburg, Ural Branch of the Russian Academy of
Sciences Publ., 2014, 574~p. (In Russian)

24.\,  Lappin S., Fox C.\; {\it The handbook of contemporary
semantic theory}. 2nd ed.\; Wiley-Blackwell, Wiley Press, 2015,
776~p.

\vskip1.5mm Received:  February 28, 2018.

Accepted: December 18, 2018.


\vskip6mm A\,u\,t\,h\,o\,r's \ i\,n\,f\,o\,r\,m\,a\,t\,i\,o\,n:%

\vskip2mm \textit{Andrey V. Orekhov} --- Senior Lecturer;
A\_V\_Orehov@mail.ru

}
