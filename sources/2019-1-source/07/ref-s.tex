
{\normalsize

\vskip 6mm

\noindent{\bf Modeling of pipe flows$\,^*$}

}

\vskip 2mm

{\small

\noindent{\it V. A. Pavlovsky$\,^1$%
, A. L. Chistov$\,^2$%
, D. M. Kuchinsky$\,^1$%

 }

\vskip 2mm

\efootnote{
%
\vspace{-3mm}\parindent=7mm
%%
\vskip 0.1mm $^{*}$ The work is supported by Russian Fundamental
Research (grand N 16-08-00890).\par
%
}

%%%%%%%%%%%%%%%%%%%%%%%%%%%%%%%%%%%%%%%%%%%%%%%%%%%%%%%%%%%%%%%%%%

{\footnotesize

\noindent%
$^1$~%
St. Petersburg State Marine Technical University, 3, Locmanskaya
ul., St. Petersburg,

\noindent%
\hskip2.45mm%
190121, Russian Federation

\noindent%
$^2$~%
St.\,Petersburg State University, 7--9, Universitetskaya nab.,
St.\,Petersburg,

\noindent%
\hskip2.45mm%
199034, Russian Federation

}

%%%%%%%%%%%%%%%%%%%%%%%%%%%%%%%%%%%%%%%%%%%%%%%%%%%%%%%%%%%%%%%%%%

\vskip3mm


\noindent \textbf{For citation:}  Pavlovsky V. A., Chistov A. L.,
Kuchinsky D. M. Modeling of pipe flows. {\it Vestnik of
Saint~Petersburg University. Applied Mathematics. Computer
Science. Control Processes},\,\issueyear,
vol.~15,~iss.~\issuenum,~pp.~\pageref{p7}--\pageref{p7e}.
\doivyp/\enskip%
\!\!\!spbu10.\issueyear.\issuenum08  (In Russian)

\vskip3mm

{\leftskip=7mm\noindent Lots of technical devices use flows in
pipes and channels caused by pressure drop, along with one's axis,
which is energy consuming and has to be estimated.    For the
estimation resistant coefficient, dependent on flow regime and
streamlined surface roughness, is required. Turbulence $f$-model
applicable for calculation for both laminar and turbulent flow and
smooth and rough walls is used for investigation. The problem of
incompressible viscous liquid steady flow in a smooth round pipe
is considered for different Reynolds numbers. First integrals for
velocity profile and turbulence measure are obtained in form
of transcendental equations and solved by Newton's method for
algebraic equation system. Calculated results are compared with
data from alternative theoretical approaches and experiments.\\[1mm]
\textit{Keywords}: pipe flow, viscosity, $f$-model of turbulence,
Reynolds number, pressure difference, differential equations,
boundary conditions, velocity profile, resistance coefficient.
\par}

\vskip5mm

\noindent \textbf{References} }

\vskip 2mm

{\footnotesize

1. Lojczyanskij L. G. \emph{Mehanika zhidkosti i gaza}
[\emph{Fluid and gas mechanics}]. Moscow, Drofa Publ., 2003,
840~p. (In Russian)

2. Schlichting H. \emph{Grenzschicht Theorie} [\emph{Boundary
layer theory}]. Berlin, Verlag G. Braun Publ., 1965, 736~p. (Rus.
ed.: Schlichting H. \emph{Teoriya pogranichnogo sloya}. Moscow,
Nauka Publ., 1974, 711~p.)

3. Pavlovskij V. A., Nikushhenko D. V. \emph{Vychislitelnaya
gidrodinamika. Teoreticheskie osnovy} [\emph{Computational Fluid
Dynamics. Theoretical fundamentals}]. Saint Petersburg, Lan'
Publ., 2018, 368~p. (In Russian)

4. Robertson J. M., Martin J. D., Burkhurt T. H. Turbulent flow in
rough pipes. \emph{Ind. Eng. Chem. Fundam.}, 1963, no.~7,
pp.~253--265.

5. Jimenz J. Turbulent flow over rough walls. \emph{Annu. Rev.
Fluid Mech.}, 2004, no.~36, pp.~173--196.

6. Sedova O., Pronina Y. A new model for the mechanochemical
corrosion of a thin spherical shell. \emph{EPJ Web of
Conferences}, 2016, vol.~108. 02040, pp.~1--6.

7. Patel V. C. Perspective: flow at high Reynolds number and over
rough surfaces --- Achilles heel of CFD. \emph{J. Fluids Eng}.,
1998, no.~120, pp.~434--444.

8. Moody L. F. Friction factors for pipe flow. \emph{Trans. ASME},
November 1944, vol.~66, pp.~671--684.

9. Tullis J. P., Wang J.-S. Turbulent flow in the entry region of
a rough pipe. \emph{ASME J. Fluids Eng}., 1974, no.~75,
pp.~62--68.

10. Pavlovskij V. A. Ob odnoj fenomenologicheskoj alternative
gipoteze dliny puti peremeshivaniya [About a phenomenological
alternative to the mixing length hypothesis]. \emph{Modeli
mehaniki sploshnoj sredy}. \emph{Sb. Fizicheskaya mehanika.
Vyp.~7} [\emph{Models of continuum mechanics}. \emph{Physical
mechanics digest. release. Iss.~7}]. Pod red. B.~V.~Filippova.
Saint Petersburg, Saint Petersburg State University Publ., 1998,
pp.~21--35. (In Russian)

11. Pavlovskij V. A. Uchet sherohovatosti stenki dlya edinoj
fenomenologicheskoj modeli techeniya vyazkoj zhidkosti pri
proizvolnyh chislah Rejnoldsa [Accounting for wall roughness for a
unified phenomenological model of viscous fluid flow at arbitrary
Reynolds numbers].     \emph{Problemy ekonomii
toplivno-energeticheskih resursov na predpriyatiyah i TES:
mezhvuz. sb. nauch. trudov} [\emph{Problems of saving fuel and
energy resources in enterprises and thermal power plants: an
intercollege digest of scientific papers}]. Saint Petersburg, SPb
GTU RP Publ., 2002, pp.~11--17. (In Russian)

12. Nikuradze J. \emph{Laws of flow in rough pipes} (English
translation of Str\"{o}mungsgesetze in rauhen Rohren,
VDI-Forschungsheft 361, Ausgabe B, Bd 4, pp.~1--22, July/August
1933), p.~63. Washington, NACA Technical Memo 1292 Press, November
1950.

13. Korotkin A. I., Rogovoj Yu. A. \emph{Metod rascheta prodol'nyh
srednih skorostej v pristennyh turbulentnyh techeniyah
neszhimaemoj zhidkosti}     [\emph{Calculation method for
longitudinal average velocities of near-wall turbulent
incompressible fluid flows}].  Saint Petersburg, MorVest Publ.,
2009, 121~p. (In Russian)


\vskip1.5mm Received:  May 8, 2018.

Accepted: December 18, 2018.


\vskip6mm A\,u\,t\,h\,o\,r's \ i\,n\,f\,o\,r\,m\,a\,t\,i\,o\,n:%

\vskip2mm \textit{Valery A. Pavlovsky}~--- Dr. Sci. in Physics and
Mathematics, Professor; v.a.pavlovsky@gmail.com

\vskip2mm \textit{Alexey L. Chistov}~--- PhD in Physics and
Mathematics, Associate Professor; a.chistov@spbu.ru

\vskip2mm \textit{Dmitry M. Kuchinsky}~--- PhD in Technics,
Associate Professor; kuchinskiy-dm@bk.ru

}
