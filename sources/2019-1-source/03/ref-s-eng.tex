

{\small



\vskip6mm

\noindent \textbf{References} }

\vskip 2mm

{\footnotesize

1. Altman E., Barman D.,  El Azouzi R., Jimenez T. A game
theoretic approach for delay minimization in slotted ALOHA.
\textit{2004 IEEE Intern. Conference on Com\-mu\-ni\-ca\-tions}
(IEEE Cat. no. 04CH37577), 2004, vol.~7, pp.~3999--4003.


2.  Marb\'{a}n S., van de Ven P., Borm P., Hamers H. ALOHA
networks: a game-theoretic approach. \textit{Math. Meth. Oper.
Res.}, 2013, vol.~78, iss.~2, pp.~221--242.


3. Sagduyu Y. E., Ephremides A. A game-theoretic look at simple
relay channel.  \textit{Wireless Networks}, 2006, vol.~12, no.~5,
pp.~545--560.

4. Bure V. M.,  Parilina E. M.  Stohasticheskie modeli peredachi
dannyh v setyah s razlichnymi topologiyami [Stochastic models of
data transmission in networks with different topologies].
\textit{Large-Scale Systems Control}, 2017, vol.~68, pp.~6--29.
(In Russian)

5. Bure V. M.,  Parilina E. M. Igra ``Mnozhestvennyj dostup''\ s
nepolnoj informaciej [Multiple access game with imperfect
information]. \textit{Mathematical Game Theory and
Ap\-pli\-ca\-tions}, 2017, vol.~9(4), pp.~3--17.  (In Russian)

6.  Inaltekin  H., Wicker S. B. The analysis of Nash equilibria of
the one-shot random-access game for wireless networks and the
behavior of selfish nodes. \textit{IEEE/ACM Transactions on
Networking}, 2008, vol.~16, no.~5, pp.~1094--1107.

7. Afghah F., Razi A.,  Abedi A. Stochastic game theoretical model
for packet forwarding in relay network. \textit{Telecommunication
Systems}, 2013, vol.~52, iss.~4, pp.~1877--1893.


8. Fink A. M. Equilibrium in a stochastic $n$-person game.
\textit{J. Sci. Hirosima Univ. Series A-I}, 1964, vol.~28,
pp.~89--93.

9. Koutsoupias E., Papadimitriou C. Worst-case equilibria.
\textit{Proceedings of the 16th Annual Symposium on Theoretical
Aspects of Computer Science}, 1999, pp.~404--413.

10.  Herings P. J.-J., Peeters R. Homotopy methods to compute
equilibria in game theory. \textit{Econ. Theory}, 2010, vol.~42,
pp.~119--156.

11. Lemke C. E., Howson J. T. Equilibrium points of bimatrix
games. \textit{J. Soc. Indust. Appl. Math.}, 1964, vol.~12,
pp.~413--423.

12. Raghavan T. E. S., Filar J. A. Algorithms for stochastic games
--- a survey. \textit{ZOR --- Methods and Models of Operations
Research}, 1991, vol.~35, pp.~437--472.


\vskip 1.5mm

%\noindent Recommendation: prof. L. A. Petrosyan.
%
%\vskip 1.5mm

%\noindent

Received:  October 18, 2018.

Accepted: December 18, 2018.

\vskip6mm A\,u\,t\,h\,o\,r's \ i\,n\,f\,o\,r\,m\,a\,t\,i\,o\,n:

\vskip2mm\textit{Vladimir M. Bure} --- Dr. Sci. in Physics and
Mathematics, Professor; v.bure@spbu.ru

\vskip2mm\textit{Elen� M. Parilina} --- PhD in Physics and
Mathematics, Associate Professor; e.parilina@spbu.ru

}
