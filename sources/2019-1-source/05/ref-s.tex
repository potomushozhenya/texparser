
{\normalsize

\vskip 6mm

\noindent{\bf Analysis of the dynamics of charged particles in an
ideal Penning trap\\ with a rotating field and a buffer gas}

}

\vskip 2mm

{\small

\noindent{\it A. D. Ovsyannikov%$\,^1$%
%, I.~�. Famylia%$\,^2$%
%, I.~�. Famylia%$\,^2$%

 }

\vskip 2mm

%%%%%%%%%%%%%%%%%%%%%%%%%%%%%%%%%%%%%%%%%%%%%%%%%%%%%%%%%%%%%%%%%%

{\footnotesize

\noindent%
%$^1$~%
St.\,Petersburg State University, 7--9, Universitetskaya nab.,
St.\,Petersburg,

\noindent%
%\hskip2.45mm%
199034, Russian Federation

%\noindent%
%$^2$~%
%St.\,Petersburg State University, 7--9, Universitetskaya nab.,
%St.\,Petersburg,

%\noindent%
%\hskip2.45mm%
%199034, Russian Federation

}

%%%%%%%%%%%%%%%%%%%%%%%%%%%%%%%%%%%%%%%%%%%%%%%%%%%%%%%%%%%%%%%%%%

\vskip3mm


\noindent \textbf{For citation:}  Ovsyannikov A. D. Analysis of
the dynamics of charged particles in an ideal Penning trap with a
rotating field and a buffer gas. {\it Vestnik of Saint~Petersburg
University. Applied Mathematics. Computer Science. Control
Processes},\,\issueyear,
vol.~15,~iss.~\issuenum,~pp.~\pageref{p5}--\pageref{p5e}.
\doivyp/\enskip%
\!\!\!spbu10.\issueyear.\issuenum05  (In Russian)

\vskip3mm

{\leftskip=7mm\noindent The paper deals with particle dynamics in
a Penning trap with a rotating electric dipole field and a buffer
gas. Electromagnetic traps are widely used for the accumulation
and storage of charged particles of matter and antimatter for
further use in various experiments. In this paper, a general
analytical criterion is established, which must satisfy the
parameters of the type of trap under investigation in order to
provide compression or expansion modes of the trajectory beam.
These modes correspond to the cases of asymptotic stability or
instability of the system under study. The most effective
combinations of parameters were determined, providing the maximum
possible degree of stability (or, accordingly, instability) of the
system with the minimum possible amplitudes of the rotating
electric field. Analytical solutions are constructed for the rapid
calculation and analysis of the behavior of individual particles
or envelopes of an ellipsoidal beam of trajectories and an
estimate of the radius of the accumulated cloud. The proposed
approach is applicable to the analysis of the system for any
values of the parameters of the
studied model of particle dynamics in a trap.\\[1mm]
\textit{Keywords}: Penning trap, Rotating Wall,
Penning---Malmberg---Surko trap, charged particle dynamics,
stability.
\par}

\vskip5mm

\noindent \textbf{References} }

\vskip 2mm

{\footnotesize

1. Isaac C. A., Baker C. J., Mortensen T., van der Werf D. P.,
Charlton M. Compression of positron clouds in the independent
particle regime. \textit{Physical Review Letters}, 2011, vol.~107,
pp.~033201(1--4).

2.  Isaac C. A. Motional sideband excitation using rotating
electric fields. \textit{Physical Review A}, 2013, vol.~87,
pp.~043415(1--7).

3.  Eseev M. K., Meshkov I. N.  Traps for storing charged
particles and antiparticles in high precision experiments.
\textit{Phys. Usp.}, 2016, vol.~59, pp.~304--317.

4.  Meshkov I. N., Eseev M. K., Ovsyannikov A. D., Ovsyannikov D.
A., Ponomarev V. A. Analysis of the particle dynamics stability in
the Penning---Malmberg---Surko trap. \textit{Proc. XXV Russian
Particle Accelerator Conference}. Saint Petersburg, 2016, pp.~64.
doi: 10.18429/JACoW-RuPAC2016-WECAMH03


5.  Meshkov I. N., Ovsyannikov A. D., Ovsyannikov D. A., Eseev M.
K. Study of the stability of charged particle dynamics in a
Penning---Malmberg---Surko trap with a rotating field.
\textit{Papers of Academy of Science RAN}, 2017, vol.~476, no.~6,
pp.~630--634. doi: 10.7868/S0869565217300065


\vskip1.5mm Received:  November 9, 2018.

Accepted: December 18, 2018.


\vskip6mm A\,u\,t\,h\,o\,r's \ i\,n\,f\,o\,r\,m\,a\,t\,i\,o\,n:%

\vskip2mm \textit{Alexander D. Ovsyannikov} --- PhD in Physics and
Mathematics, Associate Professor; ovs74@mail.ru

}
