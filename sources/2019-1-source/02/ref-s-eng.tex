
{\small

\vskip6mm

\noindent \textbf{References} }

\vskip 2mm

{\footnotesize

1. Chaubey N., Aggarwal A., Gandhi S., Jani K. A.  Performance
analysis of TSDRP and AODV routing protocol under black hole
attacks in MANETs by varying network size. {\it Advanced Computing
$\&$  Communication Technologies (ACCT)},  IEEE, 2015,
pp.~320--324.

2. Anjum S. S., Noor R. M., Anisi M. H. Survey on MANET based
communication scenarios for search and rescue operations. {\it IT
Convergence and Security (ICITCS)},  IEEE,  2015, pp.~1--5.

3. Erim O., Wright C. Optimized mobility models for disaster
recovery using UAVs. {\it 2017 IEEE\linebreak 28th\-An\-nual
Intern. Symposium on Personal, Indoor, and Mobile Radio
Communications (PIMRC)}. Montreal, QC, 2017, pp.~1--5. doi:
10.1109/PIMRC.2017.8292716

4. Benkaouha H., Abdelli A., Badache N., Ben-Othman J., Mokdad L.
Towards improving failure detection in mobile ad hoc networks.
{\it IEEE Global Communications Conference (GLOBECOM)}, IEEE,
2015, pp.~1--6.

5. Son T. T., Le Minh H., Sexton G., Aslam N., Boubezari R. A new
mobility, energy and congestion aware routing scheme for MANETs.
{\em Communication Systems, Networks $\&$  Digital Signal
Processing (CSNDSP)}, IEEE, 2014, pp.~771--775.

6. Blakeway S., Pullin A. The effect of node density on routing
protocol performance in mobile Ad-hoc Networks.  {\em Proc. of the
Convergence of Telecommunications, Networking and Broadcasting},
2007,  pp.~237--240.

7. Resende~C., Almulla~M., Boukerche~A. The use of Erasure Coding
for video streaming unicast over Vehicular Ad-hoc Networks. {\em
Local Computer Networks}, 2013, no.~38, pp.~715--718.

8. Sahingoz O. K. Mobile networking with UAVs: Opportunities and
challenges, {\em Intern. Conference on Unmanned Aircraft Systems
(ICUAS)}, 2013, pp.~933--941. doi: 10.1109/ICUAS.2013.6564779

9. Gupta L., Jain R., Vaszkun G. Survey of important issues in UAV
communication networks. {\em IEEE Communications Surveys $\&$
Tutorials}, 2016, vol.~18, no.~2, pp.~1123--1152. doi:
10.1109/COMST.2015.2495297

10. Zhu M., Cai Z., Zhao D., Wang J., Xu M. Using multiple
unmanned aerial vehicles to maintain\linebreak connectivity of
MANETs. {\em 23rd Intern. Conference on Computer Communication and
Networks (ICCCN)}. Shanghai, 2014, pp.~1--7. doi:
10.1109/ICCCN.2014.6911826

11. Xuelin C., Zuxun S. An overview of slot assignment (SA) for
TDMA. {\em Signal Processing, Communications and Computing
(ICSPCC)}, IEEE, 2015, pp.~1--5.

12. McKinsey J.~C.~C. \textit{Introduction to the Theory of
Games}. Dover, Dover Publ., 2003, 384~p.\newpage

13. Fudenberg D., Tirole J. \textit{Game Theory}. Cambridge, The
MIT Press, 1991, 604~p.

14. Poongothai T.,  Jayarajan K. A noncooperative game approach
for intrusion detection in Mobile Ad-hoc Networks. {\em Computing,
Communication and Networking (ICCCn)}, IEEE, 2008, pp.~1--4.

15. Wang F., Mo Y., Huang B. Defending reputation system against
false recommendation in Mobile Ad-hoc Network.  {\em Networking,
Sensing and Control}, IEEE, 2008, pp.~488--493.

16. Li F., Yang Y., Wu J. Attack and flee: game-theory-based
analysis on interactions among nodes in MANETs.  {\em Systems,
Man, and Cybernetics. Pt B. Cybernetics, IEEE Transactions on},
2010, vol.~40, no.~3, pp.~612--622.

17. Wang X., Feng R., Wu Y., Che S., Ren Y. A game theoretic
malicious nodes detection model in MANETs.  {\em Mobile Ad-hoc and
Sensor Systems}, IEEE, 2012, pp.~1--6.

18. Wang K., Wu M., Ding C., Lu W. Game-based modelling of node
cooperation in Ad-hoc Networks. {\em Wireless and Optical
Communications Conference (WOCC)}, 2010, pp.~1--5.

19. Ermolin N. A., Mazalov V. V., Pechnikov A. A.
Teoretiko-igrovye metody nakhozhdeniia soobshchestv v
akademicheskom Vebe [Game-theoretic methods for finding
communities in the academic Web]. {\em Trudy SPIIRAN}, 2017,
no.~55, pp.~237--254. (In Russian)

20. Gubanov D. A., Novikov D. A., Chkharteshvili A. G. {\it
Sotsialnye seti: modeli informatsionnogo vliianiia, upravleniia i
protivoborstva} [{\em Social networks: models of information
influence, control and confrontation}]. Moscow, Phizmatlit Publ.,
2010, 228~p. (In Russian)

21. Jackson M.~O. {\em Social and Economic Networks}. Princeton,
Princeton Univ. Press, 2008, 520~p.

22. Petrosyan L. A., Sedakov A. A. Multistage network games with
complete information. {\em Autom. Remote Control}, 2014, vol.~75,
no.~8, pp.~1532--1540.

23. Bulgakova M. A., Petrosyan L. A. Kooperativnye setevye igry s
poparnymi vzaimodeistviiami [Cooperative network games with
pairwise interactions]. {\em Mathematical game theory and its
applications}, 2015, vol.~7,  no.~38, pp.~7--18. (In Russian)

24. Parilina E. M. Cooperative game on sending data in the
wireless network.  {\em UBS}, 2010,  vol.~31, no.~1, pp.~191--209.

25. Novikov D. A. Games and networks.  {\em Automation and Remote
Control}, 2014, vol.~75, no.~6, pp.~1145-1154.

26. Han~Z., Niyato~D., Saad~W., Basar~T., Hjorungnes~A. {\em Game
Theory in Wireless and Communication Networks. Theory, Models, and
Applications}. New York, Cambridge University Press, 2012, 530~p.
 %%%%% baz

27. Bazenkov N. I. Double best response dynamics in topology
formation game for Ad-hoc Networks. {\em Autom. Remote Control},
2015, vol.~76, pp.~323--335. doi: 10.1134/S0005117915020125


28. Gromova~E., Gromov~D., Timonin~N., Kirpichnikova~A.,
Blakeway~S. A dynamic game of Mobile Agent placement in a MANET.
{\em Systems Informatics, Modelling and Simulation (SIMS)}, 2016,
pp.~153--158. doi: 10.1109/SIMS.2016.25

29. Plekhanova~T., Gromova~E., Gromov~D., Kirpichnikova~A.,
Blakeway~S.  The strategic placement of Mobile Agents on a
hexagonal graph using game theory. {\em Proc. of the IEEE
conference ICAT}, 2017. doi: 10.1109/ICAT.2017.8171635

30. Albert~R., Jeong~H., Barabasi~A. L. Diameter of the World-Wide
Web. {\em Nature}, 1999, vol.~401, pp.~130--131.

31. Petrosyan L., Zenkevich N., Shevkoplyas E.  {\it Teoriia igr}
[\textit{Game Theory}]. Saint Petersburg, BHV-Petersburg Publ.,
2012, 480~p. (in Russian)

32. Petrosyan L., Kuzyutin D.  {\it Ustoichivye resheniia
pozitsionnykh igr} [{\em Consistent solutions of positional
games}]. Saint Petersburg, Izd. Dom Saint Petersburg University
Press, 2008, 326~p. (in Russian)

33. Kuhn H. W. Extensive games and the problem of information,
{\it Contributions to the Theory of Games}, 1953, vol.~2, no.~28,
pp.~193--216.

34. Sanjab A., Saad W., Basar T. Prospect theory for enhanced
cyber-physical security of drone delivery systems: A network
interdiction game. {\em IEEE Intern. conference on Communications
(ICC)}. Paris, 2017, pp.~1--6.

35. Cormen T. H., Leiserson C. E.,  Rivest R. L., Stein C.
\textit{Introduction to algorithms}. 2nd ed. Cambridge, Cambridge
MIT Press, 2001, 1314~p.

36. Wehrle K., Gune\c{s} M., Gross J. (editors). \textit{Modeling
and Tools for Network Simulation.} Berlin, Springer Verlag Publ.,
2010, 547~p.

37. Gromova E. V., Plekhanova T. M. On the regularization of a
cooperative solution in a multi-\linebreak stage game with random
time horizon. {\em Discrete Applied Mathematics}, 2018, vol.~255,
pp.~40--55. https://doi.org/10.1016/j.dam. 2018.08.008


\vskip 1.5mm

%\noindent Recommendation: prof. L. A. Petrosyan.
%
%\vskip 1.5mm

%\noindent
Received:  June 27, 2018.

Accepted: December 18, 2018.

\vskip6mm A\,u\,t\,h\,o\,r's \ i\,n\,f\,o\,r\,m\,a\,t\,i\,o\,n:

\vskip1.5mm\textit{Blakeway Stewart} --- PhD, Lecturer;
s.blakeway@glyndwr.ac.uk

\vskip1.5mm\textit{Dmitry V. Gromov} --- Dr.-Ing., Associate
Professor; d.gromov@spbu.ru

\vskip1.5mm\textit{Ekaterina V. Gromova} --- Dr. Sci. in  Physics
and Mathematics, Professor; e.v.gromova@spbu.ru

\vskip1.5mm\textit{Anna S. Kirpichnikova} --- PhD, Lecturer;
anya@cs.stir.ac.uk

\vskip2mm\textit{Taissia M. Plekhanova} --- postgraduate student;
taisiiaplekhanova@gmail.ru


}
