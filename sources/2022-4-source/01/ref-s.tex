
{\normalsize

\vskip 6mm

\noindent{\bf Convergence conditions for continuous and discrete models\\ of population dynamics%$^{*}$%
}

}

\vskip 2mm

{\small

\noindent{\it A. Yu. Aleksandrov %

}

\vskip 2mm

%%%%%%%%%%%%%%%%%%%%%%%%%%%%%%%%%%%%%%%%%%%%%%%%%%%%%%%%%%%%%%%%%%

%\efootnote{
%%
%\vspace{-3mm}\parindent=7mm
%%
%\vskip 0.1mm $^{*}$ This work was supported by Russian %
%Foundation for Basic Research (project N 20-07-00531 A).%\par
%%
%%\vskip 2.0mm
%%
%%\indent{\copyright} �����-������������� ���������������
%%�����������, \issueyear%
%%
%}

%%%%%%%%%%%%%%%%%%%%%%%%%%%%%%%%%%%%%%%%%%%%%%%%%%%%%%%%%%%%%%%%%%

{\footnotesize


\noindent%
%$^3$~%
St\,Petersburg State University, 7--9, Universitetskaya nab.,
St\,Petersburg,

\noindent%
%\hskip2.45mm%
199034, Russian Federation



}

%%%%%%%%%%%%%%%%%%%%%%%%%%%%%%%%%%%%%%%%%%%%%%%%%%%%%%%%%%%%%%%%%%

\vskip2mm%3mm


\noindent \textbf{For citation:} Aleksandrov A. Yu.
Convergence conditions for continuous and discrete models of population dynamics. {\it Vestnik of Saint~Petersburg Uni\-ver\-si\-ty.
Ap\-plied Mat\-he\-matics. Computer Science. Control
Processes},\,\issueyear,
vol.~18,~iss.~\issuenum,~pp.~\pageref{p1}--\pageref{p1e}. \\
\doivyp/\enskip%
\!\!\!spbu10.\issueyear.\issuenum01 (In Russian)

\begin{hyphenrules}{english}

\vskip2mm

{\leftskip=7mm\noindent Some classes of continuous  and discrete  generalized Volterra
models of population dynamics are considered.
It is supposed that there are
relationships of the type
``symbiosis��, ``compensationism�� or ``neutralism�� between any two species in a biological community.
The objective of the work is to obtain conditions under which the investigated models
possess the  convergence property. This means that
the studying system admits a bounded solution that is globally asimptotically
stable.
To determine the required conditions, the V. I. Zubov's approach and its discrete-time counterpart are
used. Constructions of Lyapunov functions are  proposed, and with the aid of these functions,
the convergence problem for the considered models is reduced to the problem of
the existence of  positive solutions for some
 systems of linear algebraic  inequalities.
In the case where parameters of models are almost periodic functions, the fulfilment of
the derived conditions implies that limiting bounded   solutions are almost periodic, as well.
An example is presented  illustrating the obtained theoretical conclusions.\\[1mm]
\textit{Keywords}: population dynamics,  convergence,
almost periodic oscillations, asymptotic stability,  Lyapunov functions. \par}

\vskip6mm%5

\end{hyphenrules}

%\newpage

\noindent \textbf{References} }

\vskip 2mm

{\footnotesize

1. {Zubov V. I.} {\it Kolebanija v nelinejnyh i upravljaemyh sistemah} [{\it Oscillations in nonlinear and controlled systems}]. Leningrad, Sudpromgiz Publ., 1962, 631~p. (In Russian)

2. {Provotorov V. V., Sergeev S. M., Hoang V. N.} Point control of a dif\-fe\-rential-difference system with distributed parameters on the graph.
{\it Vestnik of Saint Petersburg University. Applied Mathematics. Computer Science. Control Processes},  2021, vol.~17, iss.~3, pp.~277--286. \\ https://doi.org/10.21638/11701/spbu10.2021.305

3. {Zhabko A. P., Provotorov V. V.,  Ryazhskikh V. I.,  Shindyapin A. I.}
Optimal control of a differential-difference parabolic systems with distributed parameters on the graph. {\it Vestnik of Saint Petersburg University. Applied Mathematics. Computer Science. Control Processes}, 2021, vol.~17, iss.~4, pp.~433--448. https://doi.org/10.21638/11701/spbu10.2021.411

4. Tkhai V. N. Model with coupled subsystems. {\it Automation and Remote Control}, 2013, vol.~74, no.~6, pp. 919--931.


5. Sedighi H. M.,  Daneshmand F. Nonlinear transversely vibrating beams by the homotopy perturbation method with an auxiliary term. {\it Journal of Applied and Computational Mechanics}, 2015, vol.~1,
no.~1, pp.~1--9.

6. {Park Y., Lee C.} Dynamic investigation of non-linear behavior of hydraulic cylinder in mold oscillator using PID control process. {\it Journal of Applied and Computational Mechanics}, 2021, vol.~7, no.~1,
pp.~270--276.

7. Aleksandrov A. Yu., Stepenko N. A. Stability analysis of gyroscopic systems with delay under synchronous and asynchronous switching.
{\it Journal of Applied and Computational Mechanics}, 2022, vol.~8, no.~3, pp.~1113--1119.

8. Demidovich B. P. {\it Lekcii po matematicheskoj teorii ustojchivosti}
[{\it Lectures on mathematical stability theory}].  Moscow, Nauka Publ.,
1967, 472~p. (In Russian)

9. Pavlov A., Pogromsky A., van de Wouw N., Nijmeijer H. Convergent dynamics, a tribute to Boris Pavlovich Demidovich. {\it Syst. Control Lett.}, 2004, vol.~52, pp.~257--261.

10. Ruffer B., van de Wouw N., Mueller M. Convergent systems vs. incremental stability. {\it Syst. Control Lett.}, 2013, vol.~62, pp.~277--285.

11. {Chen F.} Some new results on the permanence and extinction of
nonautonomous Gilpin\,---\,Ayala type competition model with delays. {\it
Nonlinear Analysis: Real World Applications}, 2006, vol.~7, pp.~1205--1222.

12. Aleksandrov A., Aleksandrova E. Convergence conditions for some
classes of nonlinear systems. {\it Syst. Control Lett.}, 2017, vol.~104, pp.~72--77.


13. Mei W., Efimov D., Ushirobira R., Aleksandrov A. On convergence conditions for generalized Persidskii systems. {\it Intern. Journal of Robust Nonlinear Control}, 2022, vol.~32, no.~6, pp.~3696--3713.

14. Pliss V. A. {\it Nonlocal problems of the theory of oscillations}. London, Academic Press, 1966, 306~p.

15. Ataeva N. N. Svojstvo konvergencii dlja raznostnyh sistem [The convergence property for difference systems]. {\it Vestnik of Saint Petersburg University. Series 10. Applied Mathematics. Computer Science. Control Processes}, 2004, iss.~4, pp.~91--98. (In Russian)

16. Yakubovich V. Matrix inequalities in stability theory for nonlinear control systems. III. Absolute stability of forced vibrations.
{\it Automation and Remote Control}, 1964, vol.~7, pp.~905--917.

17. {Nguen D. H.} Uslovija konvergencii nekotoryh klassov nelinejnyh
raznostnyh sistem [Convergence conditions for some classes
of nonlinear difference systems]. {\it Vestnik of Saint Petersburg University. Series 10. Applied Mathematics. Computer Science. Control Processes}, 2011, iss.~2, pp.~90--96. (In Russian)

18.  Aleksandrov A. Yu. Some convergence and stability conditions for nonlinear systems. {\it Differ. Equ.}, 2000, vol.~36, no.~4, pp.~613--615.

19. Kosov A. A., Shchennikov V. N. On the convergence phenomenon in complex almost periodic systems. {\it Differ Equ.} 2014, vol.~50, no.~12, pp.~1573--1583.

20. Strekopytov S. A., Strekopytova M. V. O konvergencii dinamicheskih kvaziperiodicheskih system [On the convergence of dynamic quasiperiodic systems]. {\it Vestnik of Saint Petersburg University.
Applied Mathematics. Computer Science. Control Processes}, 2022, vol.~18, iss.~1, pp.~79--86.\\ https://doi.org/10.21638/11701/spbu10.2022.106  (In Russian)

21. {Pykh Yu. A.} {\it Ravnovesie i ustojchivost' v modeljah populjacionnoj dinamiki} [{\it Equilibrium and stability in models of population dynamics}]. Moscow, Nauka Publ., 1983, 182~p. (In Russian)

22. {Hofbauer J., Sigmund K.} {\it Evolutionary games and population dynamics.} Cambridge, Cambridge University Press, 1998, 323~p.

23. {Britton N. F.} {\it Essential mathematical biology}.
London, Berlin, Heidelberg, Springer Publ., 2003, 335~p.

24. {Aleksandrov A. Yu.,  Aleksandrova E. B.,  Platonov A. V.}
 Ultimate boundedness conditions for a hybrid model of population dynamics. {\it Proceedings of 21$^{st}$ Mediterranean conference on Control and Automation}, June~25--28, 2013. Platanias-Chania, Crite, Greece, 2013, pp.~622--627.

25. {Khalanai A., Vexler D.} {\it Kachestvennaja teorija impul'snyh sistem} [{\it Qualitative theory of impulsive systems}]. Translated from Romanian, ed. by V.~P.~Rubanik. Moscow, Mir Publ., 1971, 312~p. (In Russian)


\vskip1.5mm

Received:  April 23, 2022.

Accepted: August 01, 2022.


\vskip6mm A~u~t~h~o~r's \ i~n~f~o~r~m~a~t~i~o~n:%

\vskip2mm \textit{Alexander Yu. Aleksandrov} --- Dr. Sci. in Physics and
Mathematics, Professor;  a.u.aleksandrov@spbu.ru \par%
%
%
}
