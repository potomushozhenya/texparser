
{\normalsize

\vskip 6mm
%\newpage

\noindent{\bf Equilibrium in the problem of choosing the meeting time for $N$ persons$^{*}$
 }

}

\vskip 3mm

{\small

\noindent{\it V. V. Mazalov$^{1,2}$, V. V. Yashin$^1$%
%, I.~�. Famylia%$\,^2$%

 }

\vskip 3mm

%%%%%%%%%%%%%%%%%%%%%%%%%%%%%%%%%%%%%%%%%%%%%%%%%%%%%%%%%%%%%%%%%%

\efootnote{
%%
\vspace{-3mm}\parindent=7mm
%%
\vskip 0.1mm \indent~$^{*}$ This work was supported by the Russian Science Foundation (grant N 22-11-00051,\\ https://rscf.ru/project/22-11-00051/).\par%
%%
%%\vskip 2.0mm
%%
%%\indent{\copyright} �����-������������� ���������������
%%�����������, \issueyear%
%%
}

%%%%%%%%%%%%%%%%%%%%%%%%%%%%%%%%%%%%%%%%%%%%%%%%%%%%%%%%%%%%%%%%%%

{\footnotesize

\noindent%
$^1$~%
Institute of Applied Mathematical Research, Karelian Research Centre of Russian Academy of Sciences,

\noindent%
\hskip2.45mm%
11, ul. Pushkinskaya, Petrozavodsk, 185910, Russian Federation


\noindent%
$^2$~%
St\,Petersburg State University, 7--9, Universitetskaya nab., St\,Petersburg,

\noindent%
\hskip2.45mm%
199034, Russian Federation

}

%%%%%%%%%%%%%%%%%%%%%%%%%%%%%%%%%%%%%%%%%%%%%%%%%%%%%%%%%%%%%%%%%%

\vskip 3%
mm


\noindent \textbf{For citation:}  Mazalov V. V., Yashin V. V. Equilibrium in the problem of choosing the meeting time for $N$ persons. {\it Vest\-nik of Saint Pe\-ters\-burg Uni\-ver\-si\-ty. Ap\-plied Mathe\-ma\-tics. Com\-pu\-ter
Scien\-ce. Cont\-rol Pro\-cesses}, %\,
\issueyear, vol.~18, iss.~\issuenum, pp.~\pageref{p5}--\pageref{p5e}. \\
\doivyp/\enskip%
\!\!\!spbu10.\issueyear.\issuenum05 (In Russian)

\vskip2.2mm

\begin{hyphenrules}{english}

%\pagebreak

{\leftskip=7mm\noindent A game-theoretic model of competitive decision  on a meet time is considered. There are $n$ players who are negotiating the meeting time. The objective is to find a meet time that satisfies all participants.  The players' utilities are represented by linear unimodal functions $u_i(x),\, x\in[0, 1],\, i=1,2,...,n$. The maximum values of the utility functions are located at the points $i/(n-1), ~i=0,...,n-1$. Players take turns $1 \rightarrow 2\rightarrow 3 \rightarrow \ldots\rightarrow (n-1) \rightarrow n\rightarrow 1\rightarrow\dots ~. $
Players can indefinitely insist on a profitable solution for themselves. To prevent this from happening, a discounting factor $\delta<1$ is introduced to limit the duration of nego\-tiations. We will assume that after each negotiation session, the utility functions of all play\-ers will decrease proportionally to $\delta$. Thus, if the players have not come to a decision before time $t$, then at time $t$ their utilities are represented by the functions $\delta^{t-1}u_i(x), ~i = 1, 2,..., n.$ We will look for a solution in the class of stationary strategies, when it is assumed that the decisions of the players will not change during the negotiation time, i. e. the player $i$ will make the same offer at step $i$ and at subsequent steps $n+i, 2n+i, \ldots$ . This will allow us to limit ourselves to considering the chain of sentences $1 \rightarrow 2 \rightarrow 3 \rightarrow \ldots \rightarrow(n-1) \rightarrow n\rightarrow 1.$
We will use the method of backward induction. To do this, assume that player $n$ is looking for his best responce, knowing player 1's proposal, then player $(n-1)$ is looking for his best responce, knowing player $n$'s solution, etc. In the end, we find the best responce of the player $1, $ and it should coincide with his offer at the beginning of the procedure. Thus, the reasoning in the method of backward induction has the form $1 \leftarrow 2\leftarrow 3\leftarrow \ldots\leftarrow(n-1)\leftarrow n\leftarrow 1.$
The   subgame perfect equilibrium in the class of stationary strategies is found in analytical form.   It is shown that  when $\delta$ changes from $1$ to $0$, the optimal offer of player 1 changes from $\frac{1}{2}$ to $1$. That is, when the value of $\delta$ is close to 1, the players have a lot of time to negotiate, so the offer of player 1 should be fair to everyone. If the discounting factor is close to 0, the utilities of the players decreases rapidly and they must quickly make a decision that is beneficial to player 1.
\\[1mm]
\textit{Keywords}: optimal timing, linear utility functions, sequential bargaining, Rubinstein bargaining model,   subgame perfect equilibrium, stationary strategies, backward induction.\par}

\vskip6mm

\end{hyphenrules}
%\pagebreak

\noindent \textbf{References} }

\vskip 2.0mm

{\footnotesize

1.  Rubinstein A. Perfect equilibrium in a Bargaining Model.  {\it Econometrica,} 1982,
vol.~50\,(1), pp.~97--109.
https://doi.org/10.2307/1912531

2. Baron D., Ferejohn J. Bargaining in legislatures. {\it American Political Science Association,} 1989,
vol.~83\,(4), pp.~1181--1206.
https://doi.org/10.2307/1961664

3. Eraslan Y. Uniqueness of stationary equilibrium payoffs in the Baron\,---\,Ferejohn model. {\it Journal of Economic Theory,} 2002, vol.~103, pp.~11--30.

4. Cho S., Duggan J. Uniqueness of stationary equilibria in a one-dimensional model of bargaining.
{\it Journal of Economic Theory,} 2003, vol.~113\,(1), pp.~118--130. \\
https://doi.org/10.1016/S0022-0531(03)00087-5

5. Banks J. S., Duggan J. A general bargaining model of legislative policy-making. {\it  Quarterly Journal of Political Science,} 2006, vol.~1, pp.~49--85. https://doi.org/10.2307/2586381

6. Predtetchinski A. One-dimensional bargaining. {\it  Games and Economic Behavior}, 2011, vol.~72\,(2), pp.~526--543.

7. Mazalov V. V., Nosalskaya T. E. Stohasticheskij dizajn v zadache o delezhe piroga [Stochastic design in the cake division problem]. {\it Matematicheskaya teoriya igr i eio prilozheniya} [{\it Matematic theory and supplement}], 2012, vol.~4\,(3),  pp.~33--50. (In Russian)

8. Mazalov V. V., Nosalskaya T. E., Tokareva J. S. Stochastic Cake Division Protocol. {\it International Game Theory Review}, 2014, vol.~16\,(2), no.~1440009.

9. Mazalov V. V., Tokareva J. S.  Teoretiko-igrovye modeli provedeniya konkursov [Game-theoretic models of tender design]. {\it Matematicheskaya teoriya igr i eio prilozheniya} [{\it Matematic theory and sup\-plement}], 2014, vol.~2\,(2), pp.~66--78. (In Russian)

10. Bure V. M. Ob odnoj teoretiko-igrovoj modeli tendera [�ne game-theoretical tender model]. {\it Vestnik of Saint~Petersburg University. Se\-ries~10. Applied Mathematics. Computer Science. Control Processes}, 2015, iss.~1, pp.~25--32. (In Russian)

11. Gubanov D. A., Novikov D. A., Chartishvili A. G. {\it Social'nye seti: modeli informacionnogo vliyaniya, upravleniya i protivoborstva} [{\it Informational influence and informational control models in social networks}]. Moscow, Fizmatlit Publ., 2010, 229~p. (In Russian)

12. Bure V. M., Parilina E. M., Sedakov A. A. Konsensus v social'noj seti s dvumya centrami vliyaniya [Consensus in a social network with two principals]. {\it Automation and Remote Control}, 2016, vol.~1, pp.~21--28.  (In Russian)

13. Sedakov A. A., Zhen  M. Opinion dynamics game in a social network with two influence nodes. {\it Vestnik of Saint Petersburg University. Applied Mathematics. Computer Science. Control Processes}, 2019, vol.~15, iss.~1, pp.~118--125. https://doi.org/10.21638/11701/spbu10.2019.10

\newpage

14. Burkov V. N., Korgin N. A., Novikov D. A. {\it Vvedenie v teoriyu upravleniya organizacionnymi sistemami} [{\it Introduction to the theory of management of organizational systems}]. Moscow, Librocom Publ., 2009, 264~p. (In Russian)

15. Novikov D. A. {\it Teoriya upravleniya organizacionnymi sistemami} [{\it Theory of management of organizational systems}]. Moscow, Moscow Psychological and Social Institute Publ., 2005, 584~p. (In Russian)

16. Cardona D., Ponsati C. Bargaining one-dimensional social choices.
{\it Journal of Economic Theory}, 2007, vol.~137\,(1), pp.~627--651.
https://doi.org/10.1016/j.jet.2006.12.001

17. Cardona D., Ponsati C. Uniqueness of stationary equilibria in bargaining one-dimentional polices under (super) majority rules. {\it Game and Economic Behavior}, 2011, vol.~73\,(1), pp.~65--67. \\ https://doi.org/10.1016/j.geb.2011.01.006

18. Breton M., Thomas A., Zaporozhets V. Bargaining in River Basin Committees: Rules versus. {\it IDEI working papers}, 2012, vol.~732, pp.~1--38.


\vskip1.5mm Received:  August 08, 2022.

Accepted: September 01, 2022.

\vskip6mm
A\,u\,t\,h\,o\,r\,s' \, i\,n\,f\,o\,r\,m\,a\,t\,i\,o\,n:%
%
\vskip2mm \textit{Vladimir V. Mazalov} --- Dr. Sci. in Physics and Mathematics, Professor; vmazalov@krc.karelia.ru
%
\vskip2mm \textit{Vladimir V. Yashin} --- yashinv@krc.karelia.ru \par
%
%
}
