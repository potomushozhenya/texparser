
{\normalsize

\vskip 4.5%6
mm
%\newpage

\noindent{\bf Recovering complex reflection coefficient using the reference layer algorithm for multilayer systems with non-collinear magnetic ordering$^{*}$
 }

}

\vskip 2.5%3
mm

{\small

\noindent{\it Yu. A. Salamatov, E. S. Nikova, D. I. Devyaterikov, E. A. Kravtsov %
%, I.~�. Famylia%$\,^2$%

 }

\vskip 2.5%3
mm

%%%%%%%%%%%%%%%%%%%%%%%%%%%%%%%%%%%%%%%%%%%%%%%%%%%%%%%%%%%%%%%%%%

\efootnote{
%%
\vspace{-3mm}\parindent=7mm
%%
\vskip 0.1mm \indent~$^{*}$ The research was carried out within the state assignment of Ministry of Science and Higher Education of the Russian Federation (theme ``Spin�� N 22021000036-3), neutronographic experiments was carried out with the financial support of the Ministry of Science and Higher Education of the Russian Federation, Agreement N 075-15-2022-830 dated May 27, 2022 (continuation of Agreement N 075-15-2021-1358 dated October 12, 2021).\par%
%%
%%\vskip 2.0mm
%%
%%\indent{\copyright} �����-������������� ���������������
%%�����������, \issueyear%
%%
}

%%%%%%%%%%%%%%%%%%%%%%%%%%%%%%%%%%%%%%%%%%%%%%%%%%%%%%%%%%%%%%%%%%

{\footnotesize


\noindent%
%$^2$~%
M. N. Mikheev lnstitute of Metal Physics of Ural Branch of Russian Academy of Sciences,

\noindent%
%\hskip2.45mm%
18, ul. Sofii Kovalevskoy, Ekaterinburg, 620137, Russian Federation


%\noindent%
%%$^2$~%
%St\,Petersburg State University, 7--9, Universitetskaya nab., %St\,Petersburg,
%
%\noindent%
%%\hskip2.45mm%
%199034, Russian Federation

}

%%%%%%%%%%%%%%%%%%%%%%%%%%%%%%%%%%%%%%%%%%%%%%%%%%%%%%%%%%%%%%%%%%

\vskip 3%
mm


\noindent \textbf{For citation:}  Salamatov Yu. A., Nikova E. S., Devyaterikov D. I., Kravtsov E. A. Recovering complex reflection coefficient using the reference layer algorithm for multilayer systems with non-collinear magnetic ordering. {\it Vest\-nik of Saint Pe\-ters\-burg Uni\-ver\-si\-ty. Ap\-plied Mathe\-ma\-tics. Com\-pu\-ter
Scien\-ce. Cont\-rol Pro\-cesses}, %\,
\issueyear, vol.~18, iss.~\issuenum, pp.~\pageref{p12}--\pageref{p12e}. \\
\doivyp/\enskip%
\!\!\!spbu10.\issueyear.\issuenum12 (In Russian)

\vskip2.2mm

\begin{hyphenrules}{english}

%\pagebreak

{\leftskip=7mm\noindent This work analyzes two mathematical algorithms for processing experimental curves in polarized neutron reflectometry, one of which makes it possible to determine the complex reflection coefficient. The approbation was carried out for a Fe/Cr type superlattice with an irregular non-collinear ordering of the magnetic moments of the Fe layers. The processing of the experiment was carried out both by direct refinement of the structure parameters and by the method of calculating the module and phase of the reflectometry signal using the Gd reference layer. The results obtained with different methods are compared with each other. To clarify the structural and magnetic characteristics, the Levenberg\,---\,Marquardt algorithm was applied in both cases. The obtained data on the magnetic structure is in agreement with the theoretical model of magnetization of a layered antiferromagnet of finite dimensions in a weak field. The presented modification of the reference layer method can be considered as the phase problem solution in polarized neutron reflectometry.
\\[1mm]
\textit{Keywords}: polarized neutron reflectometry, phase-amplitude function method, Runge\,---\,Kutta method, complex reflection coefficient, multilayer nanoheterostructures, phase problem, non-collinear magnetic ordering, reference layer, Levenberg\,---\,Marquardt algorithm.\par}

\vskip6mm

\end{hyphenrules}
%\pagebreak

\noindent \textbf{References} }

\vskip 2.0mm

{\footnotesize

1. Baibich M. N., Broto J. M., Fert A., Nguyen Van Dau F., Petroff~F., Etienne~P., Creuzet~G., Friederich~A., Chazelas~J. Giant magnetoresistance of (001)Fe/(001)Cr magnetic superlattices. \textit{Physical Review Letters}, 1988, vol.~61, iss.~21, pp.~2472--2475.

2. Binasch G., Grunberg P., Saurenbach F., Zinn W. Enhanced magnetoresistance in layered magnetic structures with antiferromagnetic interlayer exchange. \textit{Physical Review B}, 1989, vol.~39, iss.~7, pp.~4828--4830.

3.  Kuranov D. Yu., Andreeva T. A., Bedrina M. E. Raschet potenciala ionizacii ftalocianinatov cinka i grafena na poverhnosti dielektrikov [Calculation of the ionization potential of zinc and graphene phthalocyaninates on the surface of dielectrics]. \textit{Vestnik of Saint Petersburg University. Applied Mathematics. Computer Science. Control Processes}, 2022, vol.~18, iss.~1, pp.~52--62. \\ https://doi.org/10.21638/11701/spbu10.2022.104 (In Russian)

4. Ustinov V. V. Spin-flop transition scenario in finite layered antiferromagnets. \textit{Journal of Mag\-ne\-tism and Magnetic Materials}, 2007, vol.~310, iss.~2, pp.~2219--2221.

5.  Lauter-Pasyuk V., Lauter H., Toperverg B., Romashev L., Ustinov V. Transverse and lateral structure of the spin-flop phase in Fe/Cr antiferromagnetic superlattices. \textit{Physical Review Letters}, 2002, vol.~89, iss.~16, p.~167203.

6. Babikov V. V. \textit{Metod fazovyh funkcij v kvantovoj mekhanike $[$Phase function method in quantum mechanics$]$}. Moscow, Nauka Publ., 1976, 288~p. (In Russian)

7. Lekner J. \textit{Theory of reflection of electromagnetic and particle waves}. Dordrecht, Springer Science+Business Media Publ., 1987, 281~p.

8. Zimmerman K. M. \textit{Advanced analysis techniques for X-ray reflectivities: theory and application}. Doctoral dissertation in chemistry. Karlsruhe, University of Dortmund Press, 2005, 190~p.

9. De Haan V. O., van Well A. A., Sacks P. E., Adenwalla S., Felcher G. P. Toward the solution of the inverse problem in neutron reflectometry. \textit{Physica B: Physics of Condensed Matter},  1996, vol.~221, iss.~1--4, pp.~524--532.

10. Lynn J. E., Seeger P. A. Resonance effects in neutron scattering lengths of rare-earth nuclides. \textit{Atomic Data and Nuclear Data Tables}, 1990, vol.~44, iss.~2, pp.~191--207.

11. Nikova E. S., Salamatov Yu. A., Kravtsov E. A., Bodnarchuk V. I., Ustinov V. V. Experimental determination of gadolinium scattering characteristics in neutron reflectometry with reference layer. \textit{Physica B: Physics of Condensed Matter}, 2019, vol.~552, pp.~58--61.

12. Aksenov V. L., Lauter-Pasyuk V. V., Lauter H., Nikitenko Yu. V., Petrenko~A.~V. Polarized neutrons at pulsed sources in Dubna. \textit{Physica B: Physics of Condensed Matter},  1996, vol.~335, iss.~1--4, pp.~147--152.

13. Nikova E. S., Salamatov Yu. A., Kravtsov E. A., Ustinov~V.~V. Application of a Gd reference layer for the study of magnetic metallic nanostructures by neutron reflectometry. \textit{Journal of Surface Investigation. X-ray, Synchrotron and Neutron Techniques}, 2021, vol.~15, no.~5, pp. 899--902.


\vskip1.5mm Received:  July 27, 2022.

Accepted: September 01, 2022.

\vskip6mm
A\,u\,t\,h\,o\,r\,s' \, i\,n\,f\,o\,r\,m\,a\,t\,i\,o\,n:%
%

\vskip2mm \textit{Yuri A. Salamatov} --- PhD in Physics and Mathematics; salamatov@imp.uran.ru

\vskip2mm \textit{Ekaterina S. Nikova} ---   nikova@imp.uran.ru

\vskip2mm \textit{Denis I. Devyaterikov} --- devidor@imp.uran.ru

\vskip2mm \textit{Evgeny A. Kravtsov} ---  Dr. Sci. in Physics and Mathematics; kravtsov@imp.uran.ru \par
%
%
}
