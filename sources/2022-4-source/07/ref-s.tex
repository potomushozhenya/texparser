
{\normalsize

\vskip 6mm
%\newpage

\noindent{\bf Power generalization of the linear constitutive equations of heat\\ and mass transfer and the variants of writing the equations\\ of momentum transfer, heat and diffusion arising from them$^{*}$
 }

}

\vskip 3mm

{\small

\noindent{\it V. A. Pavlovsky%
%, I.~�. Famylia%$\,^2$%

 }

\vskip 3mm

%%%%%%%%%%%%%%%%%%%%%%%%%%%%%%%%%%%%%%%%%%%%%%%%%%%%%%%%%%%%%%%%%%

\efootnote{
%%
%\vspace{-3mm}\parindent=7mm
%%
\vskip 0.1mm \indent~$^{*}$ This study was carried out within the framework of the state task  by Ministry of Science and Higher Education of the Russian Federation for the implementation of research works
N~075-03-2020-094/1 of June 10, 2020.\par%
%%
%%\vskip 2.0mm
%%
%%\indent{\copyright} �����-������������� ���������������
%%�����������, \issueyear%
%%
}

%%%%%%%%%%%%%%%%%%%%%%%%%%%%%%%%%%%%%%%%%%%%%%%%%%%%%%%%%%%%%%%%%%

{\footnotesize

\noindent%
%$^2$~%
St Petersburg State Marine Technical University, 3, Locmanskaya ul., St Petersburg,

\noindent%
%\hskip2.45mm%
190121, Russian Federation



%\noindent%
%$^2$~%
%St\,Petersburg State University, 7--9, Universitetskaya nab., %St\,Petersburg,
%
%\noindent%
%\hskip2.45mm%
%199034, Russian Federation

}

%%%%%%%%%%%%%%%%%%%%%%%%%%%%%%%%%%%%%%%%%%%%%%%%%%%%%%%%%%%%%%%%%%

\vskip 3%
mm


\noindent \textbf{For citation:}  Pavlovsky V. A. Power generalization of the linear constitutive equations of heat and mass transfer and the variants of writing the equations of momentum transfer, heat and diffusion arising from them. {\it Vest\-nik of Saint Pe\-ters\-burg Uni\-ver\-si\-ty. Ap\-plied Mathe\-ma\-tics. Com\-pu\-ter
Scien\-ce. Cont\-rol Pro\-cesses}, %\,
\issueyear, vol.~18, iss.~\issuenum, pp.~\pageref{p7}--\pageref{p7e}. \\
\doivyp/\enskip%
\!\!\!spbu10.\issueyear.\issuenum07 (In Russian)

\vskip2.2mm

\begin{hyphenrules}{english}

%\pagebreak

{\leftskip=7mm\noindent Currently, when solving problems of heat and mass transfer, linear constitutive equations are used --- in hydrodynamics, the viscous stress tensor is proportional to the strain rate tensor (Newton's rheological ratio), in heat transfer, the heat flux density is linearly related to the temperature gradient (Fourier's heat conduction law), in mass transfer, the diffusion flux density proportional to the concentration gradient (Fick's law). When writing these linear governing equations, proportionality coefficients are used, which are called the visco\-sity coefficient, thermal conductivity coefficient and diffusion coefficient, respectively. Such constitutive equations are widely used to describe the processes of heat and mass transfer in a laminar flow regime. For turbulent flows, these equations are unsuitable, it is necessary to introduce into consideration the empirical turbulent coefficients of viscosity $\mu _t$, thermal conductivity $\lambda_t$ and diffusion $D_t$. However, to describe turbulent flows, it is possible to go in another way --- to modify the linear constitutive relations by giving them a nonlinear power-law form. Two-parameter power-law generalizations of Newton's, Fourier's and Fick's formulas for shear stress, heat flux density and diffusion, which, depending on the value of the exponents, can be used to describe the processes of heat and mass transfer both in la\-mi\-nar and turbulent fluid flow. Also, this generalization can be used to describe the behavior of power-law fluids and flows of polymer solutions exhibiting the Toms effect.
\\[1mm]
\textit{Keywords}: hydrodynamics, heat transfer, diffusion, Newton's, Fourier's, Fick's formulas, power generalizations, turbulence.\par}

\vskip6mm

\end{hyphenrules}
%\pagebreak

\noindent \textbf{References} }

\vskip 2.0mm

{\footnotesize

1. {Kutateladze S. S} {\it Osnovi teorii teploobmena} [{\it Fundamentals of the theory of heat transfer}]. Moscow, Atomizdat Publ., 1979, 234~p. (In Russian)

2. {Pavlovsky V. A.} Power-law generalization of Newton�s formula for shear stress in a liquid in the form of a tensor rheological relation. \textit{Vestnik of Saint Petersburg University. Mathematics}, 2022, vol.~55, iss.~2, pp.~229--234.
https://doi.org/10.1134/S1063454122020091

3. {Pavlovsky V. A., Kabrits S. A. } {Raschyot tyrbulentnogo pogranichnogo sloya ploskoy plastini} [{Calculation of turbulent boundarylayer of a flat plate}]. \textit{Vestnik of Saint Petersburg University. Applied Mathematics. Computer Sciences. Control Processes}, 2021, vol.~17, iss.~4, pp.~370--380. (In Russian) \\
https://doi.org/10.21638/11701/spbu10.2021.405

4. {Nikushchenko D. V., Pavlovsky V. A., Nikushchenko E. A.} Fluid flow development in a pipe as a demonstration of a sequential 402 Change in its rheological properties. \textit{Applied Sciences}, 2022, no.~12~(6).
https://doi.org/10.3390/app12063058

5. {Pavlovsky V. A.}  {Stepennoe obobshchenie formyli teploprovodnosti Fyr'e i vitekaushchie iz nego varianti dlya zapisi yravneniya energii} [{Power-law generalization of the Fourier formula for heat conduction and variants arising from it for writing the energy equation}].  \textit{Marine intelligent technologies}, 2022, vol.~2~(4), no.~2, pp.~133--138. (In Russian)

6. {Isachenko V. P., Osipova V. A., Sukomel A. S.} {\it Teploperedacha} [{\it Heat transfer}]. Moscow, Energoizdat Publ., 1981, 416~p. (In Russian)

7. {Popov P. V.} {\it Diffyzia} [{\it Diffusion}]. Moscow, Moscow Institute of Physics and Technology Publ., 2016, 94~p.  (In Russian)

8. {Pavlovsky V. A., Nikushhenko D. V.} {\it  Vychislitelnaya gidrodinamika. Teoreticheskie osnovy} [{\it Computational fluid dynamics. Theoretical fundamentals}]. St Petersburg, Lan� Publ., 2018, 368~p. (In Russian)


\vskip1.5mm Received:  August 08, 2022.

Accepted: September 01, 2022.

\vskip6mm
A\,u\,t\,h\,o\,r's \, i\,n\,f\,o\,r\,m\,a\,t\,i\,o\,n:%
%
\vskip2mm \textit{Valery A. Pavlovsky} --- Dr. Sci. in Physics and Mathematics, Professor; v.a.pavlovsky@gmail.com \par
%
%
}
