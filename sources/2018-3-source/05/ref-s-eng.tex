

{\small



\vskip6mm

\noindent \textbf{References} }

\vskip 2mm

{\footnotesize

1. Anggai S. The design and implementation of social networking at
virtual museum of Indonesia (a~case study museum of geology).
Bandung, Indonesia, {Bandung Institute of Technology}, 2012, 60~p.
(see pp.~1--60).

2. Foo S. Online virtual exhibitions: concepts and design
considerations. \emph{Journal of Library and Information
Technology}, 2008, vol.~28, pp.~1--19.

3. Champion E. Entertaining the similarities and distinctions
between serious games and virtual heritage projects.
\emph{Entertainment Computing}, 2016, vol.~14, pp.~67--74.

4. Palombini A. Storytelling and telling history. Towards a
grammar of narratives for cultural heritage dissemination in the
digital era. \emph{Journal of Cultural Heritage}, 2017, vol.~24,
pp.~134--139.\newpage

5. Deerwester S., Dumais S. T., Furnas G. W., Landauer T. K.,
Harshman R. Indexing by latent semantic analysis. \emph{Journal of
the American Society for Information Science}, 1990, vol.~41,
pp.~391--407.

6. Foltz P. W. Using latent semantic indexing for information
filtering. \emph{SIGOIS Bull.}, 1990, vol.~11, pp.~40--47.

7. Hofmann T. Probabilistic latent semantic indexing.
\emph{Proceedings of the 22nd Annual International ACM SIGIR
conference on Research and Development in Information Retrieval}.
Berkeley, California, USA, 1999, pp.~50--57.

8. Hofmann T. Unsupervised learning by probabilistic latent
semantic analysis. \emph{Machine Learning}, 2001, vol.~42,
pp.~177--196.

9. Blei D. M., Ng A. Y., Jordan M. I. Latent dirichlet allocation.
\emph{The Journal of Machine Learning Research}, 2003, vol.~3, no.
2, pp.~993--1022.

10. Yan X., Guo J., Lan Y., Cheng X. A biterm topic model for
short texts. \emph{Proceedings of the 22nd International
conference on World Wide Web (WWW'13)}. Rio de Janeiro, Brazil,
2013, pp.~1445--1456.

11. Cheng X., Yan X., Lan Y., Guo J. Topic modeling over short
texts. \emph{IEEE Transactions on Knowledge and Data Engineering},
2014, pp.~2928--2941.

12. Xu J., Liu P., Wu G., Sun Z., Xu B., Hao H. A fast matching
method based on semantic similarity for short texts. \emph{Natural
Language Processing and Chinese Computing}, 2013, vol.~400,
pp.~299--309.

13. Wang P., Zhang H., Liu B. X., Hao H. Short text feature
enrichment using link analysis on topic-keyword graph.
\emph{Natural Language Processing and Chinese Computing}, 2014,
vol.~496, pp.~79--90.

14. Griffiths T. {\it Gibbs sampling in the generative model of
latent dirichlet allocation}. {Stanford technical report}.
Stanford, California, USA, 2002, pp.~1--3.

15. He X., Xu H., Li J., He L., Yu L. FastBTM: reducing the
sampling time for biterm topic model. \emph{Knowledge-Based
Systems}, 2017, vol.~132, pp.~11--20.

16. Mimno D., Wallach H. M., Talley E., Leenders M., McCallum A.
Optimizing semantic coherence in topic models. \emph{Proceedings
of the Conference on Empirical Methods in Natural Language
Processing}. Edinburgh, United Kingdom, 2011, pp.~262--272.

17. Stevens K., Kegelmeyer P., Andrzejewski D., Buttler D.
Exploring topic coherence over many models and many topics.\emph{
Proceedings of the 2012 Joint Conference on Empirical Methods in
Natural Language Processing and Computational Natural Language
Learning}. Jeju Island, Korea, 2012, pp.~952--961.

18. Xia Y., Tang N., Hussain A., Cambria E. Discriminative Bi-Term
topic model for headline-based social news clustering.\emph{ The
Twenty-Eighth International Florida Artificial Intelligence
Research Society Conference}. Florida, North America, 2015,
pp.~311--316.

19. Manning C. D., Raghavan P., Schutze H. Introduction to
Information Retrieval. New York, USA, {Cambridge University
Press}, 2008, 544 p. (see pp.~61--123).

20. Zhang W., Yoshida T., Tang X. A comparative study of TF{*}IDF,
LSI and multi-words for text classification. \emph{Expert Systems
with Applications}, 2011, vol.~38, no. 2, pp.~2758--2765.

21. Anggai S., Blekanov I. S., Sergeev S. L. Index data structure,
functionality and microservices in thematic virtual museums.
\emph{Vestnik of Saint Petersburg University. Applied Mathematics.
Computer Science. Control Processes}, 2018, vol.~14, iss.~1,
pp.~31--39. DOI: 10.21638/11701/spbu10.2018.104


%\vskip3mm%
%\noindent \textbf{For citation:} Tregubov V. P., Rutkina
%S. V. Mathematical modelling of pulsativ� blood flow in deformable
%arteries. {\it Vestnik SPbSU. Applied Mathematics. Computer
%Science. Control Processes}, \issueyear, vol.~14, iss.~\issuenum,
%pp.~\pageref{p8}--\pageref{p8e}.
%\doivyp/spbu10.\issueyear.\issuenum08




\vskip 1.5mm


%\noindent Recommendation: prof. L. A. Petrosyan.
%
%\vskip 1.5mm

%\noindent
Received:  10.03.2018; accepted: 14.06.2018.




\vskip6mm A\,u\,t\,h\,o\,r's \ i\,n\,f\,o\,r\,m\,a\,t\,i\,o\,n:
\vskip2mm\textit{Sajarwo Anggai} --- postgraduate student;
sajarwo@gmail.com

\vskip2mm\textit{Ivan S. Blekanov} --- PhD Sci. in Technics,
Associate Professor; i.blekanov@spbu.ru


\vskip2mm\textit{Sergei L. Sergeev} --- PhD Sci. in Physics and
Mathematics, Associate Professor; slsergeev@yandex.ru




}
