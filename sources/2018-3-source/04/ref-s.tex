
{\normalsize

\vskip 6mm

\noindent{\bf Approximation-evaluation tests for a stress-strain
state of deformable solids}

}

\vskip 1.5mm

{\small

\noindent{\it A.\ V.\ Orekhov}

\vskip 1.5mm

{\footnotesize \noindent St.\,Petersburg State University, 7--9,
Universitetskaya nab., St.\,Petersburg, 199034, Russian Federation

}

\vskip3mm

\noindent \textbf{For citation:}  Orekhov A. V.
Approximation-evaluation tests for a stress-strain state of
deformab-\linebreak le solids. {\it Vestnik of Saint~Petersburg
University. Applied Mathematics. Computer Science. Control
Processes}, \issueyear, vol.~14, iss.~\issuenum,
pp.~\pageref{p4}--\pageref{p4e}.
\doivyp/\enskip%
\!\!\!spbu10.\issueyear.\issuenum04

\vskip3mm

{\leftskip=7mm\noindent When analyzing some applied problems, it
is of interest to obtain certain statistical criteria. These
criteria could determine the moment when a monotonically
increasing quantity, given in the form of a table and whose
analytical form is unknown, changes the linear increasing to the
nonlinear one. In this paper, we consider
``approximation-evaluation tests'', which allows us to determine
the point, when the type of increase in the monotone sequence of
numerical parameters of deformable solid, characterizing its
stress-strain state, is changed from linear to parabolic. This
point could be considered as a harbinger of the strength loss.
This criterion is based on the comparison of the quadratic errors
of the linear and the incomplete parabolic approximations.
Approximating functions are constructed locally, not overall
values of the sequence, but only over several of them. These
points are located in the left half-neighborhood of the
investigated point. An inverse problem is solved in which the
critical values of the sequence are calculated, for which the
quadratic errors of the linear and incomplete parabolic
approximations are equal. The example shows that a simple
comparison of finite differences cannot
be used to determine the point at which a linear increase becomes parabolic.\\[1mm]
\textit{Keywords}: stress-strain state, least-squares method.
\par}

\vskip5mm

\noindent \textbf{References} }

\vskip 2mm

{\footnotesize


1.\;{Pestrikov V. M., Morozov Ye. M.} \emph{Mekhanika razrusheniya
tverdykh tel} [\emph{Mechanics of destruction of solids}].  Saint
Petersburg, Professiya Publ., 2002, 320~p. (In Russian)

2.\;{Teregulov I. G.} \emph{Soprotivleniye materialov i osnovy
teorii uprugosti i plastichnosti} [\emph{Resistance of materials
and the foundations of the theory of elasticity and plasticity}].
Moskow, Vysshaya shkola Publ., 1964, 472~p. (In Russian)

3.\;{Pavilaynen G. V., Yushin R. U.} Analiz ucheta uprugoy
transversal'noy izotropii i plasticheskoy anizotropii pri izgibe
kruglykh plastin  [Analysis of the account of elastic transversal
isotropy and plastic anisotropy in the bending of circular
plates]. \emph{Vestnik of Saint Peterburg University. Series~1.
Mathematics. Mechanics. Astronomy}, 2011, iss.~1, pp.~122--131.
(In Russian)

4.\;{Pavilaynen G. V., Yushin R. U.} An approximate solution of
elastic-plastic problem of circular strength different (SD)
plates. \emph{Constructive Nonsmooth Analysis and Related Topics}.
Abstracts of the International conference. Dedicated to the memory
of professor V.~F.~Demyanov. Saint Petersburg, VVM Publ., 2017,
pp.~207--209.

5.\;{Pavilaynen G. V.} Matematicheskoye modelirovaniye
uprugo-plasticheskogo izgiba balki, material kotoroy obladayet
effektom SD [Mathematical simulation of the elastic-plastic
bending of a beam whose material has the SD effect]. \emph{Trudy
seminara ``Komp'yuternyye metody v mekhanike sploshnoy sredy''.
2014--2015 gg.} [\emph{Proceedings of the seminar ``Computer
Methods in Continuum Mechanics''. 2014--2015}]. Saint Petersburg,
Saint Petersburg University Publ., 2015, pp.~49--62. (In Russian)

6.\;{Vansovich K. A.} Model' rosta ustalostnykh poverkhnostnykh
treshchin za tsikl �nagruzka---razgruzka�  [The model of growth of
fatigue surface cracks for the ``load---unload'' cycle].
\emph{Omskiy nauchnyy vestnik} [\emph{Omskiy scientific vestnik}],
2017, no.~3(153), pp.~49--53.  (In Russian)

7.\;{Tikhomirov V. M.} Rost treshchiny pri znakoperemennom tsikle
nagruzheniya [Fissure growth under alternating loading cycle].
\emph{Prikladnaya mekhanika i tekhnicheskaya fizika}
[\emph{Applied Mathematics and Technical Physics}], 2008, vol.~49,
no.~5(291), pp.~190--198.  (In Russian)

8.\;{Treshchev A. A., Poltavets P. A.} K teorii plastichnosti
materialov, chuvstvitel'nykh k navo-\linebreak dorozhivaniyu  [On
the theory of plasticity of materials sensitive to hydrogenation].
\emph{Problemy ma-\linebreak shinostroyeniya i avtomatizatsii}
[\emph{Problems of Machine-building and Automatization}], 2006,
no.~2, pp.~60--67. (In Russian)\newpage

9.\;{Orekhov A. V.} Kriteriy otsenki napryazhenno-deformiruyemogo
sostoyaniya SD-materialov [Crite-\linebreak rion for estimating
the stress-strain state of SD materials].   \emph{Mezhdunarodnaya
nauchnaya konferentsiya po mekhanike ``Vos'myye Polyakhovskiye
chteniya''}. Tezisy dokladov [\emph{International scientific
conference on mechanics ``Eight Polyakhovsky Readings''}. Thesis
of paper]. Saint Petersburg, Saint Petersburg University Publ.,
2018, pp.~221--222. (In Russian)

10.\;{Orekhov A. V.} Dva kriteriya napryazhenno-deformiruyemogo
sostoyaniya tverdogo tela [Two criteria for the stress-strain
state of a solid]. \emph{XXIII Peterburgskiye chteniya po
problemam prochnosti, posvyashchennyye 100-letiyu FTI im.
A.~I.~Ioffe i 110-letiyu so dnya rozhdeniya chl.-korr. AN SSSR
A.~V.~Stepanova} [\emph{XXIII Petersburg readings on the problems
of strength, dedicated to the 100th anni-\linebreak versary of the
Ioffe Physics and Technical Institute and 110th anniversary from
the date of birth of cor-\linebreak responding member of the USSR
Academy of Sciences A.~V.~Stepanov}]. Saint Petersburg, VVM Publ.,
2018, pp.~208--210. (In Russian)

11.\;{Orekhov A. V.} Criterion for estimation of stress-deformed
state of SD-materials. \emph{AIP Conference Proceedings}, 2018,
vol.~1959, pp.~070028. DOI: 10.1063/1.5034703

12.\;{Ivanov V. A., Chemodanov B. K., Medvedev I. S.}
\emph{Matematicheskiye osnovy teorii avtoma- ticheskogo
regulirovaniya} [\emph{Mathematical foundations of the theory of
automatic regulation}]. Moskow, Vysshaya shkola Publ., 1971,
808~p. (In Russian)


\vskip6mm A\,u\,t\,h\,o\,r's \ i\,n\,f\,o\,r\,m\,a\,t\,i\,o\,n:
\vskip2mm \textit{Andrey V. Orekhov}~--- senior lecturer;
A\_V\_Orehov@mail.ru



}
