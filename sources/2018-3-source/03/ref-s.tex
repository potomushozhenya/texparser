
{\normalsize

\vskip 6mm

\noindent{\bf A family of sixth-order methods with six stages}

}

\vskip 2mm

{\small

\noindent{\it I.~V.~Olemskoy%$^{1,2}$
, N.~A.~Kovrizhnykh%$^1$

}

\vskip 2mm

{\footnotesize \noindent% $^1$~
St.\,Petersburg State University,
7--9, Universitetskaya nab., St.\,Petersburg,

\noindent%\hskip2.45mm
199034, Russian Federation

%\noindent $^2$~Solomenko Institute of Transport Problems, 13, 12th
%line of V. O., St.\,Petersburg,
%
%\noindent\hskip2.45mm 199178, Russian Federation

}

\vskip3mm

\noindent \textbf{For citation:}  Olemskoy~I.~V.,
Kovrizhnykh~N.~A. A family of sixth-order methods with six stages.
{\it Vestnik of Saint~Petersburg University. Applied Mathematics.
Computer Science. Control Processes},\,\issueyear,
vol.~14,~iss.~\issuenum,~pp.~\pageref{p3}--\pageref{p3e}.
\doivyp/\enskip%
\!\!\!spbu10.\issueyear.\issuenum03

\vskip3mm

{\leftskip=7mm\noindent The paper deals with construction of an
efficient explicit sixth order method for structurally partitioned
systems of ordinary differential equations. The general scheme of
the method is presented, which algorithmically uses structural
properties of a system of differential equations. The order
conditions and the simplifying conditions for the considered
explicit six-stage method are written down and their consistency
is determined. The general solution with seven free parameters is
obtained and a computational scheme for certain values of free
parameters is constructed. Its performance on test
problems is compared to three other explicit sixth-order methods.\\[1mm]
\textit{Keywords}: order,  order conditions, simplifying
conditions.
\par}

\vskip5mm

\noindent \textbf{References} }

\vskip 2mm

{\footnotesize

1. Olemskoy I. V. Fifth-order four-stage method for numerical
integration of spe\-ci\-al sys\-tems.  {\it Comput. Math. Math.
Phys.}, 2002, vol.~42, pp.~1135--1145.

2. Olemskoy I. V. Structural approach to the design of explicit
one-stage methods. {\it Comput. Math. Math. Phys.}, 2003, vol.~43,
pp.~918--931.

3. Olemskoy I. V. Modifikatsiya algoritma vydeleniya strukturnykh
osobennostei [Modi\-fi\-ca\-tion of structural properties
detection algorithm]. {\it Vestnik of Saint Petersburg University.
Series~10. Applied Mathematics. Computer Sciences. Control
Processes}, 2006, iss.~2, pp.~55--64. (In Russian)

4. Olemskoy I. V., Eremin A. S., Ivanov A. P.  Sixth order method
with six stages for integrating special systems of ordinary
differential equations. {\it 2015 Intern. Conference on
``Stability and Control Processes'' in memory of V.~I.~Zubov.
SCP--2015. Proceedings}, 2015, pp.~110--113.

5. Butcher J. C. On Runge---Kutta processes of high order. {\it
Journal of the Australian Mathematical Society}, 1964, vol.~4,
pp.~179--194.

6. Tsitouras Ch., Famelis I. Th. On phase-fitted modified
Runge---Kutta pairs of order 6(5). {\it Intern. conference of
Numerical Analysis and Applied Mathematics. Extended Abstracts}.
Crete, 2006, pp.~1962--1965.

7. Chammud~(Hammud) G. M. A three-dimensional family of seven-step
Runge---Kutta methods of order 6. Numerical Methods and
Programming. {\it Advanced Computing}, 2001, vol.~2, pp.~159--166.

8. Hairer E., Nersett S. P., Wanner G. {\it Solving ordinary
differetial equation~I: Non-stiff problems}. 2nd ed. Berlin,
Heidelberg, Springer-Verlag Publ., 2008, 528~p. (Springer Series
in Computational mathematics.) (Russ. ed.: {\it Hairer~E.,
Nersett~S.~P., Wanner~G.} {\it Reshenie obyknovennykh differen-
tsial'nykh uravnenii. Nezhestkie zadachi}. Moscow, Mir Publ.,
1990, 512~p.)

9. Shymanchuk D. V., Shmyrov A. S. Postroenie traektorii
vozvrascheniya v okretnost' kollinearnoy tochki libratsii sistemy
Solntse---Zemlya [Construction of the return trajectory to the
neighborhood of the collinear libration point of the Sun---Earth
system]. {\it Vestnik of Saint Petersburg University. Series~10.
Applied Mathematics. Computer Sciences. Control Processes}, 2013,
iss.~2, pp.~75--84. (In Russian)

\vskip6mm A\,u\,t\,h\,o\,r's \ i\,n\,f\,o\,r\,m\,a\,t\,i\,o\,n:%

\vskip2mm \textit{Igor V. Olemskoy}~--- Dr. Sci. in Physics and
Mathematics, Professor; i.olemskoj@spbu.ru

\vskip2mm \textit{Nikolay A. Kovrizhnykh}~--- postgraduate
student; sagoyewatha@mail.ru
%
}
