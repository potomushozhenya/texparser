
{\normalsize

\vskip 6mm

\noindent{\bf Alpha-sets in finite-dimensional Euclidean spaces\\
and their applications in control theory}

}

\vskip 2mm

{\small

\noindent{\it V. N. Ushakov, A. A. Uspenskii, A. A. Ershov}

\vskip 2mm

{\footnotesize \noindent Krasovskii Institute of Mathematics and
Mechanics Ural Branch of Russian Academy Sciences,\\ 16, S.
Kovalevskaya ul., Yekaterinburg, 620990, Russian Federation

}

\vskip3mm

\noindent \textbf{For citation:}  Ushakov V. N., Uspenskii A. A.,
Ershov A. A. Alpha-sets in finite-dimensional Euclidean spaces and
their applications in control theory. {\it Vestnik of
Saint~Petersburg University. Applied Mathematics. Computer
Science. Control Processes}, \issueyear, vol.~14, iss.~\issuenum,
pp.~\pageref{p7}--\pageref{p7e}.
\doivyp/\enskip%
\!\!\!spbu10.\issueyear.\issuenum07

\vskip3mm

{\leftskip=7mm\noindent In this paper, a technique for
investigating nonconvex sets that occur when describing the
evolution of wave fronts, in the construction of generalized
solutions of boundary value problems for equations of
Hamilton---Jacobi type, in the formation of resolving structures
in the problems of dynamic control is developed. An estimate is
obtained for the Hausdorff distance between such sets and their
convex hulls. The estimate is based on the concept of a measure of
nonconvexity $\alpha$. It is shown that for small $\alpha$,
nonconvex $\alpha$-sets are close to convex. An example of a
solution
of the optimal control problem on the basis of $\alpha$-sets is~give.\\[1mm]
\textit{Keywords}: $\alpha$-set, convex hull, Hausdorff distance,
control, performance, Hamilton---Jacobi equation.
\par}

\vskip5mm

\noindent \textbf{References} }

\vskip 2mm

{\footnotesize


1. Ushakov V. N., Uspenskii A. A., Fomin A. N.  {\it
$\alpha$-mnojestva i ih svoistva} [\textit{$\alpha$-sets and their
properties}]. Yekaterinburg,  Institute of Mathematics and
Mechanics, Ural Branch of the Russian Academy of Sciences Publ.,
2004, 62~p. (In Russian).

2. Lee E. B., Markus L. {\it Foundations of optimal control
theory.}  New York etc., Wiley Press, 1967, 576~p. (Russ. end.:
Lee E. B., Markus L. \textit{Osnovy teorii optimal'nogo
upravlenija}. Moscow, Nauka Publ., 1972, 574~p.)

3. Patsko V. S., Pyatko S. G., Fedotov A. A. Trehmernoe mnozhestvo
dostizhimosti nelinejnoj upravljaemoj sistemy [Three-dimensional
reachability set for a nonlinear control system].
\textit{Proceedings of the Russian Academy of Sciences. Theory and
Systems Control}, 2003, vol.~42, no.~3, pp.~8--16. (In Russian)
DOI: 10.1134/S1064230706030075

4. Michael E. Paraconvex sets. \textit{Math. Scand.}, 1959, vol.
7, no.~2, pp.~312--315. DOI: 10.7146/ math.scand.a-10583

5. Semenov P. V.\, Funkcional'no paravypuklye mnozhestva
[Functionally paraconvex sets]. \textit{Mathe-\linebreak matical
Notes}, 1993, vol.~54, iss.~6, pp.~1236--1240. (In Russian)

6. Ivanov G. E. {\it Slabo vypuklye mnozhestva i funkcii: teorija
i prilozhenija} [{\it Weak convex sets and functions: theory and
applications}]. Moscow, Fizmatlit Publ., 2006, 352~p. (In Russian)

7. Uspenskii A. A., Lebedev P. D. Geometrija i asimptotika
volnovyh frontov [Geometry and asymptotics of wavefronts].
\textit{Proceedings of Higher educational Institutions.
Mathematics}, 2008, vol.~52, no.~3, pp.~27--37. (In Russian) DOI:
10.3103/S1066369X08030031

8. Lebedev P. D., Uspenskii A. A., Ushakov V. N. Postroenie
minimaksnogo reshenija uravnenija tipa jejkonala [Construction of
a minimax solution for an eikonal-type equation].
\textit{Proceedings of the Steklov Institute of Mathematics and
Mechanics Ural Branch of the Russian Academy of Sciences}, 2008,
vol.~14, no.~2, pp.~182--191. (In Russian) DOI:
10.1134/S0081543808060175

9. Uspenskii A. A., Lebedev P. D. Postroenie funkcii optimal'nogo
rezul'tata v zadache bystrodejstvija na osnove mnozhestva
simmetrii [Construction of the optimal outcome func\-ti\-on for a
time-optimal problem on the basis of a symmetry set].
\textit{Automation and Remote Control}, 2009, vol.~70, iss.~7,
pp.~50--57. (In Russian) DOI: 10.1134/S0005117909070054

10. Ushakov V. N., Uspenskii A. A., Lebedev P. D. Geometrija
singuljarnyh krivyh dlja odnogo klassa zadach bystrodejstvija
[Geometry of singular curves of a class of time-optimal problems].
\textit{Vestnik of Saint Petersburg University. Series~10. Applied
Mathematics. Computer Sciences. Control Processes}, 2013, iss.~3,
pp. 157--167. (In Russian)

11. Ushakov V. N., Uspenskii A. A. $\alpha$-mnozhestva v
konechnomernyh evklidovyh prost\-ranst\-vah i ih svojstva
[$\alpha$-sets in finite dimensional Euclidean spaces and their
properties]. \textit{Vestnik of Udmurtsk University. Mathematics.
Mekhanics. Computer Sciences}, 2016,  vol.~26, iss.~1,
pp.~95--120. (In Russian) DOI: 10.20537/vm160109

12. Subbotin A. I. {\it Minimaksnyye neravenstva i uravneniya
Gamil'tona---Yakobi} [{\it Minimax inequa-\linebreak lities and
the Hamilton---Jacobi equations}]. Moscow, Nauka Publ., 1991,
214~p. (In Russian)

13. Dem'yanov V. F., Vasil'yev L. V. {\it Nedifferentsiruyemaya
optimizatsiya} [{\it Non\-dif\-fe\-ren\-ti\-ab\-le opti-\linebreak
mization}]. Moscow, Nauka Publ., 1981, 384~p. (In Russian)

14. Uspenskii A. A., Lebedev P. D. Analiticheskoe i chislennoe
konstruirovanie funkcii optimal'nogo rezul'tata dlja odnogo klassa
zadach bystrodejstvija [Analytical and numerical design of the
optimal result function for one class of performance problems].
\textit{Applied Ma\-the\-ma\-tics and Informatics. Proceedings of
the Faculty of Computational Mathematics and Cybernetics
M.~V.~Lomonosov Moscow State University}, 2007, no.~27,
pp.~65--79. (In Russian)

15. Uspenskii A. A. Neobhodimye uslovija sushhestvovanija
psevdovershin kraevogo mno\-zhest\-va v zadache Dirihle dlja
uravnenija jejkonala [Necessary conditions for the existence of
pseudo-vertices of a boundary set in the Dirichlet problem for the
eikonal equation]. \textit{Proceedings of the Institute of
Mathe\-ma\-tics and Mechanics Ural Branch of the Russian Academy
of Sciences}, 2015, vol.~21, no.~1, pp.~250--263. (In Russian)

16. Starr~R.~M. Quasi-equilibria in markets with non-convex
preferences. \textit{Econometrica}, 1969,  vol.~37, iss.~1,
pp.~25--38. DOI: 10.2307/1909201



\vskip6mm A\,u\,t\,h\,o\,r's \ i\,n\,f\,o\,r\,m\,a\,t\,i\,o\,n:
\vskip1.5mm \textit{Vladimir N. Ushakov} --- Dr. Sci. in Physics
and Mathematics, Professor; ushak@imm.uran.ru

\vskip2mm \textit{Aleksander A. Uspenskii} --- Dr. Sci. in Physics
and Mathematics; uspen@imm.uran.ru

\vskip2mm \textit{Aleksander A. Ershov} --- PhD Sci. in Physics
and Mathematics; ale10919@yandex.ru



}
