
{\normalsize

\vskip 6mm

\noindent{\bf On the calculation of surfaces glaciation in
seawater}

}

\vskip 1.5mm

{\small

\noindent{\it G.~I. Kurbatova}

\vskip 1.5mm

{\footnotesize \noindent St.\,Petersburg State University, 7--9,
Universitetskaya nab., St.\,Petersburg, 199034, Russian Federation

}

\vskip3mm

\noindent \textbf{For citation:}   Kurbatova G.~I. On the
calculation of surfaces glaciation in seawater. {\it Vestnik of
Saint~Petersburg University. Applied Mathematics. Computer
Science. Control Processes}, \issueyear, vol.~14, iss.~\issuenum,
pp.~\pageref{p1}--\pageref{p1e}.
\doivyp/\enskip%
\!\!\!spbu10.\issueyear.\issuenum01

\vskip3mm

{\leftskip=7mm\noindent Selection of a method for the numerical solution
to the Stefan problem is considered, which permits one to calculate the
ice boundary with accepted accuracy during a long period of time, and can be
extended to multidimensional glaciation problems. A comparison of the different
versions of continuous schemes with an exact analytical solution to the classical
Stefan problem is presented. The criterion that permits estimation, under given
assumptions, of the error behavior of calculating the phase boundary in numerical
solution to the Stefan problem using a continuous calculation scheme is suggested.
The advantage of this continuous calculation scheme for phase boundary calculation
in problems of surface glaciation in seawater is demonstrated.\\[1mm]
\textit{Keywords}: Stefan problem, glaciation, gas pipeline,
versions of the coefficient smoothing method, numerical solution,
computational examples.
\par}

\vskip5mm

\noindent \textbf{References} }

\vskip 2mm

{\footnotesize

1. Vasilev F. P. On finite difference methods for the solution of
Stefan's single-phase problem.  {\it USSR. Comput. Math. Math.
Phys.}, 1963, vol.~3(5),  pp.~1175--1191.

2.  Kurbatova G. I.,  Ermolaeva N. N. Analysis of the cylinder
glaciation models in seawater. {\it Applied Mathematics and
Information Sciences}, 2017, vol.~11 (3),  pp.~925--930.

3. Samarskii A. A., Moiseyenko B. D. An economic continuous
calculation scheme for the stefan multidi\-mensional problem. {\it
USSR. Comput. Math. Math. Phys.}, 1965, vol.~5 (5), pp.~43--58.

4. Budak B. M.,  Sobol'eva E. N.,  Uspenskii A. B. A difference
method with coefficient smoothing for the solution of Stefan
probkems. {\it USSR Comput. Math. Math. Phys.}, 1965, vol.~5 (5),
pp.~59--76.

5. Eyres N. R.,  Hartree D. R.,   Ingham J. et al.  The
calculation of variable heat flow in solids. {\it Philosophical
Transactions of the Royal Society. Series A}, 1946, vol.~240,
pp.~1--57.

6. Ermolaeva N. N., Kurbatova G. I. Nestacionarnaja model
narastania morskogo l�da [Nonstationary model of the sea ice
growing]. {\it Vestnik of Saint Petersburg University of
Technology and Design. Series Natural and Technical Sciences},
2017, iss.~1, pp.~3--8. (In Russian)

7. Tichonov A. N., Samarskii A. A. \textit{ Equations of
mathematical physics}. New York et all., USA, Pergamon Press Ltd.,
1963, 781~p.

8. Oleinik O. A. Ob odnom metode reshenia obshei zadachi Stefana
[On one solution method of the general Stefan problem]. {\it Paper
of AN USSR}, 1960, vol.~135, no. 5, pp.~1054--1057. (In Russian)

9. Budak B. M., Vasilev F. P., Uspenskii A. B. Raznostnie metodi
reshenia nekotorih kraevih zadach tipa Stefana [Difference methods
of the solution some boundary  Stefan problems]. {\it Chislennie
metodi v gasovoj dinamiki} [{\it Numerical methods in
gas-dynamics}]. Moscow, Calculating Centre of Moscow State
University Publ., 1965, pp.~139--183. (In Russian)

10. Krivovichev G. V., Ermolaeva N. N., Kurbatova G. I., Miheev S.
A. Mathematical modelling of the glaciation process. {\it
International Journal of Pure and Applied Mathematics}, 2017,
vol.~113 (5), pp.~609--616.

11. Shamsundar N., Sparrow E. M. Analysis of multidimensional
conduction phase change via the Enthalpy Model. {\it J. Heat
Transfer}, 1975, vol.~97(3), pp.~333--340.

12. Vasilev V. I.,  Maksimov A. M., Petov E. E., Cipkin G. G.
\textit{Teplomassoperenos v promerzajushih i protaivajushih
gruntah} [{\it The heat and mass transfer in the freezing
ground}]. Moscow, Nauka, Physmatlit Publ., 1996, 224~p. (In
Russian)



\vskip5mm A\,u\,t\,h\,o\,r's \ i\,n\,f\,o\,r\,m\,a\,t\,i\,o\,n:

\vskip1.5mm \textit{Galina I. Kurbatova}~--- Dr. Sci. in Physics
and Mathematics, Professor; g.kurbatova@spbu.ru





}
