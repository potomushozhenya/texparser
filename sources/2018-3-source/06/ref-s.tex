
{\normalsize

\vskip 6mm

\noindent{\bf Evaluation of the volume of ordering of goods while
possible demand drop}

}

\vskip 2mm

{\small

\noindent{\it V.~M.~Bure, V. V. Karelin, A. V. Bure}

\vskip 2mm

{\footnotesize%

%\noindent%
%$^1$~%
%St.\,Petersburg State University, 7--9,
%Universitetskaya nab., St.\,Petersburg,%

%\noindent%
%\hskip2.45mm %
%199034, Russian Federation


\noindent%
%$^2$~%
St.\,Petersburg State University, 7--9,
Universitetskaya nab., St.\,Petersburg,%

\noindent%
%\hskip2.45mm %
199034, Russian Federation

}

\vskip3mm

\noindent \textbf{For citation:}  Bure V. M., Karelin V. V., Bure
A. V. Evaluation of the volume of ordering of goods while possible
demand drop. {\it Vestnik of Saint~Petersburg University. Applied
Mathematics. Computer Science. Control Processes}, \issueyear,
vol.~14, iss.~\issuenum, pp.~\pageref{p6}--\pageref{p6e}.
\doivyp/\enskip%
\!\!\!spbu10.\issueyear.\issuenum06

\vskip3mm

{\leftskip=7mm\noindent In this work, a mathematical model,
designed to determine the optimal strategy of the trading firm's
behavior is constructed under conditions of random demand. As a
result of special marketing research, it is determined that at
some random time $T$ there will be a sharp and strong drop in
demand. It is assumed that the trading company uses the following
scheme of the wholesale order of the goods. All ordered goods are
divided into two parts, the first consignment of goods arriving
immediately, and it must be sold within a certain period of time
$T_1$, if the demand does not drop, then the sale of the second
consignment of goods starts at a discount. The delivery of the
second batch to the buyers occurs at time $T$. The article
considers various situations that arise as a result of a drop in
demand after the moment $T$, either until the time $T_1$ or in the
period between the time moments $T_1$ and $T$. It is necessary to
consider such a wholesale order scheme. Firstly, the warehouses of
the trading company have a limited scope and cannot accommodate
the entire ordered volume of goods, and secondly, the producer
cannot immediately supply the entire ordered lot of goods, since
not all goods could be produced in the initial (zero) time when
the order is made. For the trading company, the moments of time
$T_1$ and $T$ are important. At time $T_1$, the trading company
will completely sell the first shipment of the goods and receive
the funds, part of which it will pay to the firm for the
manufacturer. The time $T$ is also extremely important for the
trading company, as it will mean the successful completion of the
full realization of the entire purchased product. The choice of
time points $T_1$ and $T$ allows determining the volume of the
first consignment of ordered goods and the total volume of all
ordered goods from the manufacturer. In this work a mathematical
model is proposed that allows choosing the optimal ordering
strategy for a trading company in conditions of a possible
drop in demand at a random time.\\[1mm]
\textit{Keywords}: stock level of the goods, random demand,
shortage of goods, discount.
\par}

\vskip5mm

\noindent \textbf{References}

}

\vskip 2mm

{\footnotesize

1. {Bure V. M., Karelin V. V., Polyakova L. N., Yagolnik I. V.}
Modelirovanie processa zakaza dlya kusochno-linejnogo sprosa s
nasyshcheniem [Modeling of the ordering process for
piecewise-linear demand with saturation].{\it Vestnik of Saint
Peterburg University. Applied Mathematics. Computer Science.
Control Processes}, 2017, vol.~13(2), pp.~138--146. (In Russian)

2. {Bure V. M., Karelin V. V., Myshkov S. K., Polyakova L. N.}
Determination of order QUANTITY with piecewise-linear demand
function with saturation. {\it Intern. Journal of Applied
Engineering Research},  2017, vol.~12, no.~18, pp.~7857--7862.

3.  {Giri B. C.,   Sharma S.} Optimal ordering policy for an
inventory system with linearly increasing demand and alowable
shortages under two levels trade credit financing. {\it Oper. Res.
Intern. Journal}, 2016, vol.~16, pp.~25--50.

4. {Aggarwal S. P., Jaggi C. K.}  Ordering policies of
deteriorating items under permissible delay in payments. {\it
Oper. Res. Intern. Journal},  1995, vol.~46(5), pp.~658--662.

5. {Dave U.}  Letters and viewpoints on  economic order quantity
under conditions of permissible delay in payments. {\it  Journal
Oper. Res. Soc.},   1985, vol.~46(5), pp.~1069--1070.

6. {Chen S. C., Teng J. T., Skouri K.}  Economic production
quantity models for deteriorating items with up-stream full trade
credit and down-stream partial trade credit. {\it Intern. Journal
Prod. Econ.}, 2013, vol.~155, pp.~302--309.\newpage

7. {Giri B. C., Sharma S.} An integrated inventory model for a
deteriorating item with allowable shortages and credit linked
wholesale price. {\it Optim. Lett.}, 2015, vol.~37, pp.~624--637.
DOI:10.1007/s11590-014-0810-2.

8.    Goyal S. K. Economic order quantity under conditions of
permissible delay in payments. {\it J. Oper. Res. Soc.}, 1985,
vol.~36(4), pp.~335--338.

9. {Huang Y. F.}    Optimal retailer�s ordering policies in the
EOQ model under trade credit financing.  {\it  J. Oper. Res.
Soc.}, 2003, vol.~54(9), pp.~1011--1015.

10. {Huang Y. F., Hsu K. H.}    An EOQ model under retailer
partial trade credit policy in supply chain. {\it  Intern. Journal
Prod. Econ.}, 2008, vol.~112(2), pp.~655--664.

11. {Jamal A. M. M., Sarker B. R., Wang S.}    An ordering policy
for deteriorating items with allowable shortages and permissible
delay in payment. {\it  J. Oper. Res. Soc.}, 1997, vol.~48(8),
pp.~826--833.

12. {Khanra S., Ghosh S. K., Chaudhuri K. S.}    An EOQ model for
a deteriorating item with time dependent quadratic demand under
permissible delay in payment. {\it  Appl. Math. Comput.}, 2011,
vol.~218(1), pp.~1--9.

13. {Khanra S., Mandal B., Sarkar B.}  An inventory model with
time dependent demand and shortages under trade credit policy.
{\it Econ. Model.}, 2013, vol.~35, pp.~349--355.

14. {Maihami R., Abadi I. N. K.}   Joint control of inventory and
its pricing for non-instantaneously deteriorating items under
permissible delay in payments and partial backlogging. {\it  Math.
Comp. Model.}, 2012, vol.~55(5--6), pp.~1722--1733.

15. {Bure V.~M., ~Karelin V.~V., Polyakova L. N.} Probabilistic
model of terminal services. {\it    Applied Mathematical
Sciences}, 2016, vol.~10 (39),   pp.~1945--1952.

16. {Bure V.~M., ~Karelin V.~V., Polyakova L. N.}  The problem of
resource allocation between the protection system and constructing
redundant components. {\it  Applied Mathematical Sciences},  2015,
vol.~9(93--96),   pp.~4771--4779.

17. {Karelin V.~V.}  Probabilistic model of terminal services.
{\it     Automation and Remote Control},  2004, vol.~65(3),
pp.~483--492.

\vskip6mm A\,u\,t\,h\,o\,r's \ i\,n\,f\,o\,r\,m\,a\,t\,i\,o\,n:%

\vskip2mm \textit{Vladimir M. Bure}  --- Dr. Sci. in Technics,
Professor; vlb310154@gmail.com

\vskip2mm \textit{Vladimir V. Karelin}  --- PhD Sci. in Physics
and Mathematics, Associate Professor; vlkarelin@mail.ru

\vskip2mm \textit{Artem V. Bure}  --- bure.artem@gmail.com

}
