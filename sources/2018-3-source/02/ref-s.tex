
{\normalsize

\vskip 6mm

\noindent{\bf The maximum likelihood method for detecting
communities\\ in communication networks}

}

\vskip 2mm

{\small

\noindent{\it V. V. Mazalov$^{1,2}$, N. N. Nikitina$^2$}

\vskip 2mm

{\footnotesize

\noindent $^1$~St.\,Petersburg State University, 7--9,
Universitetskaya nab., St.\,Petersburg, %

\noindent\hskip2.45mm 199034, Russian Federation

\noindent $^2$~Federal Research Center ``Karelian Research Center %

\noindent\hskip2.45mm of the Russian Academy of Sciences'', 11,
Pushkinskaya ul., Petrozavodsk,


\noindent\hskip2.45mm 185910, Russian Federation


}

\vskip3mm

\noindent \textbf{For citation:}  Mazalov V. V., Nikitina N. N.
The maximum likelihood method for detecting commu\-nities in
communication networks. {\it Vestnik of Saint~Petersburg
University. Applied Mathematics. Computer Science. Control
Processes}, \issueyear, vol.~14, iss.~\issuenum,
pp.~\pageref{p2}--\pageref{p2e}.
\doivyp/\enskip%
\!\!\!spbu10.\issueyear.\issuenum02

\vskip3mm

{\leftskip=7mm\noindent The community detection in social and communication
networks is an important problem in many applied fields: biology, sociology,
social networks. This is especially true for networks that are represented by
large graphs. In this paper, we propose a method for community detection based
on the maximum likelihood method for the random formation of a network with
given parameters of the tightness of connections within the community and
between different communities.  A numerical algorithm for finding the maximum
of the objective function over all possible network partitions is described.
The algorithm is implemented and tested on real networks of small dimension.%
\\[1mm]
\textit{Keywords}: network communities, detecting communities in a
network, maximum likelihood method, Gibbs sampling.
\par}

\vskip5mm

\noindent \textbf{References} }

\vskip 2mm

{\footnotesize

1. Freeman L. C. A set of measures of centrality based on
betweenness. \textit{Sociometry}, 1977, vol.~40, pp.~35--41.

2. Levin D. A., Peres Y. \textit{Markov chains and mixing times}.
Providence, Rhode Island, Amer. Mathematical Soc. Publ., 2017,
447~p.

3. Page L., Brin S., Motwani R., Winograd T. \textit{The PageRank
citation ranking: Bringing order to the web}. Technical Report.
Stanford, Stanford InfoLab Publ., 1998, 17~p.

4. Pons P., Latapy M. Computing communities in large networks
using random walks. \textit{Journal of Graph Algorithms and
Applications}, 2006, vol.~10(2), pp.~191--218.

5. Avrachenkov K. E., Mazalov V. V., Tsynguev B. T. Beta current
flow centrality for weighted networks. \textit{Proceedings of
CSoNET--2015}, LNCS, 2015, vol.~9197, pp.~216--227.

6. Brandes U., Fleischer D. Centrality measures based on current
flow. \textit{Proceedings of the 22nd annual conference on
Theoretical Aspects of Computer Science}, 2005, pp.~533--544.

7. Mazalov V., Tsynguev B. Kirchhoff centrality measure for
collaboration network. \textit{CSoNet--2016}, LNCS, 2016,
vol.~9795, pp.~147--157.

8. Bogomolnaia A., Jackson M. O. The stability of hedonic
coalition structures. \textit{Games and Economic Behavior}, 2002,
vol.~38(2), pp.~201--230.

9. Mazalov V. V., Avrachenkov K. E., Trukhina L. I., Tsynguev B.
T. Game-theoretic centrality measures for weighted graphs.
\textit{Fundamenta Informaticae}, 2016, vol.~145(3), pp.~341--358.

10. Fortunato S., Barthelemy M. Resolution limit in community
detection. \textit{Proceedings of the National Academy of Sciences
USA}, 2007, vol.~104(1), pp.~36--41.

11. Girvan M., Newman M.\,E.\,J. Community structure in social and
biological networks. \textit{Proceedings of the National Academy
of Sciences USA}, 2002, vol.~99(12), pp.~7821--7826.

12. Fortunato S. Community detection in graphs. \textit{Physics
Reports}, 2010, vol.~486(3), pp.~75--174.

13. Zachary W. W. An information flow model for conflict and
fission in small groups. \textit{Journal of Anthropological
Research}, 1977, vol.~33(4), pp.~452--473.

14. Jackson M. O. \textit{Social and economic networks}.
Princeton, Princeton University Press, 2010, 520~p.

15. Kaur R., Singh S. A survey of data mining and social network
analysis based anomaly detection techniques. \textit{Egypt Inf.
Journal}, 2016, vol.~17(2), pp.~199--216.

16. Newman M. E., Girvan M. Finding and evaluating community
structure in networks. \textit{Physical Review E}, 2004,
vol.~69(2), pp.~026113.

17. Meila M., Shi J. A Random Walks View of spectral segmentation.
\textit{Proceedings of AISTATS}, 2001, pp.~1--6.

18. Myerson R. B. Graphs and cooperation in games. \textit{Math.
Oper. Res.}, 1977, vol.~2, pp.~225--229.

19. Avrachenkov K., Kondratev A., Mazalov V. Cooperative Game
Theory approaches for network partitioning. \textit{Computing and
Combinatorics}. Eds. by Y. Cao, J. Chen. COCOON, 2017, LNCS, 2017,
vol.~10392, pp.~591--603.

20. Copic J., Jackson M., Kirman A. Identifying community
structures from network data via maxi-\linebreak mum likelihood
methods. \textit{The B.~E. Journal of Theoretical Economics},
2009, vol.~9, iss.~1, pp.~1935--1704.

21. {\it Wolfram Research$,$ Inc}. (www.wolfram.com). Mathematica
Online. Champaign, IL., 2018.

22. Vinh N. X., Epps J., Bailey J. Information theoretic measures
for clusterings comparison: Variants, properties, normalization
and correction for chance. \textit{Journal of Machine Learning
Research}, 2010, vol.~11, pp.~2837--2854.

23. Kvalseth T. O. Entropy and correlation: Some comments.
\textit{Systems, Man and Cybernetics, IEEE Transactions on}, 1987,
vol.~17(3), pp.~517--519.

24. Strehl A., Ghosh J. Cluster ensembles --- a knowledge reuse
framework for combining multiple partitions. \textit{Journal of
Machine Learning Research}, 2002, vol.~3, pp.~583--617.

25. Hubert L., Arabie P. Comparing partitions. \textit{Journal of
Classification}, 1985, vol.~2(1), pp.~193--218.

26. Steinley D. Properties of the Hubert-Arabie adjusted Rand
index. \textit{Psychol. Methods}, 2004, vol.~9(3), pp.~386--396.

27. Lusseau D., Schneider K., Boisseau O. J. et al. The bottlenose
dolphin community of Doubtful Sound features a large proportion of
long-lasting associations. \textit{Behav. Ecol. Sociobiol.}, 2003,
vol.~54, pp.~396--405.



\vskip6mm A\,u\,t\,h\,o\,r's \ i\,n\,f\,o\,r\,m\,a\,t\,i\,o\,n:
\vskip2mm \textit{Vladimir V. Mazalov}~--- Dr. Sci. in Physics and
Mathematics, Professor; vmazalov@krc.karelia.ru

\vskip2mm \textit{Natalia N. Nikitina}~--- PhD Sci. in Technics;
nikitina@krc.karelia.ru
\par}
