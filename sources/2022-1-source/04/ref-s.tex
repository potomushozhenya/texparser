
{\normalsize

\vskip 6mm

%\newpage


\noindent{\bf Calculation of the ionization potential of zinc and graphene phthalocyaninates on the surface of dielectrics$^*$%
 }

}

\vskip 2mm

{\small

\noindent{\it D. Yu. Kuranov, T. A. Andreeva, M. E. Bedrina%
%, I.~�. Famylia%$~^2$%

 }

\vskip 2mm

%%%%%%%%%%%%%%%%%%%%%%%%%%%%%%%%%%%%%%%%%%%%%%%%%%%%%%%%%%%%%%%%%%

\efootnote{
%%
%\vspace{-3mm}\parindent=7mm
%%
%\vskip 0.0mm
\hskip 4mm$^{*}$ This work was supported by the Russian Foundation for Basic Research (project N 20-07-01086).%\par
%%
%%\vskip 2.0mm
%%
%\indent{\copyright} �����-������������� ���������������
%�����������, \issueyear%
%%
}

%%%%%%%%%%%%%%%%%%%%%%%%%%%%%%%%%%%%%%%%%%%%%%%%%%%%%%%%%%%%%%%%%%

{\footnotesize


\noindent%
%$^2$~%
St~Petersburg State University, 7--9, Universitetskaya nab.,
St~Petersburg,

\noindent%
%\hskip2.45mm%
199034, Russian Federation

}

%%%%%%%%%%%%%%%%%%%%%%%%%%%%%%%%%%%%%%%%%%%%%%%%%%%%%%%%%%%%%%%%%%
%\newpage
\vskip3.0mm%3mm

\noindent \textbf{For citation:} Kuranov D. Yu., Andreeva T. A., Bedrina M. E. Calculation of the ionization potential of zinc and graphene phthalocyaninates on the surface of dielectrics. {\it Vestnik of Saint~Pe\-ters\-burg Uni\-versity.
Applied Mathe\-ma\-tics. Computer Science. Control Pro\-cesses},
\issueyear,
vol.~18, iss.~\issuenum, pp.~\pageref{p4}--\pageref{p4e}.  %\\
\doivyp/\enskip%
\!\!\!spbu10.\issueyear.\issuenum04 (In Russian)

\vskip2.0%3
mm

\begin{hyphenrules}{english}

{\leftskip=7mm\noindent A mathematical model is proposed for calculating the ionization potentials of molecules on the surface of dielectrics in order to quantify changes in the electronic characteristics of materials on a substrate. The semiconductor and photoelectronic properties of nanosystems based on phthalocyanine derivatives are determined by the electronic structure of molecules. Based on the zinc phthalocyaninate molecule  ZnC$_3$$_2$N$_8$H$_1$$_6$, model structures are constructed that increase this molecule by attaching benzene rings ZnC$_{48}$N$_8$H$_{24}$, ZnC$_{64}$N$_8$H$_{32}$ and a model simulating the film structure of Zn$_4$C$_{120}$N$_{32}$ H$_{48}$. Graphene was considered as a nanostructure modeling a fragment of a monomer lm. The ionization potentials of these compounds on the surface of magnesium oxide, sodium chloride and silicon are calculated. In the presence of a substrate, the ionization potentials of all nanostructures decrease, while the values of the surface ionization potentials remain fundamentally dierent in their magnitude for all compounds. The compound ZnC$_{64}$N$_8$H$_{32}$ sprayed onto the surface exhibits the best photoelectronic properties, its surface ionization potential is comparable to graphene.\\[1mm]
\textit{Keywords}: phthalocyanine zinc, graphene, structure, dielectric surface, ionization potential. \par}

\end{hyphenrules}

\vskip 6mm

\noindent \textbf{References} }

\vskip 2 mm

{\footnotesize

1. Egorov N. V., Vinogradova E. M. Mathematical modeling of triode system on the basis of field emitter with ellipsoid shape. {\it Vestnik of Saint Petersburg University. Applied
Mathematics. Computer Science. Control Processes}, 2021, vol.~17, iss.~2, pp.~131--136. \\ https://doi.org/10.21638/11701/spbu10.2021.203 %(In Russian)

2. Simon J., Andre J.-J. {\it Molecular Semiconductors: Photoelectrical
Properties and solar cells}.  Eds. by J. M. Lehn, C. W. Rees. Reprint of the original $1^{\rm st}$ ed. 1985. Springer Publ., 2011, 306~p. (Rus. ed.: Simon Zh., Andre Zh. Zh. {\it Molekuliarnye poluprovodniki. Fotoelektricheskie svoistva i solnechnye elementy}. Moscow, Mir Publ., 1988, 342~p.).

3. Xie D., Pan W., Jiang Y. D., Li Y. R. Erbium bis[phthalocyaninato] complex LB film gas sensor. {\it Materials Letters}, 2003, vol.~57, iss.~16--17,  pp.~2395--2398.

4. Dini D., Hanack M.  {\it Physical properties of phthalocyanine-based materials}. The porphyrin handbook. Eds. by K. M. Kadish, K. M. Smith, R. Guilard. Netherlands, Elsevier Science Publ., 2003, vol.~17, ch.~107, pp.~1--36.

5. Dini D., Calvete M. J. F., Hanack M.  Nonlinear optical materials for the smart filtering of optical radiation.  \textit {Chem. Rev.}, 2016, vol.~116, iss.~22,  pp.~13043--13233.

6. Mroz P., Tegos G., Gali H.  Photodynamic therapy with fullerenes. \textit{Photochemical $\&$ Photobiological Sciences}, 2007, vol.~6, iss.~11,  pp.~1139--1149.

7. Yourre T. A., Rudaya L. I., Klimova N. V. {\it Polymers, phosphors, and voltaics for radioisotope microbatteries}. Boca Raton, CRC Press, 2002, 504~p.

8. Wohrle D., Schnurpfeil G., Makarov S. G., Kazarin A., Suvorova O. N. Practical applications of phthalocyanines --- from dyes and pigments to materials for optical, electronic and photo-electronic devices. \textit{Macroheterocycles}, 2012, vol.~5, iss.~3,  pp.~191--202.

9. Kruchinin V. N., Klyamer D. D., Spesivcev E. V., Ryhlickij S. V., Basova T. V.  Opticheskie svojstva tonkih plenok ftalocianinov cinka po dannym spektralnoj ellipsometrii [Optical properties of thin films of zinc phthalocyanines according to spectral ellipsometry]. \textit{Optika i spektroskopiia}  [\textit{Optics and spektroscopy}], 2018, vol.~125, iss.~6, pp.~825--829. (In Russian)

10. Mirabito T., Huet B., Briseno A. L., Snyder D. W. Physical vapor deposition of zinc phthalocyanine nanostructures on oxidized silicon and graphene substrates. \textit{Journal of Crystal Growth}, 2020, vol.~533, pp.~2--6.

11. Semenov S. G. Kvantovo-himicheskaya model molekuly v polyarizuyushchej srede [Quantum-chemical model of a molecule in a polarizing medium]. \textit{Zhurnal strukturnoj himii} [\textit{Journal of Structural Chemistry}], 2001, vol.~42, iss.~3, pp.~582--586. (In Russian)

12. Kuranov D. Yu., Bedrina M. E. Modelirovanie vzaimodejstviya nanostruktur s poverhnostyu [Modeling the interaction of nanostructures with a surface]. \textit{Nano- i mikrosistemnaya tekhnika} [\textit{Nano- and microsystems technology}], 2019, vol.~21, iss.~2, pp.~83--88. (In Russian)

13. Koch W., Holthausen M. {\it A chemist's guide to density functional theory.} Ed.~2. Weinheim, Wiley-VCH Press, 2002, 293~p.

14. Bedrina M. E., Egorov N. V., Kuranov D. Yu., Semenov S. G.  Raschet ftalocianinatov cinka na vysokoproizvoditelnom vychislitelnom komplekse [Calculation metalphthalocyaninatozinc on the high-efficiency computer complex]. {\it Vestnik of Saint Petersburg University. Series 10. Applied
Mathematics. Computer Science. Control Processes}, 2011, iss.~3, pp.~13--21. (In Russian)

15. Becke A. D. Density-functional thermochemistry. 3. The role of exact exchange. {\it Thin Solid Films}, 1993, vol.~98, iss.~7,  pp.~5648--5652.

16. Vasiliev A. A., Bedrina M. E., Andreeva T. A. Zavisimost' rezul'tatov rascheta po metodu funktsionala elektronnoi plotnosti ot sposoba predstavleniia volnovoi funktsii [The dependence of calculation results on the density functional theory from the means of presenting the wave function]. \textit{Vestnik of Saint Petersburg University. Applied Mathematics. Computer Science. Control Processes}, 2018, vol.~14, iss.~1, pp.~51--58. https://doi.org/10.21638/11701/spbu10.2018.106 (In Russian)

17. Andreeva T. A., Bedrina M. E., Ovsyannikov D. A. Sravnitel'nyi analiz raschetnykh metodov elektronnoi spektroskopii [Comparative analysis of calculation methods in electron spectroscopy].  \textit{Vestnik of Saint Petersburg University. Applied Mathematics. Computer Science. Control Processes,} 2019, vol.~15, iss.~4, pp.~518--528. https://doi.org/10.21638/11701/spbu10.2019.408 (In Russian)

18. Frisch M. J., Trucks G. W., Schlegel H. B. et al.  \textit{GAUSSIAN-09, Rev. C.01}. Wallingford, CT, Gaussian Inc. Press, 2010.

19. Novoselov K. S., Geim A. K., Morozov S. V., Jiang D., Zhang Y., Dubonos S. V., Gri\-go\-rie\-va~I.~V.,  Fir\-sov A. A.  Electric field effect in atomically thin carbon films. \textit{Science}, 2004, vol.~306, iss.~5696,  pp.~666--669.

20. Novoselov K. S., Geim A. K., Morozov S. V., Jiang D., Kats\-nel\-son M. I., Gri\-go\-rie\-va~I.~V., Du\-bo\-nos S. V., Fir\-sov A. A. Two-dimensional gas of massless Dirac fermions in graphene. \textit{Nature}, 2005, vol.~438, iss.~7065,  pp.~197--200.

21. Vilesov F. I., Zagrubskij A. A., Garbuzov D. E. Vneshnij fotoeffekt s poverhnosti organicheskih poluprovodnikov [External photoelectric effect from the surface of organic semiconductors].  \textit{Fizika tverdogo tela} [\textit{Physics of the Solid State}], 1963, vol.~5, iss.~7, pp.~2000--2006. (In Russian)



\vskip 1.5mm

Received:  November 06, 2021.

Accepted: February 01, 2022.


\vskip6 mm A~u~t~h~o~r~s'\,  i~n~f~o~r~m~a~t~i~o~n:


\vskip2 mm \textit{Dmitry Yu. Kuranov} --- PhD in Physics and Mathematics, Associate Professor; d.kuranov@spbu.ru \par%
%
\vskip2 mm \textit{Tatiana A. Andreeva} --- PhD in Physics and Mathematics, Associate Professor; t.a.andreeva@spbu.ru \par
%
\vskip2 mm \textit{Marina E. Bedrina} --- Dr. Sci. in Physics and Mathematics,  Professor; m.bedrina@spbu.ru \par
%
}
