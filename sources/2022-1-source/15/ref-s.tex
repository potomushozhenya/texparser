
{\normalsize

\vskip 6%
mm

\noindent{\bf Procedure for regularization of bilinear optimal control problems based\\ on a finite-dimensional model$^{*}$
 }

}

\vskip 3%
mm

{\small

\noindent{\it A.~V.~Arguchintsev, V.~A.~Srochko %
%, I.~�. Famylia%$\,^2$%

 }

\vskip 3%
mm

%%%%%%%%%%%%%%%%%%%%%%%%%%%%%%%%%%%%%%%%%%%%%%%%%%%%%%%%%%%%%%%%%%

\efootnote{
%%
\vspace{-3mm}\parindent=7mm
%%
\vskip 0.1mm \indent~$^{*}$ This project was supported by the Vladimir Potanin Foundation (grant GSAD-0022/21).\par
%%
%%\vskip 2.0mm
%%
%%\indent{\copyright} �����-������������� ���������������
%%�����������, \issueyear%
%%
}

%%%%%%%%%%%%%%%%%%%%%%%%%%%%%%%%%%%%%%%%%%%%%%%%%%%%%%%%%%%%%%%%%%

{\footnotesize


\noindent%
%$^2$~%
Irkutsk State University, 1, ul. K. Marksa, Irkutsk,

\noindent%
%\hskip2.45mm%
664003, Russian Federation



%\noindent%
%%$^2$~%
%St.\,Petersburg State University, 7--9, Universitetskaya nab., %St.\,Petersburg,
%
%\noindent%
%%\hskip2.45mm%
%199034, Russian Federation

}

%%%%%%%%%%%%%%%%%%%%%%%%%%%%%%%%%%%%%%%%%%%%%%%%%%%%%%%%%%%%%%%%%%

\vskip 3mm


\noindent \textbf{For citation:}  Arguchintsev A.~V., Srochko V.~A. Procedure for regularization of bilinear optimal control problems based on a finite-dimensional model. {\it Vest\-nik of Saint Petersburg
University. Applied Mathe\-ma\-tics. Computer
Science. Cont\-rol Pro\-cesses}, %\,
\issueyear,
vol.~18, iss.~\issuenum,~pp.~\pageref{p15}--\pageref{p15e}. \\
\doivyp/\enskip%
\!\!\!spbu10.\issueyear.\issuenum15 (In Russian)

\vskip3mm

%\pagebreak

\begin{hyphenrules}{english}

{\leftskip=7mm\noindent An optimization problem of a linear system of ordinary differential equations on a set of piecewise continuous scalar controls with two-sided restrictions is considered. The cost functional contains the bilinear part (control, state) and a control square with a parameter, which plays the role of a regularization term. An approximate solution of the optimal control problem is carried out on a subset of piecewise constant controls with a non-uniform grid of possible switching points. As a result of the proposed parametrization, reduction to the finite-dimensional problem of quadratic programming was carried out with the parameter in the objective function and the simplest restrictions. In the case of a strictly convex objective function, the finite-dimensional problem can be solved in a finite number of iterations by the method of special points. For strictly concave objective functions, the corresponding problem is solved by simple or specialized brute force methods. In an arbitrary case, parameter conditions and switching points are found at which the objective function becomes convex or concave. At the same time, the corresponding problems of mathematical programming allow a global solution in a finite number of iterations. Thus, the proposed approach allows to approximate the original non-convex variation problem with a finite-dimensional model that allows to find a global solution in a finite number of iterations.
\\[1mm]
\textit{Keywords}: linear phase system,  bilinear-quadratic functional, finite-dimensional model, finite iterative methods, global solution.\par}

\end{hyphenrules}

\vskip6mm
%\pagebreak

\noindent \textbf{References} }

\vskip 3mm

{\footnotesize

1. Boltyansky V. G. {\it Optimalnoe upravlenie diskretnymi sistemami}
[{\it Optimal control of discrete systems}]. Moscow, Nauka Publ., 1973, 448~p. (In Russian)

2. Propoi A. I. {\it Elementy teorii optimal'nykh diskretnykh protsessov} [{\it Elements of the theory of optimal discrete processes}]. Moscow, Nauka Publ., 1973, 255~p. (In Russian)

3. Poswiata A. Optimal discrete processes, nonlinear in time intervals: theory and selected applications.
{\it Cybernetics and Physics}, 2012, vol.~1, no.~2, pp.~120--127.

4. Kotina E. D., Ovsyannikov D. A. Matematicheskaya model sovmestnoy optimizatsii programmnogo i vozmushennogo dvizheniy v diskretnykh sistemakh  [Mathematical model of joint optimization of programmed and perturbed motions in discrete systems]. {\it Vestnik of Saint Petersburg University. Applied Mathematics. Computer Science. Control Processes}, 2021, vol.~17, iss.~2, pp.~213--224. \\
https://doi.org/10.21638/11701/spbu10.2021.210 (In Russian)

5. Rao A. V. A survey of numerical methods for optimal control. {\it Adv. Astron. Sci.}, 2009, vol.~135, pp.~1--32.

6. Golfetto W. A., Silva Fernandes S. A review of gradient algorithms for numerical computation of optimal trajectories. {\it J. Aerosp. Technol. Manag.}, 2012, vol.~4, pp.~131--143. \\ https://doi.org/10.5028/JATM.2012.04020512

7. Gabasov R., Dmitruk N. M., Kirillova F. M. Chislennye metody optimizatsii nestatsionarnykh mnogomernykh sistem s poliedral'nymi ogranicheniiami [Numerical optimization of time-dependent multidimensional systems under polyhedral constraints]. {\it Zhurnal vychislitel�noi matematiki i ma\-te\-ma\-ti\-cheskoi fiziki} [{\it Comput. Math. Math. Phys.}], 2005, vol.~45, no.~4, pp.~617--636. (In Russian)

8. Gorbunov V. K., Lutoshkin I. V. Development and experience of using the para\-met\-eri\-za\-tion
method in singular problems of dynamic optimization. {\it J. Comput. Syst. Sci. Int.}, 2004, vol.~43, no.~5, pp.~725--742.

9. Korsun O. N., Stulovskii A. V. Direct method for forming the optimal open loop control of aerial vehicles.
{\it J. Comput. Syst. Sci. Int.}, 2019, vol.~58, no.~2, pp.~229--243. \\
https://doi.org/10.1134/S1064230719020114

10. Chernov A. V.  O primenenii funktsii Gaussa dlia chislennogo resheniia zadach optimal'nogo upravleniia [On application of Gaussian functions to numerical solution of optimal control problems].
{\it Avtomatika i telemekhanika}  [{\it Autom. Remote Control}], 2019, no.~6, pp.~1026--1040. \\
https://doi.org/10.1134/S0005231019060035 (In Russian)

11. Popkov A. S. Optimal program control in the class of quadratic splines for linear systems.
{\it Vestnik of Saint Petersburg University. Applied Mathematics. Computer Science. Control Processes}, 2020, vol.~16, iss.~4, pp.~462--470.  %\\
https://doi.org/10.21638/11701/spbu10.2020.411

12. Fominyh A. V., Karelin V. V., Polyakova L. N., Myshkov S. K., Tregubov V. P. Metod kodifferentsial'nogo spuska v zadache nakhozhdeniya globalnogo minimuma kusochno-afinnogo tselevogo funktsionala [The codifferential descent method in the problem of finding the global minimum of a piecewise affine objective functional in linear control systems].
{\it Vestnik of Saint Petersburg University. Applied Mathematics. Computer Science. Control Processes}, 2021, vol.~17, iss.~1, pp.~47--58. \\ https://doi.org/10.21638/11701/spbu10.2021.105
   (In Russian)


13. Srochko V. A., Aksenyushkina E. V. Parametrizatsiya nekotorykh zadach upravleniya lineynymi sistemami  [Parameterization of some linear systems control problems]. {\it The Bulletin of Irkutsk State University. Series Mathematics}, 2019, vol.~30, pp.~83--98. \\ https://doi.org/10.26516/1997-7670.2019.30.83 (In Russian)

14. Arguchintsev A. V., Dykhta V. A., Srochko V. A. Optimal control: nonlocal conditions, computational methods, and the variational principle of maximum. {\it Russian Math.}, 2009, vol.~53, no.~1, pp.~1--35.

15. Antonik V. G., Srochko V. A.  Usloviia optimal'nosti tipa printsipa maksimuma v bilineinykh zadachakh upravleniia [Optimality conditions of the maximum principle type in bilinear control problems].
{\it Zhurnal vychislitel�noi matematiki i matematicheskoi fiziki}   [{\it Comput. Math. Math. Phys.}], 2016, vol.~56, no.~12, pp.~2054--2064.
https://doi.org/10.7868/S0044466916120024 (In Russian)

16. Izmailov A. F., Solodov M. V. {\it Chislennye metody optimizatsii} [{\it Numerical methods of optimization}].
Moscow, Fizmatlit Publ., 2005, 304~p. (In Russian)

17. Srochko V. A., Aksenyushkina E. V., Antonik V. G. Reshenie lineyno-kvadratichnoy zadachi optimal'nogo upravleniya na osnove konechnomernoy modeli  [Resolution of a linear-quadratic optimal control problem based on finite-dimensional models]. {\it The Bulletin of Irkutsk State University. Series Mathematics}, 2021, vol.~37, pp.~3--16.
https://doi.org/10.26516/1997-7670.2021.37.3 (In Russian)

\vskip1.5mm Received:  December 29, 2021.

Accepted: February 01, 2022.

\vskip6mm A\,u\,t\,h\,o\,r\,s' \, i\,n\,f\,o\,r\,m\,a\,t\,i\,o\,n:%

\vskip2mm \textit{Alexander V. Arguchintsev} --- Dr. Sci. in Physics and Mathematics, Professor; arguch@math.isu.ru \par
%
\vskip2mm \textit{Vladimir A. Srochko} --- Dr. Sci. in Physics and Mathematics, Professor; srochko@math.isu.ru \par
%
%
}
