
{\normalsize

\vskip 6mm
%\newpage

\noindent{\bf On the convergence of dynamic quasi-periodic systems %$^{*}$
}

}

\vskip 3%
mm

{\small

\noindent{\it S. A. Strekopytov, M. V. Strekopytova%$\,^2$%
%, I.~�. Famylia%$\,^2$%

 }

\vskip 3%
mm

%%%%%%%%%%%%%%%%%%%%%%%%%%%%%%%%%%%%%%%%%%%%%%%%%%%%%%%%%%%%%%%%%%

%\efootnote{
%%
%\vspace{-3mm}\parindent=7mm
%%
%\vskip 0.1mm \indent\indent~$^{*}$ This work was  supported by the%
%Russian Foundation for Basic Research (grant N 19-01-00146-a).\par
%%
%%\vskip 2.0mm
%%
%%\indent{\copyright} �����-������������� ���������������
%%�����������, \issueyear%
%%
%}

%%%%%%%%%%%%%%%%%%%%%%%%%%%%%%%%%%%%%%%%%%%%%%%%%%%%%%%%%%%%%%%%%%

{\footnotesize



\noindent%
%$^2$~%
St\,Petersburg State University, 7--9, Universitetskaya nab., St\,Petersburg,

\noindent%
%\hskip2.45mm%
199034, Russian Federation

}

%%%%%%%%%%%%%%%%%%%%%%%%%%%%%%%%%%%%%%%%%%%%%%%%%%%%%%%%%%%%%%%%%%

\vskip 2,5%3%
mm


\noindent \textbf{For citation:}  Strekopytov S. A., Strekopytova M. V. On the convergence of dynamic quasi-periodic systems. {\it Vest\-nik of Saint Petersburg
University. Applied Mathe\-ma\-tics. Computer
Science. Cont\-rol Pro\-cesses}, %\,
\issueyear,
vol.~18, iss.~\issuenum,~pp.~\pageref{p6}--\pageref{p6e}. \\
\doivyp/\enskip%
\!\!\!spbu10.\issueyear.\issuenum06 (In Russian)

\vskip3mm

%\pagebreak

\begin{hyphenrules}{english}

{\leftskip=7mm\noindent The convergence problem for non-autonomous systems of differential equations with a quasi-periodic right-hand side in an independent argument is considered. It is proposed to replace the consideration of the set of solutions of the system of differential equations under consideration by considering the movements of a dynamic quasi-periodic system generated by these differential equations. Necessary and sufficient conditions are obtained when a dynamic quasi-periodic system has the convergence property, and a proof is given.
\\[1mm]
\textit{Keywords}: convergence, dynamic quasi-periodic system, quasi-periodic motion.\par}

\end{hyphenrules}

\vskip6mm
%\pagebreak

\noindent \textbf{References} }

\vskip 3mm

{\footnotesize

1. Birkhoff G. {\it Dinamicheskie sistemi} [{\it Dynamical systems}]. Ijevsk, Udmurtskiy University Press, 1999, 408~p. (In Russian)

2. Zubov V. I. {\it Kolebaniya v nelineynih i upravlyaemih sistemah} [{\it Oscillations in nonlinear and controlled systems}]. Leningrad, Sudpromgiz Publ., 1962, 632~p. (In Russian)

3. Zubov V. I. {\it Kolebaniya i volni} [{\it Oscillations and waves}]. Leningrad, Leningrad University Press, 1989, 416~p. (In Russian)

4. Pliss V. A. {\it Nelokalnye problemi teorii kolebaniy} [{\it Nonlocal problems of the theory of oscillations}]. Moscow, Nauka Publ., 1964, 368~p. (In Russian)

5. Pliss V. A. {\it Integralnie mnojestva periodicheskih sistem differencialnih uravneniy} [{\it Integral sets of periodic systems of differential equations}]. Moscow, Nauka Publ., 1977, 304~p. (In Russian)

6. Yakubovich V. A., Starzhinskii V. M. {\it Lineynie differencialnie uravneniya s periodicheskimi koefficientami i ih prilojeniya} [{\it Linear differential equation with periodic coefficients and applications}]. Moscow, Nauka Publ., 1972, 720~p. (In Russian)

7. Levitan B., Zhikov V. {\it Almost periodic functions and differential equations}. New York, Cambridge University Press, 1982, 211~p.

8. Kosov A. A. Issledovanie konvergencii slojnih pochti periodicheskih sistem s pomosh'u vektor-funkciy sravneniya s komponentami v vide form chetnoy stepeni [Investigation of convergence of large scale almost periodic systems by means of comparison vector functions
with components as forms of even degrees]. {\it Izvestiia vysshikh uchebnykh zavedenii. Matematika} [{\it Russian Mathematics}], 2015, vol.~59, no.~7, pp.~25--35. (In Russian)

9. Kosov A. A., Shchennikov V. N. On the convergence phenomenon in complex almost periodic systems. {\it Differential Equations},
2014, vol.~50, iss.~12, pp.~1573--1583.

10. Aleksandrov A., Aleksandrova E. Convergence conditions for some classes of nonlinear systems. {\it Systems $\&$ Control Letters},
2017, vol.~104, pp.~72--77.

11. Strekopytov S. A. {\it Teoriya kvasiperiodicheskih sistem} [{\it Theory of quasi-periodic systems}]. St~Pe\-ters\-burg, VVM Publ., 2014, 157~p. (In Russian)

12. Ataeva N. N. Svoiystvo konvergencii dlya raznostnih sistem [Property of convergence for difference systems]. {\it Vestnik of Saint Petersburg University. Series 10. Applied Mathematics. Computer Science. Control Processes}, 2004, iss.~4, pp.~91--98. (In Russian)


13. Ekimov A. V., Zhabko A. P., Yakovlev P. V. Ustoiychivost' differencial'no-raznostnih sistem s lineyno vozrastayushim zapazdivaniem [The stability of differential-difference
equations with proportional time delay]. {\it Vestnik of Saint Petersburg University. Applied Mathematics. Computer Science. Control Processes}, 2020, vol.~16, iss.~3, pp.~316--325. https://doi.org/10.21638/11701/spbu10.2020.308
 (In Russian)

\vskip1.5mm Received:  January 13, 2022.

Accepted: February 01, 2022.

\vskip6mm A\,u\,t\,h\,o\,r\,s' \,\ i\,n\,f\,o\,r\,m\,a\,t\,i\,o\,n:%


\vskip2mm \textit{Sergey A. Strekopytov} --- PhD in Physics and Mathematics, Associate Professor; sastrek@yandex.ru \par

\vskip2mm \textit{Maria V. Strekopytova} --- PhD in Physics and Mathematics; mariya-str@yandex.ru \par
%
%
}
