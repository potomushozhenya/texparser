
{\normalsize

\vskip 6mm

\noindent{\bf Mathematical modeling of malignant ovarian tumors% $^{*}$%
 }

}

\vskip 3mm

{\small

\noindent{\it A. B. Goncharova$^1$, E. P. Kolpak$^1$, M. Yu. Vil'$^1$, A. V. Abramova$^2$, E. A. Busko$^{1,2}$%
%, I.~�. Famylia%$\,^2$%

}

\vskip 3mm

%%%%%%%%%%%%%%%%%%%%%%%%%%%%%%%%%%%%%%%%%%%%%%%%%%%%%%%%%%%%%%%%%%

%\efootnote{
%%
%\vspace{-3mm}\parindent=7mm
%%
%\vskip 0.1mm $^{*}$ This work  was supported by the Russian %Foundation for Basic Research (project N 19-31-90033).%\par
%%
%%\vskip 2.0mm
%%
%%\indent{\copyright} �����-������������� ���������������
%%�����������, \issueyear%
%%
%}

%%%%%%%%%%%%%%%%%%%%%%%%%%%%%%%%%%%%%%%%%%%%%%%%%%%%%%%%%%%%%%%%%%

{\footnotesize

\noindent%
$^1$~%
St\,Petersburg State University, 7--9, Universitetskaya nab.,
St\,Petersburg,

\noindent%
\hskip2.45mm%
199034, Russian Federation


\noindent%
$^2$~%
N. N. Petrov National Medical Research Center of Oncology of the Ministry  of Healthcare

\noindent%
\hskip2.45mm%
of the Russian Federation, 68, ul. Leningradskaya, pos. Pesochny, St Petersburg,

\noindent%
\hskip2.45mm%
197758, Russian Federation


}

%%%%%%%%%%%%%%%%%%%%%%%%%%%%%%%%%%%%%%%%%%%%%%%%%%%%%%%%%%%%%%%%%%

\vskip3mm


\noindent \textbf{For citation:} Goncharova A. B., Kolpak E. P., Vil' M. Yu., Abramova A. V., Busko E. A. Mathematical modeling of malignant ovarian tumors. {\it Vestnik of Saint~Petersburg Uni\-ver\-si\-ty.
Ap\-plied Mat\-he\-matics. Computer Science. Control
Processes},\,\issueyear,
vol.~18,~iss.~\issuenum,~pp.~\pageref{p10}--\pageref{p10e}. \\
\doivyp/\enskip%
\!\!\!spbu10.\issueyear.\issuenum10 (In Russian)

\vskip3mm

\begin{hyphenrules}{english}

{\leftskip=7mm\noindent The article explores modeling the development of ovarian cancer, the treatment of this oncologicaldisease in women, the assessment of the time to achieve remission, and the assessment of the time of the onset of relapse. The relevance of the study is that ovarian cancer is one of the most common cancers in women and has the highest mortality rate among all gynecological diseases. Modeling the process of the development of the disease makes it possible to better understand the mechanism of the development of the disease, as well as the time frame of the onset of each stage, as well as the assessment of the survival time. The aim of the work is to develop a model of an  ovarian tumor. It is based on a model of competition between two types of cells: epithelial cells (normal cells) and tumor cells (dividing cells). The mathematical interpretation of the competition model is the Cauchy problem for a system of ordinary differential equations. Treatment is seen as the direct destruction of tumor cells by drugs. The behavior of solutions in the vicinity of stationary points is investigated by the eigenvalues of the Jacobi matrix of the right side of the equations. On the basis of this model, the distribution of conditional patients by four stages of the disease is proposed. Biochemical processes that stimulate the accelerated growth of the tumor cell population are modeled by a factor that allows tumor cells to gain an advantage in a competitive relationship with epithelial cells. The spatio-temporal dynamics of an ovarian tumor leads to a modification of the competition model due to the introduction of additional factors into it, taking into account the presence of increased nutrition of ovarian tumors, the exit of the tumor from the plane of the ovary, as well as the effect of treatment on tumor cells. The new model describes the interaction conditions with a system of second-order partial differential equations. The results of computer modeling demonstrate an assessment of the distribution of conditional patients by stages of the disease, the time of onset of relapse, the duration of remission, the obtained theoretical results of modeling are compared with the  real data.\\[1mm]
\textit{Keywords}: mathematical modeling, treatment model, malignant tumor of the ovary, reproductive system.\par}

\end{hyphenrules}

\vskip6mm

\noindent \textbf{References} }

\vskip 2mm

{\footnotesize

1. Aksel'~Ye.\:M., Vinogradova~N.\:N. Statistika zlokachestvennykh novoobrazovaniy zhenskikh re\-pro\-duk\-tiv\-nykh organov [Statistics of malignant neoplasms of female reproductive organs]. \textit{On\-ko\-gi\-ne\-ko\-lo\-giia} [\textit{On\-co\-gy\-ne\-co\-lo\-gy}], 2018, vol.~3(27), pp.~64--78.  (In Russian)
		
2. {Siegel R.\:L., Miller~K.\:D., Jemal~A.} Cancer statistics, 2020.  \textit{CA Cancer J. Clin}, 2020, vol.~70(1), pp.~7--30.  https://doi.org/10.3322/caac.21590
	
3. Kit~O.\:I., Frantsiyants~E.\:M., Moiseenko~T.\:I., Verenikina~E.\:V., Che\-rya\-ri\-na~N.\:D., Koz\-lo\-va~L.\:S., Po\-go\-re\-lo\-va~Yu.\:A. Faktory rosta tkaney razlichnykh stadiy raka yaichnikov [Growth factors in tussues of ovarian cancer at various]. \textit{Meditsinskii Vestnik Severnogo Kavkaza} [\textit{Medical Bulletin of the North Caucasus}], 2017,  vol.~12, no.~1,  pp.~48--52. https://doi.org/10.14300/mnnc.2017 (In Russian)
	
4. {Yin~A., Moes~D. J. A. R., van Hasselt~J. G. C., Swen~J. J., Guchelaar  H. J.} A review of mathematical models for tumor dynamics and treatment resistance evolution of solid tumors. \textit{CPT Pharmacometrics Syst. Pharmacol},  2019,  vol.~8, pp.~720--737. https://doi.org/10.1002/psp4.12450
	
5. {Das~P., Mukherjee~S., Das~Pr.} An investigation on Michaelis --- Menten kinetics based complex dynamics of tumor-immune interaction. \textit{Chaos, Solitons and Fractals},  2019, vol.~128, pp.~297--305. \\ https://doi.org/10.1016/j.chaos.2019.08.006
	
6. {Mahlbacher~G.\:E., Reihmer~K.\:C., Frieboes~H.\:B.} Mathematical modeling of tumor-immune cell interactions. \textit{J. of Theoretical Biology}, 2019, vol.~469, pp.~47--60. \\ https://doi.org/10.1016/j.jtbi.2019.03.002
	
7. {Liu~P., Liu~X.} Dynamics of a tumor-immune model considering targeted chemotherapy. \textit{Chaos, Solitons and Fractals}, 2017, vol.~98,  pp.~7--13.  https://doi.org/10.1016/j.chaos.2017.03.002
	
	
8. {Pang~L., Shen~L., Zhao~ Z.} Mathematical modelling and analysis of the tumor treatment regimens with pulsed immunotherapy and chemotherapy. \textit{Computational and Mathematical Methods in Medicine}, 2016, vol.~1, pp.~1--12.  https://doi.org/10.1155/2016/6260474
	
	
9. Busko~E.\:A., Goncharova~A.\:B., Rozhkova~N.\:I., Semiglazov~V.\:V., Shishova~A.\:S., Zhil\-tso\-va~E.\:K., Zi\-no\-viev~G.\:V., Be\-lo\-bo\-ro\-do\-va~K.\:A., Kri\-vo\-rot\-ko~P.\:V. Model' sistemy prinyatiya diagnosticheskikh resheniy na osnove mul'tiparametricheskikh ul'trazvukovykh pokazateley obrazovaniya molochnoy zhelezy [Model for making diagnostic decisions in multiparametric ultrasound of breast lesions]. \textit{Voprosy onkologii} [\textit{Problems in oncology}], 2020, vol.~66, no.~6, pp.~653--658. \\ https://doi.org/10.37469/0507-3758-2020-66-6-653-658 (In Russian)
	
10. {Chaplain~M.\:A.\:J., Sleeman~B.\:D.}  A mathematical model for the growth and classification of a solid tumor: a new approach via nonlinear elasticity theory using strain-energy functions. \textit{Mathematical biosciences}, 1992, vol.~111, no.~2, pp.~169--215. https://doi.org/10.1016/0025-5564(92)90070-D

11. {Hathout~L., Ellingson~B.\:M., Cloughesy~T.\:F., Pope~W.\:P.} Patient-specific characterization of the invasiveness and proliferation of low-grade gliomas using serial MR imaging and a mathematical model of tumor growth. \textit{Oncology Reports}, 2015, vol.~33,  no.~6,  pp.~2883--2888.  https://doi.org/10.3892/or.2015.3926


12. {Karaman~M.\:M., Sui~Y., Wang~H., Magin~R.\:L.,  Li~Y.,  Zhou~X.\:J.} Differentiating low and high grade pediatric brain tumors using a continuous time randomwalk diffusion model at high b-values. \textit{Magnetic resonance in medicine},  2016,  vol.~76, no.~4, pp.~1149--1157.  https://doi.org/10.1002/mrm.26012
	
13. {Ozhiganova~I.\:N.} Morfologiya raka yaichnikov v klassifikatsii VOZ 2013 goda [Morphology of ovarian cancer in the WHO classification of 2013]. \textit{Prakticheskaia onkologiia} [\textit{Practical Oncology}], 2014, vol.~15, no.~4, pp.~143--152. (In Russian)
	
	
	
14. {Zhordania~K.\:I., Kalinicheva~E.\:V., Moiseev~A.\:A. } Rak yaichnikov: epidemiologiya, mor\-fo\-lo\-giya i gistogenez [Ovarian cancer: epidemiology, morphology and histogenesis]. \textit{Onkoginekologiia} [\textit{On\-co\-gy\-ne\-cology}], 2017, vol.~3(23), pp.~26--32. (In Russian)
	
	
	
15. {Zhordania~K.\:I., Khokhlova~S.\:V.}  Ranniy rak yaichnikov: nash vzglyad na problemu [Early ovarian cancer: our view of the problem]. \textit{Opukholi zhenskoi reproduktivnoi sistemy} [\textit{Tumors of the female reproductive system}], 2011, vol.~3, pp.~56--64. (In Russian)
	
	
	
16. {Kozlova~L.\:S., Pogorelova~Yu.\:A.} Faktory rosta v tkani razlichnykh stadiy raka yaichnikov [Growth factors in tissue of different stages of ovarian cancer]. \textit{Meditsinskii Vestnik Severnogo Kavkaza}  [\textit{Medical Bulletin of the North Caucasus}], 2017,  vol.~12, no.~1,  pp.~42--44. (In Russian)

	
17. {Zheng~X., Sweidan~M.} A mathematical model of angiogenesis and tumor growth: analysis and application in anti-angiogenesis therapy.  \textit{J. of Mathematical Biology}, 2018, vol.~77, pp.~1589--1622. https://doi.org/10.1007/s00285-018-1264-4
	
18. {Khokhlova~S.\:V.} Rol' ingibitorov sosudistogo endotelial'nogo faktora rosta v lechenii raka yaichnikov [The role of vascular endothelial growth factor inhibitors in the treatment of ovarian cancer]. \textit{Opukholi zhenskoi reproduktivnoi sistemy} [\textit{Tumors of the female reproductive system}], 2010,  vol.~3,  pp.~35--44. (In Russian)

	
19. {Goncharova~A.\:B., Kolpak~E.\:P., Rasulova~M.\:M., Shmeleva~A.\:A} Matematicheskoye modelirovaniye onkologicheskogo zabolevaniya [Mathematical modeling of oncological disease]. \textit{Perspektivy nauki}  [\textit{Prospects of science}],  2020,  vol.~12(135), pp.~20--26. (In Russian)
	
	
20. {Goncharova~A.\:B., Kolpak~E.\:P., Rasulova~M.\:M., Abramova~A.\:V.} Matematicheskoye modelirovaniye lecheniya onkologicheskogo zabolevaniya [Mathematical modeling of cancer treatment]. \textit{Vestnik of Saint Petersburg University. Applied Mathematics. Computer Science. Control Processes}, 2020, vol.~16, iss.~4, pp.~437--446. https://doi.org/10.21638/11701/spbu10.2020.408 (In Russian)

	
21. {Domschke~P., Trucu~D., Gerisch~A., Chaplain~M.\:A.\:J.} Mathematical modeling of cancer invasion: Implications of cell adhesion variability for tumour infiltrative growth patterns. \textit{J. of Theoretical Biology}, 2014, vol.~361, pp.~41--60. https://doi.org/10.1016/j.jtbi.2014.07.01

22. {Malinzi~J., Sibanda~P., Mambili-Mamboundou~H.} Response of immunotherapy to yumour-TICLs interactions: a travelling wave analysis. \textit{Abstract and Applied Analysis},  2014, pp.~1--10. https://doi.org/10.1155/2014/137015
	
23. {Kuznetsov~M.\:B., Kolobov~A.\:V.} Vliyaniye khimiotepapii na progressiyu biklonal'noy opukho\-li~--- analiz metodom matematicheckogo modelipovaniya [Influence of chemotherapy on the progression of biclonal tumor --- analysis by the method of mathematical modeling]. \textit{Biofizika}   [\textit{Biophysics}],  2019,  vol.~64, iss.~2, pp.~371--387. https://doi.org/10.1134/S0006302919020170 (In Russian)

	
24. {Kuznetsov~M.\:B., Gubernov~V.\:V., Kolobov~A.\:V.} Vliyaniye dinamiki interstitsial'noy zhidkosti na rost i terapiyu angiogennoy opukholi. Analiz c pomoshch'yu matematicheskoy modeli [Influence of the dynamics of the interstitial fluid on the growth and therapy of angiogenic tumor. Analysis using a mathematical model]. \textit{Biofizika}   [\textit{Biophysics}],  2017, vol.~62, iss.~1, pp.~151--160. (In Russian)

	
25. Gorodnova~T.\:V., Baranov~S.\:B., Tyatkov~S.\:A., Shevkunov~L.\:N., Sokolenko~A.\:P., Ko\-tiv~Kh.\:B., Imya\-ni\-tov~E.\:N., Ber\-lev~I.\:V. Opyt ispol'zovaniya luchevoy terapii pri brca-pozitivnom rake yaichnikov [Experience of using radiation therapy for brca-positive ovarian cancer]. \textit{Sibirskii onkologicheskii zhurnal} [\textit{Siberian Journal of Oncology}],  2017, vol.~16(4), pp.~103--107. (In Russian)
	
26.	\text{Chu~E., De Vita~VT~Jr.} A history of cancer chemotherapy. \textit{Cancer Res.},  2008, vol.~68(21), pp.~8643--8653. https://doi.org/10.1158/0008-5472


\vskip1.5mm Received:  October 03, 2021.

Accepted: February 01, 2022.


\vskip6mm A\,u\,t\,h\,o\,r\,s'\,  i\,n\,f\,o\,r\,m\,a\,t\,i\,o\,n:%


\vskip1.5mm \textit{Anastasiya B. Goncharova} --- PhD in Physics and Mathematics, Associate Professor;\\ a.goncharova@spbu.ru \par%

\vskip1.5mm \textit{Eugeny P. Kolpak} --- Dr. Sci. in Physics and Mathematics, Professor; e.kolpak@spbu.ru \par%

\vskip1.5mm \textit{Maria Yu. Vil'} --- Magistrant; st054723@student.spbu.ru \par%

\vskip1.5mm \textit{Alina V. Abramova} --- Oncologist; alinochkamv1991@gmail.com \par%

\vskip1.5mm \textit{Ekaterina A. Busko} --- Dr. Sci. in Medicine, Associate Professor, Oncologist; e.busko@spbu.ru \par%
%
}
