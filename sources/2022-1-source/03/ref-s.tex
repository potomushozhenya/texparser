
{\normalsize

\vskip 4.5%6%
mm

\noindent{\bf Extraction of common properties of objects %\\
for creation of a logic ontology$^{*}$
}

}

\vskip 2.2%3%
mm

{\small

\noindent{\it T. M. Kosovskaya, N. N. Kosovskii%$\,^2$%
%, I.~�. Famylia%$\,^2$%

 }

\vskip 2.2%3%
mm

%%%%%%%%%%%%%%%%%%%%%%%%%%%%%%%%%%%%%%%%%%%%%%%%%%%%%%%%%%%%%%%%%%

\efootnote{
%%
\vspace{-3mm}\parindent=7mm
%%
\vskip 0.1mm \indent~$^{*}$ This work was carried out with the financial support of St Petersburg State University (project N 73555239).\par
%%
%%\vskip 2.0mm
%%
%%\indent{\copyright} �����-������������� ���������������
%%�����������, \issueyear%
%%
}

%%%%%%%%%%%%%%%%%%%%%%%%%%%%%%%%%%%%%%%%%%%%%%%%%%%%%%%%%%%%%%%%%%

{\footnotesize



\noindent%
%$^2$~%
St\,Petersburg State University, 7--9, Universitetskaya nab., St\,Petersburg,

\noindent%
%\hskip2.45mm%
199034, Russian Federation

}

%%%%%%%%%%%%%%%%%%%%%%%%%%%%%%%%%%%%%%%%%%%%%%%%%%%%%%%%%%%%%%%%%%

\vskip 2,5%3%
mm


\noindent \textbf{For citation:}  Kosovskaya T. M., Kosovskii N. N.  Extraction of common properties of objects for creation of a logic ontology. {\it Vest\-nik of Saint Pe\-ters\-burg
University. Applied Mathe\-ma\-tics. Computer
Science. Cont\-rol Pro\-cesses}, %\,
\issueyear,
vol.~18, iss.~\issuenum,~pp.~\pageref{p3}--\pageref{p3e}. \\
\doivyp/\enskip%
\!\!\!spbu10.\issueyear.\issuenum03 (In Russian)

\vskip3mm

\begin{hyphenrules}{english}

%\pagebreak

{\leftskip=7mm\noindent The paper describes an approach to the formation of ontology based on descriptions of objects in terms of the predicate calculus language. With this approach, an object is represented as a set of its elements, on which a set of predicates is defined that defines the properties of these elements and the relationship between them. A description of an object is a conjunction of literals that are true on elements of an object. In the present work, ontology is understood as an oriented graph with descriptions of subsets as nodes and such that the elements of a set at the end of an oriented edge have the properties of the elements of the set at the beginning of this edge. Three settings of an ontology construction problem are considered: $1)$~all predicates are binary and subsets of the original set of objects are given; $2)$~all predicates are binary and it is required to find subsets of the original set; $3)$~among the predicates there are many-valued ones and subsets of the original set of objects are given. The main tool for construction such a description is to extract an elementary conjunction of literals of predicate formulas that is isomorphic to subformulas of some formulas. The definition of an isomorphism of elementary conjunctions of atomic predicate formulas is given. The method of ontology construction is formulated. An illustrative example is provided.
\\[1mm]
\textit{Keywords}: logic ontology, predicate formula, isomorphism of predicate formulas.\par}

\vskip6mm
%\pagebreak

\end{hyphenrules}

\noindent \textbf{References} }

\vskip 3mm

{\footnotesize

1. Noy N. F.,  McGuinness D. L.  {\it Ontology development 101: A guide to creating your first ontology.} Stanford Knowledge Systems Laboratory Technical Report KSL-01-05 and Stanford Medical Informatics Technical Report SMI-2001-0880. Stanford, March 2001.

2. Beniaminov E.~M. Nekotorye problemy shirokogo vnedreniia ontologii v IT i napravleniia ikh reshenii [Some problems of the widespread introduction of ontologies in IT and the directions of their solutions].   {\it Simpozium ``Ontologicheskoe modelirovanie��}. Sbornik nauchnykh trudov [{\it Symposium ``Ontological modeling''}. A collection of scientific papers]. Moscow, Institute IPI RAN Publ.,  2008, pp.~71--82. (In Russian)

3. Kogalovskii M.~R., Parinov S.~I. Semanticheskoe strukturirovanie kontenta nauchnykh elektronnykh bibliotek na osnove ontologii [Semantic structuring of the content of scientific electronic libraries based on ontologies]. {\it Sovremennye tekhnologii integratsii informatsionnykh resursov.} Sbornik nauchnykh trudov [{\it Modern technologies for the integration of information resources.} A collection of scientific papers]. St~Pe\-tersburg, President Library Publ., 2011, iss.~2, pp.~1--13. (In Russian)

4. Diachenko O.~O., Zagorulko Yu. A. Podkhod k kollektivnoi razrabotke ontologii i baz znanii [An approach to the collective development of ontologies and knowledge bases]. {\it Znaniia\,---\,Ontologii\,---\,Teorii. Vserossiiskaia konferentsiia s mezhdunarodnym uchastiem} [{\it Knowledge\,---\,Ontologies\,---\,Theories. All-Russian Conference with international participation}]. Eds. by D.~E.~Palchunov. Novosibirsk, S.~L.~Sobolev Institute of Mathematics SO~RAN Publ., 2013, pp.~141--149. (In Russian)

5.  Mykhailiuk A., Petrenko M. Machine learning and ontologies as two approaches for building intellectual information systems. {\it  Intern. J. ``Information Technologies $\&$ Knowledge��}, 2019, vol.~13, no.~1, pp.~55--75.

%%%%%%%%%%%%%%%%%%%%%%
6.  Karpov A.~G., Klemeshev V.~A, Kuranov D. Yu. Opredelenie pabotosposobnosti sistemy, struktura kotoroy zadana grafom  [Determining the ability to work of the system, the structure of which is given using graph]. {\it Vestnik of Saint Petersburg University. Applied Mathematics. Computer Science. Control Processes},  2020, vol.~16, iss.~1, pp.~41--49.  https://doi.org//10.11702/spbu10.2020.104 (In Russian)


7. Goncharova A. B. Postanovka predvaritelnogo meditsinskogo diagnoza na osnove teorii nechetkih mnozhestv s ispolzovaniem mery Sugeno [Preliminary medical diagnostics based on the fuzzy sets theory using the Sugeno measure]. {\it Vestnik of Saint Petersburg University. Applied Mathematics. Computer Science. Control Processes},  2019, vol.~15, iss.~4, pp.~529--543. \\ https://doi.org//10.21638/11702/spbu10.2019.409 (In Russian)


%%%%%%%%%%%%%%%%%%%%%%%%%%%%%

8. Kosovskaya T. Predicate calculus as a tool for AI problems solution: Algorithms and their complexity.
{\it Intelligent System. Pt 3. Open access peer-reviewed}. Ed. vol.  Chatchawal, Chatchawal Wongchoosuk Kasetsart University Press, 2018, pp.~1--20.

9.  Kosovskaya T. M. Podhod k resheniyu zadachi postroeniya mnogourovnevogo opisaniya klassov na yazyke ischisleniya predikatov [An approach to the construction of a level description of classes by means of a predicate calculus language]. {\it SPIIRAS Proceedings}, 2014, no.~3(34), pp.~204--217. (In Russian)

10. Kosovskaya T. M., Kosovskii N. N. Polinomialnaya ekvivalentnost zadach izomorfizm predikatnyh formul i  izomorfizm grafov [Polynomial equivalence of the problems predicate formulas isomorphism and graph isomorphism]. {\it Vestnik of Saint Petersburg University. Mathematics. Mechanics. Astronomy}, 2019, vol.~6(64), iss.~3, pp.~430--439.  https://doi.org//10.21638/11701/spbu10.2019.308 (In Russian)


11. Babai L. {\it Graph isomorphism in quasipolynomial time (Version 2.1. Unfinished Revision of Version 2 Posted on arXiv May 23, 2017)}. Available at:\\
http://people.cs.uchicago.edu/~laci/17groups/version2.1.pdf (accessed: March 21, 2019).

12. Kosovskaya T. M., Petrov D. A. Vydelenie naibolshey obschty podformuly predikatnyh formul dlya resheniya ryada zadach iskusstvennogo intellekta [Extraction of a maximal common sub-formula of predicate formulas for the solving of some artificial intelligence problems]. {\it Vestnik of Saint Petersburg University. Applied Mathematics. Computer Science. Control Processes}, 2017, vol.~13, iss.~3, pp.~250--263. https://doi.org//10.21638/11702/spbu10.2017.303 (In Russian)


\vskip1.5mm Received:  August 21, 2021.

Accepted: February 01, 2022.

\vskip6mm A\,u\,t\,h\,o\,r\,s' \, i\,n\,f\,o\,r\,m\,a\,t\,i\,o\,n:%

\vskip2mm \textit{Tatiana M. Kosovskaya} --- Dr. Sci. in Physics and Mathematics, Professor;  kosovtm@gmail.com \par

\vskip2mm \textit{Nikolai N. Kosovskii} --- PhD in Physics and Mathematics, Associate Professor;  kosovnn@gmail.com \par
%
%
}
