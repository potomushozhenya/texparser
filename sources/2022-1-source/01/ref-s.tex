
{\normalsize

\vskip 6mm

\noindent{\bf Approximation of supremum and infimum processes as a stochastic approach\\ to the providing of homeostasis$^{*}$%
}

}

\vskip 2mm

{\small

\noindent{\it G. I. Beliavsky, N. V. Danilova, G. A. Ougolnitsky%

}

\vskip 2mm

%%%%%%%%%%%%%%%%%%%%%%%%%%%%%%%%%%%%%%%%%%%%%%%%%%%%%%%%%%%%%%%%%%

\efootnote{
%%
\vspace{-3mm}\parindent=7mm
%%
\vskip 0.1mm $^{*}$ This work was supported by Russian Science Foundation (project N 17-19-01038).%\par
%%
%%\vskip 2.0mm
%%
%%\indent{\copyright} �����-������������� ���������������
%%�����������, \issueyear%
%%
}

%%%%%%%%%%%%%%%%%%%%%%%%%%%%%%%%%%%%%%%%%%%%%%%%%%%%%%%%%%%%%%%%%%

{\footnotesize

\noindent%
%$^1$~%
Southern Federal University, 105/42, Bolshaya Sadovaya ul., Rostov-on-Don,

\noindent%
%\hskip2.45mm%
344006, Russian Federation


%\noindent%
%$^3$~%
%St.\,Petersburg State University, 7--9, Universitetskaya nab.,
%St.\,Petersburg,
%
%\noindent%
%\hskip2.45mm%
%199034, Russian Federation


}

%%%%%%%%%%%%%%%%%%%%%%%%%%%%%%%%%%%%%%%%%%%%%%%%%%%%%%%%%%%%%%%%%%

\vskip2mm%3mm


\noindent \textbf{For citation:} Beliavsky G. I., Danilova N. V., Ougolnitsky G. A. Approximation of supremum and infimum processes as a stochastic approach to the providing of homeostasis. {\it Vestnik of Saint~Petersburg Uni\-ver\-si\-ty.
Ap\-plied Mat\-he\-matics. Computer Science. Control
Processes},\,\issueyear,
vol.~18,~iss.~\issuenum,~pp.~\pageref{p1}--\pageref{p1e}.  %\\
\doivyp/\enskip%
\!\!\!spbu10.\issueyear.\issuenum01 (In Russian)

\begin{hyphenrules}{english}

\vskip2mm

{\leftskip=7mm\noindent We consider the calculation of bounded functional of the trajectories of a stationary diffusion process. Since an analytical solution to this problem does not exist, it is necessary to use numerical methods. One possible direction for obtaining
the numerical method is applying the Monte Carlo (MC)
method. This involves reproducing the trajectory of a
random process with subsequent averaging over the trajectories. To simplify the reproduction of the trajectory, the Girsanov transform is used in this paper. The main goal is to approximate the   supremum and infimum processes, which allows us to more accurately compute the mathematical expectation of a function depending on the values of the supremum and infimum processes at the end of the time interval compared to the classical method. The method is based on randomly dividing the interval of the time axis by stopping times passages of the Wiener process, approximating the density to replace the measure, and using the MC method to calculate the expectation. One of the applications of the method is the task of keeping a random process in a given area --- the problem of homeostasis.\\[1mm]
\textit{Keywords}: diffusion, Monte-Carlo method, Girsanov transform, homeostasis. \par}

\vskip6mm%5

\end{hyphenrules}



\noindent \textbf{References} }

\vskip 2mm

{\footnotesize

1. Fishman G. {\it Monte-Carlo: concepts, algorithms and applications}. New York, Springer Publ., 1995, 722~p.

2. Kudryavtsev O. Approximate Wiener\,---\,Hopf factorization and Monte-Carlo methods for Levy processes. {\it Probability Theory and its Applications}, 2019, vol.~64(2), pp.~186--208.

3. Girsanov I. On the transformation of a class of random processes using a completely continuous measure replacement. {\it Probability Theory and its Applications}, 1960, vol.\,5, pp.\,314--330.

4. Novikov~�. Ob odnom tozhdestve dlia stokhasticheskikh integralov [On one identity for stochastic integrals]. \emph{Teoriia veroiatnostei i ee primeneniia} [\emph{Probability Theory and its Applications}], 1972, no.~4, pp.~761--765. (In Russian)

5. Kloeden P., Platen E. \emph{Numerical solution of stochastic differential equations}. New York, Springer Publ., 1995, 632~p.

6. Carr P. Randomization and American put. \emph{Rev. Financ. Stud.}, 1996, no.~11, pp.~597--626.

7. Kuznetsov~A., Kyprianou~A.\,E., Pardo~J.\,C., van Schaik~K. A Wiener\,---\,Hopf Monte-Carlo simulation technique for L\'{e}vy processes. \emph{Ann. Appl. Prob.}, 2011, no.~21, pp.~2171--2190.

8. Ferreiro-Castilla~A., Kyprianou~A.\,E., Scheichl~R., Suryanarayana~G.   Multilevel Monte-Carlo simulation for L\'{e}vy processes based on the Wiener\,---\,Hopf factorization. \emph{Stoch. Process. Appl.}, 2014. no.~124, pp.~985--1010.

9. Beliavsky~G., Danilova~N., Ougolnitsky~G. Calculation of probability of the exit of a stochastic process from a band by Monte-Carlo method: A Wiener\,---\,Hopf factorization. \emph{Mathematics}, 2019, no.~7, pp.~581--597.

10. Beliavsky~G.\,I., Danilova~N.\,V. The combined Monte-Carlo method to calculate the capital of the optimal portfolio in nonlinear models of financial indexes. \emph{Siberian Electronic Mathematical Reports}, 2014, no.~11, pp.~1021--1034.

11. Aubin J.-P. \emph{Viability theory}. New York, USA, Springer-Verlag Publ., 1991, 342~p.

12. Ougolnitsky G.  \emph{Sustainable management}. New York, USA, Nova Science Publ., Hauppauge, 2011, 287~p.

13. Shiryaev~�.\,N. O martingal'nykh metodakh v~zadachakh o~peresechenii granits brounovskim dvizheniem [On martingale methods in problems of boundary crossing by Brownian motion]. \emph{Sovremennye problemy matematiki} [\emph{Modern mathematics problems}], 2007, iss.~8, pp.~3--78. (In Russian)

14. Casella~G., Robert~C.~P., Wells~M.~T.  Generalized accept-reject sampling chemes. \emph{Institute of Mathematical Statistics}, 2004, pp.~342--347.

15. Kamchatkin~�. �., Stepenko~N. �., Chitrov~G. �. K teorii konstruktivnogo postroeniia lineinogo reguliatora  [On the theory of constructive construction of a linear regulator]. \emph{Vestnik of Saint Petersburg University. Applied Mathematics. Computer Sciences. Control Processes}, 2020, vol.~16, iss.~3, pp.~326--344. https://doi.org/10.21638/11701/spbu10.2020.309  (In Russian)

\vskip1.5mm

Received:  June 03, 2021.

Accepted: February 01, 2022.


\vskip6mm A~u~t~h~o~r~s' \ i~n~f~o~r~m~a~t~i~o~n:%

\vskip2mm \textit{Grigory I. Beliavsky}~--- Dr. Sci. in Technics, Professor; gbelyavski@sfedu.ru\par%
%
\vskip2mm \textit{Natalia V. Danilova}~--- PhD in Physics and Mathematics, Associate Professor; nvdanilova@sfedu.ru\par%
%
\vskip2mm \textit{Gennady A. Ougolnitsky}~--- Dr. Sci. in Physics and Mathematics, Professor; gaugolnickiy@sfedu.ru\par%
%
%
}
