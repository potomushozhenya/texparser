
{\normalsize

\vskip 5%6
mm

\noindent{\bf On the effective elastic properties of a material with mutually perpendicular \\ systems of parallel cracks$^{*}$%
}

}

\vskip 2mm

{\small

\noindent{\it A.\,M. Abakarov, Yu.\,G. Pronina%
% , I.~J.~Favilia$^1$%

}

\vskip 2mm

%%%%%%%%%%%%%%%%%%%%%%%%%%%%%%%%%%%%%%%%%%%%%%%%%%%%%%%%%%%%%%%%%%

\efootnote{
%%
\vspace{-3mm}\parindent=7mm
%%
\vskip 0.1mm $^{*}$ This work was supported by the Russian Science Foundation (grant N 21-19-00100).%\par
%%
%%\vskip 2.0mm
%%
%%\indent{\copyright} �����-������������� ���������������
%%�����������, \issueyear%
%%
}

%%%%%%%%%%%%%%%%%%%%%%%%%%%%%%%%%%%%%%%%%%%%%%%%%%%%%%%%%%%%%%%%%%

{\footnotesize


\noindent%
%$^1$~%
St\,Petersburg State University, 7--9, Universitetskaya nab.,
St\,Petersburg,

\noindent%
%\hskip2.45mm%
199034, Russian Federation


}

%%%%%%%%%%%%%%%%%%%%%%%%%%%%%%%%%%%%%%%%%%%%%%%%%%%%%%%%%%%%%%%%%%

\vskip 3mm


\noindent \textbf{For citation:} Abakarov A. M., Pronina Yu. G. On the effective elastic properties of a material with mutually perpendicular systems of parallel cracks. {\it Vestnik of Saint~Petersburg Uni\-ver\-si\-ty.
Ap\-plied Mat\-he\-matics. Computer Science. Control
Processes},\,\issueyear,
vol.~18,~iss.~\issuenum,~pp.~\pageref{p9}--\pageref{p9e}.  \\ %\\
\doivyp/\enskip%
\!\!\!spbu10.\issueyear.\issuenum09 (In Russian)

\begin{hyphenrules}{english}

\vskip2mm

{\leftskip=7mm\noindent The effective properties of cracked solids are often expressed in terms of the crack density parameter or its tensor generalization, using the approximation of noninteracting cracks. This approximation remains accurate at sufficiently high crack densities, provided the location of cracks are random. The presented analysis confirms that the effective elastic moduli of a material with ordered fracture structures strongly depend on the linear dimensions of cracks and their mutual arrangement even at a constant crack density. A change in these parameters can cause a noticeable anisotropy of the effective properties of the material even when the eigenvalues of the crack density tensor are equal to each other. The effective elastic characteristics of a material with one doubly periodic system of parallel cracks are compared with those for a material with two mutually perpendicular systems of such cracks in a two-dimensional formulation. The calculations are carried out using the approximate method of M.~Kachanov for determining the mean stresses at the cracks edges, applicable for large systems of interacting cracks. Analysis of the obtained results showed that the effective compliance of the material in a certain direction is largely determined by the effects of interaction (shielding and amplification) within a system of parallel cracks perpendicular to this direction. The interaction of this system of cracks with the perpendicular system has a weak effect on the indicated properties in the case of rectangular symmetry of the system. In this case, the interaction of mutually perpendicular systems of cracks leads to a violation of the symmetry of the tensor of effective elastic constants.\\[1mm]
\textit{Keywords}: crack density, crack interaction, effective elastic properties. \par}

\vskip6mm%5

\end{hyphenrules}



\noindent \textbf{References} }

\vskip 2mm

{\footnotesize

1. {Kachanov~M.} On the effective moduli of solids with cavities and cracks. \textit{Intern. J. of Fracture}, 1993, vol.~{59}(1), pp.~R17--R21.  https://doi.org/10.1007/BF00032223

2. {Doan~T., Le-Quang~H., To~Q.-D.} Effective elastic stiffness of 2D materials containing nano\-voids of arbitrary shape. \textit{Intern. J. of Engineering Science}, 2020, vol.~150, no.~103234. \\ https://doi.org/10.1016/j.ijengsci.2020.103234

3. {Du~K., Cheng~L., Barthelemy~J.~F., Sevostianov~I., Giraud~A., Adessina~A.} Numerical computation of compliance contribution tensor of a~concave pore embedded in a transversely isotropic matrix. \textit{Intern. J. of Engineering Science}, 2020, vol.~{152}, no.~103306.  https://doi.org/10.1016/j.ijengsci.2020.103306

4. {Markov~A., Trofimov~A., Sevostianov~I.} A unified methodology for calculation of compliance and stiffness contribution tensors of inhomogeneities of arbitrary 2D and 3D shapes embedded in isotropic matrix --- open access software. \textit{Intern. J. of Engineering Science}, 2020, vol.~157, no.~103390. \\ https://doi.org/10.1016/j.ijengsci.2020.103390

5. {Sevostianov~I., Kushch~V.~I.} Compliance contribution tensor of an arbitrarily oriented ellipsoidal inhomogeneity embedded in an orthotropic elastic material. \textit{Intern. J. of Engineering Science}, 2020, vol.~149, no.~103222.   https://doi.org/10.1016/j.ijengsci.2020.103222

6. {Kachanov~M.} Elastic solids with many cracks and related problems. \textit{Advances in Applied Mechanics}, 1993, vol.~30(C), pp.~259--445. https://doi.org/10.1016/S0065-2156(08)70176-5

7. {Kachanov~M., Mishakin~V.~V.} On crack density, crack porosity, and the possibility to interrelate them. \textit{Intern. J. of Engineering Science}, 2019, vol.~142, pp.~185--189. \\ https://doi.org/10.1016/j.ijengsci.2019.06.010

8. {Bristow~J.~R.} Microcracks and the static and dynamic elastic constants of annealed and heavily cold-worked metals. \textit{British J. Appl. Phys.}, 1960, vol.~11, pp.~81--85.

9. {Vakulenko~A., Kachanov~M.} Continuum theory of medium with cracks. \textit{Mech. of Solids}, 1971, vol.~6(4), pp.~145--151.

10. {Kachanov~M.} On the problems of crack interactions and crack coalescence. \textit{Intern. J. of Frac\-tu\-re}, 2003, vol.~120(3), pp.~537--543.

11. {Sevostianov~I., Kachanov~M.} On approximate symmetries of the elastic properties and elliptic orthotropy. \textit{Intern. J. of Engineering Science}, 2008, vol.~46(3), pp.~211--223.

12. {Abakarov~A., Pronina~Y., Kachanov~M.} Symmetric arrangements of cracks with perturbed symmetry: extremal properties of perturbed configurations. \textit{Intern. J. of Engineering Science}, 2021, vol.~171(4), no.~103617. http://doi.org/10.1016/j.ijengsci.2021.103617

13. {Grekov~M. A., Sergeeva~T.~S.} Interaction of edge dislocation array with bimaterial interface incorporating interface elasticity. \textit{Intern. J. of Engineering Science}, 2020, vol.~149, no.~103233. \\ https://doi.org/10.1016/j.ijengsci.2020.103233

14. {Pronina~Yu., Maksimov~A., Kachanov~M.} Crack approaching a domain having the same elastic properties but different fracture toughness: Crack deflection vs penetration. \textit{Intern. J. of Engineering Science}, 2020, vol.~156, no.~103374. https://doi.org/10.1016/j.ijengsci.2020.103374

15. {Shuvalov~G.~M., Vakaeva~A.~B., Shamsutdinov~D.~A., Kostyrko~S.~A.} The effect of nonlinear terms in boundary perturbation method on stress concentration near the nanopatterned bimaterial interface. \textit{Vestnik of Saint Petersburg University. Applied Mathematics. Computer Science. Control Processes}, 2020, vol.~16, iss.~2, pp.~165--176. https://doi.org/10.21638/11701/spbu10.2020.208

\vskip1.5mm

Received:  June 01, 2021.

Accepted: February 01, 2022.


\vskip4.5%6
mm A~u~t~h~o~r~s' \ i~n~f~o~r~m~a~t~i~o~n:%

\vskip2mm \textit{Abdulla M. Abakarov}~--- Student; st046811@student.spbu.ru \par%
%
\vskip2mm \textit{Yulia G. Pronina}~---  Dr. Sci. in Physics and Mathematics, Professor; y.pronina@spbu.ru \par%
%
}
