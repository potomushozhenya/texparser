
{\normalsize

\vskip 4mm

\noindent{\bf Selection of homogeneous zones of agricultural field
for laying\\ of experiments using unmanned aerial vehicle}

}

\vskip 2mm

{\small




\noindent{\it V.~M.~Bure$^{1, 2}$, E.~P.~Mitrofanov$^1$,
O.~A.~Mitrofanova$^{1, 2}$, A.~F.~Petrushin$^1$}

\vskip 2mm




{\footnotesize \noindent $^1$~Agrophysical Research Institute, 14,
Grazhdanskiy pr., St.\,Petersburg, %
%
%\noindent\hskip2.45mm
195220, Russian Federation

\noindent $^2$~St.\,Petersburg State University, 7--9,
Universitetskaya nab., St.\,Petersburg, %
%
%\noindent\hskip2.45mm
199034, Russian Federation




}

\vskip3mm

\noindent \textbf{For citation:} Bure V.~M.,~Mitrofanov
E.~P.,~Mitrofanova O.~A., Petrushin A. F. Selection of homogeneous
zones of agricultural field for laying of experiments using
unmanned aerial vehicle. {\it Vestnik of Saint~Petersburg
University. Applied Mathematics. Computer Science. Control
Pro-\linebreak cesses}, \issueyear, vol.~14, iss.~\issuenum,
pp.~\pageref{p6}--\pageref{p6e}.
\doivyp/spbu10.\issueyear.\issuenum06

\vskip2mm

{\leftskip=7mm\noindent An important stage in research aimed at
solving the problems of accurate farming is the laying of field
experiments. A necessary condition for carrying out such
experiments is to ensure homogeneity of the selected land plot.
Most of the existing techniques for isolating homogeneous zones
for conducting experiments are based on costly and labor-intensive
sampling and analysis of soil samples. An alternative and
promising approach is the use of an unmanned aerial vehicle. In
work, all stages of choosing a homogeneous land plot with the help
of aerial photography are presented in sufficient detail. The
object of the study was a field with a long-term sowing of
``goat'' on the basis of the Menkovsky branch of the Agrophysical
Institute (Leningrad region). Aerial photography was carried out
in 2015--2017 with the help of an unmanned aerial vehicle
``Geoscan 401''. The received data were processed with the help of
specialized software: crosslinking and alignment were carried out
in the Agisoft PhotoScan program; the thematic processing and
allocation of homogeneous areas of the field were carried out in
the programs QGis and Saga Gis. To assess the state of vegetation,
the vegetative index NDVI (Normalized Difference Vegetation Index)
was applied. To cluster the homogeneous parts of the field in
terms of NDVI parameters the ISODATA algorithm (Iterative
Self-Organizing Data Analysis Technique Algorithm) was applied.
The paper presents the results of clustering images of the same
field in different time periods. In the course of the work,
intersections of these aerial photographs were constructed, four
clusters were identified, which are the intersection of the
corresponding homogeneous zones for the considered time periods.
Accordingly, the laying of experiments is expedient to be carried
out on these sites,
since the homogeneity present there seems more stable in dynamics.\\[1mm]
\textit{Keywords}: aerial photography, precision agriculture,
clustering, ISODATA algorithm.
\par}

\vskip4mm

\noindent \textbf{References} }

\vskip 2mm

{\footnotesize

1. {Dospekhov~B.~A.} {\it Metodika polevogo opyta $[$Method of
field experiment$]$}. Moscow, Kolos Publ., 1979, 416~p. (In
Russian)

2. {Bure~V.~M.,  Mitrofanova~O.~A.} Prognoz prostranstvennogo
raspredeleniia ekologicheskikh dannykh s primeneniem kriginga i
binarnoi regressii [Prediction of the spatial distribution of
ecological data using kriging and binary regression]. {\it Vestnik
of Saint Petersburg University. Series~10. Applied Mathematics.
Computer Science. Control Processes}, 2016, iss.~3, pp.~97--105.
(In Russian)

3. {Petrushin~A.~F., Mitrofanov~E.~P., Mitrofanova~�.~�.}
Tsifrovaia model' rel'efa mestnosti dlia monitoringa
meliorativnykh sooruzhenii [Digital terrain model for monitoring
reclamation facilities]. {\it Agroecosystems in natural and
regulated conditions: from the theoretical model to the practice
of precision control}. Saint\,Petersburg, Agrophysical Institute
Publ., 2016, pp.~47--451. (In Russian)

4. {Bure~V.~M., Mitrofanova~O.~A.} Analysis of aerial photographs
to predict the spatial distribution of ecological data. {\it
Contemporary Engineering Sciences,} 2017, vol.~10, no.~4,
pp.~157--163.

5. {Iakushev~V.~P., Kanash~E.~V., Konev~�.~�., Kovtiukh~S.~N.,
Lekomtsev~P.~V., Matveenko~D.~A., Petrushin~A.~F., Iakushev~V.~V.,
Bure~V.~M., Rusakov~D.~V., Osipov~Iu.~A.} {\it Teoreticheskie i
metodicheskie osnovy vydeleniia odnorodnykh tekhnologicheskikh zon
dlia differentsirovannogo primeneniia sredstv khimizatsii po
opticheskim kharakteristikam poseva}: prakt. posobie {\it
$[$Theoretical and methodological foundations for the separation
of homogeneous technological zones for the differentiated
application of chemicalization means based on the optical
characteristics of seeding}: practical. allowance$]$. Saint
Petersburg, Agrophysical Institute Publ., 2010, 60~p. (In Russian)

6. {Tou~J.~T, Gonzalez~R.~C.} {\it Pattern recognition
principles}. Boston, MA, USA, Addison-Wesley Publ. Company, 1974,
395~p.




\vskip5mm A\,u\,t\,h\,o\,r's \ I\,n\,f\,o\,r\,m\,a\,t\,i\,o\,n:
\vskip1.5mm \textit{Bure Vladimir M.}  --- Dr. Sci. in technics,
professor; vlb310154@gmail.com

\textit{Mitrofanov Evgenii P.}  --- postgraduate student;
mjeka@bk.ru

\textit{Mitrofanova Olga A.}  --- postgraduate student;
omitrofa@gmail.com

\textit{Petrushin Aleksei F.}  --- Dr. Sci. in technics;
apetrushin@agrophys.com






}
