
{\normalsize

\vskip 6mm

\noindent{\bf Quadratic and cubic Volterra polynomials:
identification and application}

}

\vskip 1.5mm

{\small

\noindent{\it S.~V.~Solodusha}

\vskip 1.5mm

{\footnotesize \noindent Melentiev Energy Systems Institute SB
RAS, 130, Lermontov ul., Irkutsk, 664033, Russian Federation

}

\vskip3mm

\noindent \textbf{For citation:}  Solodusha S. V. Quadratic and
cubic Volterra polynomials: identification and appli-\linebreak
cation. {\it Vestnik of Saint~Petersburg University. Applied
Mathematics. Computer Science. Control Processes}, \issueyear,
vol.~14, iss.~\issuenum, pp.~\pageref{p5}--\pageref{p5e}.
\doivyp/spbu10.\issueyear.\issuenum05

\vskip3mm

{\leftskip=7mm\noindent Volterra kernels identification is the
main problem in constructing an input-output type mathematical
model of nonlinear dynamical system by a Volterra polynomial of
$N$th order. Currently, various algorithms for solving this
problem are proposed. Usually, it is assumed that the
decomposition of the dynamical system response $y(t)$ into
components is preliminarily performed. Each of components is due
to the influence of the concrete integral term. In general, the
separation problem is invariant with respect to a particular
family of test actions, and the choice of amplitudes of the test
signals used to identify the Volterra kernels is related to the
necessary conditions for the solvability of the corresponding
multidimensional integral equations in special classes of
functions. In the present paper, existence theorems for solutions
of two-dimensional and three-dimensional Volterra integral
equations of the first kind are given. This result is obtained in
terms of the amplitudes of the test signals. This will allow us to
remove the arbitrariness in the choice of amplitudes in
construction of the quadratic and cubic Volterra polynomials in
the case when external action $x (t) = (x_1 (t), x_2 (t))^T$ is a
vector function of time. Illustrative calculations are given
through the dynamic reference systems.\\[1mm]
\textit{Keywords}: identification, Volterra kernels, integral
equations, mathematical modeling.
\par}

\vskip5mm

\noindent \textbf{References} }

\vskip 2mm

{\footnotesize

1. Fujii K., Nakao K. Identification of nonlinear dynamic systems
without self-regulation using Volterra functional series. {\it
Trans. Soc. Instr. Control Eng. (Japan)}, 1971, vol.~7, no.~2,
pp.~129--136.

2. Pavlenko V. D. Kompensatsionnyy metod identifikatsii
nelineynykh dinamicheskikh sistem v vide yader Vol'terra
[Compensation method for identification of nonlinear dynamic
systems in the form of Vol-\linebreak \newpage\noindent terra
kernels]. {\it Trudy Odesskogo politekhnicheskogo universiteta}
[{\it Proceedings of the Odessa Polytechnic University}], 2009,
vol.~2, pp.~121--129. (In Russian)

3. Masri M. M. {\it Metody i sredstva postroeniya informatsionnykh
modeley nelineynykh dinamicheskikh ob"ektov dlya tseley
diagnostiki} [{\it Methods and tools for constructing information
models of nonlinear dynamic objects for diagnostic purposes}]. PhD
tech. sci. diss. Odessa, Odessk. National Polytechnic
Uni-\linebreak versity Publ., 2015, 173~p. (In Russian)

4. Fomin A. A., Pavlenko V. D., Fedorova V. D. Metod postroeniya
mnogomernoy modeli Vol'terra glazodvigatel'nogo apparata [Method
for constructing the Volterra multidimensional model of the
eye-movement apparatus]. {\it Elektrotekhnicheskie i komp'yuternye
sistemy} [{\it Electrotechnical and Computer Systems}], 2015,
vol.~19, pp.~296--301. (In Russian)

5. Apartsyn A. S. {\it Neklassicheskie uravneniya Vol'terra I roda
v integral'nykh modelyakh dinamicheskikh sistem: teoriya,
chislennye metody, prilozheniya} [{\it Nonclassical Volterra
equations of the first kind in integral models of dynamical
systems: theory, numerical methods, applications}]. Dr.
phys.-math. sci. diss. Irkutsk, Irkutsk. Gos. University, 2000,
319~p. (In Russian)

6. Apartsyn A. S. Mathematical modelling of the dynamic systems
and objects with the help of the Volterra integral series. {\it
EPRI-SEI Joint Seminar}, Beijing, China, 1991, pp.~117--132.

7. Solodusha S. V. Chislennye metody identifikatsii
nesimmetrichnykh yader Vol'terra i ikh prilozheniya v
teploenergetike [Numerical methods for identification of
asymmetric Volterra kernels and their applications in heat power
engineering]. {\it Materialy XXIV konferentsii nauchnoy molodezhi
SEI SO RAN} [{\it Proceedings of the XXIV conference of young
scientists SEI SB RAS}]. Irkutsk, 10--11.03.1994 g. Deposited at
VINITI 30.08.1994, no.~2129-B94, pp.~76--91. (In Russian)

8. Apartsyn A. S., Tairov E. A., Solodusha S. V., Khudyakov D. V.
Primenenie integrostepennykh ryadov Vol'terra k modelirovaniyu
dinamiki teploobmennikov [Application of integro-power Volterra
series to modeling the dynamics of heat exchangers]. {\it
Izvestiya RAN. Energetika} [{\it Proceedings of the Russian
Academy of Sciences. Power Engineering}], 1994, vol.~3,
pp.~28--42. (In Russian)

9. Tairov E. A.   Nelineynoe modelirovanie dinamiki teploobmena v
kanale s odnofaznym teplo-\linebreak nositelem [{Nonlinear
modeling of the dynamics of heat transfer in a channel with single
phase coolant}]. {\it Izvestiya AN SSSR. Energetika i transport}
[{\it Proceedings of the Russian Academy of Sciences. Power
Engineering and Transport}], 1989, no.~1, pp.~150--156.  (In
Russian)

10. Solodusha S. V., Sidorov D. N. O modelirovanii nelineynoy
dinamiki teploobmennyh processov funkcional'nymi ryadami Vol'terra
[On modelling of heat-exchange process nonlinear dynamics by
functional Volterra series]. {\it Trudy Mezhdunar. konferensii
``Sredstva matematicheskogo modelirovaniya''}  [{\it Proceedings
of conference ``Mathematical Modelling Tools''}]. Saint
Petersburg, 3--6.12.1997. Saint Pe-\linebreak tersburg, SPbGTU
Publ., 1998, pp.~221--229. (In Russian)

11. Apartsin A. S., Solodusha S. V. Ob optimizatsii testovyh
signalov pri identifikatsii yader Vol'tera [Test signal amplitude
optimization for identification of the Volterra kernels]. {\it
Avtomatika i telemehanika} [{\it Automation and Remote Control}],
2004, vol.~65, no.~3, pp.~464--471. (In Russian)

12. Solodusha S. V. Chislennoe modelirovanie dinamiki teploobmena
modifitsirovannym kvadratichnym polinomom Vol'terra [Numerical
modeling of heat exchange dynamics by modified quadratic Volterra
polynomial]. {\it Vychislitel'nye tekhnologii} [{\it Computational
Technologies}], 2013, vol.~18, no.~2, pp.~83--94. (In Russian)




\vskip5mm A\,u\,t\,h\,o\,r's \ I\,n\,f\,o\,r\,m\,a\,t\,i\,o\,n:
\vskip1.5mm \textit{Solodusha Svetlana V.}~--- PhD Sci. in physics
and mathematics, associate professor, leading researcher;
solodusha@isem.irk.ru






}
