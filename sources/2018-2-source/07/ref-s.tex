
{\normalsize

\vskip 6mm

\noindent{\bf The entropy approach construction of the optimal
structure of the chain\\ of the automated identification systems
basic stations for inland waterways}

}

\vskip 1.5mm

{\small

\noindent{\it V. V. Karetnikov, A. V. Vasin}

\vskip 1.5mm

{\footnotesize \noindent Admiral Makarov State University of
Maritime and Inland Shipping,\\ 5/7, Dvinskaya ul.,
St.\,Petersburg, 198035, Russian Federation

}

\vskip3mm

\noindent \textbf{For citation:}  Karetnikov V. V., Vasin A. V.
The entropy approach construction of the optimal structure of the
chain of the automated identification systems basic stations for
inland waterways. {\it Vestnik of Saint~Petersburg University.
Applied Mathematics. Computer Science. Control Pro-\linebreak
cesses}, \issueyear, vol.~14, iss.~\issuenum,
pp.~\pageref{p7}--\pageref{p7e}.
\doivyp/spbu10.\issueyear.\issuenum07

\vskip3mm

{\leftskip=7mm\noindent The article deals with a constructive
approach based on the entropy for constructing a quasi-optimal
chain of automated identification systems (AIS) base stations on
the inland waterways. The inland waterways of Russian Federation
play an important role in providing the transport process while
transporting various types of cargo and passengers. Here is a
special way, it is worth noting the info-communication system of
the first hierarchical level of the vessel control system. This
class of AIS is the most promising for large-scale implementation
of the inland waterways of Russian Federation. For the
full-fledged operation of such info-communication systems on the
inland waterways of Russian Federation, it is necessary to form a
continuous information field, formed by sufficient overlapping of
the working zones of the operation of  AIS base stations connected
in a chain. To estimate the optimal number of AIS base stations,
we consider inland waterways as a fractal set. Therefore, it is
convenient to estimate the number of stations forming a continuous
coverage zone in terms of the Hausdorff measure. The problem is to
find the minimal  quantity  of elements of the $r$-network for
different $r$. This number is calculated by means of $r$-entropy
$H_r$ of the set under consideration. The entropy approach allows
us to take into account the phenomenon of collapse of AIS base
station coverage zone in case of interference with a useful signal
of a spectrum-centered noise. This case corresponds to the entropy
$H$
of the set (the limit of  $H_r$ when $r$ tends to 0).\\[1mm]
\textit{Keywords}: inland waterways, entropy of a set, automatic
identification system, interference, safety of navigation, relief,
coverage area of AIS base station.
\par}

\vskip5mm

\noindent \textbf{References} }

\vskip 2mm

{\footnotesize


1. Karetnikov V. V., Sikarev A. A. \textit{Topologia
differentsial'nyh polei i dal'nost' deistvia
kontrol'no-korrektiruyuschih stantsiy vysokotochnogo
mestoopredelenia na vnutrennih vodnyh putiah} [\textit{Topology of
differential fields and range of control-correcting stations for
high-precision positioning on inland waterways}]. Ed.~2.
Saint\,Petersburg, GUMRF Publ., 2013, 525~p. (In Russian)

2. Falconer K.L. \textit{The geometry of fractal sets.} Cambridge,
Cambridge University Press, 1985, 162~p.



\vskip5mm A\,u\,t\,h\,o\,r's \ I\,n\,f\,o\,r\,m\,a\,t\,i\,o\,n:
\vskip1.5mm \textit{Karetnikov Vladimir V.} --- Dr. Sci. in
technics, professor; spguwc-karetnikov@yandex.ru

\textit{Vasin Andrei V.} --- Dr. Sci. in technics,  professor;
andrejvasin@gmail.com




}
