{\footnotesize

\vskip 4mm
%\newpage

\noindent{\small \textbf{References} }

\vskip 3mm


1. Brucker P. {\it Scheduling algorithms.} Heidelberg, Springer
Press, 2007, 378~p.


2. Graham R. L., Lawner E. L., Kan R. Optimization and
approximation in deterministic sequencing and scheduling: A
survey. {\it Ann. of Disc. Math.}, 1979, vol.~5, no.~10,
pp.~287--326.



3. Lenstra J. A., Kan R., Brucker P. Complexity of machine
scheduling problems. {\it Ann. of Disc. Math.}, 1977, vol.~1,
pp.~343--362.

4. Brucker P., Garey M. R., Johnson D. S. Scheduling equal-length
tasks under tree-like precendence constraints to minimize maximum
lateness. {\it  Matematics of Operations Research}, 1977, no.~2,
pp.~275--284.


5. Zinder Y., Singh G. Preemptive scheduling on parallel
processors with due dates. {\it Asia Pacific Journal of
Operational Research}, 2005, no.~22, pp.~445--462.

6. Baker K. R. {\it Introduction to Sequencing and Scheduling.}
New York, John Wiley \& Son Press,  1974, 318~p.


7. Kanet J. J., Sridharan V. Scheduling with inserted idle time:
problem taxonomy and literature review. {\it Operations Research},
2000, vol.~48, no.~1, pp.~99--100.

8. Carlier J. The one-machine sequencing problem. {\it  European
J. Oper. Res.}, 1982, no.~11, pp.~42--47.


9. Grigoreva N. S. Algoritm vetvey i granits dlya zadachi
sostavleniya raspisaniya na parallelnykh protsessorakh [Branch and
bound algorithm  for multiprocessor scheduling problem]. {\it
Vestnik of Saint Petersburg University. Series~10. Applied
mathematics.  Comruter science. Control processes}, 2009, issue~1,
pp.~44--55. (In Russian)

10. Grigoreva N. S.  Multiprocessor scheduling with inserted idle
time to minimize the maximum lateness. {\it Proceedings of the 7th
Multidisciplinary Intern. Conference of Scheduling: Theory and
Appli-\linebreak cations.} Prague, MISTA Press, 2015,
pp.~814--816.


11. Gusfield D. Bounds for naive multiple machine scheduling with
release times and deadlines. {\it Journal of Algorithms}, 1984,
vol.~5, pp.~1--6.


12. Romanovskii I. V., Khristova N. P. Reshenie minimaksnykh
zadach metodom dikhotomii [The solution of minimax problem by the
method of dichotomy]. {\it Computational   mathematics and
mathematical physics}, 1973, vol.~13, no.~5, pp.~200--209. (In
Russian)\newpage

13. Mastrolilli M. Efficient approximation schemes for scheduling
problems with release dates and delivery times. {\it Journal of
Scheduling}, 2003, vol.~6, pp.~521--531.


14. Fernandez E., Bussell B. Bounds the number of processors and
time for multiprocessor optimal schedules. {\it IEEE Trans. on
Computers}, 1973, vol.~4, no.~11, pp.~745--751.


\vskip 2mm

{\bf For citation:}  Grigoreva N. S. Scheduling problem to
minimize the maximum lateness  for parallel processors. {\it
Vestnik of Saint Petersburg University. Series~10. Applied
mathematics. Computer science. Control processes}, \issueyear,
issue~\issuenum, pp.~\pageref{p5}--\pageref{p5e}.
\doivyp/spbu10.\issueyear.\issuenum05


}
