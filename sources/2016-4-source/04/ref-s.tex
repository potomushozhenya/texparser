

{\footnotesize

\vskip 2mm
%\newpage

\noindent {\small\textbf{References} }

\vskip 1.5mm



1. Lozkins~A., Bure~V. M. Veroyatnostniy podhod k opredeleniu
lokalno-optimalnogo chisla klasterov [The probabilistic method of
finding the local-optimum of clustering]. {\it Vestnik of Saint
Petersburg University. Series 10. Applied mathematics. Computer
science. Control processes}, 2016, issue~1, pp.~28--38. (In
Russian)

2. Aldenderfer~M. S., Blashfield~R. K. Klasternyi analiz
$[$Cluster analysis$]$. \emph{Factornyi, diskriminantnyi i
klasternyi analiz $[$Cluster, factor, discriminant function
analysis$]$}. Pod red. I.~S.~Enukova. Moskow, Finansi i statistica
Publ., 1989, 215~p. (In Russian)

3. Dmitriev~A. P., Zubriyanova~N. S. Statisticheskoe izuchenie
dinamiki pervichnoj zabolevaemosti naseleniya Penzenskoj oblasti
[Statistical research of dinamics of morbidity rate of Penza
Region population]. {\it Izvestiya vysshih uchebnyh zavedenij.
Povolzhskij region. Medicinskie nauki $[$University proceedings.
Volga region. Medical sciences$]$}, 2008, issue~2, pp.~89--98. (In
Russian)

4. Staroverova~K. U. Issledovanie dinamiki zabolevaemosti v Sankt
Peterburge [Research of behavior of Saint Petersburg morbidity
rate]. {\it Rezultatyi nauchnyih issledovaniy: sbornik statey
mezhdunarodnoy nauchno-prakticheskoy konferentsii $[$Scientific
research results: proceedings of the Intern. scientific and
research conference$]$}. Tyumen, Aeterna Publ., 2016, pp.~11--14.
(In Russian)

5. Staroverova~K. U. Klasterizatsiya vremennyih ryadov s
ispolzovaniem R [Time series clustering in R]. {\it Protsessyi
upravleniya i ustoychivost $[$Control processes and stability$]$},
2016, vol.~3, issue~1, pp.~317--323. (In Russian)

6. Montero~P., Vilar~J. TSclust: An R package for time series
clustering. {\it Journal of Statistical Software}, 2015, no.~62.1,
pp.~1--43.

7. Douzal~Chouakria A., Nagabhushan~P. N. Adaptive dissimilarity
index for measuring time series proximity. {\it Advances in Data
Analysis and Classification}, March 2007, vol.~1, issue~1,
pp.~1--43.

8. Yang~K., Shahabi~C. A PCA-based similarity measure for
multivariate time series. {\it MMDB '04 Proceedings of the 2nd ACM
Intern. workshop on Multimedia databases}, 2004, pp.~65--74.

9. Li~S. J. , Zhu~Y. L., Zhang~X. H., Wan~D. BORDA counting method
based similarity analysis of multivariate hydrological time
series. {\it Journal of Hydraulic Engineering}, 2009, vol.~40,
no.~3, pp.~378--384.

10. Wang~J., Zhu~Y., Li~S., Wang~D., Zhang~P. Multivariate time
series similarity searching. {\it The Scientific World Journal},
2014, vol.~2014, article ID~851017, 8~p.





\vskip 2mm

{\bf For citation:} Bure V. M., Staroverova K. U. Applying
clustering analysis for discovering time series heterogeneity
using Saint Petersburg morbidity rate as an illustration. {\it
Vestnik of Saint Petersburg University. Series~10. Applied
mathematics. Computer science. Control processes}, \issueyear,
issue~\issuenum, pp.~\pageref{p4}--\pageref{p4e}.
\doivyp/spbu10.\issueyear.\issuenum04



}
