


{\footnotesize

\vskip 3mm
%\newpage

\noindent {\small\textbf{References} }

\vskip 2mm

1. Arnold V. I. \emph{Zhyostkie i myagkie matematicheskie modeli
$[$Hard and soft mathematical models$]$}. Moscow, MShNMO Publ.,
2008, 32~p. (In Russian).


2. Kapitza S. P. {\it Global population blow-up and after. The
demographic revolution and information society. A report to the
club of Rome and the Global Marshall plan initiative}. Hamburg,
Tolleranza Publ., 2006, 272~p.


3. Kapitsa S. P. {\it Paradoksy rosta: zakony razvitiya
chelovechestva $[$Paradoxes of growth. Laws of humanity
development$]$}. Moscow, Alpina non-fiction Publ., 2010, 192~p.
(In Russian).


4. Aleskerov F., Andrievskaya I., Penikas G., Solodkov V. {\it
Analiz matematicheskikh modeley Bazel II $[$Analysis of
Mathematical Models of Basel II$]$}. Moscow, Fizmatlit Publ.,
2013, 269~p. (In Russian).
%Aleskerov, F., Andrievskaya, I., Penikas, G., and Solodkov, V. (2009). Analysis of Mathematical Models of Basel 2 (in Russian). Moscow: Fizmatlit.


5. Gubanov D., Novikov D., Chkhartishvili A. {\it Sotsial'nyye
seti: modeli informatsionnogo vliyaniya, upravleniya i
protivoborstva $[$Social networks: models of informational
influence, control and opposition$]$}. Moscow, Fizmatlit Publ.,
2010, 228~p. (In Russian).


6. Batov A. V., Breyer V. V., Novikov D. A., Rogatkin A. D. {\rm
Mikro- i makro modeli sotsial'nykh setey. Ch.~2. Identifikatsiya i
imitatsionnyye eksperimenty $[$Micro- and macro models of social
networks. Pt~2. Identification and simulations$]$}. {\it Problemy
upravleniya}, 2014, no.~6, pp.~45--51. (In Russian).


7. Weidlich W. {\it Sociodynamics: a systematic approach to
mathematical modelling in the social sciences}. Amsterdam, Harwood
Academic Publ., 2000. 380~p.


8. McClelland K., Fararo T. {\it Purpose, meaning and action:
Control systems theory in sociology}. London, Palgrave-Macmillan
Publ., 2006. 352~p.


9. Davydov A. A. {\rm Matematicheskaya sotsiologiya: obzor
zarubezhnogo opyta $[$Mathemati\-cal sociology. A review of
international experience$]$}. {\it Sociologicheskie
issledovaniya}, 2008, no.~4, pp.~105--111. (In Russian).


10. Akayev A. A. God bifurkatsii v dinamike mirovoy ekonomiki
[Year of the bifurcation in the dynamics of the world economy].
{\it Vestnik of Russian Academy of Sciences}, 2015, vol.~85,
issue~12, pp.~1059--1069. (In Russian)


11. Mann S. Chaos theory and strategic thought. {\it Parameters
$($US Army War College Quarterly$)$}, 1992, no.~22, pp.~54--68.


12. Lepsky B. Tekhnologii upravlyayemogo khaosa --- oruzhiye
razrusheniya subyektnosti razvitiya [Technology of controlled
chaos weapon of destruction of subjectivity]. {\it
Informacion\-nye vojny} [{\it Information wars}], 2010, no.~4,
issue~16, pp.~69--78. (In Russian)


13. Fradkov A. L., Pogromsky A. Yu. {\it Introduction to control
of oscillations and chaos}. Singapore, World Scientific Publ. Co
Inc., 1999, 391~p. (World scientific series on nonlinear science.
Series a. Monographs and treatises, vol.~35 (Book~35)).


14. Chen G., Yu X. {\it Chaos control. Theory and applications}.
Berlin, Heidelberg, Springer-Verlag Publ., 2003, 369~p.


15. Turchin P. {\it Historical dynamics$:$ Why states rise and
fall}. Princeton, Princeton University Press, 2003, 264~p.


16. Haken H. {\it Synergetics, an introduction: Nonequilibrium
phase transitions and self-organization in physics, chemistry, and
biology}. New York, Springer-Verlag Publ., 1983, 367~p.


17. Knyazeva H., Kurdyumov S. {\it Osnovaniya sinergetiki: Rezhimy
s obostreniyem, samo\-organizatsiya, tempomiry $[$Foundations of
synergetics: blow-up regimes, self-organization, tempo-worlds$]$}.
Saint Petersburg, Aletheia Publ., 2002, 414~p. (In Russian).


18. Leonov G. A. {\it Teoriya upravleniya $[$Control theory$]$}.
Saint Petersburg, Saint Petersburg University Press, 2006, 233~p.
(In Russian).


19. Andrasik L. Theory emerging instability via creative
perturbation --- the engine of socio-economic progress. {\it
Journal of knowledge society/International scientific journal},
2014, no.~1(2), pp.~11--27.


20. Davydov A. A. Arabskiye revolyutsii 2011 goda: sistemnaya
diagnostika [The Arab revolutions of 2011: system diagnostics].
{\it Sistemnyy monitoring global'nykh i regional'nykh riskov$:$
Arabskaya vesna 2011 g. $[$Systems monitoring of global and
regional risks$:$ the Arabic spring. 2011$]$}, 2012, pp.~174--187.
(In Russian)


21. Tsirel S. V. Revolyutsii, volny revolyutsiy i Arabskaya vesna
[Revolution, a wave of revolutions and the Arab Spring]. {\it
Sistemnyy monitoring global'nykh i regional'nykh riskov$:$
Arabskaya vesna 2011 g. $[$Systems monitoring of global and
regional risks$:$ the Arabic spring. 2011$]$}, 2012, pp.~128--161.
(In Russian)


22. Korotayev A. V., Khodunov A. S., Belova A. N., Malkov S. Yu.,
Khalturina D. A., Zin'kina~Yu.~V. Sotsial'no-demograficheskiy
analiz Arabskoy vesny [Socio-demographic analysis of the Arab
spring]. {\it Sistemnyy monitoring global'nykh i regional'nykh
riskov$:$ Arab\-skaya vesna 2011 g. $[$Systems monitoring of
global and regional risks$:$ the Arabic spring. 2011$]$}, 2012,
pp.~28--76. (In Russian)


23. Fituni L. L. Blizhniy Vostok: tekhnologii upravleniya
protestnym potentsialom [Middle East: control technology protest
potential]. {\it Aziya i Afrika segodnya $[$Asia and Africa
today$]$}, 2011, vol.~12, pp.~8--16. (In Russian)


24. Vasil'yev A. M. Tsunami revolyutsiy [Tsunami revolutions].
{\it Aziya i Afrika segodnya $[$Asia and Africa today$]$}, 2011,
vol.~3, pp.~2--18. (In Russian)


25. Korotayev A. V., Zin'kina Yu. V. Yegipetskaya revolyutsiya
2011 g. Strukturno-demograficheskiy analiz [The Egyptian
revolution of 2011. Structural and demographic analysis]. {\it
Aziya i Afrika segodnya $[$Asia and Africa today$]$}, 2011,
vol.~6, pp.~10--16. (In Russian)


26. Sundiyev I. Yu., Smirnov A. A. {\it Teoriya i tekhnologii
sotsial'noy destruktsii $($na primere ``tsvetnykh revolyutsiy''$)$
$[$Theory and technology of social degradation $($on example of
``color revolutions''$)$$]$}. Moscow, Institute for Economic
Strategies Publ., 2016, 433~p. (In Russian)


27. Goldstone J. Towards a fourth generation of revolutionary
theory. {\it Annual Review of Political Science}, 2001, no.~4,
pp.~139--187.


28. Goldstone J., Bates R., Epstein D., Gurr T., Lustik M.,
Marshall M., Ulfelder J., Wood-\linebreak ward~M.~A global model
for forecasting political instability. {\it The Americal Journal
of Political Science}, 2010, vol.~54, no.~1, pp.~190--208.


29. Komlos J., Nefedov S. Compact macromodel of pre-industrial
population growth. {\it Historical methods}, 2002, no.~35,
pp.~92--94.


30. Turchin P. Dynamical feedbacks between population growth and
sociopolitical instabi\-lity in agrarian states. {\it Journal of
Anthropological and Related Sciences}, 2005, no.~1, pp.~1--19.


31. Dunning T. Resource dependence, economic performance, and
political stability. {\it Journal of Conflict Resolution}, 2005,
no.~49, pp.~451--482.


32. Richardson L. {\it Weather prediction by numerical process}.
Cambridge, Cambridge University Press, 1922, 262~p. (London, UK,
reprinted by Dover, 1965, 236~p.).


33. Lynch P. The origins of computer weather prediction and
climate modeling. {\it Journal of Computational Physics}, 2008,
no.~227(7), pp.~3431--3444.


34. Leonov G. A. Dynamic principles of prognosis and control. {\it
Chaos theory: modeling, simulation and applications. Pt~1. Plenary
and keynote talks}. Eds: C.~H.~Skiadas, I.~Dimotikalis,
C.~Skiadas. Singapore, World scientific Publ. Co., 2011,
pp.~21--29.


35. Leonov G. A., Kudryashova E. V., Kuznetsov N. V. Modeling and
identification of the Tunisian social system in 2011--2014:
Bifurcation, revolution, and controlled stabilization. {\it
Preprints. 1st IFAC conference on modelling, identification and
control of nonlinear systems}, 2015, pp.~735--739.


36. {\it Sustainable Society Foundation}. Full Report
``Sustainable Society Index 2012'' (2012). Available at:
http://www.ssfindex.com/ssi2014/wp-content/uploads/pdf/ssi2012.pdf
(accessed: 28.03.2016).


37. Arnold V., Afraimovich Y., Ilyashenko Y., Shilnikov L. Teoriya
bifurkatsiy [Bifurcation theory]. {\it Dinamicheskiye sistemy-5.
Itogi nauki i tekhniki. Ser. Sovremennyye problemy matema\-tiki,
Fundam. napravleniya $[$Dynamic systems-5. Sums of science and
technology. Modern of mathematical problems. Fundamental
directions$]$}. Moscow, VINITI, 1986, vol.~5, pp.~5--218. (In
Russian)


38. Filatov S. Tunisskiy bunt. Bessmyslennyy i besposhchadnyy
[Tunisian riot. Senseless and merciless]. {\it Russkij
obozrevatel} [{\it Russian reviewer}], 2011. Available at:
http://www.rus-obr.ru/idea/9335 (accessed: 28.03.2016). (In
Russian)


39. {\it Global Food Markets Group}. U. K. Government. {\it The
2007/08 Agricultural Price Spikes: Causes and policy
implications}. London, 2008. Available at:
http://www.growthe\-nergy.org/images/reports (accessed:
28.03.2016).


40. Kashina A. A. Strategiya Tunisa v oblasti obrazovaniya
[Tunisian strategy in education]. {\it Blizhnij vostok i
sovremennost $[$Near East and the present$]$}, 2010, vol.~42,
pp.~50--73. (In Russian)


41. Kashina A. A. {\it  Sotsial'naya politika Tunisskoy respubliki
posledney treti XX veka --- nachala XXI~veka $[$Social Policy
Republic of Tunisia in the last third of XX --- beginning of XXI
centuries$]$}. Abstract of dissertation. Moscow, Institute of
Asian and African countries at Lomonosov Moscow State University
Publ., 2012, 24~p. (In Russian)


42. {\it Times Higher Education World University Rankings, 2014}.
Available at: http://www.
timeshighereducation.co.uk/world-university-rankings (accessed:
28.03.2016).


43. {\it The Academic Ranking of World Universities, 2014}.
Available at: http://www.shanghairanking. com  (accessed
28.03.2016).


44. Landau L. K probleme turbulentnosti [On the problem of a
turbulence]. {\it Dokl. Akad. Nauk SSSR}, 1944, vol.~44(8),
pp.~339--342. (In Russian)


45. Brinton C. {\it The anatomy of revolution}. New York, Vintage
Books Publ.,  1965, 320~p.



46. Leonov G. A. Dinamicheskiye printsipy prognozirovaniya i
upravleniya [Dynamic principles of prediction and control]. {\it
Problemy upravleniya $[$The problems of management$]$}, 2008,
no.~5, pp.~31--35. (In Russian)


47. Leonov G. A. {\it Strange attractors and classical stability
theory}. Saint Petersburg, Saint Petersburg University Press,
2008, 161~p.


48. Leonov G. A. O matematicheskom obrazovanii v Rossii i
Sankt-Peterburge. Proshloye, nastoyashcheye, budushcheye [On
mathematical education in Russia and Saint Petersburg. Past,
present, future]. {\it Differentsial'nyye uravneniya i protsessy
upravleniya $[$Differential equation and process management$]$},
2012, vol.~2, pp.~4--8. (In Russian).


49. Abramovich S., Kuznetsov N. V., Leonov G. A. V. A. Yakubovich
--- mathematician, ``father of the field'', and herald of
intellectual democracy in science and society. {\it
IFAC-PapersOnLine}, 2015, vol.~48, issue~11, pp.~001--003.



\vskip 1mm

{\bf For citation:}  Leonov G. A., Kuznetsov N. V., Kudryashova E.
V. Tunisia 2011�2014. Bifurcation, revolution, and controlled
stabilization. {\it Vestnik of Saint Petersburg University.
Series~10. Applied mathematics. Computer science. Control
processes}, \issueyear, issue~\issuenum,
pp.~\pageref{p9}--\pageref{p9e}. \doivyp/
spbu10.\issueyear.\issuenum08


}
