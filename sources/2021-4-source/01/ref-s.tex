
{\normalsize

\vskip 5%6
mm

\noindent{\bf Modelling and design of permanent magnet multipoles for beam transport and focusing. I. Selection of optimal design and parameters$^{*}$%
}

}

\vskip 2mm

{\small

\noindent{\it V.~M.~Amoskov$^1$, V.~N.~Vasiliev$^1$, E.~I.~Gapionok$^1$,
G.~G.~Gulbekyan$^2$, N.~S.~Edamenko$^3$, I.~A.~Iva\-nen\-ko$^2$,
N.~Yu.~Kazarinov$^2$, I.~V.~Kalagin$^2$, M.~V.~Kaparkova$^1$,
V.~P.~Kukhtin$^1$, E.~A.~Lamzin$^1$, A.~A.~Makarov$^1$,
A.~N.~Nezhentzev$^1$, D.~A.~Ovsyannikov$^3$,  D.~A.~Ovsyannikov$^{1,4}$~(Jr.), N.~F.~Osi\-pov$^2$, I.~Yu.~Rodin$^1$,
S.~E.~Sytchevsky$^{1,3}$, A.~A.~Firsov$^1$%

}

\vskip 2mm

%%%%%%%%%%%%%%%%%%%%%%%%%%%%%%%%%%%%%%%%%%%%%%%%%%%%%%%%%%%%%%%%%%

\efootnote{
%%
\vspace{-3mm}\parindent=7mm
%%
\vskip 0.1mm $^{*}$ This work was supported by the Saint Petersburg State University (project INI\_2021, ID:73371205).%\par
%%
%%\vskip 2.0mm
%%
%%\indent{\copyright} �����-������������� ���������������
%%�����������, \issueyear%
%%
}

%%%%%%%%%%%%%%%%%%%%%%%%%%%%%%%%%%%%%%%%%%%%%%%%%%%%%%%%%%%%%%%%%%

{\footnotesize

\noindent%
$^1$~%
D.~V.~Efremov Institute of Electrophysical
Apparatus, 3, Doroga na Metallostroy, St. Petersburg, 

\noindent%
\hskip2.45mm%
196641, Russian Federation


\noindent%
$^2$~%
Joint Institute for Nuclear Research, 6, ul. Zholio-Kyuri, Moscow
Region, Dubna,

\noindent%
\hskip2.45mm%
141980, Russian Federation


\noindent%
$^3$~%
St.\,Petersburg State University, 7--9, Universitetskaya nab.,
St.\,Petersburg,

\noindent%
\hskip2.45mm%
199034, Russian Federation


\noindent%
$^4$~%
St. Petersburg State University of Industrial Technologies and Design,
18, ul. Bolshaya Morskaya,

\noindent%
\hskip2.45mm%
St. Petersburg, 191186, Russian Federation


}

%%%%%%%%%%%%%%%%%%%%%%%%%%%%%%%%%%%%%%%%%%%%%%%%%%%%%%%%%%%%%%%%%%

\vskip2mm%3mm


\noindent \textbf{For citation:} Amoskov~V.~M., Vasiliev~V.~N., Gapionok~E.~I., Gulbekyan~G.~G., Edamenko~N.~S., Ivanenko~I.~A., Kazarinov~N.~Yu., Kalagin~I.~V., Kaparkova~M.~V., Kukhtin~V.~P., Lamzin~E.~A., Makarov~A.~A., Nezhentzev~A.\,N., Ovsyannikov~D.~A., Ovsyannikov~D.~A. (Jr.), Osipov~N.~F., Rodin~I.~Yu., Sytchevsky~S.~E., Firsov~A.~A.
Modelling and design of permanent magnet multipoles for beam transport
and focusing. I. Selection of optimal design and parameters. {\it Vestnik of Saint~Petersburg Uni\-ver\-si\-ty.
Ap\-plied Mat\-he\-matics. Computer Science. Control
Processes},\,\issueyear,
vol.~17,~iss.~\issuenum,~pp.~\pageref{p1}--\pageref{p1e}.  %\\
\doivyp/\enskip%
\!\!\!spbu10.\issueyear.\issuenum01 (In Russian)

\begin{hyphenrules}{english}

\vskip2mm

{\leftskip=7mm\noindent The design and specification choices are described for a PM quadrupole used to enable beam transport in a cyclotron. First an analytic study with a simplified 2D model is performed to give initial values for magnet configuration and performance. Characteristics of PM blocks and cylinders are analysed. Then a 3D parametrized model is used to solve the direct magnetostatic problem and accurately define quad specifications. Simulations are carried out with KOMPOT electromagnetic code utilizing the differential formulation. The regularization method is applied to solve the inverse problem. Magnetic characteristics, dimensions and shapes of the PM units and iron parts are determined in order to reach the specified field gradient. Possible correction of the resulting the ideal specification is discussed with respect to additional constraints put by practical implementation. Candidate PM materials are proposed. Simulated field maps are presented. The method described may serve as a basis for virtual prototyping and be integrated into end-to-end design and construction of magnet systems.\\[1mm]
\textit{Keywords}: permanent magnet, quadrupole, beam transport, analytical model, numerical model, direct and inverse problems, computed code, simulation. \par}

\vskip6mm%5

\end{hyphenrules}



\noindent \textbf{References} }

\vskip 2mm

{\footnotesize

1. Kapchinsky I.~M., Skachkov V.~S., Artemov V.~S. et al.
Opyt ispol'zovaniya neyavnopolyusnyh kwad\-rupol'\-nyh lins s postoyannymi magnitami na lineinom uskoritele I-2
[Experience of employing nonsalient-pole PM quads in the linear particle accelerator I-2].
{\em Trudy IX Vsesoyuznogo sovetshania po uskoriteliam zariazhennyh chastits} [{\em Proceedings of All-Union Conference by accelerating for charged particles}]
(Dubna, October 16--18, 1984). Dubna, Joint Inst. for Nucl. Research, 1984, vol.~2, pp.~57--60.
(In Russian)

2. Mitrofanov S., Apel P., Bashevoy V. et al.
The DC130 project: new multipurpose applied science facility for FLNR.
{\em Proceedings of 14th Intern. Conference on Heavy Ion Accelerator Technology}.
Lanzhou, China, 2018, pp.~122--124.

3. Kazarinov N., Apel P., Bekhterev V. et al.
Conceptual design of FLNR JINR radiation facility based on DC130 cyclotron.
{\em Proceedings of 61 Advanced Beam Dynamics Workshop on High-Intensity and High-Brightness Hadron Beams}.
Daejeon, Korea, 2018, pp.~324--328.

4. Thome R.~J., Tarrh J.~M. {\em MHD and fusion magnets: field and force design
concepts}. New York, Wiley Publ., 1982, 249~p. (Rus. ed.: Thome R.~J., Tarrh J.~M. {\em Magnitnye sistemy MGD-generatorov i termoiadernykh ustanovok. Osnovy rascheta polei
i sil.} Moscow, Energoatomizdat Publ., 1985, 272 p.)

5. Belyakov V.~A., Sytchevsky S.~E. Osobennosti tekhnologii chislennogo
modelirovaniya elektromagnitnyh polej termoyadernyh reaktorov na osnove tokamakov
[Aspects of EM field simulations for designing, analyzing and optimizing the
tokamak-type fusion reactors].
{\em Izvestia Rossiiskoi akademii nauk. Energetika} [{\em Proceedings of Russian Academy of Sciences. Energetics}], 2014, no.~1, pp.~141--149.
(In Russian)

6. Amoskov V.~M., Arslanova D.~N., Bazarov A.~M. et al.
Adaptatsiya vychislitel'noj tekhnologii modelirovaniya ustanovok termoyadernogo
sinteza dlya analiza i optimizatsii magnitnyh podvesov levitatsionnyh
transportnyh sistem
[Adaptation computation technology of modeling of thermonuclear syntesis
machines for analysis and optimization of magnetic suspension of levitating
vehicle]. {\em VANT}. Series Thermonuclear syntheses, 2014, vol.~37, iss.~4, pp.~84--95. (In Russian)

7. Amoskov V.~M., Belov A.~V., Belyakov V.~A. et al.
Computation technology based on KOMPOT and KLONDIKE codes for magneto static simulations in tokamaks.
{\em Plasma Devices Oper.}, 2008, vol.~16, pp.~89--103.

8. Amoskov V. M., Belov A. V., Belyakov V. A. et al.
Magnetic model MMTC-2.2 of ITER tokamak complex.
{\em Vestnik of Saint Petersburg University. Applied Mathematics.
Computer Science. Control Processes}, 2019, vol.~15, iss.~1, pp.~5--21. https://doi.org/10.21638/11702/spbu10.2019.101

9. Batygin V. V., Toptygin I. N.  {\em Problems in electrodynamics}.
London, New York, Academic Press, 1964, 587~p.
(Rus. ed.: Batygin~V.~V., Toptygin~I.~N. {\em Sbornik zadach po elektrodinamike}.
Moscow, Nauka Publ., 1970, 503~p.)

10. Tikhonov A.~N., Arsenin V. Ya. {\em Solutions of ill-posed problems}.
New York, Halsted Press, 1977, 288~p.
(Rus. ed.: Tikhonov A.~N., Arsenin V.~Ya. {\em Metody resheniya nekorrektnyh zadach}.
Moscow, Nauka Publ., 1979, 288~p.)

11. Preobrazhenskiy A.~A., Bishard E.~G. {\em Magnitnye materialy i elementy}
[{\em Magnetic materials and elements}]. �oscow, Vysshaia shkola Publ., 1986, 352~p. (In Russian)

12. Neiman L.~R., Demirchan K.~S. {\em Teoreticheskiye osnovy elektrotekhniki}.
Uchebnik dlya vuzov [{\em Theoretical foundations of electrical engineering}. Textbook
for high schools]. Leningrad, Energiya Publ., 1975, 407~p. (In Russian)

\vskip1.5mm

Received:  July 15, 2021.

Accepted: October 13, 2021.


\vskip4.5%6
mm A~u~t~h~o~r~s' \ i~n~f~o~r~m~a~t~i~o~n:%

\vskip2mm \textit{Victor M. Amoskov}  --- PhD in Physics and Mathematics; sytch@sintez.niiefa.spb.su \par%
%
\vskip2mm \textit{Vyacheslav N. Vasiliev}  --- sytch@sintez.niiefa.spb.su \par%
%
\vskip2mm \textit{Elena I. Gapionok} ---  sytch@sintez.niiefa.spb.su \par%
%
\vskip2mm \textit{Georgy G. Gulbekyan}  ---  georgy@jinr.ru \par%
%
\vskip2mm \textit{Nikolai S. Edamenko}  --- PhD in Physics and Mathematics, Associate Professor; n.edamenko@spbu.ru \par%
%
\vskip2mm \textit{Ivan A. Ivanenko} --- PhD in Physics and Mathematics;  ivan@jinr.ru \par%
%
\vskip2mm \textit{Nikolay Yu. Kazarinov}  --- PhD in Physics and Mathematics; nyk@jinr.ru \par%
%
\vskip2mm \textit{Igor V. Kalagin}  --- PhD in Physics and Mathematics; kalagin@jinr.ru \par%
%
\vskip2mm \textit{Marina V. Kaparkova} --- sytch@sintez.niiefa.spb.su \par%
%
\vskip2mm \textit{Vladimir P. Kukhtin}  --- PhD in Physics and Mathematics; sytch@sintez.niiefa.spb.su \par%
%
\vskip2mm \textit{Evgeny A. Lamzin}  --- Dr. Sci. in Physics and Mathematics; sytch@sintez.niiefa.spb.su \par%
%
\vskip2mm \textit{Anatoly A. Makarov} ---  sytch@sintez.niiefa.spb.su \par%
%
\vskip2mm \textit{Andrey N. Nezhentzev}  --- sytch@sintez.niiefa.spb.su \par%
%
\vskip2mm \textit{Dmitrij A. Ovsyannikov}  --- Dr. Sci. in Physics and Mathematics, Professor; dovs45@mail.ru \par%
%
\vskip2mm \textit{Dmitry A. Ovsyannikov} ({\it Jr.}) --- Postgraduate Student, Assistant;  d-ovs@yandex.ru \par%
%
\vskip2mm \textit{Nikolai F. Osipov}  ---  onik@jinr.ru \par%
%
\vskip2mm \textit{Igor Yu. Rodin}  --- PhD in Engineering; rodin@sintez.niiefa.spb.su \par%
%
\vskip2mm \textit{Sergey E. Sytchevsky} --- Dr. Sci. in Physics and Mathematics; sytch@sintez.niiefa.spb.su \par%
%
\vskip2mm \textit{Alexey A. Firsov}  ---  sytch@sintez.niiefa.spb.su \par%
%
}
