
{\normalsize

%\vskip 6mm
\newpage

\noindent{\bf Mathematical modeling of pressure distribution\\ during deformations of the intervertebral disc%$^{*}$
}

}

\vskip 3%
mm

{\small

\noindent{\it D.\,A. Chubarov, V.\,P. Tregubov, N.\,K. Egorova%$\,^2$%
%, I.~�. Famylia%$\,^2$%

 }

\vskip 3%
mm

%%%%%%%%%%%%%%%%%%%%%%%%%%%%%%%%%%%%%%%%%%%%%%%%%%%%%%%%%%%%%%%%%%

%\efootnote{
%%
%\vspace{-3mm}\parindent=7mm
%%
%\vskip 0.1mm \indent\indent~$^{*}$ This work was  supported by the%
%Russian Foundation for Basic Research (grant N 19-01-00146-a).\par
%%
%%\vskip 2.0mm
%%
%%\indent{\copyright} �����-������������� ���������������
%%�����������, \issueyear%
%%
%}

%%%%%%%%%%%%%%%%%%%%%%%%%%%%%%%%%%%%%%%%%%%%%%%%%%%%%%%%%%%%%%%%%%

{\footnotesize



\noindent%
%$^2$~%
St.\,Petersburg State University, 7--9, Universitetskaya nab., St.\,Petersburg,

\noindent%
%\hskip2.45mm%
199034, Russian Federation

}

%%%%%%%%%%%%%%%%%%%%%%%%%%%%%%%%%%%%%%%%%%%%%%%%%%%%%%%%%%%%%%%%%%

\vskip 2,5%3%
mm


\noindent \textbf{For citation:}  Chubarov D.\,A., Tregubov V.\,P., Egorova N.\,K. Mathematical modeling of pressure distribution during deformations of the intervertebral disc. {\it Vest\-nik of Saint Petersburg
University. Applied Mathe\-ma\-tics. Computer
Science. Cont\-rol Pro\-cesses}, %\,
\issueyear,
vol.~17, iss.~\issuenum,~pp.~\pageref{p6}--\pageref{p6e}. %\\
\doivyp/\enskip%
\!\!\!spbu10.\issueyear.\issuenum06 (In Russian)

\vskip3mm

%\pagebreak

\begin{hyphenrules}{english}

{\leftskip=7mm\noindent Based on the analysis of the mechanical models of the intervertebral disc given
in the literature, it is concluded that the finite element computational grids used
in them do not reflect the real structure of the intervertebral disc. In this regard,
a mechanical model of the intervertebral disc was built, the structure of which is
close to its real structure. The proposed model was used to determine the dynamics
of the pressure distribution in the intervertebral disc when one of the vertebrae
is rotated by a given angle. To determine the resulting bulges of each cell in the
structure of the fibrous ring, the Rayleigh method and its modification were used.
This made it possible to rationally calculate the volumes of cells. When calculating
the pressure in each cell, its linear dependence on the deflection value of the cell was used. As a result of the proposed algorithm, the pressure dynamics in each cell of the intervertebral disc model was determined when the angle between the vertebrae changed.
\\[1mm]
\textit{Keywords}: mechanical model, intervertebral disc, pressure distribution,
mathematical mo\-de\-ling.\par}

\end{hyphenrules}

\vskip6mm
%\pagebreak

\noindent \textbf{References} }

\vskip 3mm

{\footnotesize

1. {Castro A.\,P.\,G., Wilson W., Huyghe J.\,M., Ito K., Alves J.\,L.} Intervertebral disc creep behavior assessment through an open source finite element solver. {\it Journal of Bio\-me\-cha\-nics}, 2013, vol.~47, pp.~297--301.

2. {Gao X., Zhu Q., Gu W.} Prediction of glycosaminoglycan synthesis in intervertebral disc under mechanical loading. {\it Journal of Biomechanics}, 2016, vol.~49, pp.~2655--2661.

3. {Guo L.-X., Li R., Zhang M.} Biomechanical and fluid flowing characteristics of intervertebral disc of lumbar spine predicted by poroelastic finite element method. {\it Acta of Bioengineering and Biomechanics}, 2016, vol.~18, no.~2, pp.~19--29.

4. {Jacobs N.\,T., Cortes D.\,H., Peloquin J.\,M., Vresilovic E.\,J., Elliott D.\,M.} Validation and application of an intervertebral disc finite element model utilizing independently constructed tissue-level constitutive formulations that are nonlinear, anisotropic, and time-dependent. {\it Journal of Biomechanics}, 2014, vol.~47, pp.~2540--2546.

5. {Schmidt H., Bashkuev M., Galbusera F., Wilke H.-J., Shirazi-Adl A.} Finite element study of human lumbar disc nucleus replacements. {\it Computer Methods in Biomechanics and Biomedical Engineering}, 2014, vol.~17, iss.~16, pp.~1762--1776.

6. {Zanjani-Pour S., Winlove C.\,P., Smith C.\,W., Meakin J.\,R.} Image driven subject-specific finite element models of spinal biomechanics. {\it Journal of Biomechanics}, 2016, vol.~49, pp.~919--925.

7. {Gohari E., Nikkhoo M., Haghpanahi M., Parnianpour M.} Analysis of different material theories used in a FE model of a lumbar segment motion. {\it Acta of Bioengineering and Biomechanics}, 2013, vol.~15, no.~2, pp.~33--41.

8. {Schmidt H., Heuer F., Simon U., Kettler A., Rohlmann A., Claes L., Wilke H.-J.} Application of a new calibration method for a~three-dimensional finite element model of a~human lumbar annulus fibrosus. {\it Clinical Biomechanics}, 2006, vol.~21, pp.~337--344.

9. {Meroi E.\,A., Natali A.\,N., Pavan P.\,G., Skarpa K.} Chislennyj analiz mekha\-nichesko\-go povedeniya mezhpozvonkovogo diska s uchetom struktury kollagenovyh volokon [Nume\-ri\-cal analysis of the mechanical behavior of the intervertebral disc taking into account the structure]. {\it Russian Journal of Bio\-me\-cha\-nics},  2005, vol.~9, no.~1~(33), pp.~36--51. (In Russian)

10. {Zharnov A.\,M., Zharnova O.\,A.} Biomekhanicheskie processy v mezhpozvonkovom diske shejnogo otdela pozvonochnika pri ego dvizhenii [Biomechanical processes in the intervertebral disc of the cervical spine during its movement]. {\it Russian Journal of Bio\-me\-cha\-nics}, 2013, vol.~17, no.~1~(59), pp.~32--40. (In Russian)

11. {Horoshev D.\,V., Il'yalov O.\,R., Ustyuzhancev N.\,E., Nyashin Yu.\,I.} Bio\-me\-kha\-ni\-cheskoe mo\-de\-li\-ro\-va\-nie mezhpozvonochnogo diska poyasnichnogo otdela chelove\-ka --- sovremennoe sostoyanie problemy [Biomechanical modeling of the intervertebral disc of the human lumbar spine --- current state of the problem]. {\it Russian Journal of Bio\-me\-cha\-nics}, 2019, vol.~23, no.~3, pp.~411--422. (In Russian)


\vskip1.5mm Received:  July 19, 2021.

Accepted: October 13, 2021.

\vskip6mm A\,u\,t\,h\,o\,r\,s' \,\ i\,n\,f\,o\,r\,m\,a\,t\,i\,o\,n:%


\vskip2mm \textit{Dmitry A. Chubarov} --- Student; st050417@student.spbu.ru \par

\vskip2mm \textit{Vladimir P. Tregubov} --- Dr. Sci. in Physics and Mathematics, Professor;
v.tregubov@spbu.ru \par

\vskip2mm \textit{Nadezhda K. Egorova} --- Postgraduate Student; nadezhda\_ego@mail.ru \par
%
%
}
