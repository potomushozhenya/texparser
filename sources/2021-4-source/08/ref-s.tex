
{\normalsize

\vskip 4%6
mm

\noindent{\bf Exact penalty functions in the problem of choosing the optimal wholesale order in the face of rapid fluctuations in demand$^{*}$%
}

}

\vskip 2mm

{\small

\noindent{\it V. M. Bure, V. V. Karelin, L. N. Polyakova%
%, I.~�. Famylia%$\,^2$%

}

\vskip 2mm

%%%%%%%%%%%%%%%%%%%%%%%%%%%%%%%%%%%%%%%%%%%%%%%%%%%%%%%%%%%%%%%%%%

\efootnote{
%%
\vspace{-3mm}\parindent=7mm
%%
\vskip 0.1mm $^{*}$ This work  was supported by the Russian Foundation for Basic Research (project N 20-07-0108).%\par
%%
%%\vskip 2.0mm
%%
%%\indent{\copyright} �����-������������� ���������������
%%�����������, \issueyear%
%%
}

%%%%%%%%%%%%%%%%%%%%%%%%%%%%%%%%%%%%%%%%%%%%%%%%%%%%%%%%%%%%%%%%%%

{\footnotesize

\noindent%
%$^1$~%
St.\,Petersburg State University, 7--9, Universitetskaya nab.,
St.\,Petersburg,

\noindent%
%\hskip2.45mm%
199034, Russian Federation


}

%%%%%%%%%%%%%%%%%%%%%%%%%%%%%%%%%%%%%%%%%%%%%%%%%%%%%%%%%%%%%%%%%%

\vskip2mm%3mm


\noindent \textbf{For citation:} Bure V. M., Karelin V. V., Polyakova L. N. Exact penalty functions in the problem of choosing the optimal wholesale order in the face of rapid fluctuations in demand.
{\it Vestnik of Saint~Petersburg Uni\-ver\-si\-ty.
Ap\-plied Mat\-he\-matics. Computer Science. Control
Processes},\,\issueyear,
vol.~17,~iss.~\issuenum,~pp.~\pageref{p8}--\pageref{p8e}. %\\
\doivyp/\enskip%
\!\!\!spbu10.\issueyear.\issuenum08 (In Russian)

\vskip2mm

{\leftskip=7mm\noindent The current article discusses a different situation in the market, when there is a rush demand for a new product followed by a sharp drop in demand. The trading company uses the following scheme for the wholesale order of goods. The ordered product is divided into two parts, and the first batch of goods arrives immediately, it is sold over a certain period of time
$[0, T_1]$. The second batch of goods is delivered at the time $T$, but at the time interval $[0, T]$. This batch is sold at a discount and is completely sold out. Time $T$ corresponds to the end of the sale of the entire product. The time points $T_1, T$ are selected by the trading firm from the condition of maximizing income. The need to consider such a wholesale order scheme is related to the fact that, firstly, the warehouses of the trading firms have limited capacity and cannot accommodate all the ordered goods, and secondly, a manufacturer may not offer the entire ordered batch of goods, since not all goods can be produced at the initial (zero) point of time immediately after receiving the order. At the time $T_1$, the trading company completely sells the first batch of goods and receives financial
resources, part of which is paid to the manufacturer. At the moment of time $T$, the complete sale of all purchased goods is completed. The choice of time points $T_1$ and $T$ allow to determine the volume of the first batch of ordered goods and the total volume of product ordered from the manufacturer. In the article, a mathematical model is proposed that makes it possible to choose the optimal ordering strategy for a trading company in the conditions of excessive growth of demand for the new product in time $\tau_1,\tau_2$ at some unknown point in time and $\tau_*\in [\tau_3,\tau_4]$,  and the subsequent sharp drop in demand in the period of time $[\tau_3,~\tau_4]$ due to the saturation of the market with a new product. Four possible variants of optimization problems are considered. A method of exact penalty function is suggested, which allows one to find their solutions.\\[1mm]
\textit{Keywords}: inventory level, random demand, rush demand, shortage of goods, discount, maximin, method of exact penalty functions.\par}


\vskip6mm

\noindent \textbf{References} }

\vskip 2mm

{\footnotesize

1. Polyakova L. N., Bure V. M., Karelin V. V.  Maksiminnii podhod k ocenke obema zakaza tovara v usloviyah padeniya sprosa [Maximin approach in estimating of the goods order volume under condition of falling demand]. {\it Vestnik of Saint Petersburg University. Applied Mathematics. Computer Science. Control Processes}, 2018, vol.~14, iss.~4, pp.~252--260. \\ https://doi.org/10.21638/11701/spbu10.2018.408 (In Russian)


2.  Aggarwal S. P., Jaggi C. K.  Ordering policies of deteriorating items under permissible delay in payments.   {\it J. Oper. Res. Soc.},  1995, vol.~46, no.~5, pp.~658--662.

3. {Bure V. M., Karelin V. V., Bure A. V.} Ocenka ob"ema zakaza tovara pri vozmozhnom padenii sprosa [Evaluation of the volume of ordering of goods while possible demand drop].   {\it Vestnik of Saint Petersburg University. Applied Mathematics. Computer Science. Control Processes},     2018, vol.~14, iss.~3, pp.~252--260.
https://doi.org/10.21638/11701/spbu10.2018.306 (In Russian)

4.  Bure	V. M., Karelin V. V., Polyakova  L. N.  Probabilistic model of terminal services. {\it Appl. Math. Sci.}, 2016, vol.~10(39), pp.~1945--1952.
 https://doi.org/10.12988/ams.2016.63131

5. Bure V. M., Karelin V. V., Polyakova L. N., Yagolnik I. V. Modelirovanie processa zakaza dlya kusochno-linejnogo sprosa s nasyshcheniem [Modeling of the ordering process for piecewise-linear demand with saturation]. {\it Vestnik of Saint Petersburg University. Applied Mathematics. Computer Science. Control Processes}, 2017,  vol.~13, iss.~2, pp.~138--146. (In Russian)

6. Bure V. M., Karelin V. V., Myshkov S. K., Polyakova L. N.  Determination of order quantity with piecewise-linear demand function with saturation. {\it Intern. Journal of Applied Engineering Research},  2017, vol.~12, no.~18, pp.~7857--7862.

7. Chen S. C., Teng J. T., Skouri K. Economic production quantity models for deteriorating items with up-stream full trade credit and down-stream partial trade credit. {\it  Intern. Journal Prod. Econ.},  2013, vol.~155,  pp.~302--309.

8. Dave U. Letters and viewpoints on  economic order quantity under conditions of permissible delay in payments.  {\it  Journal Oper. Res. Soc.}, 1985, vol.~46,  no.~5, pp.~1069--1070.

9.  Giri B. C., Sharma S. An integrated inventory model for a deteriorating item with allowable shortages and credit linked wholesale price. {\it Optim. Lett.}, 2015, vol.~37, pp.~624�637.\\ https://doi.org/10.1007/s11590-014-0810-2

10. Giri B. C., Sharma S. Optimal ordering policy for an inventory system with linearly increasing demand and alowable shortages under two levels trade credit financing. {\it  Oper. Res. Intern. Journal}, 2016, vol.~16,  pp.~25--50.

11. 	Goyal S.  K.  Economic order quantity under conditions of permissible delay in payments. {\it J. Oper. Res. Soc.}, 1985, vol.~36(4), pp.~335--338.

12. 	Huang Y. F. Optimal retailer�s ordering policies in the EOQ model under trade credit financing.  {\it Journal Oper. Res. Soc.}, 2003, vol.~54(9), pp.~1011--1015.

13. Huang Y. F., Hsu K. H.  An EOQ model under retailer partial trade credit policy in supply chain. {\it Intern. Journal Prod. Econ.}, 2008, vol.~112(2), pp.~655--664.

14. 	Jamal A. M. M., Sarker B. R., Wang S. An ordering policy for deteriorating items with allowable shortages  and permissible delay in payment.  {\it J. Oper. Res. Soc.}, 1997, vol.~48(8), pp.~826--833.

15. 	Khanra S, Ghosh S. K., Chaudhuri K. S.  An EOQ model for a deteriorating item with time dependent quadratic demand under permissible delay in payment.  {\it Appl. Math. Comput.}, 2011, vol.~218(1), pp.~1--9.

16. 	Khanra S., Mandal B., Sarkar B. An inventory model with time
 dependent demand and shortages under trade credit policy. {\it   Econ. Model.}, 2013, vol.~35, pp.~349--355.

17. 	Maihami R., Abadi I. N. K. Joint control of inventory and its pricing for non-instantaneously deteriorating items under permissible delay in payments and partial backlogging. {\it Math. Comp. Model.}, 2012, vol.~55(5--6), pp.~1722--1733.

18.  Vasiliev F. P.  {\it Metody optimizacii. Kniga 1} [{\it Optimization methods. Book 1}].  Moscow, Moscow Centre of continuous mathematical education Publ., 2011, 624~p. (In Russian)

19. Polyak B. T. {\it Vvedenie v optimizaciyu} [{\it Introduction to optimization}]. Moscow, Nauka Publ., 1983, 384~p. (In Russian)

20.  Sukharev A. G., Timokhov A.V., Fedorov V. V. {\it Kurs metodov optimizacii} [{\it Course of optimization methods}]. �oscow, Nauka Publ., 1986,  325~p.

21.  Fedorov V. V. O metode shtrafnih funkcii v zadache opredeleniya  maksimina [On the method of penalty functions in the problem of determining maximina]. {\it Computational Mathematics and Mathematical Physics}, 1972,  no.~2, pp.~321--333. (In Russian)

22. Polyakova L. N. O metode tochnih shtrafnih kvazidifferenciruemih
funkcii [On the method of exact penalty quasi-differentiable
functions]. {\it Computational Mathematics and Mathematical Physics},  2001, vol.~41, no.~2, pp.~225--238. (In Russian)


\vskip1.5mm Received:  February 2, 2021.

Accepted: October 13, 2021.


\vskip6mm A\,u\,t\,h\,o\,r\,s'\,  i\,n\,f\,o\,r\,m\,a\,t\,i\,o\,n:%


\vskip1.5mm \textit{Vladimir M. Bure} --- Dr. Sci. in Technics, Professor; vlb310154@gmail.com \par%
%
\vskip1.5mm \textit{Vladimir V. Karelin} --- PhD   in Physics and Mathematics, Associate Professor; lkarelin@mail.ru \par%
%
\vskip1.5mm \textit{Ludmila N. Polyakova} --- Dr. Sci. in Physics and Mathematics, Professor; lnpol07@mail.ru \par%
%
}
