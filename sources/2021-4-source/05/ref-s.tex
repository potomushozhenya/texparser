
{\normalsize

%\vskip 6mm

\newpage


\noindent{\bf Calculation of the turbulent boundary layer of a flat plate$^*$%
 }

}

\vskip 2mm

{\small

\noindent{\it V. A. Pavlovsky$^1$,  S. A. Kabrits$^2$%$^{1,4}$%
%, I.~�. Famylia%$~^2$%

 }

\vskip 2mm

%%%%%%%%%%%%%%%%%%%%%%%%%%%%%%%%%%%%%%%%%%%%%%%%%%%%%%%%%%%%%%%%%%

\efootnote{
%%
%\vspace{-3mm}\parindent=7mm
%%
%\vskip 0.0mm
\hskip 4mm$^{*}$ This study was carried out within the framework of the state task for the implementation of research works
N 075-03-2020-094/1 of June 10, 2020.%\par
%%
%%\vskip 2.0mm
%%
%\indent{\copyright} �����-������������� ���������������
%�����������, \issueyear%
%%
}

%%%%%%%%%%%%%%%%%%%%%%%%%%%%%%%%%%%%%%%%%%%%%%%%%%%%%%%%%%%%%%%%%%

{\footnotesize


\noindent%
$^1$~%
St. Petersburg State Marine Technical University, 3, ul. Locmanskaya, St. Petersburg,

\noindent%
\hskip2.45mm%
190121, Russian Federation



\noindent%
$^2$~%
St.~Petersburg State University, 7--9, Universitetskaya nab.,
St.~Petersburg,

\noindent%
\hskip2.45mm%
199034, Russian Federation

}

%%%%%%%%%%%%%%%%%%%%%%%%%%%%%%%%%%%%%%%%%%%%%%%%%%%%%%%%%%%%%%%%%%
%\newpage
\vskip3.0mm%3mm

\noindent \textbf{For citation:} Pavlovsky V. A., Kabrits S. A. Calculation of the turbulent boundary layer of a flat plate. {\it Vestnik of Saint~Peters\-burg Uni\-versity.
Applied Mathe\-ma\-tics. Computer Science. Control Pro\-cesses},
\issueyear,
vol.~17, iss.~\issuenum, pp.~\pageref{p5}--\pageref{p5e}.  %\\
\doivyp/\enskip%
\!\!\!spbu10.\issueyear.\issuenum05 (In Russian)

\vskip2.0%3
mm

\begin{hyphenrules}{english}

{\leftskip=7mm\noindent The calculation of the turbulent boundary layer is performed when a steady flow of a viscous fluid flows around a flat plate. The calculation is based on a system of equations of turbulent fluid motion, obtained by generalizing Newton�s formula for the tangential stress in a fluid by giving it a power-law form followed by writing the corresponding rheological relationship in tensor form and substituting it into the equation of motion of a continuous medium in stresses. The use of this system for the problem of longitudinal flow around a flat plate after estimates of the boundary layer form made it possible to write a system of equations describing a two-dimensional fluid flow in the boundary layer of a flat plate. This system is reduced to one ordinary third-order equation, similarly to how Blasius performed it for a laminar boundary layer. When solving this equation, the method of direct reduction of the boundary value problem to the Cauchy problem was used. The results of this solution made it possible to determine expressions for the thickness of the boundary layer, displacement and loss of momentum. These values are compared with the available experimental data.\\[1mm]
\textit{Keywords}: turbulence, differential equations of turbulent flow, flat plate, boundary layer, Reynolds number, drag coefficient, boundary layer thickness, displacement thickness, momentum loss thickness. \par}

\end{hyphenrules}

\vskip 6mm

\noindent \textbf{References} }

\vskip 2 mm

{\footnotesize

1.  Pavlovsky V. A. Stepennoe obobchenie  formuli Newtona dlya  kasatelnogo napriazhenia  v zhidkosti  v forme tensornogo rheologitcheskogo sootnochenia i vitekayuchie  iz nego  varianti  postroenia modeli techenia [Power-law generalization of Newton's formula for the shear stress in a fluid in the form of a ten\-sor rheological relation and  the resulting options for constructing flow models]. {\it Thesisy  dokladov IX~Po\-liachovskich thteniy. Materialy Mezhdunarodnoiy nautchnoi konferenzii po mechanike} [{\it Thesis of papers IX~Po\-liachovsky readings. Materials of Intern. Sci. Conference on Mechanics}]. March 9--12,
2021.  Saint Petersburg, Russia. St. Petersburg, BBM Press, 2021, pp.~226--227. (In Russian)

2.  Schlichting H. {\it Grenzschicht Theorie} [{\it Boundary layer theory}]. Berlin, Verlag G. Braun Publ., 1965, 736~p. (Rus. ed.: Schlichting H. {\it Teoriya pogranichnogo sloya}. Moscow, Nauka Publ., 1974, 711~p.)

3.  Lojczyanskij L. G.  {\it Mehanika  zhidkosti  i  gaza} [{\it Fluid and gas mechanics}]. Moscow, Drofa Publ., 2003, 840~p. (In Russian)

4. Pavlovsky V. A., Chistov A. L., Kuchinsky D. M. Modelirovanie techenij v trubah [Modeling of pipe flows]. {\it Vestnik of Saint Petersburg University. Applied Mathematics. Computer Science. Control Processes}, 2019, vol.~15, iss.~1, pp.~93--106. https://doi.org/10.21638/11702/spbu10.2019.107 (In Russian)

5. Pavlovsky V. A., Nikushhenko D. V. {\it  Vychislitelnaya gidrodinamika. Teoreticheskie osnovy} [{\it Computational fluid dynamics. Theoretical fundamentals}]. St. Petersburg, Lan� Publ., 2018, 368~p. (In Russian)

6. Fediyaevsky K. K., Voitkunsky Ya. I., Faddeev Yu. I. {\it Gidromechanica} [{\it Hydromechanics}]. Leningrad, Sudostroenie Publ., 1968, 568~p. (In Russian)

7. Kabrits S. A., Kolpak E. P. Numerical study of convergence of nonlinear models of the theory of shells with thickness decrease.
 {\it AIP Conference 2015. Proceedings}, vol.~1648, no.~300005. https://doi.org/10.1063/1.4912547

8. Novozhilov V. V., Pavlovskij V. A. {\it Ustanovivchiesiya  turbulentnia  tchechenia neszhimaemoi zhidkosty}. 2-e izd. [{\it Establishment of turbulence through an incompressible fluid.} 2${^{\rm nd}}$ ed.]. St. Petersburg, St. Pe\-ters\-burg University Press, 2012, 484~p.



\vskip 1.5mm

Received:  October 3, 2020.

Accepted: October 13, 2021.


\vskip6 mm A~u~t~h~o~r~s'\,  i~n~f~o~r~m~a~t~i~o~n:


\vskip2 mm \textit{Valery A. Pavlovsky} --- Dr. Sci. in Physics and Mathematics, Professor; v.a.pavlovsky@gmail.com \par%
%
\vskip2 mm \textit{Sergey A. Kabrits} --- PhD in Physics and Mathematics, Associate Professor; s.kabrits@spbu.ru \par
%
}
