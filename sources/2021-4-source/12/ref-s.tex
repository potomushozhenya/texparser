
{\normalsize

\vskip 6%
mm

\noindent{\bf Digital control of output  variables
in a given range considering delay$^{*}$
 }

}

\vskip 3%
mm

{\small

\noindent{\it M.~V.~Sotnikova, R.~A.~Sevostyanov %$\,^2$%
%, I.~�. Famylia%$\,^2$%

 }

\vskip 3%
mm

%%%%%%%%%%%%%%%%%%%%%%%%%%%%%%%%%%%%%%%%%%%%%%%%%%%%%%%%%%%%%%%%%%

\efootnote{
%%
\vspace{-3mm}\parindent=7mm
%%
\vskip 0.1mm \indent~$^{*}$ This work is supported by Russian Foundation for Basic Research (project N 20-07-00531a).\par
%%
%%\vskip 2.0mm
%%
%%\indent{\copyright} �����-������������� ���������������
%%�����������, \issueyear%
%%
}

%%%%%%%%%%%%%%%%%%%%%%%%%%%%%%%%%%%%%%%%%%%%%%%%%%%%%%%%%%%%%%%%%%

{\footnotesize



\noindent%
%$^2$~%
St.\,Petersburg State University, 7--9, Universitetskaya nab., St.\,Petersburg,

\noindent%
%\hskip2.45mm%
199034, Russian Federation

}

%%%%%%%%%%%%%%%%%%%%%%%%%%%%%%%%%%%%%%%%%%%%%%%%%%%%%%%%%%%%%%%%%%

\vskip 3mm


\noindent \textbf{For citation:}  Sotnikova M.\,V., Sevostyanov R.\,A. Digital control of output variables in a given range considering delay. {\it Vest\-nik of Saint Petersburg
University. Applied Mathe\-ma\-tics. Computer
Science. Cont\-rol Pro\-cesses}, %\,
\issueyear,
vol.~17, iss.~\issuenum,~pp.~\pageref{p12}--\pageref{p12e}. \\
\doivyp/\enskip%
\!\!\!spbu10.\issueyear.\issuenum12 (In Russian)

\vskip3mm

%\pagebreak

\begin{hyphenrules}{english}

{\leftskip=7mm\noindent The article is devoted to the control of a dynamic object with the retention of controlled variables in the required range, taking into account the high dimension of input and output, time delay, constraints and external disturbances. The formalized statement of the problem of digital control synthesis is considered. An approach based on the use of predictive models is proposed. Its key feature is the introduction of a specialized quality functional that reflects the specifics of the problem, the error functional. This functional plays the role of a penalty for the output of controlled variables that go beyond the specified range. It is shown that the implementation of the control law is reduced to solving the problem of quadratic programming at each instant of discrete time. The results obtained are illustrated by an example of controlling the oil refining process in a distillation column.
\\[1mm]
\textit{Keywords}: digital control, predictive model, control in a range, optimization, constraints.\par}

\end{hyphenrules}

\vskip6mm
%\pagebreak

\noindent \textbf{References} }

\vskip 3mm

{\footnotesize

{1}. {Burdick D. L., Leffler W. L.} {\it Petrochemicals in nontechnical language}. Oklahoma, USA, PennWell Publ. Company, 1990, 347~p.

{2}. {Corriou J. P.} Distillation column control. {\it Process Control}. Cham, Springer Publ., 2018, pp.~793--819.

{3}. {Kouvaritakis B., Cannon M.} {\it Model predictive control: classical, robust and stochastic}. Cham, Springer Intern. Publ., 2016, 384~p.

{4}. {\it Handbook of Model Predictive Control}. Eds by S. V. Rakovi\'{c}, W. S. Levine. Basel, Birkh\"{a}user Publ., 2019, 692~p.

{5}. {Veremey E. I., Sotnikova M. V.} {\it Upravlenie s prognoziruyushchimi modelyami}. Ucheb. posobie [{\it Model predictive control}. Tutorial]. Voronezh, Nauchnaya kniga Publ., 2016, 214~p. (In Russian)

{6}. {Lahiri S. K.} {\it Multivariable predictive control: Applications in industry}. Hoboken, New York, USA, John Wiley \& Sons Publ., 2017, 304~p.

{7}. {Donzellini G., Oneto L., Ponta D., Anguita D.} {\it Introduction to digital systems design}. Cham, Springer Intern. Publ., 2019, 536~p.

{8}.	{\it Sotnikova M.} Plasma stabilization based on model predictive control. {\it Intern. Journal of Modern Physics A}, 2009, vol.~24, no.~5, pp.~999--1008.

{9}. {Aleksandrov A. Yu., Zhabko A. P.} {\it Ustoychivost dvizheniy diskretnyh dinamicheskih sistem $[$Sta\-bi\-lity of motions of discrete dynamic systems$]$}. St. Petersburg, Research institute of chemistry of St.~Pe\-tersburg University Publ., 2003, 112~p. (In Russian)

{10}. { Landau I. D., Zito G.} {\it Digital control systems: design, identification and implementation}. London, Springer-Verlag Publ., 2006, 484 p.

{11}. {Sotnikova M. V.} Sintez tsifrovogo upravleniya s prognozom dlya uderzhaniya kon\-tro\-li\-ruye\-mykh peremennyh v zadannom diapazone [Digital control design based on predictive models to keep the controlled variables in a given range]. {\it Vestnik of Saint Petersburg University. Applied Mathematics. Computer Science. Control Processes}, 2019, vol. 15, iss. 3, pp. 397--409.\\ https://doi.org/10.21638/11702/spbu10.2019.309 (In Russian)

{12}. {Veremey E. I.} {\it Lineynye sistemy s obratnoi svyaz'yu}. Ucheb. posobie [{\it Linear feedback systems}. Tutorial]. St. Petersburg, Lan Publ., 2013, 448 p. (In Russian)


\vskip1.5mm Received:  October 13, 2021.

Accepted: May 28, 2021.

\vskip6mm A\,u\,t\,h\,o\,r\,s' \, i\,n\,f\,o\,r\,m\,a\,t\,i\,o\,n:%

\vskip2mm \textit{Margarita V. Sotnikova} --- Dr. Sci. in Physics and Mathematics, Associate Professor;\\ m.sotnikova@spbu.ru \par
%
\vskip2mm \textit{Ruslan A. Sevostyanov} --- Assistant; sevostyanov.ruslan@gmail.com \par
%
%
}
