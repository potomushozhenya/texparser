
{\normalsize

\vskip 4.5mm%6mm

\noindent{\bf Radionuclide images processing with the use of
discrete systems}

}

\vskip 1.8mm%2mm

{\small

\noindent{\it E. D. Kotina, E. B. Leonova, V. A. Ploskikh%$\,^1$%
%, I.~�. Famylia%$\,^2$%

 }

\vskip 1.8mm%2mm

%%%%%%%%%%%%%%%%%%%%%%%%%%%%%%%%%%%%%%%%%%%%%%%%%%%%%%%%%%%%%%%%%%

%\efootnote{
%%
%\vspace{-3mm}\parindent=7mm
%%
%\vskip 0.1mm $^{*}$ This work is supported by Russian Science
%Foundation (project N 18-71-00006 ).\par
%%
%%\vskip 2.0mm
%%
%%\indent{\copyright} �����-������������� ���������������
%%�����������, \issueyear%
%%
%}

%%%%%%%%%%%%%%%%%%%%%%%%%%%%%%%%%%%%%%%%%%%%%%%%%%%%%%%%%%%%%%%%%%

{\footnotesize

\noindent%
%$^1$~%
St.\,Petersburg State University, 7--9, Universitetskaya nab.,

\noindent%
%\hskip2.45mm%
St.\,Petersburg, 199034, Russian Federation


}

%%%%%%%%%%%%%%%%%%%%%%%%%%%%%%%%%%%%%%%%%%%%%%%%%%%%%%%%%%%%%%%%%%

\vskip1.8mm%3mm


\noindent \textbf{For citation:}  Kotina E. D., Leonova E. B.,
Ploskikh V. A. Radionuclide images processing with the use of
discrete systems. {\it Vestnik of Saint~Petersburg University.
Applied Mathematics. Computer Science. Control
Processes},\,\issueyear,
vol.~15,~iss.~\issuenum,~pp.~\pageref{p10}--\pageref{p10e}.\\
\doivyp/\enskip%
\!\!\!spbu10.\issueyear.\issuenum10  (In Russian)

\vskip3mm

{\leftskip=7mm\noindent The article considers digital processing
of image series using the discrete systems. The problem of
displacement field calculation for image sequences is presented.
The mathematical model based on discrete optimization is proposed
as its solution. The model reflects the discrete nature of image
series acquisition. It also takes into account the intensity
change along the system trajectories. The study of integral-type
functional on the trajectories ensemble of the discrete system was
performed, and the optimization algorithm for displacement field
construction based on its results was designed. The analytical
form of functional variation is obtained and the functional
gradient is derived, which allows us to use directional
optimization methods to find the required parameters. The
algorithm can be used in digital image processing, in particular,
in nuclear medicine imaging. The method example implementation for
nuclear medicine image processing is considered.\\[1mm]
\textit{Keywords}: discrete systems, functional variation,
optimization, image processing, radio\-nuc\-lide images.
\par}

\vskip5mm

\noindent \textbf{References} }

\vskip 2mm

{\footnotesize

1.    Zhuravlev Y. I. Ryazanov V. V. O reshenii zadach
raspoznavaniya po precedentam pri bol'shom chisle klassov
[Solution of instance-based recognition problems with a large
number of classes]. \textit{Doklady Academii nauk. Mathematica}
[Papers of Academy of Science. Mathematics], 2017, vol.~96(2),
pp.~488--490. (In Russian)

2.  Zhuravlev Y. I., Laptin Yu. P., Vinogradov A. P., Zhurbenko N.
G., Lykhovyd O. P., Berezovs\-kyi~O.~A.  Linear classifiers and
selection of informative features. \textit{Pattern Recognition and
Image Analysis}, 2017, vol.~27, no.~3, pp.~426--432.

3.  Kotina �. D.  K teorii opredeleniya polya peremeshchenij na
osnove uravneniya perenosa v diskretnom sluchae [On the theory of
determining displacement field on the base of transfer equation in
discrete case]. \textit{Vestnik of Saint Petersburg University.
Series~10. Applied Mathematics. Computer Science. Control
Processes}, 2010, iss.~3, pp.~38--43. (In Russian)

4. Kotina E., Pasechnaya G.  3D velocity field for heart
tomography. \textit{Proceedings  of 2015 International Conference
on �Stability and Control Processes� in memory of V.~I.~Zubov
(SCP)}. Saint Petersburg, 2015, pp.~646--647.

5. Kotina E. D., Latypov V. N., Ploskikh V. A.  Universal system
for tomographic reconstruction on GPUs. \textit{Problems of Atomic
Science and Technology}, 2013, no.~6(88), pp.~175--178.

6. Kotina E. D., Ploskikh V. A. Data processing and quantitation
in nuclear medicine.  \textit{RuPAC 2012 Contributions to the
Proceedings --- 23rd Russian Particle Accelerator Conference},
2012, pp.~526--528.

7. Kotina E. D., Ploskikh V. A., Babin A. V.  Radionuclide methods
application in cardiac studies. \textit{Problems of Atomic Science
and Technology}, 2013, vol.~88(6), pp.~179--182.

8. Kotina E. D., Pasechnaya G. A.  Optical flow-based approach for
the contour detection in radionuclide images processing.
\textit{Cybernetics and Physics}, 2014, vol.~3, no.~2, pp.~62--65.

9. Horn B. K. P., Schunck B. G.  Determining optical flow.
\textit{Artificial Intelligence}, 1981, vol.~17, no.~11,
pp.~185--203.

10. Anandan P. A.  A computational framework and an algorithm for
the measurement of visual motion. \textit{International Journal of
Computer Vision}, 1989, vol.~2, pp.~283--310.

11. Barron J., Fleet D.   Performance of optical flow techniques.
\textit{International Journal of Computer Vision}, 1994, vol.~12,
pp.~43--77.

12. Black M. J., Anandan P. A. The robust estimation of multiple
motions. Parametric and piecewise-smooth flow fields.
\textit{CVIU}, 1996, vol.~63, pp.~75--104.

13. Fleet D., Weiss J.  Optical flow estimation.
\textit{Mathematical Models in Computer Vision}. The Handbook. Ch.
15. Berlin, Springer Press, 2005, pp.~239--258.

14. Papenberg N., Bruhn A., Brox T. et al. Highly accurate optic
flow computation with theoretically justified warping.
\textit{International Journal of Computer Vision}, 2006,
vol.~67(2), pp.~141--158.

15. Kotina E. D. O skhodimosti blochnyh iteracionnyh metodov [On
convergence of block iterative methods]. \textit{The bulletin of
Irkutsk State University. Series Mathematics}, 2012, vol.~3,
pp.~41--55. (In Russian)

16. Lucas B. D., Kanade T. An iterative image registration
technique with an application to stereo vision.
\textit{Proceedings of Imaging Understanding Workshop}, 1981,
pp.~121--130.

17. Bruhn A., Weickert J., Schnorr C. Lucas/Kanade Meets
Horn/Schunck: Combining local and global optic flow methods.
\textit{International Journal of Computer Vision}, 2005,
vol.~61(3), pp.~211--231.

18. Ovsyannikov D. A. {\it Matematicheskie metody upravleniya
puchkami} [{\it Mathematical methods of beam control}]. Leningrad,
Leningrad State University Publishing House, 1980, 228~p. (In
Russian)

19. Ovsyannikov D. A. {\it Modelirovanie i optimizaciya dinamiki
puchkov zaryazhennyh chastic} [{\it Modeling and optimization of
charged particle beam dynamics}]. Leningrad, Leningrad State
University Publishing House, 1990, 312~p. (In Russian)

20. Kotina E. D.  Discrete optimization problem in beam dynamics.
\textit{Nuclear Instruments and Methods in Physics Research.
Section A.  Accelerators, Spectrometers, Detectors and Associated
Equipment}, 2006, vol.~558, iss.~1, pp.~292--294.

21. Russell D., Koch R.\textit{ Sobel operator}. VSD, 2013, 106~p.
(In Russian)



\vskip1.5mm Received:  September 08, 2019.

Accepted: November 07, 2019.


\vskip6mm A\,u\,t\,h\,o\,r's \ i\,n\,f\,o\,r\,m\,a\,t\,i\,o\,n:%

\vskip1.5mm%2mm
\textit{Elena D. Kotina} --- Dr. Sci. in Physics and Mathematics,
Professor; e.kotina@spbu.ru

\vskip1.5mm%2mm
\textit{Ekaterina B. Leonova} --- Postgraduate Student;
st062324@student.spbu.ru

\vskip1.5mm%2mm
\textit{Viktor A. Ploskikh} --- PhD in Physics and Mathematics,
Associate Professor; v.ploskikh@spbu.ru

}
