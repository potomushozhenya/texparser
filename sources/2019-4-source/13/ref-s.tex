
{\normalsize

\vskip 4.5mm%6mm

\noindent{\bf Model for planning the number of optical disks
needed to create\\ and maintain a long-term electronic
archive%$^{*}$%
}

}

\vskip 1.8mm%2mm

{\small

\noindent{\it A. V. Chernyshov%$\,^1$%
%, I.~�. Famylia%$\,^2$%

 }

\vskip 1.8mm%2mm

%%%%%%%%%%%%%%%%%%%%%%%%%%%%%%%%%%%%%%%%%%%%%%%%%%%%%%%%%%%%%%%%%%

%\efootnote{
%%
%\vspace{-3mm}\parindent=7mm
%%
%\vskip 0.1mm $^{*}$ This work is supported by Russian Science
%Foundation (project N 18-71-00006 ).\par
%%
%%\vskip 2.0mm
%%
%%\indent{\copyright} �����-������������� ���������������
%%�����������, \issueyear%
%%
%}

%%%%%%%%%%%%%%%%%%%%%%%%%%%%%%%%%%%%%%%%%%%%%%%%%%%%%%%%%%%%%%%%%%

{\footnotesize

%\noindent%
%$^1$~%
%St.\,Petersburg State University, 7--9, Universitetskaya nab.,
%
%\noindent%
%\hskip2.45mm%
%St.\,Petersburg, 199034, Russian Federation



\noindent%
%$^2$~%
Mytischi Branch of N. E. Bauman Moscow State Technical University,
1, 1st Institutskaya ul., Mytischi,

\noindent%
%\hskip2.45mm%
Moscow region, 141005, Russian Federation

}

%%%%%%%%%%%%%%%%%%%%%%%%%%%%%%%%%%%%%%%%%%%%%%%%%%%%%%%%%%%%%%%%%%

\vskip1.8mm%3mm


\noindent \textbf{For citation:}  Chernyshov A. V. Model for
planning the number of optical disks needed to create and maintain
a long-term electronic archive. {\it Vestnik of Saint~Petersburg
University. Applied Mathematics. Computer Science. Control
Processes},\,\issueyear,
vol.~15,~iss.~\issuenum,~pp.~\pageref{p13}--\pageref{p13e}.\\
\doivyp/\enskip%
\!\!\!spbu10.\issueyear.\issuenum13  (In Russian)

\vskip1.8mm%3mm

{\leftskip=7mm\noindent Development of an analytical model that
allows to estimate the number of optical disks needed annually to
create and maintain a complete electronic archive of long-term
storage of information, as well as the time costs associated with
these processes. An analytical model is developed for archives
based on single disks (disks are not combined into RAID
structures). The model uses the following parameters as input
data: the estimated (design) capacity of the archive; the capacity
of the optical disks used; estimates of the time to write, read
and control the integrity of optical disks; the time allocated for
technical work in the archive (the actual recording of new data,
the control of previously recorded disks, the recovery of failed);
the probability of failure of the optical disk during storage; the
probability of failure of the disk during recording (marriage).
Considered 4 options for the functioning of the archive depending
on the bandwidth of the equipment and describes the stages of each
option. For each option, analytical expressions are obtained that
allow to estimate annually the required number of optical disks
(which largely determines the annual material costs), as well as
the time of changing stages. In addition, for some variants,
expressions are obtained to evaluate the possibility of achieving
the design capacity of the archive.\\[1mm]
\textit{Keywords}: long-term archives of electronic information,
optical discs, ensuring the integrity of information, annual
costs.
\par}

\vskip4mm%5mm

\noindent \textbf{References} }

\vskip 1.8mm

{\footnotesize

1. Verbatim Launches 100GB MDISC Blu-ray Storage Media. {\it
CDRinfo.} Available at: \\
https://cdrinfo.com/d7/content/verbatim-launches-100gb-mdisc-blu-ray-sto
rage-media (created: May 18, 2016; accessed: January 04, 2019).

2. Optical media longevity. {\it The X Lab.} Available at: \\
https://www.thexlab.com/faqs/ opticalmedialongevity.html
(accessed: September 29, 2016).

3. Akimova G. P., Pashkin M. A., Pashkina E. V., Solovyev A. V.
 Elektronnye arkhivy: vozmozhnye resheniya problem
dolgosrochnogo khraneniya dannykh [Electronic archives: possible
solutions to problems of long-term data storage]. {\it Trudy
Instituta sistemnogo analiza Rossiyskoy Akademii nauk
$[$Proceedings of the Institute of system analysis of the Russian
Academy of Sciences$]$}, 2013, vol.~63, no.~4, pp.~39--49. (In
Russian)

4. Smith E. {\it When discs die. Tedium.} Available at: \\
https://tedium.co/2017/02/02/disc-rot-phenomenon/ (created:
February 02, 2017; accessed: November 18, 2017).

5. Malichenko D. A. Evristicheskiy algoritm rascheta razmerov
pamyati v mnogourovnevoy sisteme khraneniya [Heuristic algorithm
for calculating the size of tiers in a hierarchical storage
system]. {\it Informacionno-upravlyayuschie sistemy $[$Information
management systems$]$}, 2015,  no.~5, pp.~100--105.
https://doi.org/10.15217/issn1684-8853.2015.5.100 (In Russian)

6. Zalaev G. Z., Kalenov N. E., Tsvetkova V. A.  Some issues of
long-term storage of electronic documents. {\it Scientific and
Technical Information Processing}, 2016, vol.~43, no.~4,
pp.~268--274. https://doi.org/10.3103/S0147688216040110

7. Lobanov A. K. Metody postroeniya sistem khraneniya dannykh [A
methods of build of data storage systems]. {\it Jet Info Online},
2003, no.~7. Available at: https://citforum.ru/hardware/data/db/
(accessed: May~05, 2016). (In Russian)

8. {\it GOST R 54989--2012 / ISO TR 18492:2005. Obespechenie
dolgovremennoy sokhrannosti elektron\-nykh dokumentov $[$Ensuring
long-term preservation of electronic records$]$.} (Entered to
force May 01, 2013). Moscow, Standartinform Publ., 2013. (In
Russian)

9. Chernyshov A. V. K voprosu o primenenii opticheskikh diskov
dlya sozdaniya dolgovremennykh elektronnykh arkhivnykh
khranilishch informatsii nebol'shikh organizatsiy [To the question
of the optical discs application for long term digital archive
storage of small organizations]. {\it Informacionnye tehnologii
$[$Information technologies$]$}, 2016, vol.~22, no.~8,
pp.~635--640. (In Russian)

10. Chernyshov A. V. Model' nadezhnosti khraneniya informatsii na
sovremennykh bibliotekakh opticheskikh diskov, ob''edinennykh v
massivy RAID--6 [Reliability model for data storage on mo\-dern
libraries of optical disks united in RAID--6]. {\it Vestnik of
Moscow State Technical University im.~N.~E.~Bauman. Series
Instrumentation}, 2017, no.~3, pp.~65--75. \\
https://doi.org/10.18698/0236-3933-2017-3-65-75 (In Russian)

11. Chernyshov A. V. Metod povysheniya nadezhnosti khraneniya
informatsii v dolgovremennykh elektronnykh khranilishchakh na
opticheskikh diskakh, organizovannykh v massivy RAID--6, za schet
smeshivaniya diskov zapasnykh kopiy [Method for improving
information storage reliability in long-term electronic storage on
optical disks arrayed in RAID--6 by mixing backup copy disks].
{\it Vestnik of Moscow State Technical University im.
N.~E.~Bauman. Series Instrumentation}, 2017, no.~4, pp.~88--97.
https://doi.org/10.18698/0236-3933-2017-4-88-97 (In Russian)

12. Chernyshov A. V. Issledovanie svoystv dolgovremennykh
elektronnykh arkhivnykh khranilishch informatsii na opticheskikh
diskakh, organizovannykh v struktury RAID--5 [The study of the
properties of long-term electronic archival information storage on
optical disks organized into a structure of RAID--5].  {\it
Informacionnye tehnologii $[$Information technologies$]$,} 2018,
vol.~24, no.~9, pp.~586--593.
https://doi.org/10.17587/it.24.586-593 (In Russian)

13. Weatherspoon H., Kubiatowicz J. Erasure Coding vs.
Replication: A~quantitative comparison. {\it Peer-to-Peer Systems,
Lecture Notes in Computer Science}, 2002, vol.~2429, pp.~328--337.

14. Thomasian A., Tang Y., Hu Y. Hierarchical RAID: Design,
performance, reliability, and recovery. {\it J. Parallel Distrib.
Comput.}, 2012, vol.~72, pp.~1753--1769.
https://doi.org/10.1016/j.jpdc.2012.07.002

15. Thomasian A. Shortcut method for reliability comparisons in
RAID. {\it Journal of Systems and Software}, 2006, vol.~79,
pp.~1599--1605.

16. Yuan D., Peng X., Liu T., Cui Z. A randomly expandable method
for data layout of Raid Storage Systems. {\it International
Journal of Innovative Computing, Information and Control}, 2018,
vol.~14, no.~3, pp.~1079--1094.

17. Pilipchuk M. I., Balakirev A. N., Dmitrieva L. V.,
Zalaev~G.~Z. {\it Rekomendacii po obespecheniyu sohrannosti
informacii, zapisannoj na opticheskih diskah (Testirovanie
vyborochnogo massiva dokumentov federal'nyh arhivov)
$[$Recommendations to ensure the safety of information recorded on
optical discs (Testing of a sample array of Federal archives
documents)$]$.} Moscow, RGANDT Publ., 2011, 52~p. (In Russian)

18. Jumasheva Ju. Ju. {\it Metodicheskie rekomendacii po
ehlektronnomu kopirovaniyu arhivnyh dokumentov i upravleniyu
poluchennym informacionnym massivom $[$Guidelines for electronic
copying of archival documents and management of the resulting
information array$]$.} Moscow, Rosarhiv, VNIIDAD Publ., 2012,
125~p. (In Russian)

19. {\it Rekomendacii po komplektovaniyu, uchyotu i organizacii
hraneniya ehlektronnyh arhivnyh dokumentov v gosudarstvennyh i
municipal'nyh arhivah $[$Recommendations on acquisition,
accounting and organization of storage of electronic archival
documents in state and municipal archives$]$.} Moscow, Federalnoe
arhives agency, VNIIDAD Publ., 2013, 49~p. (In Russian)

20. {\it Rossiyskie organizatsii nachinayut stroit' elektronnye
arkhivy na opticheskikh diskakh $[$Russian organizations are
beginning to build electronic archives on optical discs$]$.}
Associacija jelektronnyh torgovyh ploshhadok. Available at:
https://www.aetp.ru/market-news/item/400867 (created: November 06,
2015, accessed: November 13, 2015). (In Russian)


\vskip1.5mm Received:  February 06, 2019.

Accepted: November 07, 2019.


\vskip6mm A\,u\,t\,h\,o\,r's \ i\,n\,f\,o\,r\,m\,a\,t\,i\,o\,n:%

\vskip1.5mm%2mm
\textit{Alexandr V. Chernyshov} --- PhD in Technics, Associate
Professor; sch-ru@yandex.ru

}
