
{\normalsize

\vskip 6mm

\noindent{\bf Comparative analysis of calculation methods in
electron
spectroscopy%$\,^*$%
}

}

\vskip 2mm

{\small

\noindent{\it T.~A.~Andreeva$\,^1$%
, M.~E.~Bedrina$\,^1$%
, D.~A.~Ovsyannikov$\,^2$%
%, I.~�. Famylia%$\,^2$%

 }

\vskip 2mm

%%%%%%%%%%%%%%%%%%%%%%%%%%%%%%%%%%%%%%%%%%%%%%%%%%%%%%%%%%%%%%%%%%

%\efootnote{
%%
%\vspace{-3mm}\parindent=7mm
%%
%\vskip 0.1mm $^{*}$ This work is supported by Russian Science
%Foundation (project N 18-71-00006).%\par
%%
%%\vskip 2.0mm
%%
%\indent{\copyright} �����-������������� ���������������
%�����������, \issueyear%
%%
%}

%%%%%%%%%%%%%%%%%%%%%%%%%%%%%%%%%%%%%%%%%%%%%%%%%%%%%%%%%%%%%%%%%%

{\footnotesize


\noindent%
$^1$~%
St.\,Petersburg State University, 7--9, Universitetskaya nab.,
St.\,Petersburg,

\noindent%
\hskip2.45mm%
199034, Russian Federation


\noindent%
$^2$~%
St.\,Petersburg State University of industrial technologies and
design, 18, Bolshaya Morskaya ul.,

\noindent%
\hskip2.45mm%
St.\,Petersburg, 191186, Russian Federation

}

%%%%%%%%%%%%%%%%%%%%%%%%%%%%%%%%%%%%%%%%%%%%%%%%%%%%%%%%%%%%%%%%%%

\vskip3mm

\noindent \textbf{For citation:}  Andreeva T.~A., Bedrina
M.~E.,~Ovsyannikov D.~A. Comparative analysis of calculation
methods in electron spectroscopy. {\it Vestnik of
Saint~Pe\-ters\-burg University. Applied Mathematics. Computer
Science. Control Processes}, \issueyear, vol.~15, iss.~\issuenum,
pp.~\pageref{p8}--\pageref{p8e}. \\
\doivyp/\enskip%
\!\!\!spbu10.\issueyear.\issuenum08  (In Russian)

\vskip3mm

{\leftskip=7mm\noindent The possibilities of the electron density
functional method DFT with hybrid functionals B3LYP and M06-HF
with various basis sets for calculating the electronic spectra of
molecules were analyzed. It was shown that the specific form of
the basis sets 6-31G, cc-PVDZ,\linebreak 6-311 $++$ G ** are not
significantly influence on the value of the long-wave transition
band in the electronic absorption spectrum of 3,6-diamino-N-methyl
phthalimide. The choice of the hybrid potential in the method of
the non-stationary theory of the TD-DFT density\linebreak
functional and especially using CIS configuration interaction
scheme leads to noticeable differences in the calculated values of
the ($\pi-\pi*$)-transition band. For all other transitions, the
changes were not so significant. The electronic spectra of ten
compounds~--- substituted phthalimide were calculated by different
methods using the 6-31G basis set. The structure of a substance
uniquely determines the spectrum pattern. Comparing results of
calculations of these compounds by the TD-DFT method and the CIS
method, which includes single-excited states, we concluded that
the best agreement with the experiment is observed using the CIS
method and
the 6-31G basis set.\\[1mm]
\textit{Keywords}: density functional theory, basis functions,
TD-DFT, CIS, electronic spectra, phthalimides.
\par}

\vskip 5mm

\noindent \textbf{References} }

\vskip 2mm

{\footnotesize

1.  Koch W., Holthausen M.~C.  A chemist's guide to density
functional theory. Second ed. Weinheim, Germany, Wiley-VCH Verlag
GmbH Press, 2001, 306~p.

2.  Yurenev P. V., Shcherbinin A. V., Stepanov N. F. Primenimost'
metodov  TD-DFT dlya rascheta elektronnogo spektra pogloschenia
geksaamminorutenia (II) v vodnom rastvore [The applicability of
\mbox{TD-DFT} methods to calculations of the electronic absorption
spectrum of hexaamminoruthenium (II) in aqueous solution].
\itshape Russian Journal of Physical Chemistry\upshape,  2010,
vol. 84, no.~1, pp. 44--48. (In Russian)

3.  Tawada  Y., Tsuneda  T., Yanagisawa  S.  A
long-range-corrected time-dependent density functional theory.
\itshape  The Journal of Chemical Physics, \upshape 2004, vol.
120, pp. 8425.

4.  Maslov~V. Interpretation of the electronic spectra of
phthalocyanines with transition metals from quantum-chemical
calculations by the density functional method. \itshape  Optics
and Spectroscopy, \upshape 2006, vol. 101, no. 6, pp. 853--861.

5.  Safonov A., Bagaturyants A.,   Sazhnikov  V. Assessment of
TDDFT- and CIS-based methods for calculating fluorescence spectra
of (dibenzoylmethanato)boron difluoride exciplexes with aromatic
hydrocarbons. \itshape Journal of Molecular Modeling, \upshape
2017, vol. 164, no. 23, pp. 164--167.

6.  Kohn W., Becke  A.  D., Parr  R.  G.  Density functional
theory of electronicstructure. \itshape Journal Physical
Chemistry, \upshape 1996, vol. 100, no. 31, pp. 12974--12980.

7.  Kohn W., Sham L.  Self-consistent equations including exchange
and correlation effects. \itshape Physical Review, \upshape 1965,
vol. 140, no. 4A, pp. A1133--A1138.

8.  Perdew J. P., Wang Y.  Accurate and simple analytic
representation of the electron-gas correlation energy. \itshape
Physical Review, \upshape 1992, vol. B45, pp. 13244--13249.

9.  Becke A. D.  Density-functional exchange-energy approximation
with correct asymptotic behaviour. \itshape Physical Review A,
\upshape 1988, vol. 38, pp. 3098--3100.

10.  Lee C., Yang~W., Parr~R.~G.  Development of the
Colle---Salvetti correlation-energy formula into a functional of
the electron density. \itshape  Physical Review B, \upshape 1988,
vol. 37, no. 2, pp. 785--789.

11.  Zhao Y., Truhlar D.  The M06 suite of density functional for
main group thermochemistry, thermochemical kinetics, noncovalent
interactions, excited states, and transition elements: two new
functional and systematic testing of four M06-class functional and
12 other functional. {\it Theoretical Chemistry Accounts}, 2008,
vol. 120, no.~1--3, pp.~215--241.

12.  Zhao Y., Truhlar D. Density functional for spectroscopy: no
long-range self-interaction error, good performance for rydberg
and charge-transfer states, and better performance on average than
B3LYP for ground states. \itshape Journal of Physical Chemistry A,
\upshape 2006, vol. 110, no. 49, pp. 13126--13130.

13.  Adamo C., Barone V.  Toward reliable density functional
methods without adjustable parameters: The PBE0 model. \itshape
Journal of Chemical Physics, \upshape 1999, vol. 110, no. 13, pp.
6158--6170.


14.  Veselova T., Viktorova E.,  Klochkov V., Makushenko A.,
Reznikova I.,  Stolbova~O.~V. Spektral'nye i luminestentnye
svoistva ftalimidov v parah i rastvorah  [Spectral and luminescent
properties of phthalimidesin vapors and solutions]. \itshape
Optics and Spectroscopy, \upshape 1995, vol. 79, no. 1, pp.
60--76. (In Russian)

15.  Andreeva~T.~A., Bedrina~M.~E. Vliyanie gibridnyh potentsialov
metoda DFT na rezul'taty issledovaniya zhidkokristallicheskoy fazy
veschestva  [The influence of hybrid potentials of the DFT method
on the results of study of liquid crystal phase of a substance].
\itshape Vestnik of Saint Petersburg University. Se\-ries~10.
Applied Mathematics. Computer Science. Control processes, \upshape
2015, iss. 1, pp. 16--25. (In Russian)

16.  Andreeva~T.~A., Bedrina~M.~E. Zavisimost' rezul'tatov
rascheta po metodu funktsionala elektronnoi plotnosti ot sposoba
predstavlenia volnovoi funktsii  [Dependence of calculation
results on the density functional theory from the means of
presenting the wave function]. \itshape Vestnik of Saint
Petersburg University. Applied Mathematics. Computer Science.
Control processes, \upshape 2018, vol. 14, iss. 1, pp. 52--59. (In
Russian)

17.  Aristov A. V.,  Semenov S. G. Kvantomehanischeskaia
interpretatsia elektronnyh spektrov aminoftalimidov v gazovoi faze
i v rastvorah [Quantum-chemical interpretation of the electronic
spectra of aminophthalimides in the gaseous phase and in
solution]. \itshape Theoretical and Experimental Chemistry,
\upshape  1984, vol. 28, no. 6, pp. 713--717. (In Russian)


\vskip 1.5mm

Received:  September 09, 2019.

Accepted: November 07, 2019.


\vskip5mm A\,u\,t\,h\,o\,r's \ i\,n\,f\,o\,r\,m\,a\,t\,i\,o\,n:


\vskip2mm \textit{Tatiana A. Andreeva} --- PhD in Physics and
Mathematics, Associated Professor; \text{t.a.andreeva@spbu.ru}

\vskip2mm \textit{Marina E. Bedrina} --- Dr. Sci. in Physics and
Mathematics, Associated Professor; m.bedrina@spbu.ru

\vskip2mm \textit{Dmitry A. Ovsyannikov} --- Student;
\text{d-ovs@yandex.ru}\par
%
}
