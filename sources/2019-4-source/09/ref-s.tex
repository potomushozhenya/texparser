
{\normalsize

\vskip 4.5mm%6mm

\noindent{\bf Preliminary medical diagnostics based on the fuzzy
sets theory\\ using the Sugeno measure%$^{*}$%

}

}

\vskip 1.8mm%2mm

{\small

\noindent{\it A. B. Goncharova%$\,^1$%
%, I.~�. Famylia%$\,^2$%

 }

\vskip 1.8mm%2mm

%%%%%%%%%%%%%%%%%%%%%%%%%%%%%%%%%%%%%%%%%%%%%%%%%%%%%%%%%%%%%%%%%%

%\efootnote{
%%
%\vspace{-3mm}\parindent=7mm
%%
%\vskip 0.1mm $^{*}$ This work is supported by Russian Science
%Foundation (project N 18-71-00006 ).\par
%%
%%\vskip 2.0mm
%%
%%\indent{\copyright} �����-������������� ���������������
%%�����������, \issueyear%
%%
%}

%%%%%%%%%%%%%%%%%%%%%%%%%%%%%%%%%%%%%%%%%%%%%%%%%%%%%%%%%%%%%%%%%%

{\footnotesize

\noindent%
%$^1$~%
St.\,Petersburg State University, 7--9, Universitetskaya nab.,

\noindent%
%\hskip2.45mm%
St.\,Petersburg, 199034, Russian Federation


}

%%%%%%%%%%%%%%%%%%%%%%%%%%%%%%%%%%%%%%%%%%%%%%%%%%%%%%%%%%%%%%%%%%

\vskip1.8mm%3mm


\noindent \textbf{For citation:}   Goncharova A. B. Preliminary
medical diagnostics based on the fuzzy sets theo\-ry using the
Sugeno measure. {\it Vestnik of Saint~Petersburg University.
Applied Mathematics. Com\-puter Science. Control
Processes},\,\issueyear,
vol.~15,~iss.~\issuenum,~pp.~\pageref{p9}--\pageref{p9e}.\\
\doivyp/\enskip%
\!\!\!spbu10.\issueyear.\issuenum09  (In Russian)

\vskip3mm

{\leftskip=7mm\noindent When diagnoses a patient, it is necessary
for a doctor to conduct a comprehensive and systematic study of
the patient in order to collect anamnesis and objectively examine
the patient's condition, to conduct the laboratory, radiographic
and other types of analyses research, as well as to take into
account the presence of other diseases inpatient. Processing a
large amount of information about the patient and correlating this
information with the symptoms of various diseases, not always
relevant to the particular physician specialization is a difficult
task, which can be solved by the diagnostic support system, basing
on the available information. The article proposes a method of
diagnostics for given symptoms based on the theory of fuzzy sets.
For a given set of symptoms expressed, a generalized diagnosis
index is calculated using the Sugeno measure. Using the search for
the maximum of generalized indicators, the most probable diagnosis
is made.\\[1mm]
\textit{Keywords}: fuzzy sets, Sugeno measure, fuzzy set, Sugeno
measure, decision support system, medical diagnostics,
multicriteria choice, composite index.
\par}

\vskip3mm%5mm

\noindent \textbf{References} }

\vskip 1.5mm%2mm

{\footnotesize

1. Bogomolov ~A.\:I., Nevezhin ~V.\:P., Zhdanov ~G.\:A.
Iskusstvennyj intellekt i ekspertnyie sistemy v mobil'noi
meditsine [Artificial intelligence and expert systems in mobile
medicine]. \textit{Chronoeconomics}, 2018, no.~3(11), pp.~17--28.
(In Russian)

2. Rudenko ~T.\:A., Vlasenko ~M.\:A. Fuzzy logic systems in the
diagnosis of myocardial dissynchronization. \textit{Scientific
Journal ``Science Rise''}, 2015,  no.~5, pp.~52--61. \\
https://doi.org/10.15587/2313-8416.2015.43286

3. Toneeva ~D.\:V., Goncharova ~A.\:B. Ekspertnaia sistema
diagnostiki zabolevanyj [Expert system for diagnosing diseases].
\textit{EUROPEAN RESEARCH}. \textit{Collection of articles by the
winners of the VI~International scientific and practical
conference}. Penza, Science and Education Publ., 2016, pp.~34--38.
(In Russian)

4. Zadeh~L.\:A. Fuzzy sets. \textit{Information and Control},
1965, vol.~8,  no.~3,  pp.~338--353.

5. Sugeno~�. \textit{Theory of fuzzy integrals and its
application}. Doct. Thesis. Tokyo, Tokyo Institute of Technology
Publ., 1974, 50~p.

6. Keller~J.\:M., Osborn~J. Training the fuzzy integral.
\textit{International Journal of Approximate Reasoning},  1996,
vol.~15, iss.~1, pp.~1--24.
https://doi.org/10.1016/0888-613X(95)00132-Z

7. Tahani~H., Keller~J.\:M. Information fusion in computer vision
using the fuzzy integ\-ral. \textit{IEEE Trans\-actions on
Systems,
Man and Cybernetics}, 1990, vol.~20(3),  pp.~733--741.\\
https://doi.org/10.1109/21.57289

8. Pavlov~A.\:N., Sokolov~B.\:V. \textit{Priniatie reshenia v
usloviyah nechenkoy informatsii} [\textit{Decision making in the
conditions of fuzzy information}]. Textbook. St. Petersburg,
SPbGUAP Publ., 2006, 72~p. (In Russian)

9. Ropshtein ~A.\:P., Shtovba ~S.\:D. Modelirovanie nadezhnosti
cheloveka-operatora s pomoschiyu nechetkoy bazy znaniy Sugeno
 [The reliability of a human operator using a
fuzzy Sugeno knowledge base]. \textit{Automation and
telemechanics}, 2009, no.~1, pp.~180--187. (In Russian)

10.  Soldatova ~O.\:P., Shepelev ~Yu.\:M. Algoritm minimizatsii
bazy pravil nechetkoy neironnoy seti  Takagi---Sugeno---Kanga
[Algorithm for minimizing the rule base of the fuzzy neural
network Takagi---Sugeno---Kanga]. \textit{EUROPEAN RESEARCH}.
\textit{Collection of articles by the winners of the X
International Scientific and Practical Conference}. Penza, Science
and Education Publ.,  2017, p.~46--49. (In Russian)

11. Pavlov~A.\:N., Sokolov~B.\:V. \textit{Metody obrabotki
ekspertnoy informatsii} [\textit{Methods for processing expert
information}]. Textbook. St. Petersburg. SPbGUAP Publ., 2005,
34~p. (In Russian)

12. Korobova ~L.\:A., Gladkikh ~T.\:V. Razrabotka modeli priniatia
reshenia dlya postanovki diagnoza na osnove nechetkoy logiki
 [Development of a decision
making model for diagnosing diseases based on fuzzy logic].
\textit{Vestnik VGUIT $[$Proceedings of VSUET$]$}, 2018, vol.~80,
no.~4, pp.~80--89. (In Russian)
https://doi.org/10.20914/2310-1202-2018-4-80-89



\vskip1.5mm Received:  September 15, 2019.

Accepted: November 07, 2019.


\vskip4mm%6mm
 A\,u\,t\,h\,o\,r's \ i\,n\,f\,o\,r\,m\,a\,t\,i\,o\,n:%

\vskip1.3mm%2mm
\textit{Anastasiya B. Goncharova} --- PhD in Physics and
Mathematics, Senior Lecturer;\\ a.goncharova@spbu.ru

}
