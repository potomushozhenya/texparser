
{\normalsize

\vskip 4.5mm%6mm

\noindent{\bf On the direction of the steepest descent}

}

\vskip 1.8mm%2mm

{\small

\noindent{\it V.~N.~Malozemov, G.~Sh.~Tamasyan%$\,^1$%
%, I.~�. Famylia%$\,^2$%

 }

\vskip 1.8mm%2mm

%%%%%%%%%%%%%%%%%%%%%%%%%%%%%%%%%%%%%%%%%%%%%%%%%%%%%%%%%%%%%%%%%%

%\efootnote{
%%
%\vspace{-3mm}\parindent=7mm
%%
%\vskip 0.1mm $^{*}$ This work is supported by Russian Science
%Foundation (project N 18-71-00006 ).\par
%%
%%\vskip 2.0mm
%%
%%\indent{\copyright} �����-������������� ���������������
%%�����������, \issueyear%
%%
%}

%%%%%%%%%%%%%%%%%%%%%%%%%%%%%%%%%%%%%%%%%%%%%%%%%%%%%%%%%%%%%%%%%%

{\footnotesize

\noindent%
%$^1$~%
St.\,Petersburg State University, 7--9, Universitetskaya nab.,


\noindent%
%\hskip2.45mm%
St.\,Petersburg, 199034, Russian Federation


}

%%%%%%%%%%%%%%%%%%%%%%%%%%%%%%%%%%%%%%%%%%%%%%%%%%%%%%%%%%%%%%%%%%

\vskip1.8mm%3mm


\noindent \textbf{For citation:}  Malozemov V.~N., Tamasyan G.~Sh.
On the direction of the steepest descent. {\it Vest\-nik of
Saint~Petersburg University. Applied Mathematics. Computer
Science. Control Pro\-cesses}, %\,
\issueyear,
vol.~15,~iss.~\issuenum,~pp.~\pageref{p6}--\pageref{p6e}. %\\
\doivyp/\enskip%
\!\!\!spbu10.\issueyear.\issuenum06  (In Russian)

\vskip3mm

{\leftskip=7mm\noindent The article is dedicated of memory of
Professor V.~F.~Demyanov (1938---2014). The main scientific
interests of V.~F.~Demyanov lay in the field of numerical
optimization methods, where the notion of the direction of the
steepest descent plays an important role. This notion is
introduced for both smooth and nonsmooth functions, both in
constrained and unconstrained cases. This paper provides a
detailed analysis of methods for constructing the direction of
steepest descent. In all cases, it comes down to solving the
quadratic programming problem. Particular attention is paid to
nonsmooth functions, in the study of which V.~F.~Demyanov made a
significant contribution. An example of a function which is
quasidifferentiable at a point is given. This function has two
directions of the steepest descent
and two directions of the steepest ascent.\\[1mm]
\textit{Keywords}: steepest descent direction, nonsmooth analysis,
quasidifferential.
\par}

\vskip5mm

\noindent \textbf{References} }

\vskip 2mm

{\footnotesize

1. Malozemov~V.~N. \textit{Linejnaya algebra bez opredelitelej.
Kvadratichnaya funkciya} [\textit{Linear algebra without
determinants. Quadratic function}]. St. Petersburg, Saint
Petersburg University Publ., 1997, 80~p. (In Russian)

2. Malozemov~V.~N. Metod N'yutona---Rafsona dlya bezuslovnoj
minimizacii [Newton---Raphson method for unconditional
minimization]. {\it Seminar ``CNSA$\&$NDO''. Selected Papers}.
Febr.~14, 2019. Available at:
http://apmath.spbu.ru/cnsa/reps19.shtml\#0214 (accessed: October
13, 2019). (In Russian)

3. Demyanov V.~F., Rubinov A.~M. O kvazidifferenciruemyh
funkcionalah [On quasidiffe\-rentiable functionals]. {\it Papers
of Academy of Science the USSR}, 1980, vol.~250, no.~1,
pp.~21--25. (In Russian)

4. Demyanov V. F., Rubinov A. V. \textit{Osnovy negladkogo analiza
i kvazidifferencial�noe ischislenie} [\textit{Elements of
nonsmooth analysis and quasidifferential calculus}]. Moscow, Nauka
Publ., 1990, 432~p. (In Russian)

5.  Demyanov V. F. \textit{Minimaks: Differentsiruemost� po
napravleniyam} [\textit{Minimax: Directional
dif\-fe\-ren\-tiabi\-lity}]. Leningrad, Leningrad University
Publ., 1974, 112~p. (In Russian)



\vskip1.5mm Received:  November 01, 2019.

Accepted: November 07, 2019.


\vskip6mm A\,u\,t\,h\,o\,r's \ i\,n\,f\,o\,r\,m\,a\,t\,i\,o\,n:%

\vskip1.5mm%2mm
\textit{Vassili N. Malozemov}~--- Dr. Sci. in Physics and
Mathematics, Professor;  v.malozemov@spbu.ru

\vskip1.5mm%2mm
\textit{Grigoriy Sh. Tamasyan}~--- PhD in Physics and Mathematics,
Associate Professor; g.tamasyan@spbu.ru

}
