

{\small



\vskip6mm

\noindent \textbf{References} }

\vskip 2mm

{\footnotesize

1. Lyapunov A.\,M.  \textit{Obshaya            %1
zadacha ob ustojchivosti dvizheniya} [\textit{The general problem
of the stability of
motion}]. Moscow, Gostekhizdat Publ., 1950, 472~p. (In Russian) %

2. Ekimov A.\,V., Balykina Yu.\,E.,  Svirkin     %5
M.\,V. Analysis of attainability sets of bilinear control systems.
\textit{AIP Conference Proceedings (ICNAAM 2016)}. Rodos, American
Insti\-tu\-te of Physics Publ. LLC, 2017, vol.~1863, no.~170012.

3. Ermolin V.\,S. Value sets of the             %6
discrete interval length in the problem of discrete stabilization.
\textit{Avtomatika}, 1995, no.\,3, pp.\,15--21.

4. Ermolin V.\,S., Vlasova T.\,V.          %7
Identification of the domain of attraction. \textit{Proceedings
of~SCP~2015 Conference}. St.~Petersburg, IEEE Publ., 2015,
pp.~9--12.

5. Zubov A.\,V. Stabilization of program        %4
motion and kinematic trajectories in dynamic systems in case of
systems of direct and indirect control. \textit{Automation and
Remote Control}, 2007, vol.~68, no.~3, pp.\,386--398.

6. Zubov\,V.\,I. \textit{Kolebaniya v          %2
nelinejnyh i upravlyaemyh sistemah} [\textit{Fluctuations in
nonlinear and control systems}]. Leningrad, Sudpromgiz Publ.,
1962, 632~p. (In Russian)

7. Zubov V.\,I. \textit{Lekcii po teorii        %3
upravleniya} [\textit{Lectures on the theory of control}]. 2nd ed.
St. Petersburg, Lan's Publ., 2009, 496~p. (In Russian)

8. Kozlov V.\,V., Furta S.\,D. Lyapunows first         %8
method for strongly non-linear systems. \textit{Journal~of~Applied
Mathematics and Mechanics}, 1996, vol.~60, no.~1, pp.~7--18.

9. Chetaev\,N.\,G.\,\textit{Ustojchivost'           %9
dvizheniya}\,[\textit{The\,stability of
motion}].\,Moscow,\,Gostekhizdat Publ., 1955, 176~p. (In Russian)

10. Malkin I.\,G.  \textit{Teoriya                  %10
ustojchivosti dvizheniya} [\textit{Theory of stability of
motion}]. Moscow, Nauka Publ., 1966, 530~p. (In Russian)

11. Bylov B.\,F., Vinograd R.\,E., Grobman D.\,M.,        %11
Nemyckij V.\,V. \textit{Teoriya pokazateley Lyapunova i ejo
prilozheniya k voprosam ustojchivosti} [\textit{The theory of
Lyapunov char\-acter\-is\-tic numbers and their application to the
theory of stability}]. Moscow, Nauka Publ., 1966, 576~p. (In
Russian)

12. Demidovich B.\,P. \textit{Lekcii po              %12
matematicheskoy teorii ustojchivosti} [\textit{Lectures on the
ma\-the\-ma\-ti\-cal theory of stability}]. St. Petersburg, Lan's
Publ., 2008, 480~p. (In Russian)

13. Adami T.\,M., Best E., Zhu J.\,J. Stability          %13
assessment using Lyapunov's first method.   \textit{Proceedings of
the Annual Southeastern Symposium on System Theory}. Huntsville,
IEEE Publ., 2002, pp.~297--301.

14. Masarati P., Tamer A. Sensitivity of              %14
trajectory stability estimated by Lyapunov char\-acter\-is\-tic
exponents. \textit{Aerospace Science and Technology}, 2015,
vol.~47, pp.~501--510.

15. Oseledets V.\,I. A multiplicative ergodic        %15
theorem. Lyapunov characteristic numbers for~dynamical systems.
\textit{Trans. Moscow Mathematical Society Journal}, 1968,
vol.~19, pp.~197--231.

16. Cencini M., Ginelli F. Lyapunov              %16
analysis: from dynamical systems theory to appli\-ca\-tions.
\textit{Journal\,of Physics A: Mathematical and Theoretical},
2013, vol.~46, no.~250301.

17. Young L.-S. Mathematical theory of Lyapunov            %17
exponents. \textit{Journal\,of Physics A: Mathe\-ma\-ti\-cal and
Theoretical}, 2013, vol.~46, no.~254001.

18. Ermolin\,V.\,S. Invariantniye                    %18
preobrazovaniya v pervom metode Lyapunova [Invariant
trans\-for\-ma\-tions in Lyapunov's first method]. \textit{Vestnik
of Saint\,Petersburg University. Series\,10. Applied
Ma\-the\-ma\-tics. Computer Science. Control Processes}, 2014,
iss.~2, pp.~36--48. (In~Russian)

19. Ermolin\,V.\,S., Vlasova\,T.\,V. A         %19
group of invariant transformations in the stability prob\-lem via
Lyapunov's  first  method.  \textit{Proceedings  of  ICCTPEA 2014
 Conference}. St.\,Petersburg, IEEE Publ., 2014, pp.~48--49.


\vskip 1.5mm

%\noindent Recommendation: prof. L. A. Petrosyan.
%
%\vskip 1.5mm

%\noindent

Received:  February 01, 2019.

Accepted: November 07, 2019.

\vskip 4.5mm%6mm
A\,u\,t\,h\,o\,r's \ i\,n\,f\,o\,r\,m\,a\,t\,i\,o\,n:

\vskip 1.5mm%2mm
\textit{Vladislav S. Ermolin} --- PhD in Physics and Mathematics,
Associate Professor; vse40@mail.ru

\vskip 1.5mm%2mm
\textit{Tatyana V. Vlasova} --- PhD in Physics and Mathematics,
Associate Professor; t.vlasova@spbu.ru

}
