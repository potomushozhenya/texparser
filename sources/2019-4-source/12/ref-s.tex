
{\normalsize

\vskip 6mm

\noindent{\bf On the uniqueness of the solution to the problem\\
of determining the parameters of mechanical models\\ of the human
body exposed to vibration}

}

\vskip 2mm

{\small

\noindent{\it V. P. Tregubov, N. K. Egorova%$\,^1$%
%, I.~�. Famylia%$\,^2$%

 }

\vskip 2mm

%%%%%%%%%%%%%%%%%%%%%%%%%%%%%%%%%%%%%%%%%%%%%%%%%%%%%%%%%%%%%%%%%%

%\efootnote{
%%
%\vspace{-3mm}\parindent=7mm
%%
%\vskip 0.1mm $^{*}$ This work is supported by Russian Science
%Foundation (project N 18-71-00006 ).\par
%%
%%\vskip 2.0mm
%%
%%\indent{\copyright} �����-������������� ���������������
%%�����������, \issueyear%
%%
%}

%%%%%%%%%%%%%%%%%%%%%%%%%%%%%%%%%%%%%%%%%%%%%%%%%%%%%%%%%%%%%%%%%%

{\footnotesize

\noindent%
%$^1$~%
St.\,Petersburg State University, 7--9, Universitetskaya nab.,


\noindent%
%\hskip2.45mm%
St.\,Petersburg, 199034, Russian Federation


}

%%%%%%%%%%%%%%%%%%%%%%%%%%%%%%%%%%%%%%%%%%%%%%%%%%%%%%%%%%%%%%%%%%

\vskip3mm


\noindent \textbf{For citation:}  Tregubov~V.  P., Egorova~N. K.
On the uniqueness of the solution to the problem of determining
the parameters of mechanical models of the human body exposed to
vibration. {\it Vestnik of Saint~Petersburg University. Applied
Mathematics. Computer Science. Control Processes},\,\issueyear,
vol.~15,~iss.~\issuenum,~pp.~\pageref{p12}--\pageref{p12e}. %\\
\doivyp/\enskip%
\!\!\!spbu10.\issueyear.\issuenum12  (In Russian)

\vskip3mm

{\leftskip=7mm\noindent The article presents an analysis of
experimental studies of the human body behavior under vibration.
The results of these studies are the basis for the construction of
the human body mechanical models, which are used in the
construction of vibration protection systems. The main attention
is paid to solving the problem of determining the mechanical model
parameters and solving the problem of the uniqueness for this
problem solution. Deciding on a single set of model parameter
values is crucial to creating a vibration protection system.
Unfortunately, this problem has not been investigated before. In
this regard, the possibilities of obtaining a unique solution
having available the experimental seat-to-head transmissibility or
modulus of the input mechanical impedance have been studied. On
the example of mechanical models with two degrees of freedom, it
has been shown that in general this problem has no unique
solution. In addition, for each type of experimental results it
was established what additional experimental data were you need to
get for the problem to have a single solution. In particular, it
is necessary to know not only the transmission from the seat to
the head but also the transmission from the seat to each part of
the model.\\[1mm]
\textit{Keywords}: transfer function, seat-to-head
transmissibility, input mechanical impedance, mechanical model,
human body.
\par}

\vskip5mm

\noindent \textbf{References} }

\vskip 2mm

{\footnotesize

1.  Von Bekesy G. Ueber die Empfindlichkeit des stehenden und
sitzenedn Menschen gegen sinusfoermige Erschuetterungen. {\it
Akustische Zeitschrift}, 1939, vol.~4, pp.~360--369.

2.  Coermann R. R. The mechanical impedance of the human body in
sitting and standing position at low frequencies. {\it Human
factor}, 1962, vol.~4, pp.~227--253.

3. Potemkin B. A., Frolov K. V. Postroenie dinamitheskoy modeli
tela theloveka-operatora, podverzhennogo deistviju shirokopolosnih
sluthainih vibratsiy [Construction of dynamic model of human body
for man-operator exposed to broadband random vibra\-tions].{\it
Vibration-isolation of machine and vibration-protection of
man-operator}. By otv. red. K.~V.~Frolov. Moscow, Nauka Publ.,
1973, pp.~17--30. (In Russian)

4.  Griffin M. J. Vertical vibration of seated subjects: effects
of posture, vibration level, and frequency. {\it Aviation Space
and Environmental Medicine}, 1975, vol.~46, pp.~269--276.

5.  Paddan G. S., Griffin M. J. The transmission of translational
seat vibration to the head --- I. Vertical seat vibration. {\it
Journal of Biomechanics}, 1988, vol.~21, iss.~3, pp.~191--197.

6.  Fairley T. E., Griffin M. J. The apparent mass of the seated
human body. Vertical vibration.  {\it Journal of Biomechanics},
1989, vol.~22, no.~2, pp.~81--94.

7.  Glukharev K. K., Potemkin B. A., Frolov K. V., Sirenko V. N.
Elementi nelineynoy teorii kolebaniy v analize biomehaniki tela
theloveka [Elements of nonlinear theory oscillations in the
analysis\linebreak\newpage\noindent of human body biomechanics].
{\it Vibration-protection of man-operator and the issues of
modeling}. By otv. red. K.~V.~Frolov. Moscow, Nauka Publ., 1973,
pp.~17--30. (In Russian)

8.  Demic~M.\:S., Lukic~J.\:K.  Human body under two-directional
random vibration. {\it Journal of Low Frequency Noise Vibration
and Active Control}, 2008, vol.~27, no.~3. pp.~185--201.

9.  Qiu Y., Griffin M. J. Biodynamic response of the seated human
body to single-axis and dual-axis vibration: effect of backrest
and non-linearity. {\it Industrial Health}, 2012, vol.~50, no.~1,
pp.~37--51.

10. Muksian~R., Nash~C.\:D. On frequency-depended damping
coefficient in lumped-parameter models of human being. {\it
Journal of Biomechanics}, 1976, vol.~9, pp.~339--342.

11. Wan~Y., Schimmels~J.\:M. A simple model that captures the
essential dynamics of a seated human exposed to whole body
vibration. {\it Advances in Bioengineering}, 1995, vol.~31,
ASME-publ. BED (Bioengineering Division), pp.~333--334.

12. Liu~C., Qiu~Y., Griffin~M.\:J. Dynamic forces over the
interface between a seated human body and a rigid seat during
vertical whole-body vibration. {\it Journal of Biomecha\-nics},
2017, vol.~61, no.~16, pp.~176--182.

13. Bai Xian-Xu, Xu Shi-Xu, Cheng Wei, Qian Li-Jun. On
4-degree-of-freedom biodynamic models of seated occupants:
Lumped-parameter modeling. {\it Journal of Sound and Vibra\-tion},
2017, vol.~402, no.~18, pp.~122--141.

14. Abbas~W., Abouelatta~O.\:B., El-Azab~M., Elsaidy~M.,
Megahed~A. Optimization of biodynamic seated human models using
genetic algorithms. {\it Engineering}, 2010, vol.~2, pp.~710--719.

15.  Boileau~P.\:E., Rakheja~S. Whole-body vertical biodynamic
response characteristics of the seated vehicle driver: measurement
and model development. {\it International Journal of Ind.
Ergonomics}, 1998, vol.~22, pp.~449--472.

16. Zhang~E.,\, \,Xu~L.\:A.,\, Liu~Z.\:H.,\,\, Li~X.\:L.\,\,
Dynamic \,modeling \,and \,vibration\, characteristics
 \,of multi-DOF\, upper\,part\, system\, of \,seated \, human body.\,
 {\it Chine Journal of Engineering Design}, 2008, vol.~15, pp.~244--249.



\vskip1.5mm Received:  August 07, 2019.

Accepted: November 07, 2019.


\vskip6mm A\,u\,t\,h\,o\,r's \ i\,n\,f\,o\,r\,m\,a\,t\,i\,o\,n:%


\vskip2mm \textit{Vladimir P. Tregubov} --- Dr. Sci. in Physics
and Mathematics, Professor; v.tregubov@spbu.ru

\vskip2mm \textit{Nadezhda K. Egorova} --- Postgraduate Student;
nadezhda\_ego@mail.ru

}
