
{\normalsize

\vskip 4.5%6
mm

\noindent{\bf Mathematical model of joint optimization of programmed\\ and perturbed motions in discrete systems%$^*$%
}

}

\vskip 1.5%2
mm

{\small

\noindent{\it E. D. Kotina, D. A. Ovsyannikov%$^{2}$%
%, I.~�. Famylia%$~^2$%

 }

\vskip 2mm

%%%%%%%%%%%%%%%%%%%%%%%%%%%%%%%%%%%%%%%%%%%%%%%%%%%%%%%%%%%%%%%%%%

%\efootnote{
%%
%%\vspace{-3mm}%
%\parindent=7mm
%%
%%\vskip 0.0mm
%\hskip 4mm$^{*}$ This work was supported by the Russian Science% %Foundation (project N~19-11-00223).%\par
%%
%%\vskip 2.0mm
%%
%%\indent{\copyright} �����-������������� ���������������
%%�����������, \issueyear%
%%
%}

%%%%%%%%%%%%%%%%%%%%%%%%%%%%%%%%%%%%%%%%%%%%%%%%%%%%%%%%%%%%%%%%%%

{\footnotesize



\noindent%
%$^2$~%
St.~Petersburg State University, 7--9, Universitetskaya nab.,
St.~Petersburg,

\noindent%
%\hskip2.45mm%
199034, Russian Federation

}

%%%%%%%%%%%%%%%%%%%%%%%%%%%%%%%%%%%%%%%%%%%%%%%%%%%%%%%%%%%%%%%%%%

\vskip2.0mm%3mm

\noindent \textbf{For citation:} Kotina E. D., Ovsyannikov D. A. Mathematical model of joint optimization of programmed and perturbed motions in discrete systems. {\it Vestnik of Saint~Peters\-burg
Uni\-versity. Applied Mathematics. Computer Science. Control
Processes}, \issueyear,
vol.~17, iss.~\issuenum, pp.~\pageref{p10}--\pageref{p10e}. %\\
\doivyp/\enskip%
\!\!\!spbu10.\issueyear.\issuenum10  (In Russian)

\vskip2.0%3
mm

{\leftskip=7mm\noindent A new mathematical model for the optimization of discrete systems is constructed in the article. The program motion and the ensemble (beam) of perturbed motions are investigated. In this case, the authors consider the joint optimization of smooth and non-smooth functionals defined on the program and perturbed motions. The variation of the functional and the necessary optimality conditions are provided. The developed mathematical technique allows solving non-standard control and optimization problems in various fields of science and technology.\\[1mm]
\textit{Keywords}: discrete systems, functional variation, smooth and non-smooth functionals, optimization, optimal control.
\par}

\vskip 4.5%6
mm

\noindent \textbf{References} }

\vskip 1.8%2
mm

{\footnotesize

1. Propoy A. I.  \textit{Elementy teorii optimal'nykh diskretnykh protsessov} [\textit{Elements of the theory of optimal discrete processes}]. Moscow, Nauka Publ., 1973, 255~p. (In Russian)

2. Ovsyannikov D. A.  \textit{Matematicheskie metody upravleniya puchkami} [\textit{Mathematical methods of beam control}]. Leningrad, Leningrad University Press, 1980, 228~p. (In Russian)

3.  Ovsyannikov D. A.  \textit{Modelirovanie i optimizaciya dinamiki puchkov zaryazhennyh chastic} [\textit{Modeling and optimization of charged particle beam dynamics}]. Leningrad, Leningrad University Press, 1990, 312~p. (In Russian)

4.  Kotina E. D., Ovsyannikov A. D. On simultaneous optimization of programmed and perturbed motions in discrete systems. \textit{Proceedings of 11th International IFAC Workshop on Control Applications of Optimization (CAO 2000)}, 2000, vol.~1, pp.~183--185.

5.  Kotina E. D.  Control discrete systems and their applications to beam dynamics optimization.  \textit{Proceedings of International Conference Physics and Control (PhysCon 2003)}, 2003, vol.~3, pp.~997--1002, no.~1237041.

6.  Kotina E. D.  Discrete optimization problem in beam dynamics. \textit{Nuclear Instruments and Methods in Physics Research. Section A. Accelerators, Spectrometers, Detectors and Associated Equipment}, 2006, vol.~558, iss.~1, pp.~292--294.

7.  Kurzhanski A. B.  \textit{Upravlenie i nabludenie v usloviah neopredelennosti} [\textit{Control in case of uncertainty}]. Moscow, Fizmatlit Publ., 1977, 394~p. (In Russian)

8.  Bondarev B. I., Durkin A. P., Ovsyannikov A. D.  New mathematical optimization models for RFQ structures. \textit{Proceedings of 1999 Particle Accelerator Conference}. New York, 1999, pp.~2808--2810.

9. Ovsyannikov A. D.  Upravlenie puchkom zarjazhennih chastist s uchetom ih vzaimodeystvia [Control of charged particles beam with consideration of their interaction]. \textit{Vestnik of Saint Petersburg University. Series 10. Applied Mathematics. Computer Science. Control Processes}, 2009, iss.~2, pp.~82--92. (In Russian)

10.  Ovsyannikov A. D.  \textit{Matematicheskie modeli optimizatsii dinamiki puchkov} [\textit{Mathematical models of beam dynamics optimization}]. Saint~Pe\-ters\-burg, VVM Publ., 2014, 181~p. (In Russian)

11.  Mizintseva M., Ovsyannikov D.  On the minimax problem of beam dynamics optimization. \textit{Proceedings of the 25th Russian Particle Accelerator Conference (RuPAC 2016)}, 2016, pp.~360--362.

12.  Mizintseva M., Ovsyannikov D.  Minimax problem of simultaneous optimization of smooth and non-smooth functionals. \textit{Proceedings of 2017 Constructive non-smooth analysis and related topics (dedicated to the memory of V. F. Demyanov)}, IEEE, 2017, pp.~1--4.

13.  Ovsyannikov D. A., Mizintseva M. A., Ovsyannikov A. D.  Joint optimization of smooth and non-smooth functionals on beams of trajectories. \textit{Proceedings of the International Conference Optimal Control and Differential Games (dedicated to the 110th anniversary of L.~S.~Pontryagin)}, 2018, pp.~203--205.

14.  Ovsyannikov D. A., Mizintseva M. A., Balabanov M. Yu., Durkin A. P., Edamenko N. S., Kotina E. D., Ovsyannikov A. D.  Optimizatsiya dinamiki puchkov trayektoriy s ispol'zovaniyem gladkikh i negladkikh funktsionalov. Ch. 1 [Optimization of dynamics of trajectory bundles using smooth and non-smooth functionals. Pt 1]. \textit{Vestnik of Saint Petersburg University. Applied  Mathematics. Computer Science. Control Processes}, 2020, vol.~16, iss.~1, pp.~73--84. \\ 
https://doi.org/10.21638/11702/spbu10.2020.107 (In Russian)

15.  Demyanov V. F.  \textit{Minimaks: differentsiruyemost' po napravleniyam} [\textit{Minimax: directional dif\-fe\-rentia\-bi\-lity}]. Leningrad, Leningrad University Press, 1974, 112~p. (In Russian)

16.  Kotina �. D.  K teorii opredeleniya polya peremeshchenij na osnove uravneniya perenosa v diskretnom sluchae [On the theory of determining displacement field on the base of transfer equation in discrete case]. \textit{Vestnik of Saint Petersburg University. Series 10. Applied Mathematics. Computer Science. Control Processes}, 2010, iss.~3,
pp.~38--43. (In Russian)

17.  Kotina E. D., Leonova E. B., Ploskikh V. A.  Obrabotka radionuklidnykh izobrazheniy s is\-pol'zovaniyem diskretnykh sistem [Radionuclide images processing with the use of discrete systems]. \textit{Vestnik of Saint Petersburg University. Applied Mathematics. Computer Science. Control Processes}, 2019, vol.~15, iss.~4, pp.~544--554. %\\
https://doi.org/10.21638/11702/spbu10.2019.410 (In Russian)

18.  Golovkina A., Ovsyannikov D., Olaru S. Performance optimization of radioactive waste transmutation in accelerator driven system. \textit{Cybernetics and Physics}, 2018, vol.~7, no.~4, pp.~210--215.

19.  Kluchevskaia Yu. D., Polozov S. M. Beam dynamics simulation in a linear accelerator for Cern future circular collider. \textit{Cybernetics and Physics}, 2020, vol.~9, no.~2, pp.~98--102.

20.  Kurzhanski A. B., Varaiya P.  Optimization of output feedback control under set-membership uncertainty. \textit{Journal of Optimization Theory and Applications}, 2011, vol.~151, no.~1, pp.~11--32.

21.  Bortakovskii A. S.  Teorema razdeleniya v zadachakh upravleniya puchkami trayektoriy de\-terminirovannykh lineynykh pereklyuchayemykh system [Separation theorem in control problems of beam trajectories of deterministic linear switched systems]. \textit{Izvestiya Rossiyskoy akademii nauk. Teoriya i sistemy upravleniya} [\textit{Proceedings of the Russian Academy of Sciences. Theory and control systems}], 2020, no.~2, pp.~37--63. (In Russian)

22.  Bortakovskii A. S., Nemychenkov G. I.  Suboptimal control of bunches of trajectories of discrete deterministic automaton time-invariant systems. \textit{Journal of Computer and Systems Sciences International}, 2017, vol.~56, no.~6, pp.~914--929.

23.  Panteleev A. V., Pis�mennaya V. A.  Application of a memetic algorithm for the optimal control of  bunches of trajectories of nonlinear deterministic systems with incomplete feedback. \textit{Journal of Computer and Systems Sciences International}, 2018, vol.~57, no.~1, pp.~25--36.

\vskip 1.5mm

Received:  October 18, 2020.

Accepted: April 05, 2021.


\vskip6 mm A~u~t~h~o~r~s' \ i~n~f~o~r~m~a~t~i~o~n:


\vskip2 mm \textit{Elena D. Kotina} --- Dr. Sci. in Physics and Mathematics, Professor; e.kotina@spbu.ru

\vskip2 mm \textit{Dmitri A. Ovsyannikov} --- Dr. Sci. in Physics and Mathematics, Professor; d.a.ovsyannikov@spbu.ru\par
%
}
