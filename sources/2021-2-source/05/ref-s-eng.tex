

{\small



\vskip6 mm

\noindent \textbf{References} }

\vskip 2 mm

{\footnotesize



%\bibitem{DurieuWalterPolyak}
1. Durieu C., Walter E., Polyak B. Multi-input multi-output ellipsoidal state bounding. \emph{Journal of Optimization Theory and
Applications}, 2001, vol.~111, no.~2, pp.~273--303.

%\bibitem{Harrison}
2. Harrison G. W. Dynamic models with uncertain parameters. \emph{Proceedings of the First International Conference on Mathematical Modeling}.  Ed. by X. J. R. Avula. Rolla, University of Missouri Publ., 1977, vol.~1, pp.~295--304.

%\bibitem{MaksarovNorton}
3. Maksarov D. G., Norton J. P.  Computationally efficient algorithms for state estimation with ellipsoidal approximation. \emph{International Journal of Adaptive Control and Signal Processing}, 2002, vol.~16, no.~6, pp.~411--434.

%\bibitem{Jaulin}
4. Jaulin L. Nonlinear bounded-error state estimation of continuous-time systems. \emph{Automatica}, 2002, vol.~38, no.~6, pp.~1079--1082.

%\bibitem{ScottBarton}
5. Scott J. K., Barton P. I. {Bounds on the reachable sets of nonlinear control systems}. \emph{Automatica}, 2013, vol.~49, pp.~93--100.

%\bibitem{SingerBarton}
6. Singer A. B., Barton P. I. {Bounding the solutions of parameter dependent nonlinear ordinary differential equations}. \emph{SIAM J. Sci. Comput.}, 2006, vol.~27, no.~6, pp.~2167--2182.

%\bibitem{KiefferWalterSimeonov}
7. Kieffer M., Walter E., Simeonov I. Guaranteed nonlinear parameter
estimation for continuous-time dynamical models. \emph{Proceedings of 14th IFAC Symposium on System Identification}, 2006, pp.~843--848.

%\bibitem{DuboisPrade}
8. Dubois D., Prade H. Towards fuzzy differential calculus. Pt 3. Differentiation. \emph{Fuzzy Sets and Systems}, 1982, no.~8, pp.~225--233.

%\bibitem{ChangZadeh}
9. Chang S. L., Zadeh L. A. On fuzzy mapping and control. \emph{IEEE Trans. Systems Man Cybernet}, 1972, vol.~2, pp.~30--34.

%\bibitem{GoetschelVoxman}
10. Goetschel R., Voxman W. {Elementary fuzzy calculus}. \emph{Fuzzy Sets and Systems}, 1987, no.~24, pp.~31--43.

%\bibitem{KandelByatt}
11. Kandel A., Byatt W. J. {Fuzzy differential equations}. \emph{Proceedings of lnternational Conference on Cybernetics and Society}, 1978, pp.~1213--1216.

%\bibitem{Kaleva}
12. Kaleva O. The Cauchy problem for fuzzy differential equations. \emph{Fuzzy Sets and Systems}, 1990, no.~35, pp.~389--396.

%\bibitem{StefaniniBede}
13. Stefanini L., Bede B. Generalized Hukuhara differentiability of interval-valued functions and interval differential equations. \emph {Nonlinear Analysis}, 2009, vol.~71, pp.~1311--1328.

%\bibitem{MaFriedmanKandel}
14. Ma M., Friedman M., Kandel A. Numerical solutions of fuzzy differential equations. \emph{Fuzzy Sets and Systems}, 1990, no.~105, pp.~133--138.

%\bibitem{Pederson}
15. Pederson S., Sambandham M. Numerical solution to hybrid fuzzy systems. \emph{Mathematical and Computer Modelling}, 2007, vol.~45, pp.~113--144.

%\bibitem{Jayakumar}
16. Jayakumar T., Maheskumar D., Kanagarajan K. Numerical solution of fuzzy differential equations by Runge\,---\,Kutta method of order five. \emph{Applied Mathematical Sciences}, 2012, vol.~6, pp.~2989--3002.

%\bibitem{BandyopadhyayKar}
17. Bandyopadhyay A., Kar K. Fuzzy continuous dynamical system: a multivariate optimization technique. \emph{Annual meeting of the North American Fuzzy Information Processing Society}. Berkeley, CA, USA, 2007, pp.~1--7.

%\bibitem{AbbasbandyViranloo}
18. Abbasbandy S., Viranloo T. A. Numerical solutions of fuzzy differential equations by Taylor method. \emph{Computational Methods in Applied Mathematics}, 2002, vol.~2~(2), pp.~113--124.

%\bibitem{MorozovReviznikov}
19. Morozov A. Yu., Reviznikov D. L. Modifikacija metodov reshenija zadachi Koshi dlja sistem obyknovennyh differencial'nyh uravnenij s interval'nymi parametrami [Modification of methods for solving the Cauchy problem for systems of ordinary differential equations with interval parameters]. \emph{Proceedings of MAI Institute}, 2016, vol.~89, pp.~1--17. (In Russian)

%\bibitem{HashemiMalekinagadMarasi}
20. Hashemi M. S., Malekinagad J., Marasi H. R. Series solution of the system of fuzzy differential equations. \emph{Advances in Fuzzy Systems}, 2012, pp.~1--16.

%\bibitem{Liao}
21. Liao S. Homotopy analysis method: a new analytical technique for nonlinear problems. \emph{Communications in Nonlinear Science and Numerical Simulation}, 1997, no.~2, pp.~95--100.

%\bibitem{MoslehOtadi}
22. Mosleh M., Otadi M. Minimal solution of fuzzy linear system of differential equations. \emph{Neural Computing $\&$ Applications}, 2012, vol.~21, pp.~329--336.

%\bibitem{WuZhangChenLuo}
23. Wu J., Zhang Yu., Chen L., Luo Zh. A Chebyshev interval method for nonlinear dynamic systems under uncertainty. \emph{Applied Mathematical Modelling}, 2013, vol.~37, pp.~4578--4591.

%\bibitem{ChenWuXueLiu}
24. Chen M., Wu C., Xue X., Liu G. On fuzzy boundary value problems under uncertainty. \emph{Information Sciences}, 2008, no.~178, pp.~1877--1892.

%\bibitem{SokolovaKuzmina}
25. Sokolova S. P., Kuzmina E. A. Dinamicheskie svojstva interval'nyh sistem [Dynamic properties of interval systems]. \emph{SPIIRAS Proceedings}, 2008, vol.~7, pp.~215--221. (In Russian)

%\bibitem{Rogoza}
26. Rogoza A. A. Ob odnom podhode k postroeniju dvustoronnih ocenok mnozhestv reshenij nelinejnyh differencial'nyh uravnenij s interval'nymi parametrami na osnove proekcionnyh metodov [On an approach to constructing two-sided estimates of solution sets of non-linear differential equations with interval parameters on the basis of projective methods]. \emph{Differential Equations and Control Processes}, 2015, vol.~2, pp.~19--58. (In Russian)

%\bibitem{DobronecStrelnikova}
27. Dobronec B. S., Strelnikova E. A. Reshenija kraevyh zadach dlja ODU s interval'nymi neo\-pre\-de\-lennostjami [Solutions of boundary value problems for ordinary differential equations with interval un\-cer\-tain\-ties]. \emph{Computing Technologies}, 2005, vol.~10, pp.~16--21. (In Russian)

%\bibitem{GasilovFatullayevAmrahovb}
28. Gasilov N. A., Fatullayev A. G., Amrahovb S. E. Solution method for a non-homogeneous fuzzy linear system of differential equations. \emph{Applied Soft Computing}, 2018, vol.~70, pp.~225--237.

%\bibitem{Hullermeier}
29. Hullermeier E. An approach to modelling and simulation of uncertain dynamical systems. \emph{International Journal of Uncertainty, Fuzziness and Knowledge-Based Systems}, 1997, vol.~5, no.~2, pp.~117--137.

%\bibitem{Hanss}
30. Hanss M. The transformation method for the simulation and analysis of systems with uncertain parameters. \emph{Fuzzy Sets and Systems}, 2002, no.~130, pp.~277--289.

%\bibitem{Fominyh}
31. Fominyh A. V. A numerical method for finding the optimal solution of a differential inclusion. \emph{Vestnik St. Petersburg University. Mathematics}, 2018, no.~51, pp.~397--406.

%\bibitem{BlagodatskihFilippov}
32. Blagodatskih V. I., Filippov A. F. Differencial'nye vkljuchenija i optimal'noe upravlenie [Dif\-fe\-ren\-tial inclusions and optimal control]. \emph{Proceedings of the Steklov Institute of Mathematics}, 1985, vol.~169, pp.~194--252. (In Russian)

%\bibitem{Blagodatskih}
33. Blagodatskih V. I. \emph{Vvedenie v optimal'noe upravlenie $[$Introduction to optimal control$]$}. Moscow, Vysshaya shkola Publ., 2001, 239~p. (In Russian)


%\bibitem{BonnansShapiro}
34. Bonnans J. F., Shapiro A. \emph{Perturbation analysis of optimization problems}. New York, Springer Science+Business Media Press, 2000, 601~p.

%\bibitem{KantorovichAkilov}
35. Kantorovich~L.~V., Akilov~G.~P. \emph{ Funkcional'nyj analiz $[$Functional analysis$]$.} Moscow, Nauka Publ., 1977, 752~p. (In Russian)

%\bibitem{Penot}
36. Penot J. P. On the convergence of descent algorithms. \emph{Computational Optimization and Applications}, 2002, vol.~23, no.~3, pp.~279--284.

%\bibitem{DolgFom}
37. Dolgopolik M. V., Fominyh A. V. Exact penalty functions for optimal control problems. I: Main theorem and free-endpoint problems. \emph{Optimal Control Applications $\&$ Methods}, 2019, vol.~40, no.~6, pp.~1018--1044.



%\bibitem{Polovinkin}
%Polovinkin E. S., \emph{Multivalued analysis and differential inclusions}. Moscow, Fizmatlit, 2014. (in Russian)

%\bibitem{Vasil'ev}
38. Vasil'ev~F.~P. \emph{Metody optimizacii $[$Optimization methods$]$.} Moscow, Factorial Press, 2002, 620~p. (In Russian)


\vskip 2%1.5
mm

%\noindent Recommendation: prof. L. A. Petrosyan.
%
%\vskip 1.5mm

%\noindent

Received:  November 29, 2020.

Accepted: April 05, 2021.

\vskip 6mm

%\pagebreak

A\,u\,t\,h\,o\,r\,'\,s \ i\,n\,f\,o\,r\,m\,a\,t\,i\,o\,n:

\vskip 2mm \textit{Alexander V. Fominyh}  --- PhD in Physics and Mathematics, Associate Professor;\\ alexfomster@mail.ru \par
%
%
}
