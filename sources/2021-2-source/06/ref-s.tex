
{\normalsize

\vskip 6mm

\noindent{\bf Investigation of the frequency properties %\\
of a~stan\-dard linear body%$^{*}$%
}

}

\vskip 3%2
mm

{\small

\noindent{\it V.\, P. Tregubov, N.\, K. Egorova%$^2$%
%, I.~�. Famylia%$\,^2$%

}

\vskip 3%2
mm

%%%%%%%%%%%%%%%%%%%%%%%%%%%%%%%%%%%%%%%%%%%%%%%%%%%%%%%%%%%%%%%%%%

%\efootnote{
%%
%\vspace{-3mm}\parindent=7mm
%%
%\vskip 0.1mm $^{*}$ This work was supported by the Russian Foundation% %for Basic Research (grant N~20-07-01086).%\par
%%
%%\vskip 2.0mm
%%
%%\indent{\copyright} �����-������������� ���������������
%%�����������, \issueyear%
%%
%}

%%%%%%%%%%%%%%%%%%%%%%%%%%%%%%%%%%%%%%%%%%%%%%%%%%%%%%%%%%%%%%%%%%

{\footnotesize


\noindent%
%$^1$~%
St.\,Petersburg State University, 7--9, Universitetskaya nab.,
St.\,Petersburg,

\noindent%
%\hskip2.45mm%
199034, Russian Federation


}

%%%%%%%%%%%%%%%%%%%%%%%%%%%%%%%%%%%%%%%%%%%%%%%%%%%%%%%%%%%%%%%%%%

\vskip3mm%3mm


\noindent \textbf{For citation:}  Tregubov~V.\, P., Egorova~N.\, K. Investigation of the frequency properties of a~stan\-dard linear body. {\it Vestnik of Saint~Petersburg Uni\-ver\-si\-ty.
Ap\-plied Mat\-he\-matics. Computer Science. Control
Processes},\,\issueyear,
vol.~17,~iss.~\issuenum,~pp.~\pageref{p6}--\pageref{p6e}. \\
\doivyp/\enskip%
\!\!\!spbu10.\issueyear.\issuenum06  (In Russian)

\vskip3mm

{\leftskip=7mm\noindent It is known that the Kelvin---Voigt model does not describe stress relaxation, which is observed along with elastic properties in many polymers and biomaterials. In this regard, the standard linear body model is used to describe the properties of these materials. Studies of its properties were mainly limited to the study of its reaction to an instantaneously applied load, as well as to the determination of the time and nature of stress relaxation. At the same time, the frequency properties of the standard linear body remained unexplored. In this regard, an analysis of its frequency properties was carried out, which made it possible to study its behavior under vibration exposure. On the basis of the equation of motion, the amplitude-frequency response (AFC) was constructed, and its peculiarity was revealed, which consists in the fact that an increase in the damping coefficient leads to a decrease in the maximum value of the AFC only to a certain value greater than one. A further increase in the damping coefficient leads to an increase in the maximum frequency response up to infinity at a frequency that should also be considered resonant. Thus, the frequency response of a standard linear body always has a maximum. The subsequent increase in the damping coefficient leads to the tendency of the maximum frequency response to zero at infinity.\\[1mm]
\textit{Keywords}: standard linear body, frequency response, model parameters, uniqueness of parameter values.
\par}

\vskip6mm

\noindent \textbf{References} }

\vskip 2mm

{\footnotesize

1. Xiao~R., Sun~H., Chen~W. An equivalence between generalized maxwell model and fractional zener model. {\it Mechanics of Materials}, 2016, vol.~100, pp.~148--153.

2. Sethuraman~V., Makornkaewkeyoon~K., Khalf~A., Madihally~S.~V. Influence of scaffold forming techniques on stress relaxation behavior of polycaprolactone scaffolds. {\it Journal of Applied Polymer Science}, 2013, vol.~130, pp.~4237--4244.

3. Shazly~T.~M., Artzi~N., Boehning~F., Edelman~E.~R.  Viscoelastic adhesive mecha\-nics of aldehyde-mediated soft tissue sealants. {\it Biomaterials}, 2008, vol.~29, pp.~4584--4591.

4. Feng~Z., Seya~D., Kitajima~T., Kosawada~T., Nakamura~T., Umezu~M. Vis\-co\-elas\-tic characteristics of contracted collagen gels populated with rat fibroblasts or car\-dio\-myo\-cy\-tes. {\it  Journal of Artificial Organs}, 2010, vol.~13, pp.~139--144.

5. Tirella~A., Mattei~G., Ahluwalia~A. Strain rate viscoelastic analysis of soft and highly hydrated biomaterials. {\it Journal of Biomedical Materials Research}, 2014, vol.~102, pp.~3352--3360.

6. Capocardo~L., Costa~J., Giusti~S., Buoncompagni~L., Meucci~S., Corti~A., Mattei~G., Ahluwalia~A. Real-time cellular impedance monitoring and imaging of bio\-lo\-gi\-cal barriersin a dual-flow membrane bioreactor. {\it Biosensors and Bioelectronics}, 2019, vol.~140, pp.~1--9.

7.	Kizilova N.~N. Presser wave propagation in liquid field. {\it Fluid dynamics}, 2006, vol.~41, no.~3, pp.~434--446.

8.	Orne D., Liu~Y.~K.  A mathematical model of spinal response to impact. {\it  Journal of Biomechanics}, 1971, vol.~4, no.~1, pp.~49--71.

9.  Braunsmann C., Proksch R., Revenko I., Sch\"affer T. E. Creep compliance mapping by atomic force microscopy. {\it Polymer}, 2014, vol.~55, pp.~219--225.

10. Petit-Zeman~S. Regenerative medicine. {\it Nature Biotechnology}, 2001, vol.~19, pp.~201--206.

11. O'Brien F.~J. Biomaterials $\&$ scaffolds for tissue engineering. {\it Materials Today}, 2011, vol.~14, pp.~88--95. https://doi.org/10.1016/S1369-7021(11)70058-X

12.	Smith B.~D., Grande D.~A. The current state of scaffolds for musculoskeletal regenerative applications. {\it Nature Reviews Rheumatology}, 2015, vol.~11, pp.~213--222.

13.	Agarwal~R., Garc\'{i}a~A.~J. Biomaterial strategies for engineering implants for enhanced osseointegration and bone repair. {\it Advanced Drug Delivery Reviews}, 2015, vol.~94, pp.~53--62.

14.	Deng C.~X., Hong X., Stegemann J.~P. Ultrasound imaging techniques for spa\-tio\-tem\-po\-ral characterization of composition, microstructure, and mechanical properties in tissue engineering. {\it Tissue Engineering. Pt B, Reviews}, 2015, vol.~22, pp.~311--321.

15.	Hong X., Annamalai R.~T., Kemerer T.~S., Deng C.~X., Stegemann J.~P.  Multimode ultrasound viscoelastography for three-dimensional interrogation of microscale mechanical properties in heterogeneous biomaterials. {\it Biomaterials}, 2018, vol.~178, pp.~11--22.

16.	Argatov I.~I. Mathematical modeling of linear viscoelastic impact: Application to drop impact testing of articular cartilage. {\it Tribology International}, 2013, vol.~63, pp.~213--225.

17. Thompson G.~T. In vivo determination of mechanical properties of the human ulna by means of mechanical impedance tests: Experimental results and improved mathematical model. {\it	Medical and Biological Engineering}, 1976, vol.~14, pp.~253--262.

18.	Che Yu-Lin. Alternative form of standard linear solid model for characterizing stress relaxation and creep: Including a novel parameter for quantifying the ratio of fluids to solids of a viscoelastic solid. {\it Frontiers in Materials}, 2020, pp.~7--11. %\\ https://doi.org/10.3389/fmats.2020.00011

19. Tregubov V.~P., Egorova N.~K. Modelirovanie biomekhanicheskih sistem s necelym chislom stepenej svobody [Modeling of biomechanical systems with a non-integer number of degrees of freedom]. {\it Vestnik of Saint~Pe\-ters\-burg University. Applied Mathematics. Computer Science. Control Processes}, 2020, vol.~16, iss.~3, pp.~267--276. %\\
https://doi.org/10.21638/11701/spbu10.2020.305 (In Russian)

\vskip1.5mm Received:  December 18, 2020.

Accepted:
April 05, 2021.


\vskip6mm A~u~t~h~o~r~s' \ i~n~f~o~r~m~a~t~i~o~n:%

\vskip2mm \textit{Vladimir P. Tregubov} --- Dr. Sci. in Physics and Mathematics, Professor;
 v.tregubov@spbu.ru

\vskip2mm \textit{Nadezhda K. Egorova} --- Postgraduate Student; nadezhda\_ego@mail.ru

}
