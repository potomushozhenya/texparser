
{\normalsize

\vskip 6mm

\noindent{\bf Method for the transformation of complex automatic control systems \\ to integrable form %$^*$%
}

}

\vskip 2mm

{\small

\noindent{\it A.~M.~Kamachkin, D.~K.~Potapov, V.~V.~Yevstafyeva%$^{2}$%
%, I.~�. Famylia%$~^2$%

 }

\vskip 2mm

%%%%%%%%%%%%%%%%%%%%%%%%%%%%%%%%%%%%%%%%%%%%%%%%%%%%%%%%%%%%%%%%%%

%\efootnote{
%%
%%\vspace{-3mm}%
%\parindent=7mm
%%
%%\vskip 0.0mm
%\hskip 4mm$^{*}$ This work was supported by the Russian Science% %Foundation (project N~19-11-00223).%\par
%%
%%\vskip 2.0mm
%%
%%\indent{\copyright} �����-������������� ���������������
%%�����������, \issueyear%
%%
%}

%%%%%%%%%%%%%%%%%%%%%%%%%%%%%%%%%%%%%%%%%%%%%%%%%%%%%%%%%%%%%%%%%%

{\footnotesize



\noindent%
%$^2$~%
St.~Petersburg State University, 7--9, Universitetskaya nab.,
St.~Petersburg,

\noindent%
%\hskip2.45mm%
199034, Russian Federation

}

%%%%%%%%%%%%%%%%%%%%%%%%%%%%%%%%%%%%%%%%%%%%%%%%%%%%%%%%%%%%%%%%%%

\vskip2.0mm%3mm

\noindent \textbf{For citation:} Kamachkin A.~M., Potapov D.~K., Yevstafyeva V.~V. Method for the transformation of complex automatic control systems to integrable form. {\it Vestnik of Saint~Peters\-burg
Uni\-versity. Applied Mathematics. Computer Science. Control
Processes}, \issueyear,
vol.~17, iss.~\issuenum, pp.~\pageref{p9}--\pageref{p9e}. %\\
\doivyp/\enskip%
\!\!\!spbu10.\issueyear.\issuenum09  (In Russian)

\vskip2.0%3
mm

{\leftskip=7mm\noindent The article considers a class of automatic control systems that is described by a multi-dimensional system of ordinary differential equations. The right hand-side of the system additively contains a linear part and the product of a control matrix by a vector that is the sum of a control vector and an external perturbation vector. The control vector is defined by a nonlinear function dependent on the product of a feedback matrix by a vector of current coordinates. The authors solve the problem of constructing a matrix of a nonsingular transformation, which leads the matrix of the linear part of the system to the Jordan normal form or the first natural normal form. The variables included in this transformation allow us to vary the system settings, which are the parameters of both the control matrix and the feedback matrix, as well as to convert the system to an integrable form. Integrable form is understood as a form in which the system can be integrated in a final form or reduced to a set of subsystems of lower orders. Furthermore, the sum of the subsystem orders is equal to the order of the original system. In the article, particular attention is paid to cases when the matrix of the linear part has complex conjugate eigenvalues, including multiple ones.\\[1mm]
\textit{Keywords}: automatic control system, multidimensional nonlinear dynamic system, nonsingular transformation, Jordan's normal matrix form, first natural normal matrix form, a system's integrable form.
\par}

\vskip 4.5%6
mm

\noindent \textbf{References} }

\vskip 1.8%2
mm

{\footnotesize

1. Lur'e A.~I. {\it Nekotorye nelinejnye zadachi teorii avtomaticheskogo regulirovanija} [{\it Cer\-tain nonlinear tasks of the
automatic control theory}]. Moscow, Techn.-theor. lit. Publ., 1951, 216~p. (In Russian)

2. Troitskij V.~A. O kanonicheskikh preobrazovanijakh uravnenij teorii avtomaticheskogo regu\-li\-ro\-vanija pri nalichii kratnykh kornej
[About canonical transformations of equations of the automatic cont\-rol theory in the presence of the multiple roots].
{\it Prikladnaya matematika i mekhanika} [{\it Applied Mathematics and Mechanics}], 1957, vol.~21, iss.~4, pp.~574--577. (In Russian)

3. Letov A.~M. {\it Ustojchivost' nelinejnykh reguliruemykh sistem.} 2~izd., ispr. i dop. [{\it Stability of nonli\-near control systems}.
2nd ed.]. Moscow, Phys.-math. lit. Publ., 1962, 483~p. (In Russian)

4. Petrov V.~V., Gordeev A.~A. {\it Nelinejnye servomekhanizmy} [{\it Nonlinear ser\-vo\-\me\-cha\-nisms}]. Moscow, Machinostroenie Publ.,
1979, 472~p. (In Russian)

5. Nelepin R.~A. {\it Tochnye analiticheskie metody v teorii nelinejnykh avtomaticheskikh sistem} [{\it Exact analytical methods in the theory of nonlinear automatic systems}]. Leningrad, Sudostroenie Publ., 1967, 447~p. (In Russian)

6. Nelepin R.~A., Kamachkin A.~M., Turkin I.~I., Shamberov V.~N. {\it Algoritmicheskij sintez nelinejnykh sistem upravleniya}
[{\it Algorithmic synthesis of the nonlinear control systems}]. Leningrad, Leningrad University Press, 1990, 240~p. (In Russian)

7. DeRusso P.~M., Roy R.~J., Close C.~M., Desrochers A.~A. {\it State variables for engineers}. 2nd ed. New York, Wiley-Interscience Publ., 1998, 575~p.

8. Bohr H. Zur theorie der fast periodischen funktionen. I. Eine verallgemeinerung der theorie der fourierreihen [On the theory of almost periodic functions. I. Generalization of the theory of Fourier series]. {\it Acta Math.},
1925, vol.~45, pp.~29--127.

9. Astrom K.~J. Oscillations in systems with relay feedback. {\it Adaptive Control, Filtering and Signal Processing.}
New York, Springer-Verlag Publ., 1995, pp.~1--25.

10. Andronov A.~A., Vitt A.~A., Khaikin S.~E. {\it Teorija kolebanij} [{\it Theory of oscillators}]. Moscow, Physmatgiz Publ.,
1959, 915~p. (In Russian)

11. Kamachkin A.~M., Chitrov G.~M., Shamberov V.~N. Algebraical aspects of para\-metri\-cal decomposition method.
{\it 2015 International Conference ``Stability and Control Prosesses�� in memory of V.~I.~Zubov (SCP--2015)}. St. Petersburg, St. Petersburg State University Press, 2015, pp.~52--54.

12. Kamachkin A.~M., Shamberov V.~N., Chitrov G.~M. Special matrix transformations of essential nonlinear control systems.
{\it 2017 International Conference ``Constructive Non\-smooth Analysis and Related Topics�� dedicated to the memory of V.~F.~Demyanov
(CNSA--2017)}. St. Petersburg, St. Petersburg State University Press, 2017, pp.~1--3.

13. Kamachkin A.~M., Chitrov G.~M., Shamberov V.~N. Normal'nye formy matrits v zadachakh dekompozitsii i upravleniya mnogomernykh sistem
[Normal matrix forms to decomposition and control problems for manydimensional systems].
{\it Vestnik of Saint Petersburg University. Applied Mathematics. Computer Science. Control Processes},
2017, vol.~13, iss.~4, pp.~417--430. (In Russian)

14. Kamachkin A.~M., Potapov D.~K., Yevstafyeva V.~V. Solution to second-order differential equations with discontinuous right-hand side.
{\it Electron. Journal Differential Equations}, 2014, no.~221, pp.~1--6.

15. Evstaf'eva V.~V. Ob usloviyakh sushchestvovaniya dvukhtochechno-kolebatel'nogo periodicheskogo resheniya v neavtonomnoj relejnoj sisteme s
gurvitsevoj matritsej [On existence conditions for a two-point oscillating periodic solution in an non-autonomous relay system with a Hurwitz matrix].
{\it Avtomatika i telemekhanika} [{\it Automat. Remote Control}], 2015, no.~6, pp.~42--56. (In Russian)

16. Kamachkin A.~M., Potapov D.~K., Yevstafyeva V.~V. Non-existence of periodic solutions to non-autonomous second-order differential equation with discontinuous non\-li\-nea\-ri\-ty. {\it Electron. Journal Differential Equations}, 2016, no.~04, pp.~1--8.

17. Kamachkin A.~M., Potapov D.~K., Yevstafyeva V.~V. Existence of solutions for second-order differential equations with discontinuous right-hand side. {\it Electron. Journal Differential Equations}, 2016, no.~124, pp.~1--9.

18. Kamachkin A.~M., Potapov D.~K., Yevstafyeva V.~V. Existence of periodic solutions to automatic control system with relay nonlinearity and sinusoidal external influence. {\it International Journal Robust Nonlinear Control}, 2017, vol.~27, no.~2, pp.~204--211.\pagebreak

19. Kamachkin A.~M., Potapov D.~K., Yevstafyeva V.~V. Existence of subharmonic solutions to a hysteresis system with sinusoidal external influence. {\it Electron. Journal Dif\-fe\-ren\-tial Equations}, 2017, no.~140, pp.~1--10.

20. Kamachkin A.~M., Potapov D.~K., Yevstafyeva V.~V. On uniqueness and
properties of periodic solution of second-order nonautonomous system with
discontinuous nonlinearity. {\it Journal Dynamical Control Systems}, 2017, vol.~23, no.~4, pp.~825--837.

21. Evstaf'eva V.~V. Periodicheskie resheniya sistemy differentsial'nykh uravnenij s gisterezisnoj nelinejnost'yu pri nalichii nulevogo
sobstvennogo chisla [Periodic solutions of a system of differential equations with hysteresis nonlinearity in the presence of eigenvalue zero]. {\it Ukrainskij matematicheskij zhurnal} [{\it Ukrainian Mathematical Journal}], 2018, vol.~70, no.~8, pp.~1085--1096. (In Russian)

22. Kamachkin A.~M., Potapov D.~K., Yevstafyeva V.~V. Existence of periodic modes in automatic control system with a three-position relay.
{\it International Journal Control}, 2020, vol.~93, no.~4, pp.~763--770.

23. Kamachkin A.~M., Potapov D.~K., Yevstafyeva V.~V. Dinamika i sinkhronizatsiya tsik\-li\-ches\-kikh struktur ostsillyatorov s
gisterezisnoj obratnoj svyaz'yu [Dynamics and syn\-chro\-ni\-za\-tion in feed\-back cyclic structures with hysteresis oscillators].
{\it Vestnik of Saint~Pe\-ters\-burg University. Applied Ma\-the\-ma\-tics. Computer Science. Control Processes}, 2020, vol.~16, iss.~2, pp.~186--199.\\  https://doi.org/10.21638/11701/spbu10.2020.2010 (In Russian)

24. Evstaf'eva V.~V. O sushchestvovanii dvukhtochechno-kolebatel'nykh reshenij voz\-mu\-shchen\-noj relejnoj sistemy s gisterezisom
[On the existence of two-point oscillatory solutions of a perturbed relay system with hysteresis]. {\it Differentsial'nye uravneniya} [{\it Differential Equa\-tions}], 2021, vol.~57, no.~2, pp.~169--178. (In Russian)

25. Lankaster P. {\it Teoriya matrits} [{\it Theory of matrices}]. Moscow, Nauka Publ., 1973, 280~p. (In Russian)

26. Burns R.~S. {\it Advanced control engineering}. Oxford, Butterworth-Heinemann Publ., 2001, 464~p.

27. Paraskevopoulos P.~N. {\it Modern control engineering}. New York, Marcel Dekker Inc. Publ., 2002, 736~p.



\vskip 1.5mm

Received:  February 12, 2021.

Accepted: April 05, 2021.


\vskip6 mm A~u~t~h~o~r~s' \ i~n~f~o~r~m~a~t~i~o~n:


\vskip2 mm \textit{Alexander M. Kamachkin} --- Dr. Sci. in Physics and Mathematics, Professor; a.kamachkin@spbu.ru

\vskip2 mm \textit{Dmitriy K. Potapov} --- PhD in Physics and Mathematics, Associate Professor; %\\ 
d.potapov@spbu.ru

\vskip2 mm \textit{Victoria V. Yevstafyeva} --- PhD in Physics and Mathematics, Associate Professor;\\ v.evstafieva@spbu.ru \par
%
}
