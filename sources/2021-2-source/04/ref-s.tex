
{\normalsize

\vskip 6mm

\noindent{\bf On momentum flow density of the gravitational field%$^*$%
 }

}

\vskip 2mm

{\small

\noindent{\it O.~I.~Drivotin %$^{1,4}$%
%, I.~�. Famylia%$~^2$%

 }

\vskip 2mm

%%%%%%%%%%%%%%%%%%%%%%%%%%%%%%%%%%%%%%%%%%%%%%%%%%%%%%%%%%%%%%%%%%

%\efootnote{
%%
%\vspace{-3mm}%
%\parindent=7mm
%%
%\vskip 0.0mm
%\hskip 4mm$^{*}$ This work was supported by the Russian Science% %Foundation (project N 20-71-10032).%\par
%%
%%\vskip 2.0mm
%%
%\indent{\copyright} �����-������������� ���������������
%�����������, \issueyear%
%%
%}

%%%%%%%%%%%%%%%%%%%%%%%%%%%%%%%%%%%%%%%%%%%%%%%%%%%%%%%%%%%%%%%%%%

{\footnotesize


\noindent%
%$^2$~%
St.~Petersburg State University, 7--9, Universitetskaya nab.,
St.~Petersburg,

\noindent%
%\hskip2.45mm%
199034, Russian Federation

}

%%%%%%%%%%%%%%%%%%%%%%%%%%%%%%%%%%%%%%%%%%%%%%%%%%%%%%%%%%%%%%%%%%

\vskip2.0mm%3mm

\noindent \textbf{For citation:} Drivotin O.~I.
On momentum flow density of the gravitational field. {\it Vestnik of Saint~Peters\-burg Uni\-versity.
Applied Mathematics. Computer Science. Control Processes},
\issueyear,
vol.~17, iss.~\issuenum, pp.~\pageref{p4}--\pageref{p4e}.  %\\
\doivyp/\enskip%
\!\!\!spbu10.\issueyear.\issuenum04 (In Russian)

\vskip2.0%3
mm

{\leftskip=7mm\noindent Momentum is considered on the basis of the approach widely used in the calculus of variations and in the optimal control theory, where variation of a cost functional is investigated. In physical theory, it is the action functional. Action variation under Lie dragging can be expressed as a surface integral of some differential form. The momentum density flow is defined using this form. In this work, the momentum balance equation is obtained. This equation shows that the momentum field transforms into a momentum of a mass. Examples showing the momentum flow structure for a mass distribution representing a uniform thin layer are provided.\\[1mm]
\textit{Keywords}: action variation of the gravitational field, momentum flow density of the gra\-vi\-tational field, momentum balance equation, thin layer with uniform mass distribution. \par}

\vskip 4%6
mm

\noindent \textbf{References} }

\vskip 2 mm

{\footnotesize

%\bibitem{pontr}
1. Pontryagin L. S., Boltyanskiy V. G., Gamkrelidze R. V., Mischenko E. F. {\it Matematicheskaya teoriya optimal'nyh protsessov $[$Mathematical theory of optimal processes$]$}. Moscow, Nauka Publ., 1976, 392~p.
(In Russian)


%\bibitem{driv19}
2. Drivotin O.~I. O chislennom reshenii zadachi optimal'nogo upravleniya na osnove metoda, ispol'zuyushchego vtoruyu variatsiyu trayektorii [On numerical solution of the optimal control problem based on a  method using  the second variation of a trajectory].
{\it Vestnik of Saint Petersburg University.
Applied Mathematics. Computer Science. Control Processes},
2019, vol.~15, iss.~2, pp.~283--295.\\
https://doi.org/10.21638/11702/spbu10.2019.211 (In Russian)



%\bibitem{drivmono}
3. Drivotin O.~I.
{\it Matematicheskiye osnovy teorii polya
$[$Mathematical foundations of the field theory$]$}.
St.~Pe\-ters\-burg, St.~Pe\-ters\-burg  University Press, 2010, 168~p.
(In Russian)



%\bibitem{IPAC2011}
4. Drivotin O. I. Covariant formulation of the Vlasov equation.
{\it Proceedings of International Particle Accelerators Conference (IPAC'2011)}. San Sebastian, Kursaal Publ., 2011, pp.~2277--2279.

%\bibitem{RuPAC2012}
5. Drivotin O.\,I. Degenerate solutions of the Vlasov equation.
{\it  Proceedings of Russian Accelerators Conference (RuPAC'2012)}. St.~Pe\-ters\-burg, St.~Pe\-ters\-burg State University Publ., 2012, pp.~376--378.



%\bibitem{driv16}
6. Drivotin O. I. Kovariantnoe opisanie raspredeleniy v fazovom prostranstve [Covariant des\-crip\-tion of phase space dis\-tri\-bu\-tions].
{\it Vestnik of Saint Petersburg University. Series~10. Applied Mathematics. Computer Science. Control Processes}, 2016, vol.~12, iss.~3, pp.~39--52.\\
https://doi.org/10.21638/11701/spbu10.2016.304	(In Russian)


%\bibitem{abbot}
7. Abbot B.\,P., Abbot R., Abbot T.\,D. et al. LIGO scientific collaboration and Virgo collaboration. {\it Physical Review Letters}, 2016, vol.~116, iss.~6, p.~061102.

%\bibitem{ll}
8. Landau L. D., Lifshitz E. M. {\it Teoriya polya $[$The field theory$]$}. Moscow, Nauka Publ., 1973, 504~p. (In Russian)


%\bibitem{carrol}
9. Carrol S. {\it Spacetime and geometry. An introduction to general relativity}. San Fransisco, Addison Wesley Publ., 2004, 513~p.

%\bibitem{driv14}
10. Drivotin O. I. Rigorous definition of the reference frame.
{\it Vestnik of Saint~Pe\-ters\-burg University. Series~10.
Applied Mathematics. Computer Science. Control Processes},
2014, iss.~4, pp.~25--36.



\vskip 1.5mm

Received:  April 11, 2020.

Accepted: April 05, 2021.


\vskip6 mm A~u~t~h~o~r\,'\,s \ i~n~f~o~r~m~a~t~i~o~n:


\vskip2 mm \textit{Oleg I. Drivotin} --- Dr. Sci. in Physics and Mathematics,
Professor; o.drivotin@spbu.ru \par
%
}
