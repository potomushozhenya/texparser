
{\normalsize

\vskip 6mm

\noindent{\bf Theoretical foundations of probabilistic and statistical forecasting\\ of agrometeorological risks$^*$%
}

}

\vskip 2mm

{\small

\noindent{\it V.~P.~Iakushev$^1$, V.~M.~Bure$^{1,2}$, O.~A.~Mitrofanova$^{1,2}$, E.~P.~Mitrofanov$^{1,2}$%
%, I.~�. Famylia%$~^2$%

 }

\vskip 2mm

%%%%%%%%%%%%%%%%%%%%%%%%%%%%%%%%%%%%%%%%%%%%%%%%%%%%%%%%%%%%%%%%%%

\efootnote{
%%
%%\vspace{-3mm}
%%\parindent=7mm
%%
%%\vskip 0.0mm
\hskip 4mm$^{*}$ This work is supported by the Russian Federation (agreement with the Ministry of Science and Education) (project N~075-15-2020-805 dated October 02, 2020).%\par
%%
%%\vskip 2.0mm
%%
%%\indent{\copyright} �����-������������� ���������������
%%�����������, \issueyear%
%%
}

%%%%%%%%%%%%%%%%%%%%%%%%%%%%%%%%%%%%%%%%%%%%%%%%%%%%%%%%%%%%%%%%%%

{\footnotesize

\noindent%
$^1$~%
Agrophysical Research Institute, 14, Grazhdanskiy pr., St. Petersburg,

\noindent%
\hskip2.45mm%
195220, Russian Federation




\noindent%
$^2$~%
St.~Petersburg State University, 7--9, Universitetskaya nab.,
St.~Petersburg,

\noindent%
\hskip2.45mm%
199034, Russian Federation

}

%%%%%%%%%%%%%%%%%%%%%%%%%%%%%%%%%%%%%%%%%%%%%%%%%%%%%%%%%%%%%%%%%%

\vskip2.0mm%3mm

\noindent \textbf{For citation:}   Iakushev V.~P., Bure V.~M., Mitrofanova O.~A., Mitrofanov E.~P. Theoretical foundations of probabilistic and statistical forecasting of agrometeorological risks. {\it Vestnik of Saint~Peters\-burg
Uni\-versity. Applied Mathematics. Computer Science. Control
Processes}, \issueyear,
vol.~17, iss.~\issuenum, pp.~\pageref{p7}--\pageref{p7e}. %\\
\doivyp/\enskip%
\!\!\!spbu10.\issueyear.\issuenum07  (In Russian)

\vskip2.0%3
mm

{\leftskip=7mm\noindent Each model for forecasting agrometeorological risks based on the analysis of one-dimensional time series is effective for a~certain range of initial information. In addition, the values of the initial observations can differ significantly for each specific case, respectively, the widespread use of one method for the analysis of arbitrary information can lead to significant inaccuracies. Thus, the problem of choosing a~forecasting method for the initial set of agrometeorological data arises. In this regard, a universal adaptive probabilistic-statistical approach to predicting agrometeorological risks is proposed, which makes it possible to solve the problem of choosing a~model. The article presents the results of the first stage of research carried out with the financial support of the Ministry of Education and Science of the Russian Federation: a brief overview of the current state of research in this direction is presented, theoretical foundations for predicting agrometeorological risks for a~possible onset of drought and frost have been developed, including the task of generating initial information, a~description of basic forecasting models, and also a~direct description of the proposed approach with a~presentation of the general structure of an intelligent system, on the basis of which the corresponding algorithm can be developed and automated as directions for further work.\\[1mm]
\textit{Keywords}: one-dimensional time series, forecasting, droughts, frosts, agrometeorological hazards, intelligent system.
\par}

\vskip 4.5%6
mm

\noindent \textbf{References} }

\vskip 1.8%2
mm

{\footnotesize

1. IPCC, 2012: {\it Managing the risks of extreme events and disasters to advance climate change adaptation.} A special report of Working Groups I and II of the Intergovernmental Panel on Climate Change. Eds by C.~B.~Field, V.~Barros, T.~F.~Stocker, D.~Qin, D.~J.~Dokken, K.~L.~Ebi, M.~D.~Mastrandrea, K.~J.~Mach, G.-K.~Plattner, S.~K.~Allen, M.~Tignor, P.~M.~Midgley. Cambridge, UK, New York, USA, Cambridge University Press, 2012, 582~p.

2. Drobzheva Ia. V., Volobueva O. V.
{\it Meteorologicheskie prognozy i ikh ekonomicheskaia poleznost'}. Uchebnoe posobie [{\it Meteorological forecasts and their economic usefulness}. Tutorial]. Saint Petersburg, Admyral Publ., 2016, 116~p. (In Russian)

3. Vil'fand R. M., Strashnaia A. I., Bereza O. V.
 O dinamike agroklimaticheskikh pokazatelei uslovii seva, zimovki i formirovaniia urozhaia osnovnykh zernovykh kul'tur [On the dynamics of agroclimatic indicators of sowing conditions, wintering and the formation of the yield of the main grain crops]. {\it Proceedings of Hydrometcenter of Russia}, 2016, vol.~360, pp.~45--78. (In Russian)

4. Zhang F., Chen Y., Zhang J., Guo E., Wang R., Li D.
Dynamic drought risk assessment for maize based on crop simulation model and multi-source drought indices. {\it Journal of Cleaner Production}, 2019, vol.~233, pp.~100--114.

5. Liu X., Guo P., Tan Q., Xin J., Li Y., Tang Y.
Drought risk evaluation model with interval number ranking and its application. {\it Science of the Total Environment}, 2019, vol.~685, pp.~1042--1057.

6. Strashnaia A.\,I., Bereza O.\,V., Tarasova L.\,L., Maksimenkova T.\,A., Shul'gin I.\,A., Puri\-na~I.\,E., Chekulaeva T. S.
Sovremennoe sostoianie i problemy agrometeorologicheskogo obespecheniia sel'skogo khoziaistva Rossii [Current state and problems of agrometeorological support of agriculture in Russia]. {\it Hydrometeorological Research and Forecasting}, 2019, no.~4~(374), pp.~219--240. (In Russian)

7. Park S., Im J., Park S., Rhee J.
Drought monitoring using high resolution soil moisture through mul\-ti-sen\-sor satellite data fusion over the Korean peninsula. {\it Agricultural and Forest Meteorology}, 2017, vol.~237, pp.~257--269.

8. Gobbett D. L., Nidumolu U., Crimp S.
Modelling frost generates insights for managing risk of minimum temperature extremes. {\it Weather and Climate Extremes}, 2020, vol.~27, no.~100176.

9. Crimp S., Bakar K. S., Kokic P., Jin H., Nicholls N., Howden M.
Bayesian space-time model to analyse frost risk for agriculture in Southeast Australia. {\it International Journal of Climatology}, 2015, vol.~35, pp.~2092--2108.

10. Van Hinsbergen C., van Lint J., van Zuylen H.
 Bayesian committee of neural networks to predict travel times with confidence intervals. {\it Transportation Research. Pt C. Emerging Technologies}, 2009, vol.~17, pp.~498--509.

11. Xiao L., Liu L., Asseng S., Xia Y., Tang L., Liu B., Cao W., Zhu Y.
Estimating spring frost and its impact on yield across winter wheat in China. {\it Agricultural and Forest Meteorology}, 2018, vol.~260--261, pp.~154--164.

12. Frolov A. V., Strashnaia A. I.
O zasukhe 2010 goda i ee vliianii na urozhainost' zernovykh kul'tur [On the 2010 drought and its impact on grain yields]. {\it Analysis of abnormal weather conditions in Russia in the summer of 2010}. Moscow, Triada LTD Publ., 2011, pp.~22--31. (In Russian)

13. Iakushev V. P., Bure V. M., Mitrofanova O. A., Mitrofanov E. P.
K voprosu avtomatizatsii postroeniia variogramm v zadachakh tochnogo zemledeliia [On the issue of semivariograms constructing automation for precision agriculture problems]. {\it Vestnik of Saint Petersburg University. Applied Ma\-the\-matics. Computer Science. Control Processes}, 2020, vol.~16, iss.~2, pp.~177--185.  \\ https://doi.org/10.21638/11701/spbu10.2020.209 (In Russian)

14. Bure V. M., Kanash E. V., Mitrofanova O. A.
Analiz kharakteristik tsveta rastenii po aerofotosnimkam s razlichnymi faktorami kachestvennykh pokazatelei [Analysis of plants color characteristics using aerophotos with different factors of qualitative indicators]. {\it Vestnik of Saint Petersburg University. Applied Mathematics. Computer Science. Control Processes,} 2017, vol.~13, iss.~3, pp.~278--285. \\ https://doi.org/10.21638/11701/spbu10.2017.305 (In Russian) %



\vskip 1.5mm

Received:  December 27, 2020.

Accepted: April 05, 2021.


\vskip6 mm A~u~t~h~o~r~s' \ i~n~f~o~r~m~a~t~i~o~n:


\vskip2 mm \textit{Viktor P. Iakushev}  --- RAS Academician, Dr. Sci. in Agriculture; vyakushev@agrophys.com

\vskip2 mm \textit{Vladimir M. Bure}  --- Dr. Sci. in Technics, Professor; vlb310154@gmail.com

\vskip2 mm \textit{Olga A. Mitrofanova} --- PhD in Technics; omitrofa@gmail.com

\vskip2 mm \textit{Evgenii P. Mitrofanov} --- Junior Researcher; mjeka@bk.ru \par
%
}
