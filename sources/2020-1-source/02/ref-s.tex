
{\normalsize

\vskip 6mm%6mm

\noindent{\bf On the practical applicability of three CUSUM-methods\\
for structural breaks detection in EGARCH-models$^{*}$%
}

}

\vskip 2mm%2mm

{\small

\noindent{\it D. A. Borzykh$\,^1$%
, A. A. Yazykov$\,^{1,2,3}$%
%, I.~�. Famylia%$\,^2$%

}

\vskip 2mm%2mm

%%%%%%%%%%%%%%%%%%%%%%%%%%%%%%%%%%%%%%%%%%%%%%%%%%%%%%%%%%%%%%%%%%

\efootnote{
%%
\vspace{-3mm}\parindent=7mm
%%
\vskip 0.1mm $^{*}$ The reported study was funded by RFBR
according to the research project N~19-31-90169.\par
%%
%%\vskip 2.0mm
%%
%%\indent{\copyright} �����-������������� ���������������
%%�����������, \issueyear%
%%
}

%%%%%%%%%%%%%%%%%%%%%%%%%%%%%%%%%%%%%%%%%%%%%%%%%%%%%%%%%%%%%%%%%%

{\footnotesize

\noindent%
$^1$~%
National Research University Higher School of Economics, 20,
Myasnitskaya ul., Moscow,

\noindent%
\hskip2.45mm%
101000, Russian Federation


\noindent%
$^2$~%
Federal Research Center  ``Computer Science and Control'' Russian
Academy of Sciences,

\noindent%
\hskip2.45mm%
44, bldg.~2, Vavilova ul., Moscow, 119333, Russian Federation


\noindent%
$^3$~%
Moscow Institute of Physics and Technology (State University), 9,
Institutsky per., Dolgoprudny,

\noindent%
\hskip2.45mm%
141701, Russian Federation

%\noindent%
%%$^1$~%
%St.\,Petersburg State University, 7--9, Universitetskaya nab.,
%St.\,Petersburg,
%
%\noindent%
%%\hskip2.45mm%
%199034, Russian Federation

}

%%%%%%%%%%%%%%%%%%%%%%%%%%%%%%%%%%%%%%%%%%%%%%%%%%%%%%%%%%%%%%%%%%

\vskip3mm%3mm


\noindent \textbf{For citation:}  Borzykh D.\,A., Yazykov A.\,A.
On the practical applicability of three CUSUM-methods for
structural breaks detection in EGARCH-models. {\it Vestnik of
Saint~Petersburg Uni\-ver\-si\-ty. Applied Mathematics. Computer
Science. Control Processes},\,\issueyear,
vol.~16,~iss.~\issuenum,~pp.~\pageref{p2}--\pageref{p2e}. %\\
\doivyp/\enskip%
\!\!\!spbu10.\issueyear.\issuenum02  (In Russian)

\vskip3mm

{\leftskip=7mm\noindent There are three well-known CUSUM-methods
of structural breaks detection for standard GARCH-models in the
literature: (Incl�an, Tiao, 1994), (Kokoszka, Leipus, 1999) and
(Lee, Tokutsu, Maekawa, 2004). Despite the fact that these
algorithms were initially developed for standard GARCH-models,
there are theoretical arguments that CUSUM-methods can be applied
to EGARCH-models. What is more, one can find empirical research
which uses these methods to detect structural breaks in real-time
series volatility. However, we have not found any numeric
experiments which would prove the applicability of CUSUM-methods
for EGARCH-models so far. We are not aware of any controlled
experiments conducted in order to verify the applicability of
these methods for EGARCH-models. This article adds to the existing
literature in the following way. We first generate volatility
series which possess EGARCH-model with known structural breaks.
Then we run simulations and show that CUSUM-methods are weak in
detecting structural breaks on medium size samples which are close
to real ones. We conclude that the applicability of these methods
on EGARCH-models is limited. Therefore, we suggest a hybrid
algorithm which is able to improve the performance of
CUSUM-methods when detecting structural breaks in all
EGARCH-models.\\[1mm]
\textit{Keywords}: EGARCH, volatility, change points, structural
breaks, CUSUM.
\par}

\vskip6mm

\noindent \textbf{References} }

\vskip 2mm

{\footnotesize

1. Bollerslev T. Generalized autoregressive conditional
heteroskedasticity. {\it Journal of Econometrics}, 1986, vol.~31,
no.~3, pp.~307--327.

2. Francq C., Zakoian J.-M. {\it GARCH models: structure,
statistical inference and financial applications}. New York, John
Wiley \& Sons Publ., 2010, 504~p.

3. Ayvazyan S. A., Fantatstsini D. {\it Ekonometrika-2:
prodvinutyy kurs s prilozheniyami v finansakh} [{\it
Econometrics-2: advanced course with applications in finance}].
Moscow, Magistr, Infra-M Publ., 2014, 944~p. (In Russian)

4. Nelson D. Conditional heteroskedasticity in asset returns: a
new approach. {\it Econometrica}, 1991, vol.~59, no.~2,
pp.~347--370.

5. Incl\'{a}n C., Tiao G. Use of cumulative sums of squares for
retrospective detection of changes of variance. {\it Journal of
the American Statistical Association}, 1994, vol.~89, no.~427,
pp.~913--923.

6. Kokoszka P., Leipus R. Testing for parameter changes in ARCH
models. {\it Lithuanian Mathematical Journal}, 1999, vol.~39,
no.~2, pp.~182--195.

7. Lee S., Tokutsu Y., Maekawa K. The CUSUM test for parameter
change in regression models with ARCH errors. {\it Journal of the
Japanese Statistical Society}, 2004, vol.~34, no.~2, pp.~173--188.

8. Bracker K., Smith K. L. Detecting and modeling changing
volatility in the copper futures market. {\it Journal of Futures
Markets: Futures, Options, and Other Derivative Products},  1999,
vol.~19, no.~1, pp.~79--100.

9. Miralles Marcelo J. L., Miralles Quir\'{o}s J. L., Miralles
Quir\'{o}s M. M. Sudden shifts in variance in the Spanish market:
persistence and spillover effects. {\it Applied Financial
Economics}, 2008, vol.~18, no.~2, pp.~115--124.

10. Malik F. Estimating the impact of good news on stock market
volatility. {\it Applied Financial Economics}, 2011, vol.~21,
no.~8, pp.~545--554.

11. Alfreedi A. A., Isa Z., Hassan A. Regime shifts in asymmetric
GARCH models assuming heavy-tailed distribution: evidence from GCC
stock markets. {\it Journal of Statistical and Econometric
Methods}, 2012, vol.~1, no.~1, pp.~43--76.

12. Erragragui E., Hassan M., Peillex J., Khan A. Does ethics
improve stock market resilience in times of instability? {\it
Economic Systems}, 2018, vol.~42, no.~3, pp.~450--469.




\vskip1.5mm Received:  October 29, 2019.

Accepted: February 13, 2020.


\vskip6mm A\,u\,t\,h\,o\,r\,s' \ i\,n\,f\,o\,r\,m\,a\,t\,i\,o\,n:%

\vskip2mm \textit{Dmitriy A. Borzykh} --- %
%PhD in Physics and Mathematics;
borzykh.dmitriy@gmail.com

\vskip2mm \textit{Artem A. Yazykov} --- Postgraduate Student; %
%PhD in Physics and Mathematics;
ayazikov@hse.ru

}
