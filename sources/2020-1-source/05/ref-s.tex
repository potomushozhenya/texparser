
{\normalsize

\vskip 6mm

\noindent{\bf A parallel algorithm for iterating partitions
of a finite set\\ into subsets of a given cardinality%$^{*}$%

}

}

\vskip 2mm

{\small

\noindent{\it A.~M.~Kovshov%$\,^1$%
%, V.~A.~Ageev%$^1$%
%, I.~�. Famylia%$\,^2$%

 }

\vskip 1.8mm%2mm

%%%%%%%%%%%%%%%%%%%%%%%%%%%%%%%%%%%%%%%%%%%%%%%%%%%%%%%%%%%%%%%%%%

%\efootnote{
%%
%\vspace{-3mm}\parindent=7mm
%%
%\vskip 0.1mm $^{*}$ This work was supported by the Russian
%Foundation for Basic Reaserch (grant N18-010-00723).\par
%%
%%\vskip 2.0mm
%%
%%\indent{\copyright} �����-������������� ���������������
%%�����������, \issueyear%
%%
%}

%%%%%%%%%%%%%%%%%%%%%%%%%%%%%%%%%%%%%%%%%%%%%%%%%%%%%%%%%%%%%%%%%%

{\footnotesize

\noindent%
%$^1$~%
St.\,Petersburg State University, 7--9, Universitetskaya nab.,
St.\,Petersburg,


\noindent%
%\hskip2.45mm%
199034, Russian Federation

}

%%%%%%%%%%%%%%%%%%%%%%%%%%%%%%%%%%%%%%%%%%%%%%%%%%%%%%%%%%%%%%%%%%

\vskip1.8mm%3mm


\noindent \textbf{For citation:}  Kovshov~A. M. A parallel
algorithm for iterating partitions of a finite set into subsets of
a given cardinality. {\it Vestnik of Saint~Pe\-tersburg
University. Applied Mathematics. Computer Science. Control
Processes},\,\issueyear,
vol.~16, %~
iss.~\issuenum,~pp.~\pageref{p5}--\pageref{p5e}. \\
\doivyp/\enskip%
\!\!\!spbu10.\issueyear.\issuenum05  (In Russian)

\vskip3mm

{\leftskip=7mm\noindent The article describes an iterative
algorithm for searching partitions of a finite set consisting of
distinguishable elements into subsets of a given cardinality. The
cardinalities of some subsets may be the same. The entire
algorithm consists of two independent algorithms. The first
algorithm for each element of the original set determines the
cardinality of the subset that it will fall into when
partitioning. To do this, subsets of the same cardinality are
united into composite subsets. The elements of the original set
are distributed among composite subsets. An index array is created
to describe partitions. This array of indexes indicates which
composite subset each element falls into. The length of the index
array is equal to the cardinality of the original set. Each
composite subset has its own index in the index array. Iterating
over partitions of the original set into composite subsets is
reduced to iterating over all index permutations in the index
array. The second algorithm distributes elements within each
composite subset over subsets of equal cardinalities. For each
composite subset, an index array is created that describes which
subset each element of the composite subset falls into. Iterating
over all partitions of a composite subset over equally powerful
subsets is reduced to iterating over-index permutations.
Permutations must meet the following condition: the index value
must not exceed the ordinal number of its place in the
permutation. This avoids generating the same permutations.
Permutations of all index arrays are iterated in lexicographic
order. This construction of the algorithm allows to split the
entire search into independent parts and use parallel
calculations. An example is considered that shows the consistency
of the algorithm and the acceleration of obtaining the result when
using parallel calculations.\\[1mm]
\textit{Keywords}: exhaustive search algorithms, parallel
computing.
\par}

\vskip5mm

\noindent \textbf{References} }

\vskip 2mm

{\footnotesize

1. Lipsky~V. {\it Kombinatorica dlya programmistov} [{\it
Combinatorics for programmers}]. Moscow, Mir Publ., 1988, 213~p.
(In Russian)

%\bibitem{AMK_Graph}
2. Kovshov~A.\,M. The creating of shared search algorithms on the
example of graph path searching. {\it Modern science}, 2018,
no.~12-1, pp.~48--51.

%\bibitem{SetPart}
3. {Djoci\'c~B., Miyakawa~M., Sekiguchi~S., Semba~I.,
Stojmenovi\'c~I.} A fast iterative algorithm for generating set
partitions. {\it The Computer Journal}, 1989, vol.~32, no.~3,
pp.~281--282.

%\bibitem{ParallSetPart}
4. {Djoki\'c~B., Miyakawa~M., Sekiguchi~S., Semba~I.,
Stojmenovi\'c~I.} Parallel algorithms for generating subsets and
set partitions. {\it International Symposium on Algorithms}.
SIGAL, 1990,  pp.~76--85.

%\bibitem{ParallPart}
5. {Chen~G.\,H., Chern M.\,S.} Parallel generation of permutations
and combinations.   {\it BIT Numerical Mathematics}, 1986,
vol.~26,  pp.~277--283.

%\bibitem{Kokosinski}
6. {Kokosi\'nski~Z.} On generation of permutations through
decomposition of symmetric groups into cosets.
 {\it BIT Numerical Mathematics}, 1990, vol.~30,  pp.~583--591.

%\bibitem{BoraUcar}
7. {U\c{c}ar~B.} {\it Partitioning, matching, and ordering:
Combinatorial scientific computing with matrices and tensors.
Computer Science}. Lyon, ENS de Lyon Press, 2019, 191~p.

%\bibitem{Deveci}
8. {Deveci~M., Kaya~K., U\c{c}ar~B., \c{C}ataly\"urek~\"U.\,V.}
Hypergraph partitioning for multiple communication cost metrics:
Model and methods. {\it Journal of Parallel and Distributed
Computing}, 2015, vol.~77, no.~3,  pp.~69--83.

%\bibitem{JiangFengZhu}
9. {Jiang~H., Feng~H., Zhu~D.} An 5/4-approximation algorithm for
sorting permutations by short block moves. {\it Proceedings of the
25th International Symposium on Algorithms and Computation}, 2014,
pp.~491--503.

%\bibitem{Brasil}
10. {Alexandrino~A., Miranda~G., Lintzmayer~C., Dias~Z.}
Approximation algorithms for sorting permutations by
length-weighted short rearrangements. {\it Electronic Notes in
Theoretical Computer Science}, 2019, vol.~346,  pp.~29--40.

%\bibitem{GalvLeeDias}
11. {Galv\~ao~G.\,R., Lee~O., Dias~Z.} Sorting signed permutations
by short operations. {\it Algorithms for Molecular Biology}, 2015,
no.~10,  pp.~1--17.

%\bibitem{CresPerezTod}
12. {Crespelle~C., Perez~A., Todinca~I.} An
%O($\text{n}^{\!{\scriptscriptstyle2\!}}$)
O($n^2$) time algorithm for the minimal permutation completion
problem. {\it Discrete Applied Mathematics}, 2019, vol.~254,
pp.~80--95.

%\bibitem{Albert}
13. {Albert~M.\,H., Lackner~M.-L., Lackner~M., Vatter~V.} The
complexity of pattern matching for 321-avoiding and skew-merged
permutations. {\it Discrete Mathematics $\&$ Theoretical Computer
Science}, 2016, vol.~18, no.~2,  pp.~1--17.

%\bibitem{Lackner}
14. {Lackner~M.-L., Lackner~M.} A fast algorithm for permutation
pattern matching based on alternating runs. {\it Algorithmica},
2016, vol.~75, no.~1,  pp.~84--117.

%\bibitem{Molloy}
15. {Esta\~na~A., Molloy~K., Vaisset~M., Sibille~N., Sim\'eon~T.,
Bernad\'o~P, Juan Cort\'es~J.} Hybrid parallelization of a
multi-tree path search algorithm: Application to highly-flexible
biomolecules. {\it Parallel Computing}, 2018, vol.~77,
pp.~84--100.

%\bibitem{Olivera}
16. {Oliveira~A., Fertin~G., Dias,~U., Dias~Z.} Sorting signed
circular permutations by super short operations. {\it Algorithms
for Molecular Biology}, 2018, vol.~13, no.~12, pp.~13--18.

%\bibitem{GergelStrongin}
17. {Gergel~V.\,P., Strongin~R.\,G.}
 {\it Osnovy parallelnykh vychisleniy dlya mnogoprocessornykh vychislitelnykh sistem}
[{\it Fundamentals of parallel computing for multiprocessor
computing systems}]. Nizhniy Novgorod, Publ. of Lobachevskiy NNGU,
2003, 184~p. (In Russian)

%\bibitem{Voevodin02}
18. {Voevodin~V.\,V., Voevodin~Vl.\,V.}
 {\it Parallelnye vychisleniya} [{\it Parallel computing}].
Saint Petersburg, BHV-Petersburg Publ., 2002, 608~p. (In Russian)


%\bibitem{Stenly}
19. Stanley~R.
 {\it Perechislitelnaya kombinatorica} [{\it Combinatorics for programmers}].
Moscow, Mir Publ., 1990, 440~p. (In Russian)


\vskip1.5mm Received:  February 02, 2020.

Accepted: February 13, 2020.


\vskip6mm A\,u\,t\,h\,o\,r\,s' \ i\,n\,f\,o\,r\,m\,a\,t\,i\,o\,n:%

\vskip2mm \textit{Alexander M. Kovshov} --- PhD in Physics and
Mathematics, Associate Professor;  a.kovshov@spbu.ru

}
