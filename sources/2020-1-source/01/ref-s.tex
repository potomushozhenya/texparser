
{\normalsize

\vskip 5mm%6mm

\noindent{\bf A system of models for constructing a progressive income %\\
tax schedule%$\,^*$%
}

}

\vskip 2mm

{\small

\noindent{\it S.~V.~Chistyakov$\,^1$%
, A.~N.~Kvitko$\,^1$%
, D.~B.~Kichinsky$\,^2$%
, M.~E.~Vasetsov$\,^2$%
, I.~S.~Uspasskaya$\,^1$%
%, I.~�. Famylia%$\,^2$%

 }

\vskip 2mm

%%%%%%%%%%%%%%%%%%%%%%%%%%%%%%%%%%%%%%%%%%%%%%%%%%%%%%%%%%%%%%%%%%

%\efootnote{
%%
%\vspace{-3mm}\parindent=7mm
%%
%\vskip 0.1mm $^{*}$ This work is supported by Russian Science
%Foundation (project N 18-71-00006).%\par
%%
%%\vskip 2.0mm
%%
%\indent{\copyright} �����-������������� ���������������
%�����������, \issueyear%
%%
%}

%%%%%%%%%%%%%%%%%%%%%%%%%%%%%%%%%%%%%%%%%%%%%%%%%%%%%%%%%%%%%%%%%%

{\footnotesize

\noindent%
$^1$~%
St.\,Petersburg State University, 7--9, Universitetskaya nab.,
St.\,Petersburg,

\noindent%
\hskip2.45mm%
199034, Russian Federation


\noindent%
$^2$~%
LANIT-TERCOM LLC, 2, lit. �, Chicherinskaya ul., Petergof,
St.\,Petersburg,

\noindent%
\hskip2.45mm%
198504, Russian Federation


%\noindent%
%$^2$~%
%St.\,Petersburg State University, 7--9, Universitetskaya nab.,
%St.\,Petersburg,

%\noindent%
%\hskip2.45mm%
%199034, Russian Federation

}

%%%%%%%%%%%%%%%%%%%%%%%%%%%%%%%%%%%%%%%%%%%%%%%%%%%%%%%%%%%%%%%%%%

\vskip3mm

\noindent \textbf{For citation:}   Chistyakov~S.~V., Kvitko~A.~N.,
Kichinsky~D.~B., Vasets\-ov~M.~E., Uspass\-kaya~I.~S. A system of
models for constructing a progressive income tax schedule. {\it
Vestnik of Saint~Peters\-burg Uni\-versity. Applied Mathematics.
Computer Science. Control Processes}, \issueyear, vol.~16,
iss.~\issuenum, pp.~\pageref{p1}--\pageref{p1e}.
\doivyp/\enskip%
\!\!\!spbu10.\issueyear.\issuenum01  (In Russian)

\vskip3mm

{\leftskip=7mm\noindent Modernization of the system of models
described earlier is presented. The new game-theoretic model of
constructing the average income tax rate schedule does not assume
the existence of a non-taxable minimum income, which is more
consistent with the real-world practice of using income tax in
recent decades. At the same time, the presented proofs of the main
statements related to this model are more elementary, in
particular, they do not rely on\linebreak Pontryagin's maximum
principle. For its part, in the new approximation model for
con\-structing progressive income tax schedules of marginal rates,
which has undergone the most radical modernization, excessively
rigid nonlinear restrictions on the class of acceptable
approximations are excluded, due to which, under weak assumptions
about the input parameters of both models, it was
possible to guarantee the non-emptiness of this class.\\[1mm]
\textit{Keywords}: progressive income tax, average tax rates
schedule, game-theoretic model, marginal tax rates schedule,
approximation model.
\par}

\vskip 6mm

\noindent \textbf{References} }

\vskip 2mm

{\footnotesize

1. Chistyakov~S.~V., Ishkhanova~M.~V. \emph{Matematicheskie modeli
vybora nalogovyh shkal} [\emph{Mathematical models for selection
of tax scales}]. Textbook. Saint Petersburg, Saint Peters\-burg
University Publ., 1998, 52 p. (In Russian)
%\bibitem{}

2. Aleksashenko~S.~V., Kiseljov~D.~A., Teplukhin~P.~M.,
Yasin~E.~G. Nalogovye shkaly: funkcii, svojstva, metody
upravlenija [Tax scales: functions, properties, control methods].
\emph{Economics and Mathematical Methods}, 1989, vol. 25, iss. 3,
pp. 389--395. (In Russian)
%\bibitem{}

3. Luce~R.~D., Raiffa~H. \emph{Games and decisions: Introduction
and critical survey}. New York, John Wiley and Sons Press, 1957,
531 p.
%\bibitem{}

4. Moulin~H. \emph{Theorie des jeux pour l�economie et la
politique} [\emph{Game theory with examples from mathematical
economy}]. Paris, Hermann Press, 1981, 262 p. (Russ. ed: Moulin~H.
\emph{Teoriya igr s primerami iz matematicheskoj ekonomiki}.
Moscow, Mir Publ., 1985, 199 p.)
%\bibitem{}

5.  Chistyakov~S.~V., Uspasskaya~I.~S., Kvitko~A.~N.,
Kichinsky~D.~B. A system of models for constructing a progressive
income tax scale. \emph{IFAC POL}, 2018, vol. 51, iss. 32, pp.
474--478.
%\bibitem{}

6. Smirnov~R.~O. \emph{Modelirovanie instrumentov
bjudzhetno-nalogovoj politiki gosudarstva} [\emph{Modeling tools
for fiscal policy of the state}]. Saint Petersburg, Saint
Petersburg University Publ., 2009, 112 p. (In Russian)
%\bibitem{}

7. Mirrlees~J.~A. An exploration in the theory of optimal income
taxation. \emph{Review of Economic Studies}, 1971, vol. 2, no. 38,
pp. 175--208.
%\bibitem{}

8. Nekipelov~D.~N. \emph{Raspredelitel'nye svojstva i
iskazhajushee vozdejstvie nalogov na indivi\-dual'nye dohody v
Rossii} [\emph{Distribution features and distorting effect of
taxes on individual incomes in Russia}]. Moscow, Gaidar Institute
for Economic Policy Publ., 2005, 175 p. (In Russian)
% \bibitem{}

9. Chistyakov~S.~V. Konflikty glazami matematikov [Conflicts
through the eyes of math\-e\-mat\-icians]. \emph{Saint Petersburg
State University}, 2001, no. 28, iss.~3582, November~25, p.~15.
(In Russian)
% \bibitem{}

10.  Chistyakov~S.~V., Andersen~A.~A., Vishnevskii~V.~E. A
game-theoretic model of a regressive profit tax. \emph{Applied
Mathematical Sciences}, 2015, vol. 9, no. 85, pp. 4201--4209.
% \bibitem{}

11. Niehans~J. Zur preisbildung bei ungewissen erwartungen [Price
formation with uncer\-tain expectations]. \emph{Schweizerische
Zeitschrift f$\ddot{u}$r Volkswirtschaft und Statistik}, 1948,
vol. 5, no. 5, pp. 433--456. (In German)
% \bibitem{}

12. Savage~L.~J. The theory of statistical decision. \emph{Journal
of the American Statisti\-cal Association}, 1951, vol. 46, no.
253, pp. 55--67.
%\bibitem{}

13. Andersen~A.~A., Chistyakov~S.~V. Metodologicheskie osnovy
razrabotki interaktivnoj systemy postrojenija shkaly stavok
podokhodnogo naloga [Methodological basis for con\-struct\-ing an
interactive system of income tax rate scale]. \emph{Vestnik of
Saint Petersburg Uni\-ver\-si\-ty. Series 10. Applied Mathematics.
Computer Science. Control Processes}, 2013, iss. 3, pp. 9--19. (In
Russian)

\vskip 1.5mm

Received:  October 29, 2019.

Accepted: February 13, 2020.


\vskip6mm A\,u\,t\,h\,o\,r\,s' \ i\,n\,f\,o\,r\,m\,a\,t\,i\,o\,n:


\vskip2mm \textit{Sergei V. Chistyakov} --- Dr. Sci. in Physics
and Mathematics, Professor; svch50@mail.ru

\vskip2mm \textit{Aleksandr N. Kvitko} --- Dr. Sci. in Physics and
Mathematics, Professor; alkvit46@mail.ru

\vskip2mm \textit{Dmitry B. Kichinsky} --- Specialist;
dmitry.kichinsky@gmail.com

\vskip2mm \textit{Matvei E. Vasetsov} --- PhD in Physics and
Mathematics; matvey\_v@yahoo.com

\vskip2mm \textit{Irina S. Uspasskaya} --- Magistrant;
irinausp023@gmail.com\par
%
}
