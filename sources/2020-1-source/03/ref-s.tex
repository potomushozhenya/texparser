
{\normalsize

\vskip 6mm%6mm

\noindent{\bf The use of geostatistical methods to analyze the transition feasibility\\
to the differential application of agrochemicals technologies$^{*}$%
}

}

\vskip 2mm%2mm

{\small

\noindent{\it V.~P.~Iakushev$\,^1$%
, V.~M.~Bure$\,^{1,2}$%
, O.~A.~Mitrofanova$\,^{1,2}$%
, E.~P.~Mitrofanov$\,^{1,2}$%
%, I.~�. Famylia%$\,^1$%

}

\vskip 2mm%2mm

%%%%%%%%%%%%%%%%%%%%%%%%%%%%%%%%%%%%%%%%%%%%%%%%%%%%%%%%%%%%%%%%%%

\efootnote{
%%
\vspace{-3mm}\parindent=7mm
%%
\vskip 0.1mm $^{*}$ This work was supported by the Russian
Foundation for Basic Reaserch (grant N 19-29-05184 mk).\par
%%
%%\vskip 2.0mm
%%
%%\indent{\copyright} �����-������������� ���������������
%%�����������, \issueyear%
%%
}

%%%%%%%%%%%%%%%%%%%%%%%%%%%%%%%%%%%%%%%%%%%%%%%%%%%%%%%%%%%%%%%%%%

{\footnotesize

\noindent%
$^1$~%
Agrophysical Research Institute, 14, Grazhdanskiy pr.,
St.\,Petersburg,

\noindent%
\hskip2.45mm%
195220, Russian Federation

\noindent%
$^2$~%
St.\,Petersburg State University, 7--9, Universitetskaya nab.,
St.\,Petersburg,

\noindent%
\hskip2.45mm%
199034, Russian Federation

}

%%%%%%%%%%%%%%%%%%%%%%%%%%%%%%%%%%%%%%%%%%%%%%%%%%%%%%%%%%%%%%%%%%

\vskip3mm%3mm

\noindent \textbf{For citation:}  Iakushev V.~P., Bure V.~M.,
Mitrofanova O.~A., Mitrofanov E.~P. The use of geo\-statistical
methods to analyze the transition feasibility to the differential
application of agro\-che\-micals technologies. {\it Vestnik of
Saint~Petersburg University. Applied Mathematics. Computer
Science. Control Processes},~\issueyear,
vol.~16,~iss.~\issuenum,~pp.~\pageref{p3}--\pageref{p3e}. \\
\doivyp/\enskip%
\!\!\!spbu10.\issueyear.\issuenum03  (In Russian)

\vskip3mm

{\leftskip=7mm\noindent In recent years, the tasks of precision
farming, including the technology of differential application of
agrochemicals, which significantly save resources, increase yield
and product quality, while reducing environmental impact, have
been particularly relevant and promising. However, for each
specific agricultural territory, an assessment of the prospects
for the transition to such technologies is necessary; it will not
always be justified. The paper proposes a method for solving this
problem, based on the use of variogram analysis. The idea is based
on a geostatistical model of territory heterogeneity, where the
spatial variability of the studied parameter $Z$ is represented as
the sum of three components: $m$~--- mac\-ro\-com\-po\-nent
reflecting global spatial variations of a parameter caused, for
example, by landscape features; $s$~--- me\-so\-com\-po\-nent that
describes the variability of the parameter within the scale of the
agricultural field; $\varepsilon$~--- mic\-ro\-com\-po\-nent
characterizing random micro-scale variability of the parameter. It
is assumed that for the effectiveness of the transition to the
differential application of agrochemicals in a specific
agricultural territory, it is necessary that the in-field
variation of agroecological indicators make a significant
contribution to the overall picture of field heterogeneity. The
paper also presents a computational experiment demonstrating the
efficiency of the proposed method, using aerial photographs, GIS
programs, and the programming
language R.\\[1mm]
\textit{Keywords}: variogram analysis, precision agriculture,
precision technology, geostatistical model.
\par}

\vskip6mm

\noindent \textbf{References} }

\vskip 2mm

{\footnotesize

1. {Bure~V.~M., Kanash~E.~V., Mitrofanova~O.~A.} Analiz
kharakteristik tsveta rastenii po aerofotosnimkam s razlichnymi
faktorami kachestvennykh pokazatelei [Analysis of plants color
characteristics using aerophotos with different factors of
qualitative indicators]. {\it Vestnik of Saint Petersburg
University. Applied Mathematics. Computer Science. Control
Processes,} 2017, vol.~13, iss.~3, pp.~278--285. (In Russian)

2. {Iakushev~V.~V.} {\it Tochnoe zemledelie: teoriia i praktika}
[{\it Precision agriculture: theory and practice}]. Saint
Petersburg, Agrophysical Institute Publ., 2016, 364~p. (In
Russian)

3. {Iakushev~V.~P., Kanash~E.~V., Konev~�.~�., Kovtiukh~S.~N.,
Lekomtsev~P.~V., Matveenko~D.~A., Petrushin~A.~F., Iakushev~V.~V.,
Bure~V.~M., Rusakov~D.~V., Osipov~Iu.~A.} {\it Teoreticheskie i
metodicheskie osnovy vydeleniia odnorodnykh tekhnologicheskikh zon
dlia differentsirovannogo primeneniia sredstv khimizatsii po
opticheskim kharakteristikam poseva.} Prakt. posobie  [{\it
Theoretical and methodological foundations for the separation of
homogeneous technological zones for the differentiated application
of chemicalization means based on the optical characteristics of
seeding.}  Practical allowance]. Saint Petersburg, Agrophysical
Institute Publ., 2010, 60~p. (In Russian)

4. {Bure V. M., Mitrofanov E. P., Mitrofanova O. A., Petrushin A.
F.} Vydelenie odnorodnykh zon sel'skokhoziaistvennogo polia dlia
zakladki opytov s pomoshch'iu bespilotnogo letatel'nogo apparata
[Selection of homogeneous zones of agricultural field for laying
of experiments using unmanned aerial vehicle]. {\it Vestnik of
Saint Petersburg University. Applied Mathematics. Computer
Science. Control Processes,} 2018, vol.~14, iss.~2, pp.~145--150.
(In Russian)

5. {Afanasyev R. A.} Use of the regularities of within-field
variability of arable soil fertility in precision
agrotechnologies. {\it Science Journal of Volgograd State
University. Natural sciences,} 2015, vol.~11, no.~1, pp.~42--51.

6. {Van Meirvenne M.} Is the soil variability within the small
fields of flanders structured enough to allow precision
agriculture? {\it Precision Agriculture,} 2003, vol.~4, iss.~2,
pp.~193--201.

7. {Iakushev V. P., Zhukovskii E. E., Petrushin A. F., Iakushev V.
V.} {\it Variogrammnyi analiz prostranstvennoi neodnorodnosti
sel'skokhoziaistvennykh polei dlia tselei tochnogo zemledeliia.}
Metodicheskoe posobie [{\it A variogram analysis of the spatial
heterogeneity of agricultural fields for precision agriculture}.
Toolkit]. Saint Petersburg, Agrophysical Institute Publ., 2010,
52~p. (In Russian)

8. {Dem'ianov~V.~V., Savel'eva~E.~A.} {\it Geostatistika: teoriia
i praktika} [{\it Geostatistics: theory and practice}]. Moscow,
Nuclear safety institute of the Russian Academy of Sciences, Nauka
Publ., 2010, 327~p. (In Russian)



\vskip1.5mm Received:  January 06, 2020.

Accepted: February 13, 2020.


\vskip6mm A\,u\,t\,h\,o\,r\,s' \ i\,n\,f\,o\,r\,m\,a\,t\,i\,o\,n:%

\vskip2mm \textit{Viktor P. Iakushev}  --- RAS academician, Dr.
Sci. in Agriculture; vyakushev@agrophys.com

\vskip2mm \textit{Vladimir M. Bure}  --- Dr. Sci. in Technics,
Professor; vlb310154@gmail.com

\vskip2mm \textit{Olga A. Mitrofanova}  --- Junior Researcher;
omitrofa@gmail.com

\vskip2mm \textit{Evgenii P. Mitrofanov}  --- Junior Researcher;
mjeka@bk.ru

}
