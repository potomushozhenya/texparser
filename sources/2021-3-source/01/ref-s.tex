
{\normalsize

\vskip 5%6
mm

\noindent{\bf Random search method for estimating the parameters of the emission\\ system signal$^{*}$%
}

}

\vskip 2mm

{\small

\noindent{\it M.\,I. Varayun', E.\,M. Vinogradova, A.\,Yu. Antonov%$\,^2$%

}

\vskip 2mm

%%%%%%%%%%%%%%%%%%%%%%%%%%%%%%%%%%%%%%%%%%%%%%%%%%%%%%%%%%%%%%%%%%

\efootnote{
%%
\vspace{-3mm}\parindent=7mm
%%
\vskip 0.1mm $^{*}$ This work was supported by the Russian Foundation for Basic Research (project N~20-07-01086).%\par
%%
%%\vskip 2.0mm
%%
%%\indent{\copyright} �����-������������� ���������������
%%�����������, \issueyear%
%%
}

%%%%%%%%%%%%%%%%%%%%%%%%%%%%%%%%%%%%%%%%%%%%%%%%%%%%%%%%%%%%%%%%%%

{\footnotesize

\noindent%
%$^1$~%
St.\,Petersburg State University, 7--9, Universitetskaya nab.,
St.\,Petersburg,

\noindent%
%\hskip2.45mm%
199034, Russian Federation


}

%%%%%%%%%%%%%%%%%%%%%%%%%%%%%%%%%%%%%%%%%%%%%%%%%%%%%%%%%%%%%%%%%%

\vskip2mm%3mm


\noindent \textbf{For citation:} Varayun' M.\,I., Vinogradova E.\,M., Antonov A.\,Yu. Random search method for estimating the parameters of the emission system signal. {\it Vestnik of Saint~Petersburg Uni\-ver\-si\-ty.
Ap\-plied Mat\-he\-matics. Computer Science. Control
Processes},\,\issueyear,
vol.~17,~iss.~\issuenum,~pp.~\pageref{p1}--\pageref{p1e}.  %\\
\doivyp/\enskip%
\!\!\!spbu10.\issueyear.\issuenum01 (In Russian)

\vskip2mm

{\leftskip=7mm\noindent Within the framework of a computer statistical experiment, the test problem of identifying the parameters of a field electron emission signal using a regression model based on the Fowler\,---\,Nordheim law is considered. Two approaches to determining the parameter estimates are used in the work: ordinary least squares and random search with training. It is shown that the random search error can be neglected if, for the considered ratios of the noise level to the signal power, the number of statistical tests is approximately $10^3$. The result allows us to expand the class of functionals used to identify the response without changing the method. The article notes the advantages of the random search method for the problem under consideration and the prospects of its applicability to tasks in a more general setting.\\[1mm]
\textit{Keywords}: field electron emission, current-voltage characteristic, regression model, least-squa\-res method, random search method. \par}

\vskip4.5%6
mm

\noindent \textbf{References} }

\vskip 2mm

{\footnotesize

1.~Bugaev A.\,S., Vinogradova E.\,M., Egorov N.\,V., Sheshin E.\,P. \textit{Avtojelektronnye katody i pushki} [\textit{Autoelectronic cathodes and guns}]. Dolgoprudny, ``Intellekt'' Publ., 2017, 288~p. (In Russian)

2.~Vinogradova E.\,M., Starikova A.\,V., Varayun� M.\,I.  Multipole electrostatic system mathematical modeling. \textit{Vestnik of Saint Petersburg University. Applied Mathematics. Com\-pu\-ter Science. Control Processes}, 2017, vol.~13, iss.~4, pp.~365--371. %\\
{https://doi.org/10.21638/11701/spbu10.2017.403}

3.~Egorov N.\,V., Sheshin E.\,P. \textit{Avtojelektronnaja jemissija. Principy i pribory} [\textit{Field emission. Principles and Instruments}]. Dolgoprudny, ``Intellekt'' Publ., 2011, 704~p. (In Russian)

4.~Fomenko V.\,S. \textit{Jemissionnye svojstva materialov.} Spravochnik [\textit{Emission properties of materials.} Reference book]. 4th ed. Kiev, Naukova Dumka Publ., 1981, 340~p. (In Russian)

5.~Egorov N.\,V., Antonov A.\,Yu., Varayun� M.\,I. Analysis of confidence intervals for regression model parameters, based on the Fowler\,---\,Nordheim law. \textit{Journal of Surface In\-ves\-ti\-ga\-tion: X-ray, Synchrotron and Neutron Techniques}, 2020, vol.~14, no.~4, pp.~730--737.

6.~Fowler R.\,H., Nordheim L.\,W. Electron emission in intense electric fields.  \textit{Proceedings of the Royal Society A. Mathematical, Physical and Engineering Sciences}, 1928, vol.~119, no.~781, pp.~173--181.

7.~Antonov A.\,Yu., Varayun� M.\,I., Egorov N.\,V. Linearizovannaja trjohparametricheskaja re\-gres\-sion\-naja model' dlja signala polevoj jelektronnoj jemissii [Linearized three-parameter regression model for the field electron emission signal]. \textit{Nano- and Microsystems Technology}, 2019, vol.~21, no.~2, pp.~103--110. (In Russian)

8.~Egorov N.\,V., Varayun� M.\,I., Bure V.\,M., Antonov A.\,Yu. Regression models for the field electron emission signal. \textit{Journal of Surface Investigation: X-ray, Synchrotron and Neutron Techniques}, 2020, vol.~14, no.~6, pp.~1394--1402.


9.~Egorov N.\,V., Antonov A.\,Yu., Gribkova I.\,M. Statistical test of a single semiempirical work function model. \textit{Journal of Surface Investigation: $X$-ray, Synchrotron and Neutron Techniques}, 2014, vol.~8, pp.~138--143.

10.~Egorov N.\,V., Antonov A.\,Yu., Varayun' M.\,I. Analysis of the emission characteristics of field cathodes using regression models. \textit{Journal of Surface Investigation: $X$-Ray, Synchrotron and Neutron Techniques}, 2018, vol.~12, pp.~1005--1012.

11.~Draper N., Smith H. \textit{Applied regression analysis}. 2nd ed. New York, Wiley Publ.,  1981, 736~p. (Rus. ed.: Draper N., Smith H. \textit{Prikladnoj regressionnyj analiz}. In 2 books. Book~1. 2nd ed. Moscow, Finansy i statistika Publ., 1986, 366~p.)

12.~Bure V.\,M., Parilina E.\,M., Sedakov A.\,A. \textit{Metody prikladnoj statistiki v R i Excel} [\textit{Methods of applied statistics in R and Excel}]. St. Petersburg, Lan' Publ., 2016, 152~p. (In Russian)


13.~Antonov A.\,Yu., Varayun� M.\,I., Gribkova I.\,M., Pigul� E.\,Yu. Mathematical modelling of the work function distribution on a monocrystalline cathode surface. \textit{2015 International Conference ``Stability\linebreak and Control Processes'' in memory of V.~I.~Zubov (SCP)}. St. Petersburg,  Publ. House Fedorova~G.~V.,  2015, pp.~144--147.


14.~Vladimirova L.\,V., Ovsjannikov D.\,A., Rubcova I.\,D. \textit{Metody Monte-Karlo v prikladnyh zadachah} [\textit{The Monte Carlo method in applied tasks}]. St. Petersburg, VVM Publ., 2015, 167~p. (In Russian)

15.~Sobol'~I.\,M. \textit{Chislennye metody Monte-Karlo} [\textit{Numerical Monte Carlo methods}]. Moscow, Nauka Publ., 1973, 312~p. (In Russian)

16.~Bolshev L.\,N., Smirnov N.\,V. \textit{Tablicy matematicheskoj statistiki} [\textit{Tables of mathema\-tical statistics}]. Moscow, Nauka Publ., 1983, 416~p. (In Russian)

17.~Lehmann E.\,L. Consistency and unbiasedness of certain nonparametric tests. \textit{The Annals of Mathematical Statistics}, 1951, vol.~22, no.~2, pp.~165--179.

18.~Rosenblatt M. Limit theorems associated with variants of the von Mises statistic. \textit{The Annals of Mathematical Statistics}, 1952, vol.~23, no.~4, pp.~617--623.

19.~Anderson T.\,W. On the distribution of the two-sample Cram\'{e}r\,---\,von Mises criterion. \textit{The Annals of Mathematical Statistics}, 1962, vol.~33, no.~3, pp.~1148--1159.

20.~Tyurin Yu.\,N., Makarov A.\,A. \textit{Analiz dannyh na komp'jutere} [\textit{Data analysis on the computer}]. Ed. by V.\,E.~Figurnov. Moscow, INFRA-M Publ., 2003, 544~p. (In Russian)

21.~Prohorov I.\,D., Surtaeva M.\,N., Antonov A.\,Yu. Ispol'zovanie ravnomernyh pos\-le\-do\-va\-tel'\-nos\-tej dlja parametricheskoj identifikacii signala polevoj jelektronnoj jemissii [Use of uniform sequences for the parametric identification of the field electron emission signal]. \textit{Control Processes and Stability}, 2018, vol.~5, no.~1, pp.~191--195. (In Russian)

22.~Hu Zh., Yang R.\,C. A new distribution-free approach to constructing the confidence region for multiple parameters. \textit{PLoS One}, 2013, vol.~8, no.~12, pp.~1--13.

\vskip1.5mm

Received:  March 18, 2021.

Accepted: June 04, 2021.


\vskip4.5%6
mm A~u~t~h~o~r~s' \ i~n~f~o~r~m~a~t~i~o~n:%

\vskip2mm \textit{Marina I. Varayun'}  --- PhD in Physics and Mathematics, Associate Professor; m.varayuan@spbu.ru \par%
%
\vskip2mm \textit{Ekaterina M. Vinogradova} --- Dr. Sci. in Physics and Mathematics, Professor;\\ e.m.vinogradova@spbu.ru \par%
%
\vskip2mm \textit{Andrei Yu. Antonov}  --- PhD in Physics and Mathematics, Associate Professor; a.antonov@spbu.ru \par%
%
}
