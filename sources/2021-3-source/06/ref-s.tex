
{\normalsize

%\vskip 6mm
\newpage

\noindent{\bf An example of an internal function for the SPONGE scheme$^{*}$%
}

}

\vskip 3%2
mm

{\small

\noindent{\it R. M. Ospanov, Ye. N. Seitkulov, N. M. Sissenov, B. B. Yergaliyeva%$^2$%
%, I.~�. Famylia%$\,^2$%

}

\vskip 3%2
mm

%%%%%%%%%%%%%%%%%%%%%%%%%%%%%%%%%%%%%%%%%%%%%%%%%%%%%%%%%%%%%%%%%%

\efootnote{
%%
\vspace{-3mm}\parindent=7mm
%%
\vskip 0.1mm $^{*}$ This work was supported by the �Ministry of Digital Development, Innovations and Aerospace In\-dustry of the Kazakhstan Republic� (project N~AP06851124).%\par
%%
%%\vskip 2.0mm
%%
%%\indent{\copyright} �����-������������� ���������������
%%�����������, \issueyear%
%%
}

%%%%%%%%%%%%%%%%%%%%%%%%%%%%%%%%%%%%%%%%%%%%%%%%%%%%%%%%%%%%%%%%%%

{\footnotesize


\noindent%
%$^1$~%
Gumilyov Eurasian National University, 2, ul. Satpayeva, Nur-Sultan,

\noindent%
%\hskip2.45mm%
010000, Kazakhstan


%\noindent%
%%$^1$~%
%St.\,Petersburg State University, 7--9, Universitetskaya nab.,
%St.\,Petersburg,
%
%\noindent%
%%\hskip2.45mm%
%199034, Russian Federation


}

%%%%%%%%%%%%%%%%%%%%%%%%%%%%%%%%%%%%%%%%%%%%%%%%%%%%%%%%%%%%%%%%%%

\vskip3mm%3mm


\noindent \textbf{For citation:}  Ospanov R. M., Seitkulov Ye. N., Sissenov N. M., Yergaliyeva B. B. An example of an internal function for the SPONGE scheme. {\it Vestnik of Saint~Petersburg Uni\-ver\-si\-ty.
Ap\-plied Mat\-he\-matics. Computer Science. Control
Processes},\,\issueyear,
vol.~17,~iss.~\issuenum,~pp.~\pageref{p6}--\pageref{p6e}. \\
\doivyp/\enskip%
\!\!\!spbu10.\issueyear.\issuenum06 (In Russian)

\vskip3mm

{\leftskip=7mm\noindent The article discusses a new version of the internal function underlying the perspective modern scheme for constructing cryptographic hash functions Sponge (cryptographic sponge). The con\-si\-dered example of an internal function is similar to the Keccak permutation, but it has a number of main differences. The inner function operates on a 2048-bit state $S$, which can be viewed as a three-dimensional bit array of $4 \times 8 \times 64$ size. The structure of the internal function is made up of 5 transformations similar to Keccak. However, firstly, in this example, instead of a 5-bit $S$-box, an 8-bit one is used. In this regard, the parameters of the three-dimensional representation of the state have been changed. Secondly, instead of a linear feedback shift register, a dictionary shift register with ring carry feedback is used to generate round constants. The properties of these transformations are analyzed in the work.\\[1mm]
\textit{Keywords}: information security, cryptography, hash function, Sponge modification, sym\-met\-ric encryption.
\par}

\vskip6mm

\noindent \textbf{References} }

\vskip 2mm

{\footnotesize

1. Bertoni G., Daemen J., Peeters M., Assche G. V. {\itshape Sponge functions}. Barcelona, Ecrypt Hash Workshop Press, 2007, 22~p. Available at: http://keccak.team/files/SpongeFunctions.pdf (accessed: No\-vember 15, 2020).
	
2. Bertoni G., Daemen J., Peeters M., Van Assche G. V. {\itshape Cryptographic sponge functions}. Version 0.1. January 14, 2011.  Available at: http://keccak.team/files/CSF-0.1.pdf (accessed: November 15, 2020).
	
3. Bertoni G., Daemen J., Peeters M., Van Assche G.  V. {\itshape The Keccak reference}. SHA-3 competition (round 3), 2011. Available at: http://keccak.team/sponge\_duplex.html (accessed: November 15, 2020).
	
4. Arnault F., Berger T. P., Lauradoux C., Minier M., Pousse B. A new approach for FCSRs. {\itshape Cryptology ePrint Archive}. Report 2009/167, 2009. Available at: http://eprint.iacr.org/2009/167 (accessed: November 15, 2020).
	
5. Nizam Chew L. C., Ismail E. S. $S$-box construction based on linear fractional transformation and permutation function. {\itshape Symmetry}, 2020, vol.~12, Art. no.~826. https://doi.org/10.3390/sym12050826
	
6. Zahid A. H., Arshad M. J. An innovative design of substitution-boxes using cubic polynomial mapping. {\itshape Symmetry}, 2019, vol.~11, Art. no.~437. https://doi.org/10.3390/sym11030437
	
7. Altaleb A., Saeed M. S., Hussain I., Aslam M. An algorithm for the construction of substitution box for block ciphers based on projective general linear group. {\itshape AIP Advances}, 2017, vol.~7, Art. no.~035116. https://doi.org/10.1063/1.4978264
	
8. Hussain S., Jamal S. S., Shah T., Hussain I. A power associative loop structure for the construction of non-linear components of block cipher. {\itshape IEEE Access}, 2020, vol.~8, pp.~123492--123506.
	
9. Gao W., Idrees B., Zafar S., Rashid T. Construction of nonlinear component of block cipher by action of modular group $PSL(2, Z)$ on projective line $PL(GF(2^8))$. {\itshape IEEE Access}, 2020, vol.~8, pp.~136736--136749.
	
10. Kazymyrov O. {\it Metody i sredstva generatsii nelineinykh uzlov zameny dlia simmetrichnykh krip\-toalgoritmov.} Dissertatsiia na soiskanie uchenoi stepeni kandidata tekhnicheskikh nauk  [{\itshape Methods and tools to generate nonlinear substitution boxes for symmetric cryptographic algorithms}. Diss. PhD in Technics]. Kharkiv, Kharkiv National University of Radio Electronics, 2013, 190~p. (In Russian)\pagebreak
	
11. Rodinko M., Oliynykov R., Gorbenko Y. {\itshape Optimization of the high nonlinear $S$-boxes generation method}. Bratislava, Tatra Mountains Mathematical Publ., Mathematical Institute, Slovak Academy of Sciences, 2017, vol.~70, iss.~1, pp.~93--105.
	
12. Ivanov G., Nikolov N., Nikova S. Cryptographically strong $S$-boxes generated by modified immune algorithm. {\itshape Cryptography and Information Security in the Balkans (BalkanCryptSec 2015)}. Eds by E.~Pasalic, L.~Knudsen. Cham, Springer Publ., 2016, pp.~31--42. (Lecture Notes in Computer Science, vol.~9540.)
	
13. Gorbenko I., Kuznetsov A., Gorbenko Y., Pushkar�ov A., Kotukh Y., Kuz\-net\-sova~K. Random $S$-boxes generation methods for symmetric cryptography // {\itshape 2019 IEEE 2nd Ukraine Conference on Electrical and Computer Engineering (UKRCON)}. Lviv, Ukraine, 2019, pp.~947--950.
	
14. Easttom C. A generalized methodology for designing non-linear elements in symmetric cryptographic primitives. {\itshape 2018 IEEE 8th Annual Computing and Communication Workshop and Con\-fe\-ren\-ce (CCWC)}. Las Vegas, NV Press, 2018, pp.~444--449.

\vskip1.5mm
Received:  January 08, 2021.

Accepted:
June 04, 2021.


\vskip6mm A~u~t~h~o~r~s' \ i~n~f~o~r~m~a~t~i~o~n:%


\vskip2mm \textit{Ruslan M. Ospanov} --- Elder Researcher; ospanovrm@gmail.com

\vskip2mm \textit{Yerzhan N. Seitkulov} --- PhD in Physics and Mathematics, Associate Professor;\\ yerzhan.seitkulov@gmail.com

\vskip2mm \textit{Nurbek M. Sissenov} --- Researcher; nurbek9291@mail.ru

\vskip2mm \textit{Banu B. Yergalieva} --- Researcher; ergalieva\_banu@mail.ru

}
