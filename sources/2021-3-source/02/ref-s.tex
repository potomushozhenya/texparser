
{\normalsize

\vskip 6%
mm

\noindent{\bf 3/2-approximation algorithm for a single machine scheduling problem %$^{*}$
}

}

\vskip 3%
mm

{\small

\noindent{\it N. S. Grigoreva%$\,^2$%
%, I.~�. Famylia%$\,^2$%

 }

\vskip 3%
mm

%%%%%%%%%%%%%%%%%%%%%%%%%%%%%%%%%%%%%%%%%%%%%%%%%%%%%%%%%%%%%%%%%%

%\efootnote{
%%
%\vspace{-3mm}\parindent=7mm
%%
%\vskip 0.1mm \indent\indent~$^{*}$ This work was  supported by the%
%Russian Foundation for Basic Research (grant N 19-01-00146-a).\par
%%
%%\vskip 2.0mm
%%
%%\indent{\copyright} �����-������������� ���������������
%%�����������, \issueyear%
%%
%}

%%%%%%%%%%%%%%%%%%%%%%%%%%%%%%%%%%%%%%%%%%%%%%%%%%%%%%%%%%%%%%%%%%

{\footnotesize



\noindent%
%$^2$~%
St.\,Petersburg State University, 7--9, Universitetskaya nab., St.\,Petersburg,

\noindent%
%\hskip2.45mm%
199034, Russian Federation

}

%%%%%%%%%%%%%%%%%%%%%%%%%%%%%%%%%%%%%%%%%%%%%%%%%%%%%%%%%%%%%%%%%%

\vskip 2,5%3%
mm


\noindent \textbf{For citation:}  Grigoreva N. S.  3/2-approximation algorithm for a single machine scheduling prob\-lem. {\it Vest\-nik of Saint Petersburg
University. Applied Mathe\-ma\-tics. Computer
Science. Cont\-rol Pro\-cesses}, %\,
\issueyear,
vol.~17, iss.~\issuenum,~pp.~\pageref{p2}--\pageref{p2e}. %\\
\doivyp/\enskip%
\!\!\!spbu10.\issueyear.\issuenum02 (In Russian)

%\vskip3mm

\pagebreak

{\leftskip=7mm\noindent The problem of minimizing the maximum delivery times while scheduling tasks on a single processor is a classical combinatorial optimization problem. Each task $u_i$  must be processed without interruption for $ t (u_i)$ time units on the machine, which can process at most one task at time. Each task $u_i$ has a release time  $r (u_i)$, when the task is ready for processing, and a delivery time $q (u_i)$. Its delivery begins immediately after processing has been completed. The objective is to minimize the time, by which all jobs are delivered. In the Graham notation this problem is denoted by $1|r_j,q_j|C_{\max},$  it has many applications and it is NP-hard in a strong sense. The problem is useful in solving owshop and jobshop scheduling problems. The goal of this article is to propose a new 3/2-approximation algorithm, which runs in $O(n\log n)$ times for scheduling problem $1|r_j,q_j|C_{\max}$. An example is provided which shows that the bound of 3/2 is accurate. To compare the effectiveness of proposed algorithms, random generated problems of up to 5000 tasks were tested.
\\[1mm]
\textit{Keywords}: single-machine scheduling problem, realize and delivery times,  approximation al\-go\-rithm,  guarantee approximation ratio.\par}

\vskip6mm
%\pagebreak

\noindent \textbf{References} }

\vskip 3mm

{\footnotesize

1. {Graham R. L., Lawler E. L., Lenstra J. K., Rinnooy~Kan A.~H.~G.} Optimization and approximation in deterministic sequencing and scheduling: A survey. {\it  Annals of Discrete Mathematics}, 1979, vol.~5, no.~10, pp.~287--326.


2. {Lenstra J. K., Rinnooy~Kan A.~H.~G.,  Brucker P.} Complexity of machine scheduling problems. {\it  Annals of Discrete Mathematics}, 1977, vol.~1,  pp.~343--362.

3. {Lazarev �. �., Sadykov R. R., Sevastyanov S. V.} Shema priblizhennogo resheniya zadachi   $1|r_i|L_{\max}$  [A scheme of approximation solution of problem  $1|r_i|L_{\max}$]. {\it Diskretny analiz i issedovanie operatcii} [{\it Discrete analyze and operation research}]. {\it Series~2},  2006, vol.~13, no.~1, pp.~57--76. (In Russian)

4. {Baker K. R.} {\it Introduction to sequencing and scheduling}. New York, John Wiley $\&$ Son Publ., 1974, 318~p.

5. {Kanet J. J., Sridharan V.} Scheduling with inserted idle time: problem taxonomy and literature review. {\it Operations Research}, 2000, vol.~48, no.~1, pp.~99--110.

6. Grigoreva N. S. Algoritm vetvey i granits dlya zadachi sostavleniya raspisaniya na parallel'nykh pro\-ces\-sorah [Branch and bound algorithm for multiprocessor scheduling problem]. {\it Vestnik of Saint Petersburg University. Series 10. Applied Mathematics. Computer Science. Control Processes}, 2009, iss.~1, pp.~44--55. (In Russian)

7. Grigoreva N. S. Zadacha  minimizatcii makcimalnogo vremennogo smeceniya  dlya parallel'nykh processorov [Scheduling problem to minimize the maximum lateness for parallel processors]. {\it Vestnik of Saint Petersburg University. Series 10. Applied Mathematics.  Computer Science. Control Processes}, 2016, iss.~4, pp.~51--65. (In Russian)

8. {Grigoreva N. S.}  Multiprocessor scheduling with inserted idle time to minimize the maximum lateness. {\it Proceedings of the 7th Multidisciplinary International Conference of Scheduling. Theory and Applications.} Prague, MISTA Publ., 2015, pp.~814--816.

9. {Grigoreva N. S.}   Single machine inserted idle time scheduling
with release times and due da\-tes. {\it Proceedings. DOOR2016.} Vladivostok, Russia, September 19--23, 2016. Ceur-WS, 2016, vol.~1623,\linebreak pp.~336--343.

10. {Schrage L.} {\it Optimal solutions to resource constrained network scheduling problems}. Unpublished manuscript, 1971.

11. {Kise H., Ibaraki T., Mine H.} Performance analysis of six approximation algorithms for the one-machine maximum  lateness scheduling problem with  ready times. {\it  Journal of the Operations  Research Society of Japan}, 1979, no.~22, pp.~205--224.

12. {Potts C. N.} Analysis of a heuristic for one machine sequencing with release dates and delivery times. {\it Operations Research}, 1980, vol.~28, pp.~1436--1441.

13. {Hall L. A., Shmoys D. B.} Jackson's rule for single-machine scheduling: making a good heuristic better. {\it Mathematics of Operations Research}, 1992, vol.~17, no.~1, pp.~22--35.

14. {Nowicki E., Smutnicki C.} An approximation algorithm for a single-machine scheduling problem with release times and delivery times.  {\it Discrete Applied Mathematics}, 1994, vol.~48, pp.~69--79.

15. {Hall  L. A.,  Shmoys D. B.} Approximation algorithms for constrained  scheduling problems. {\it  Proceedings of the 30th IEEE Symposium on Foundations of Computer Science}, IEEE Computer Society Press, 1989, pp.~134--139.

16. {Mastrolilli  M.} Efficient approximation schemes for scheduling problems with release dates and delivery times. {\it  Journal of Scheduling}, 2003, vol.~6, pp.~521--531.

17. {Carlier J.} The one machine sequencing problem. {\it European Journal of Operational Research}, 1982, vol.~11,  pp.~42--47.


\vskip1.5mm Received:  February 12, 2021.

Accepted: June 04, 2021.

\vskip6mm A\,u\,t\,h\,o\,r'\,s \,\ i\,n\,f\,o\,r\,m\,a\,t\,i\,o\,n:%

\vskip2mm \textit{Natalia S. Grigoreva} --- PhD in Physics and Mathematics, Associate Professor; n.s.grig@gmail.com \par
%
%
}
