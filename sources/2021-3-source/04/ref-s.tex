
{\normalsize

\vskip 6mm

\noindent{\bf PC-solutions and quasi-solutions of the interval system\\ of linear algebraic equations%$^*$%
 }

}

\vskip 2mm

{\small

\noindent{\it S. I. Noskov$^1$,  A. V. Lakeyev$^2$%$^{1,4}$%
%, I.~�. Famylia%$~^2$%

 }

\vskip 2mm

%%%%%%%%%%%%%%%%%%%%%%%%%%%%%%%%%%%%%%%%%%%%%%%%%%%%%%%%%%%%%%%%%%

%\efootnote{
%%
%\vspace{-3mm}%
%\parindent=7mm
%%
%\vskip 0.0mm
%\hskip 4mm$^{*}$ This work was supported by the Russian Science% %Foundation (project N 20-71-10032).%\par
%%
%%\vskip 2.0mm
%%
%\indent{\copyright} �����-������������� ���������������
%�����������, \issueyear%
%%
%}

%%%%%%%%%%%%%%%%%%%%%%%%%%%%%%%%%%%%%%%%%%%%%%%%%%%%%%%%%%%%%%%%%%

{\footnotesize


\noindent%
$^1$~%
Irkutsk State Transport University, 15, ul. Chernyshevskogo, Irkutsk,

\noindent%
\hskip2.45mm%
664074, Russian Federation


\noindent%
$^2$~%
Matrosov Institute for System Dynamics and Control Theory of the Siberian Branch

\noindent%
\hskip2.45mm%
of the Russian Academy of Sciences, 134, ul. Lermontova, Irkutsk, 664033, Russian Federation




%\noindent%
%$^2$~%
%St.~Petersburg State University, 7--9, Universitetskaya nab.,
%St.~Petersburg,
%
%\noindent%
%\hskip2.45mm%
%199034, Russian Federation

}

%%%%%%%%%%%%%%%%%%%%%%%%%%%%%%%%%%%%%%%%%%%%%%%%%%%%%%%%%%%%%%%%%%
\newpage
%\vskip2.0mm%3mm

\noindent \textbf{For citation:} Noskov  S. I.,  Lakeyev A. V. PC-solutions and quasi-solutions of the interval system of linear algebraic
equations. {\it Vestnik of Saint~Peters\-burg Uni\-versity.
Applied Mathe\-ma\-tics. Computer Science. Control Processes},
\issueyear,
vol.~17, iss.~\issuenum, pp.~\pageref{p4}--\pageref{p4e}.  \\
\doivyp/\enskip%
\!\!\!spbu10.\issueyear.\issuenum04 (In Russian)

\vskip2.0%3
mm

{\leftskip=7mm\noindent The problem of solving the interval system of linear algebraic equations (ISLAEs) is one of the well-known problems of interval analysis, which is currently undergoing intensive development. In general, this solution represents a set, which may be given differently, de\-pen\-ding on which quantifiers are related to the elements of the left and right sides of this system. Each set of solutions of ISLAE to be determined is described by the domain of compatibility of the corresponding system of linear inequalities and, normally, one nonlinear condition of the type of complementarity. It is difficult to work with them when solving specific problems. Therefore, in the case of nonemptiness in the process of solving the problem it is recommended to find a so-called PC-solution, based on the application of the technique known in the theory of multi-criterial choice, that presumes maximization of the solving capacity of the system of inequalities. If this set is empty, it is recommended to find a quasi-solution of ISLAE. The authors compare the approach proposed for finding PC- and/or quasi-solutions to the approach proposed by S. P. Shary, which is based on the application of the recognizing functional.\\[1mm]
\textit{Keywords}: interval system of linear algebraic equations, AE-solutions, ��-solution, quasi-solution, recognizing functional, problem of linear programming. \par}

\vskip 4%6
mm

\noindent \textbf{References} }

\vskip 2 mm

{\footnotesize

1.  Kearfott R. B.,  Nakao M. T.,  Neumaier A., Rump S. M., Shary S. P., Hentenryck P. Standardized notation in interval analysis.  {\it Computational Technologies},  2010, vol.~15, no.~1, pp.~7--13. Available at: http://www.ict.nsc.ru/jct/getfile.php?id=1345 (accessed: June 27, 2021).
	
2.  Shary S. P. {\it  Konechnomerny intervalny analiz} [{\it  Finite-dimensional interval analysis}]. Novosibirsk, XYZ Publ., 2020, 643~p. Available at:
http://www.nsc.ru/interval/Library/InteBooks/SharyBook.pdf (accessed: June 27, 2021). (In Russian)
	
3. Shary S. P. Algebraic solutions to interval linear equations and their applications. {\it  Numerical Methods and Error Bounds. Proceedings of IMACS-GAMM International Sym\-po\-si\-um on Numerical Methods and Error Bounds}. Oldenburg, Germany, July 9--12, 1995.  Eds by G.~Alefeld, J.~Herzberger. (Mathematical Research, vol.~89).  Berlin, Akademie Verlag Publ., 1996, pp.~224--233. Available at: http://interval.ict.nsc.ru/shary/Papers/Herz.pdf (accessed: June 27, 2021).
	
4. Shary S. P. Outer estimation of generalized solution sets to interval linear systems. {\it 	Reliable Computing}, 1999, vol.~5, pp.~323--335. Available at: http://interval.ict.nsc.ru/shary/Papers/GOuter.pdf (accessed: June 27, 2021).
	
5.  Shary S. P. Vneshneye otsenivaniye obobshennikh mnozhestv reshenyi intervalnikh lineynykh system [External assessment of generlized sets of solutions of interval linear systems]. {\it       Vychislitelniye tekhnologii} [{\it  Computational Terchnologies}], 1999, vol.~4, no.~4, pp.~82--110. (In Russian)
	
6. Oettli W., Prager  W.  Compatibility  of  approximate
solution  of  linear equations  with  given  error   bounds   for
coefficients and right-hand  sides. {\it  Num.
Math.}, 1964, vol.~6, pp.~405--409.

7. Oettli W. On the solution set of a linear system
with inacurate coefficients. {\it  SIAM J. Numer. Anal.}, 1965, vol.~2, pp.~115--118.

8. Beeck H. \"Uber structur und abschatzungen der
loesungsmenge von linear gle\-ichungs\-systemen mit
intervall koeffizienten [About the structure and estimates of the
set of solutions of systems of linear equations with
interval coefficients]. {\it  Computing}, 1972, vol.~10, pp.~231--244.

9. Rohn J. Interval linear systems. {\it  Freiburger Intervall-Berichte}. Freiburg, Albert-Ludwigs-Uni\-ver\-si\-taet Publ.,  1984, no.~84/7, pp.~33--58. 	
	
10.  Shary S. P. {\it  O kharakterizatsii obyedinennogo mnozhestva reshenii  intervalnoy lineynoy al\-geb\-rai\-ches\-koy systemy} [{\it  On characterization of an integrated set of solutions of an interval linear algebraic system}]. Deposited in VINITI, 1990, no.~726--891, 20~p. (In Russian)
	
11. Nuding E.,  Wilhelm J. \"Uber  Gleichungen und \"uber  L\"osungen [About equations and about solutions]. {\it Zeitschrift f\"ur
An\-ge\-wan\-dte Mathematik und Mechanik}, 1972, vol.~52, pp.~T188--T190.

12. Rohn J. Inner solutions of linear interval systems. {\it  Interval Mathematics 1985}. Berlin, Heidelberg, Springer-Verlag Publ., 1986, pp.~157--158. (Lecture Notes in Computer Science, vol.~212.)

13. Deif A. {\it Sensitivety analysis in linear systems}. Berlin, Heidelberg, Springer-Verlag Publ., 1986, 224~p.

14.  Neumaier A. Tolerance analysis with interval arithmetic. {\it Freiburger Intervall-Berichte}.  Freiburg, Albert-Ludwigs-Universit\"at Publ., 1986, no.~86/9, pp.~5--19.
	
15.  Shaidurov V. V., Shary S. P. {\it  Resheniye intervalnoy algebraicheskoy zadachi o dopus\-kakh}. Preprint VC Academii Nauk SSSR [{\it  Solving an interval algebraic problem of tolerances}.  Preprint of the Computing Center of USSR Acad. Sci.]. Krasnoyarsk, Acad. Sci. of USSR Press, 1988, no.~5, 27~p. (In Russian)
	
16.  Shary S. P. O razreshimosti lineynoy zadachi o dopuskakh [On solvability of the linear problem of tolerances]. {\it  Intervalniye vychisleniya} [{\it Interval Computations}], 1991, no.~1, pp.~92�98. (In Russian)
	
17.  Khlebalin N. A., Shokin Yu. I.  Intervalnii variant metoda modalnogo upravleniya [The interval variant of the method of modal control]. {\it Doklady Academii nauk SSSR} [{\it USSR Mathematical Papers}], 1991, vol.~316, no.~4, pp.~846--850. (In Russian)
	
18.  Lakeyev A. V., Noskov S. I. Opisaniye mnozhestv resheniy lineynogo intervalnogo uravneniya v uporyadochennom prostranstve [Description of sets of solutions of linear interval equation in the ordered set]. {\it  Sbornik trudov Mezhdunarodnoy konferentsii po intervalnym i stokhasicheskim metodam v nauke i tekhnike} [{\it Proceedings of the International conference on interval and stochastic methods in science and engineering}] (INTERVAL-92, September 22--26).  Moscow, Moscow Energy Institute Publ., 1992, vol.~1,  pp.~87--89. (In Russian)
	
19. Shary S. P. On controlled solution set of interval algebraic systems.
{\it Interval Com\-pu\-ta\-ti\-ons},  1992, vol.~4~(6), pp.~66--75.
	
20. Shary S. P. Controllable solution sets to interval static systems.
{\it Applied Mathematics and Computation}, 1997, vol.~86, no.~2--3, pp.~185--196.
	
21. Rohn J. E-mail letter to S. P. Shary and A. V. Lakeyev of November 18, 1995. Available at:
http://www.cs.cas.cz/~rohn/publist/LettShaLak.ps (accessed: June 27, 2021).
	
22. Rohn J.  $(Z,z)$-{\it solutions}. Technical Report no.~1159.  Prague, Institute of Computer Science, Academy of Sciences of the Czech Republic Publ., 2012, 3~p. Available at:\\ http://uivtx.cs.cas.cz/$\sim$rohn/publist/zzsols.pdf  (accessed: June 27, 2021).	
	
23.  Lakeyev A. V. Computational complexity of estimation of gen�ralized solution sets for interval linear systems. {\it Vychislitelnye tekhnologii} [{\it Computational Terchnologies}], 2003, vol.~8, no.~1, pp.~12--23.
	
24.   Khachiyan L. G. Polynomialniye algoritmy v lineynom programmirovanii [Polynomial algorithms in linear programming]. {\it Zhurnal vychislitelnoy matematiki i matematicheskoy fiziki} [{\it Journal of Com\-pu\-ta\-tional Mathematics and Mathematical Physics}], 1980, vol.~20, no.~1, pp.~51--68.  (In Russian)
	
25. Karmarkar N. A new polynomial-time algorithm for linear programming.
{\it Com\-bi\-na\-to\-ri\-ca},  1984, vol.~4,  pp.~373--395. https://doi.org/10.1007/BF02579150
	
26.  Lakeyev A. V., Noskov S. I. Opisaniye mnozhesva reshenii lineynogo uravneniya s intervalno zadannymi operatorom i pravoy chastyu [Description of a set of solutions for a linear equation with an interval given operator and a right-hand side]. {\it Doklady Akademii nauk USSR} [{\it USSR Mathematical Papers}], 1993, vol.~330~(4), pp.~430--433. (In Russian)
	
27.  Lakeyev A. V., Noskov S. I. O mnozhestve reshenii lineynogo uravneniya s intervalno zadannymi operatorom i pravoy chastyu [On a set of solutions of a linear equation with an interval given operator and the right-hand side]. {\it Sibirskii Matematicheskii Zhurnal} [{\it Siberian Mat\-he\-ma\-ti\-cal Journal}], 1994, vol.~35, no.~5, pp.~1074--1084. (In Russian)
	
28. Lakeyev A. V. On the computational complexity  of the solution of  linear systems with moduli. {\it Reliable Computing},  1996,  vol.~2, no.~2, pp.~125--131.
	
29.  Noskov S. I. {\it Tekhnologiya modelirovanita obyektov s nestabilnym funktsionirovaniyem i neo\-pre\-de\-len\-nos\-tyu v dannykh} [{\it A technology of modeling objects with unstable functioning and uncertainty in data}]. Irkutsk, Oblformpechat Press, 1996, 320~p.
Available at:\\
http://www.researchgate.net/publication/340570185 (accessed: June 27, 2021). (In Russian)
	
30.  Noskov S. I. Tochechnaya kharakterizatsiya mnozhestv reshenyi intervalnikh system lineynikh algebraicheskikh uravnenii [Pointwise characterization of the sets of solutions of interval systems of li\-near algebraic equations]. {\it Informatsionnyiye tekhnologii i ma\-te\-ma\-ti\-ches\-ko\-ye modelirovaniye v upravlenii slozhnymi sistemami} [{\it Information Technologies and Mathematical Modeling in Control of Complex Sys\-tems}], 2018, no.~1~(1), pp.~8--13.  (In Russian)
	
31.  Vassilyev S. N., Seledkin A. P.
Sintez funkcii effektivnosti v mnogokriterial'nyh zadachah pri\-nya\-tiya reshenij [Synthesis of the function of efficiency in multi-criterion problems of decision making]. {\it Iz\-ves\-tiya Aca\-demii nauk USSR. Tekhnicheskaya kibernetika} [{\it Proceedings of USSR Academy of Sciences. Tech\-ni\-cal Cybernetics}], 1980, no.~3, pp.~186--190. (In Russian)
	
32.  Noskov S. I., Protopopov V. A. Otsenka urovnya uyazvimosti obyektov transportnoy in\-fra\-struk\-tury: Formalizovanni podhod [Assessment of the level of vulnerability of the objects of transport infrastructure: A formalized approach]. {\it Sovremenniye Tekhnologii. Sistemny Analiz. Modelirovaniye} [{\it Con\-tem\-porary Technologies. Systems Analysis. Modeling}], 2011, no.~4 (32), pp.~241--244. (In Russian)
	
33. Cottle R. W., Pang J. H., Stone R. E. {\it The linear complementarity problem}.  Boston, Academic Press,  1992, 761~p.
	
34.  Tikhonov A. N., Arsenin V. Ya. {\it Metody resheniya nekorrektnykh zadach} [{\it Methods for solving incorrect problems}]. Moscow, Nauka Publ., 1986, 288~p. (In Russian)
	
35.  Tikhonov A. N., Goncharskij A. V., Stepanov V. V., Yagola A. G.
%Tikhonov A.N. et al.
{\it Regulyariziruyushiye algoritmy i apriornaya informatsiya} [{\it Re\-gu\-la\-ting algorithms and a priori information}].  Moscow, Nauka Publ., 1983, 200~p. (In Russian)
	
36.  Shary S. P. {\it O nekotorykh metodakh resheniya lineynoy zadachi o dopuskakh.} Preprint VC Aca\-demii Nauk SSSR [{\it On some methods of solving the linear problem of tolerances}.  Preprint of the Com\-pu\-ting Center of USSR Acad. Sci.]. Krasnoyarsk, USSR  Acad. Sci. Publ.,  1989, no.~6, 45~p. (In Russian)
	
37.  Shary S. P. Resheniye intervalnoy lineynoy zadachi o dopuskakh [Solution of the linear problem of tolerances]. {\it Avtomatika i telemekhanika} [{\it Automatics and Telemechanics}], 2004, no.~10, pp.~147--162. (In Russian)
	
38.  Shary S. P., Sharaya I. A. Raspoznavaniye razreshimosti intervalnikh uravneniy i yego pri\-lo\-zhe\-niya k analizu dannykh [Recognition of solvability of interval equations and its applications to data ana\-ly\-sis]. {\it Vychislitelniye tekhnologii} [{\it Computational Technologies}], 2013, vol.~18, no.~3, pp.~80--109. (In Russian)
	
39. Sharaya I. A., Shary S. P. Reserve of characteristic inclusion as recognizing functional for interval linear systems. {\it Scientific Computing, Computer Arithmetic, and Validated Nu\-me\-rics. 	16th International Symposium (SCAN 2014)}. W\"urzburg, Germany, September, 21--26, 2014. Revised selected papers. Eds by M.~Nehmeier, J.~W.~von~Gudenberg, W.~Tucker. Heidelberg, Springer Publ., 2016, pp.~148--167.
	
40.  Shary S. P. Silnaya soglasovannost v zadache vosstanovleniya zavisimostey pri in\-ter\-val\-noy neop\-re\-delen\-nosti dannykh [Strong consistency in the problem of restoration  of dependencies under interval data uncertainty]. {\it Vychislitelniye tekhnologii} [{\it Computational Technologies}], 2017, vol.~22, no.~2, pp.~150--172. (In Russian)
	
41.  Vulikh B. Z. {\it Vvedeniye v teoriyu poluuporyadochennykh prostranstv} [{\it Introduction to the theory of semi-ordered spaces}]. Moscow, GIFML Press, 1961, 407~p. (In Russian)
	
42. Rohn J. Systems of  linear  interval  equations.
{\it Linear Algebra and its Applications}, 1989, vol.~126, pp.~39--78.
	
43. Barth W., Nuding E. Optimale L\"{o}sung von Intervallgleichungssystemen [Optimal solution of systems of interval equations ]. {\it Computing}, 1974, vol.~12, no.~2, pp.~117--125.\\	
https://doi.org/10.1007/BF02260368
	
44. Hansen E. On linear algebraic equations with interval coefficients. {\it Topics in Interval Analysis}. Ed. by E.~Hansen.  Oxford, Oxford University Press, 1969, pp.~33--46.
	
45.  Rockafellar R. {\it Vypukly analiz} [{\it Convex analysis}]. Moscow, Mir Publ., 1973, 469~p. (In Russian)



\vskip 1.5mm

Received:  October 03, 2020.

Accepted: June 04, 2021.


\vskip6 mm A~u~t~h~o~r~s'\,  i~n~f~o~r~m~a~t~i~o~n:


\vskip2 mm \textit{Sergei I. Noskov} --- Dr. Sci. in Technics, Professor; sergey.noskov.57@mail.ru \par%
%
\vskip2 mm \textit{Anatoly V. Lakeyev} --- Dr. Sci. in Physics and Mathematics, Chief Researcher;  lakeyev@icc.ru \par
%
}
