
{\normalsize

\vskip 4%6
mm

\noindent{\bf Construction of reachability and controllability sets in a special linear\\ control problem$^{*}$%
}

}

\vskip 2mm

{\small

\noindent{\it A.\,S. Popkov%
%, I.~�. Famylia%$\,^2$%

}

\vskip 2mm

%%%%%%%%%%%%%%%%%%%%%%%%%%%%%%%%%%%%%%%%%%%%%%%%%%%%%%%%%%%%%%%%%%

\efootnote{
%%
\vspace{-3mm}\parindent=7mm
%%
\vskip 0.1mm $^{*}$ This work  was supported by the Russian Foundation for Basic Research (project N 19-31-90033).%\par
%%
%%\vskip 2.0mm
%%
%%\indent{\copyright} �����-������������� ���������������
%%�����������, \issueyear%
%%
}

%%%%%%%%%%%%%%%%%%%%%%%%%%%%%%%%%%%%%%%%%%%%%%%%%%%%%%%%%%%%%%%%%%

{\footnotesize

\noindent%
%$^1$~%
St.\,Petersburg State University, 7--9, Universitetskaya nab.,
St.\,Petersburg,

\noindent%
%\hskip2.45mm%
199034, Russian Federation


}

%%%%%%%%%%%%%%%%%%%%%%%%%%%%%%%%%%%%%%%%%%%%%%%%%%%%%%%%%%%%%%%%%%

\vskip2mm%3mm


\noindent \textbf{For citation:} Popkov A.\,S. Construction of reachability and controllability sets in a special linear control problem. {\it Vestnik of Saint~Petersburg Uni\-ver\-si\-ty.
Ap\-plied Mat\-he\-matics. Computer Science. Control
Processes},\,\issueyear,
vol.~17,~iss.~\issuenum,~pp.~\pageref{p7}--\pageref{p7e}. \\
\doivyp/\enskip%
\!\!\!spbu10.\issueyear.\issuenum07 (In Russian)

\vskip2mm

{\leftskip=7mm\noindent The article considers the problem of constructing reachability and controllability sets for a control problem. The motion of an object is described by a linear system of or\-di\-na\-ry differential equations, and control is selected from the class of piecewise-constant func\-tions. Straight boundaries are also set on the controls. The article provides definitions of rea\-cha\-bi\-li\-ty and controllability sets. It is shown that the problems of constructing these sets are equi\-va\-lent and can be reduced to the problem of linear mapping of a multidimensional cube. The properties of these sets are also given. In addition, the existing approaches to solving the problem are analyzed. Since they are all too computationally complex, the question of creating a more efficient algorithm arises. The work proposes an algorithm for constructing\linebreak\newpage\noindent the required sets as a system of linear inequalities. A proof of the theorem showing the cor\-rect\-ness of the algorithm is provided. The complexity of the presented approach is estimated.\\[1mm]
\textit{Keywords}: control, optimal control, piecewise-constant control, reachability set, controllability set, linear mapping, Fourier\,---\,Motzkin elimination.\par}


\vskip6mm

\noindent \textbf{References} }

\vskip 2mm

{\footnotesize

1. {Balashevich N.\,V., Gabasov R., Kirillova F.\,M.}
Chislennye metody programmnoy i po\-zi\-tsion\-noy optimizatsii lineynyh sistem upravleniya
[Numerical methods of program and positional optimization of linear control systems].
{\it Journal of Computational Mathematics and Mathematical Physics}, 2000, vol.~40, no.~6, pp.~838--859. (In Russian)

2. {Rawlings J.\,B., Mayne D.\,Q.} {\it Model predictive control: Theory} \& {\it design.} Madison, WI, Nob Hill Pub\-li\-shing, 2009, 533~p.

3. {Gr{\"u}ne L., Pannek J.} {\it Nonlinear model predictive control: Theory and algorithms.} 2nd ed. Cham, Springer International Publishing, Communication and Control Engineering, 2017, 473~p.


4. {Ponomarev A.\,A.} Postroenie suboptimalnogo upravleniya v regulyatore ``prediktor\,---\,korrektor��
[Suboptimal control construction for the model predictive controller].{\it Vestnik of Saint Petersburg
Uni\-ver\-sity. Series~10. Applied Mathematics. Computer Sciences. Control Processes,} 2014, iss.~3, pp.~141--153. (In Russian)

5. {Pontryagin L.\,S., Boltyanskii V.\,G., Gamkrelidze R.\,V., Mishechenko E.\,F.} {\it Matematicheskaya teoriya optimal'nykh protsessov} [{\it The mathematical theory of optimal processes}]. Moscow, Nauka Publ., 1969, 384~p.  (In Russian)

6. {Formalskii A.\,M.} {\it Upravlyaemost' i ustoychivost' sistem s ogranichennymi resursami} [{\it Con\-trol\-la\-bility and stability of systems with restricted control resources}]. Moscow, Nauka Publ., 1974, 368~p. (In Russian)

7. {Chernousko F. L.} {\it Ocenivanie fazovyh sostojanij dinamicheskih sistem. Metod jellipsoidov} [{\it Es\-ti\-ma\-tion of phase states of dynamical systems. Ellipsoid method}]. Moscow, Nauka Publ., 1988, 319~p. (In Russian)

8. {Ekimov A.\,V., Balykina Yu.\,E., Svirkin M.\,V.}  On the estimation of the attainability set of nonlinear control systems. {\it AIP Conference Proceedings}, 2015, p.~1648, no.~450008.

9. {Dantzig G.\,B., Eaves B.\,C.}  Fourier\,---\,Motzkin elimination and its dual. {\it J. Combinatorial Theory. Series~A~14}, 1973, pp.~288--297.

10. {Popkov A.\,S.} Application of the adaptive method for optimal stabilization of a nonlinear object.
{\it 2016 International Conference ``Stability and Oscillations of Nonlinear Control Systems'' (Pyatnitskiy's Conference)}, 2016, pp.~1--3.

11. {Popkov A.\,S., Smirnov N.\,V., Smirnova T.\,E.} On modification of the positional optimization method for a class of nonlinear systems.
{\it ACM International Conference. Proceeding Series}, 2018,\linebreak pp.~46--51.


\vskip1.5mm Received:  May 11, 2021.

Accepted: June 04, 2021.


\vskip6mm A\,u\,t\,h\,o\,r'\,s\, \ i\,n\,f\,o\,r\,m\,a\,t\,i\,o\,n:%

\vskip1.5mm \textit{Alexander S. Popkov} ---  Postgraduate Student; alexandr.popkoff@gmail.com \par%
%
}
