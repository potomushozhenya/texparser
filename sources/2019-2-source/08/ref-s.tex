
{\normalsize

\vskip 6mm

\noindent{\bf Modeling of the impact of information on tax audits
on the risk statuses\\ and evasions of individuals%$\,^*$}

}

\vskip 2mm

{\small

\noindent{\it E.~A. Gubar, E.~M. Zhitkova, S.~Sh. Kumacheva,
G.~A. Tomilina%$\,^1$%
%, A. L. Chistov$\,^2$%
%, D. M. Kuchinsky$\,^1$%

 }

\vskip 2mm

%%%%%%%%%%%%%%%%%%%%%%%%%%%%%%%%%%%%%%%%%%%%%%%%%%%%%%%%%%%%%%%%%%

%\efootnote{
%%
%\vspace{-3mm}\parindent=7mm
%%
%\vskip 0.1mm $^{*}$ The work is supported by Russian Fundamental
%Research (grand N 16-08-00890).\par
%%
%}

%%%%%%%%%%%%%%%%%%%%%%%%%%%%%%%%%%%%%%%%%%%%%%%%%%%%%%%%%%%%%%%%%%

{\footnotesize

\noindent%
%$^2$~%
St.\,Petersburg State University, 7--9, Universitetskaya nab.,
St.\,Petersburg,

\noindent%
%\hskip2.45mm%
199034, Russian Federation

}

%%%%%%%%%%%%%%%%%%%%%%%%%%%%%%%%%%%%%%%%%%%%%%%%%%%%%%%%%%%%%%%%%%

\vskip3mm


\noindent \textbf{For citation:}  Gubar E.~A., Zhitkova E.~M.,
Kumacheva S.~Sh., Tomilina G.~A. Modeling of the impact of
information on tax audits on the risk statuses and evasions of
individuals. {\it Vestnik of Saint~Petersburg University. Applied
Mathematics. Computer Science. Control Processes},\,\issueyear,
vol.~15,~iss.~\issuenum,~pp.~\pageref{p8}--\pageref{p8e}.
\doivyp/\enskip%
\!\!\!spbu10.\issueyear.\issuenum08  (In Russian)

\vskip3mm

{\leftskip=7mm\noindent Tax system accomplishes several significant
functions, concerned with different fields of its activities,
such as fiscal, social, regulatory, distributive, incentive and
control. The correctness of work of fiscal system depends on
tax control; in turn, the most important part of tax control is
tax audit. However, the previous studies have shown that the
implementation of tax audits with an optimal probability is an
expensive procedure. Due to the restricted budget of the tax
authorities, such audits are not always feasible. Therefore,
to increase tax collection, it is necessary to apply additional
ways to stimulate taxpayers to fair tax payments. In this paper,
we investigate a model of tax control taking into account the
dissemination information about possible audits over the population
of taxpayers as a tool to stimulate tax collection. We use an
evolutionary approach to describe the behavior of taxpayers
in total population, where each taxpayer has one of three
different risk statuses: risk-loving, risk-neutral, and
risk-avoiding. The idea of disseminating information in
a population has been studied based on the Markov process
in networks with different structures. We estimate the influence
of the propagated information on the final distribution
of risk statuses and the evasions of taxpayers. We design
a series of numerical simulations with the different topology
of the network and artificial centers of influence which can
be included or not included in the network of contacts
to corroborate our theoretical arguments. As well as we present
a comparative analysis of the received numerical simulations.\\[1mm]
\textit{Keywords}: tax auditing, risk propensity, information
dissemination, Markov processes on the network, dynamic of
opinions.
\par}

\vskip5mm

\noindent \textbf{References} }

\vskip 2mm

{\footnotesize

1. Chander P., Wilde L. L.  {A general characterization of optimal
income tax enforcement.} {\it Review of Economic Studies}, {1998},
vol.~65, pp.~165--183.

2. Vasin A. A., Morozov V. V. \emph{Teoriya igr i modeli
matematicheskoj ekonomiki} [\emph{Game theory and models of
mathematical economics}]. Moscow, Moscow University Publ., {2005}.
272~p. (In Russian)

3. Bure V. M., Kumacheva S. Sh. {Model' audita s ispol'zovaniem
statisticheskoj informacii o dohodah nalogoplatel'shchikov [A
model of audit with using of statistical information about
taxpayers� income].} {\it Vestnik of Saint Peterburg University.
Series 10. Applied Mathematics. Computer Science. Control
Processes}, {2005}, iss.~1--2, pp.~140--145. (In Russian)

4. Bure V. M., Kumacheva S. Sh. {Teoretiko-igrovaya model'
nalo\-govyh proverok s ispol'\-zovaniem statisticheskoj informacii
o nalogoplatel'shchikah [A game theoretical model of tax auditing
with using a statistical information about taxpayers].} \emph
{Vestnik of Saint Peterburg University. Series 10. Applied
Mathematics. Computer Science. Control Processes}, {2010}, iss.~4,
pp.~16--24.  (In Russian)

5. Samuelson P. A., Nordhaus W. D.  \emph{Economics}. 18th ed. New
York, McGraw Hill Publ., {2005}, 776~p. (Russ. ed.: Samuelson P.
A., Nordhaus W. D. {\it Economica}. Moscow, Wiliams Publ., 2007,
1358~p.)
%Economics / P. A. Samuelson, W. D. Nordhaus. - Boston et al., 2005.

6. Kumacheva S. Sh., Gubar E. A. {Evolutionary model of tax
auditing.} \emph{Contributions to Game Theory and Management},
2015, vol.~8, pp.~164--175.

7.  Antoci A., Russu P., Zarri L. {Tax evasion in a behaviorally
heterogeneous society: An evolutionary analysis.} \emph{Economic
Modelling}, 2014, vol.~10, no.~42, pp.~106--115.

8. Antunes L., Balsa J., Urbano P., Moniz L., Roseta-Palma C. {Tax
compliance in a simulated heterogeneous multi-agent society.}
\emph{Lecture Notes in Computer Science}, 2006, vol.~3891,
pp.~147--161.

9. Gubar E., Kumacheva S., Zhitkova E., Kurnosykh Z. {Evolutionary
behavior of tax\-payers in the model of information
dissemination.} {\it 2017 Constructive Nonsmooth Analysis and
Related Topics (Dedicated to the memory of V.~F. Demyanov)}.
\emph{CNSA 2017, Proceedings}. IEEE Conference Publ., 2017,
pp.~1--4.

10. Gubar E., Kumacheva S., Zhitkova E., Kurnosykh Z., Skovorodina
T. {Modelling of information spreading in the population of
taxpayers: Evolutionary approach}. \emph{Contributions to Game
Theory and Management}, 2017, vol.~10, pp.~100--128.

11. Gubar E.~A., Kumacheva S.~Sh., Zhitkova E.~M., Porokhnyavaya
O.~Yu. {Propagation of information over the network of taxpayers
in the model of tax auditing.}  \emph{2015 Intern. Conference on
Stability and Control Processes in memory of V.~I.~Zubov. SCP
2015, Proceedings}. IEEE Conference Publ., 2015, INSPEC Accession
no.~15637330, pp.~244--247.

12. DeGroot M. H. Reaching a consensus. \emph{Journal of the
American Statistical Association}, {1974}, vol.~69, no.~345,
pp.~118--121.

13. Borovkov A. A. \emph {Teoriya veroyatnostej} [\emph{Theory of
probability}]. Moscow, Edutorial URSS Publ., {2003}, 472~p. (In
Russian)

14. Bure V. M., Parilina E. M., Sedakov A. A. Konsensus v
social'noy seti s dvumia centrami vliyania [{Consensus in a social
network with two principals}].  \emph{Problemy upravlenia
$[$Autom. Remote Control$]$}, 2017, vol.~78, no.~8,
pp.~1489--1499.  (In Russian)

15. Gubanov D. A., Novikov D. A., Chkhartishvili A. G. {\it
Sotsial'nye seti: modeli informatsionnogo vliyania, upravlenia i
protivoborstva} [\emph {Social networks: Models of informational
influence, control and opposition}]. Moscow, Fizmatlit Publ.,
{2010}, 228~p. (In Russian)

16. Kumacheva S. Sh. {Tax auditing using statistical information
about taxpayers.} \emph{Con\-tri\-bu\-tions to Game Theory and
Management}, 2012, vol.~5, pp.~156--167.

17. Tomilina G. A. {Modelirovanie scenariev rasprostraneniya
informacii o proverkah v seti nalogoplatel'shchikov [Modeling of
scenarios for the dissemination of information on audits in the
taxpayers network].} \emph{Control Processes and Stability}, 2018,
vol.~5, no.~1, pp.~497--501. (In Russian)

18. Niazashvili A.~G., Hashchenko V.~A. {Psihologicheskaya
gotovnost' k ehkonomicheskomu risku} [Psychological readiness for
economic risk]. \emph {Psihologiya i ehkonomika. Trudy 1-st
Vseros. nauch.-praktich. konferencii RPO $[$Psychology and
Economics. All-Russian scientific and practical conference of the
Russian Psychological Society$]$}. Otv. red. O.~G.~Posypanov,
V.~V.~Spasennikov. Moscow, Kaluzhskij filial MGEI, KGPU Publ.,
{2000}, pp.~160--162. (In Russian)

19. {\itshape Web site} \emph{Federal state statistics center}.
Available at: http://www.gks.ru/ (accessed: 02.03.2018).

20. Kendall M. G., Stuart A. \emph{The advanced theory of
statistics}. London, Charles Griffin and Company Limited Publ.,
{1963}, 585~p. (Russ. ed.: Kendall~M.~G., Stuart~A. \emph{Teoria
raspredeleniy}. Moscow, Nauka Publ., 1966, 588~p.)

21. Kumacheva  S., Gubar  E., Zhitkova  E.,  Tomilina  G.
{Evolution of risk-statuses in one model of tax control}.
\emph{Static and Dynamic Game Theory$:$ Foundations and
Applications}, 2018, pp.~121--138.


\vskip1.5mm Received:  September 12, 2018.

Accepted: March 15, 2019.


\vskip6mm A\,u\,t\,h\,o\,r's \ i\,n\,f\,o\,r\,m\,a\,t\,i\,o\,n:%

\vskip2mm \textit{Elena A. Gubar} --- PhD in Physics and
Mathematics, Associate Professor; e.gubar@spbu.ru

\vskip2mm \textit{Ekaterina M. Zhitkova} --- PhD in Physics and
Mathematics, Researcher; zhitkovakaterina@mail.ru

\vskip2mm \textit{Suria Sh. Kumacheva} --- PhD in Physics and
Mathematics, Senior Lecturer; s.kumacheva@spbu.ru

\vskip2mm \textit{Galina A. Tomilina} --- Student;
g.tomilina@yandex.ru

}
