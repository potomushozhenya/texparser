
{\normalsize

\vskip 6mm

\noindent{\bf Modeling of influence among the members of the
educational team$\,^*$}

}

\vskip 2mm

{\small

\noindent{\it V. V. Mazalov$\,^{1,2}$%
, J. A. Dorofeeva$\,^{2,3}$%
, E. N.
Konovalchikova$\,^{4}$%
%, I.~�. Famylia%$\,^2$%

 }

\vskip 2mm

%%%%%%%%%%%%%%%%%%%%%%%%%%%%%%%%%%%%%%%%%%%%%%%%%%%%%%%%%%%%%%%%%%

\efootnote{
%%
\vspace{-3mm}\parindent=7mm
%%
\vskip 0.1mm $^{*}$ This work is supported by Russian Science
Foundation (project N~17-11-01079).%\par
%%
%%\vskip 2.0mm
%%
%\indent{\copyright} �����-������������� ���������������
%�����������, \issueyear%
%
}

%%%%%%%%%%%%%%%%%%%%%%%%%%%%%%%%%%%%%%%%%%%%%%%%%%%%%%%%%%%%%%%%%%%

{\footnotesize

\noindent%
$^1$~%
St.\,Petersburg State University, 7--9, Universitetskaya nab.,
St.\,Petersburg,

\noindent%
\hskip2.45mm%
199034, Russian Federation

\noindent%
$^2$~%
Institute of Applied Mathematical Research, Federal Research
Center ``Karelian Research Center

\noindent%
\hskip2.45mm%
of the Russian Academy of Sciences'', 11, Pushkinskaya ul.,
Petrozavodsk,

\noindent%
\hskip2.45mm%
185910, Russian Federation

\noindent%
$^3$~%
Petrozavodsk State University, 33, pr. Lenina, Petrozavodsk,

\noindent%
\hskip2.45mm%
185910, Russian Federation

\noindent%
$^4$~%
Zabaikalsky State University, 30, Alexandro-Zavodskaya ul., Chita,

\noindent%
\hskip2.45mm%
672039, Russian Federation

%\noindent%
%$^2$~%
%St.\,Petersburg State University, 7--9, Universitetskaya nab.,
%St.\,Petersburg,

%\noindent%
%\hskip2.45mm%
%199034, Russian Federation

}

%%%%%%%%%%%%%%%%%%%%%%%%%%%%%%%%%%%%%%%%%%%%%%%%%%%%%%%%%%%%%%%%%%

\vskip 3mm


\noindent \textbf{For citation:}  Mazalov V. V., Dorofeeva J. A.,
Konovalchikova E. N.  Modeling of influence among the members of
the educational team. {\it Vestnik of Saint~Petersburg University.
Applied\linebreak Mathematics. Computer Science. Control
Processes},\,\issueyear,
vol.~15,~iss.~\issuenum,~pp.~\pageref{p9}--\pageref{p9e}.\\
\doivyp/\enskip%
\!\!\!spbu10.\issueyear.\issuenum09  (In Russian)

\vskip 3mm

{\leftskip=7mm\noindent We study the problem of determining the
rating among members of educational groups of different levels:
secondary schools, colleges, universities. Models of ratings of
teachers and students  are described. The method is based on the
DeGroot model using an influence matrix calculated for different
types of educational groups are represented by principals and
students. In the process of training, the participants of the
educational team influence each other by discussing issues,
exchanging views, etc. For the principal, feedback from students
is important, which is the degree of influence. It is focused on
in this work. Another important aspect is the weight of the
participants in the team. It is determined by the limit vector for
the matrix. Submitted to the consideration of several scenarios,
namely: the training team with one or two principals and various
subgrops students according to level of training. The influence of
the principal in different scenarios also varies. It can affect
all participants in the same way or in different ways. It depends
on the rating of students. The interpretation of the obtained
values of ratings is offered, the results of numerical modeling
for various matrices of influence are given. The  presented model
can be used to determine the professional orientation of any
educational group
at different levels of education --- school, college, university.\\[1mm]
\textit{Keywords}: rating, educational team, reputation model,
influence matrix.
\par}

\vskip 5mm

\noindent \textbf{References} }

\vskip 2mm

{\footnotesize

1. DeGroot M. H. Reaching a consensus. \textit{Journal of the
American Statistical Association}, 1974, vol.~69, no.~345,
pp.~118--121.

2. Buechel B., Hellmann T., Klobner  S. Opinion dynamics and
wisdom under conformity. \textit{Journal of Economic Dynamics $\&$
Control}, 2015, vol.~52, pp.~240--257.

3. Gubanov   D. A., Novikov D. A., Chhartrshvili A. G.
\textit{Social'nie seti: modeli informatsionnogo vliyania,
upravlenia i protivoborstva $[$Social networks: models of
information influence, management and confrontation$]$}. Moscow,
Fizmatlit Publ., 2010, 228~p.  (In Russian)

4. Chebotarev P. Yu., Agaev R. P. Ob asimptotike v modeliyah
konsensusa [On asymptotics in consensus models].
\textit{Upravlenie bol'shimi sistemami $[$Management of large
systems$]$}, 2013, vol.~43, pp.~55--77. (In Russian)

5. Bure V. M., Ekimov A. V., Svirkin M. V. Imitatsionnaya model'
formirovania profilya mneniy vnutri kollektiva [Imitation model of
formation of a profile of opinions in collective]. \textit{Vestnik
of Saint Petersburg University. Series 10. Applied Mathematics.
Computer Sciences. Control Processes}, 2014, iss.~3, pp.~93--98.
(In Russian)

6. Lefevr V. A. {\it Teoria refleksivnyh igr} [\textit{Theory of
reflexive games}]. �oscow, Nauka Publ., 2009, 218~p. (In Russian)

7. Grabisch M., Rusinowska A. A model of influence based on
aggregation functions. \textit{Mathematical Social Sciences},
2013, vol.~66, pp.~316--330.

8. Bauso D., Cannon M. Consensus in opinion dynamics as repeated
game. \textit{Automatica}, 2018, vol.~90, pp.~204--211.

9. Golub B., Jackson M. O. Naive learning in social networks and
the wisdom of crowds. \textit{American Economic Journal.
Microeconomics}, 2010, vol.~2, no.~1, pp.~112--149.

10. Hegselmann R., Krause U. Opinion dynamics and bounded
confidence models, analysis, and simulation. \textit{Journal of
Artificial Societies and Social Simulation}, 2002, vol.~5, no.~3,
pp.~1--33.

11. Parilina E., Sedakov A. Stable cooperation in graph-restricted
games.  \textit{Contributions to Game Theory and Management},
2014, vol.~7, pp.~271--281.

12. Bure V. M., Parilina E. M., Sedakov A. A. Consensus in a
Social Network with two principals. \textit{Automation and Remote
Control}, 2017, vol.~78, no.~8, pp.~1489--1499.


\vskip2mm %1.5mm %
Received:  January 14, 2019.

Accepted: March 15, 2019.


\vskip6mm A\,u\,t\,h\,o\,r's \ i\,n\,f\,o\,r\,m\,a\,t\,i\,o\,n:%

\vskip2mm \textit{Vladimir V. Mazalov} --- Dr. Sci. in Physics and
Mathematics, Professor; vmazalov@krc.karelia.ru

\vskip2mm \textit{Julia A. Dorofeeva} --- Chief Lecturer,
Postgraduate Student; julana2008@yandex.ru

\vskip2mm \textit{Elena N. Konovalchikova} --- PhD in Physics and
Mathematics, Associate Professor;\\ konovalchikova\_en@mail.ru %
\par
%
}
