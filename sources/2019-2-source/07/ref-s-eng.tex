
{\small

\vskip4mm%6mm

\noindent \textbf{References} }

\vskip 1.5mm%2mm

{\footnotesize

1. Gomaa W. H., Fahmy A. A. A survey of text similarity
approaches. {\it Intern. Journal of Computer Applications}, 2013,
vol.~68, no.~13, pp.~13--18.

2. Freire J., Pinheiro V., Feitosa D. LEC UNIFOR no ASSIN:
FlexSTS-Um framework para Similaridade Semantica Textual. {\it
PROPOR-Intern. conference on the Computational Processing of
Portuguese}. Tomar, Portugal, 2016. Available at:
http://proper206.di.fc.ul.pt/ (accessed: 03.08.2018).

3. Barbosa L., Cavalin P., Kormaksson M., Guimaraes V. Blue man
mroup at ASSIN: Using distributed representations for semantic
similarity and entailment recognition. {\it PROPOR-Intern.
conference on the Computational Processing of Portuguese}. Tomar,
Portugal, 2016. Available at: http://proper206.di.fc.ul.pt/
(accessed: 03.08.2018).

4. Ferreira R., Lins R. D., Simske S. J., Freitas F., Riss M.
Assessing sentence similarity through lexical, syntactic and
semantic analysis. {\it Computer Speech $\&$ Language}, 2016,
vol.~39, pp.~1--28.

5. Hartmann N. S. Solo queue at ASSIN: Combinando abordagens
tradicionais e emergentes. {\it Linguam{\'a}tica}, 2016, vol.~8,
no.~2, pp.~59--64.\newpage

6. Cer D., Diab M., Agirre E., Lopez-Gazpio I., Specia L. {\it
Semeval-2017 Task 1: Semantic Textual Similarity-multilingual and
cross-lingual focused evaluation.} arXiv preprint
arXiv:1708.00055, 2017. doi: 10.18653/v1/S17-2001

7. Hartmann N., Fonseca E., Shulby C., Treviso M., Rodrigues J.,
Aluisio S. {\it Portuguese word embeddings: Evaluating on word
analogies and natural language tasks.} arXiv preprint
arXiv:1708.06025, 2017.

8. Silva A., Rigo S., Alves I. M., Barbosa J. Avaliando a
similaridade sem{\^{a}}ntica entre frases curtas atrav{\'{e}}s de
uma abordagem h{\'{i}}brida. {\it Proceedings of the 11th
Brazilian Symposium in Information and Human Language Technology},
Uberl{\^{a}}ndia, 2017, pp.~93--102.

9. Pradhan N., Gyanchandani M., Wadhvani R. A review on text
Similarity Technique used in IR and its application. {\it Intern.
Journal of Computer Applications}, 2015, vol.~120, no.~9,
pp.~29--34.

10. Chen F., Lu C., Wu H., Li M. A semantic similarity measure
integrating multiple conceptual relationships for web service
discovery. {\it Expert Systems with Applications}, 2017, vol.~67,
pp.~19--31.

11. Berrahou S. L., Buche P., Dibie J., Roche M. Xart: Discovery
of correlated arguments of $n$-ary relations in text. {\it Expert
Systems with Applications}, 2017, vol.~73, pp.~115--124.

12. Ferreira R., Cavalcanti G. D., Freitas F., Lins R. D., Simske
S. J., Riss M. Combining sentence similarities measures to
identify paraphrases. {\it Computer Speech $\&$ Language}, 2018,
vol.~47, pp.~59--73.

13. Yanaka H., Mineshima K., Martinez-Gomez P., Bekki D. {\it
Determining Semantic Textual Similarity using natural deduction
proofs.} arXiv preprint arXiv:1707.08713, 2017.

14. Kajiwara T., Bollegala D., Yoshida Y., Kawarabayashi K. I. An
iterative approach for the global estimation of sentence
similarity. {\it PloS one}, 2017, vol.~12, no.~9, pp.~e0180885.

15. Brychc{\'{i}}n T., Svoboda L. UWB at Semeval-2016 Task 1:
Semantic Textual Similarity using lexical, syntactic, and semantic
information. {\it Proceedings of the 10th Intern. Workshop on
Semantic Evaluation (SemEval-2016)}. San Diego, 2016,
pp.~588--594.

16. Kashyap A., Han L., Yus R., Sleeman J., Satyapanich T., Gandhi
S., Finin T. Robust semantic text similarity using LSA, machine
learning, and linguistic resources. {\it Language Resources and
Evaluation}, 2016, vol.~50, no.~1, pp.~125--161.

17. Oliveira Alves A., Rodrigues R., Gon{\c{c}}alo Oliveira H.
ASAPP: Alinhamento Sem{\^{a}}ntico Autom{\'{a}}tico de Palavras
aplicado ao Portugu{\^{e}}s (eng. ASAPP: Automatic semantic
alignment for phrases applied to portuguese). {\it
Linguam{\'{a}}tica}, 2016, vol.~8, no.~2, pp.~43--58.

18. Cavalcanti A. P., de Mello R. F. L., Ferreira M. A. D.,
Rolim~V.~B., Ten{\'{o}}rio~J.~V.~S. Statistical and semantic
features to measure sentence similarity in Portuguese. {\it
Intelligent Systems (BRACIS), 2017 Brazilian conference on.} IEEE,
2017, pp.~342--347.

19. Mikolov T., Chen K., Corrado G., Dean J. {\it Efficient
estimation of word representations in vector space.} arXiv
preprint arXiv:1301.3781, 2013.

20. Fialho P., Marques R., Martins B., Coheur L., Quaresma P. {\it
Medi{\c{c}}{\~{a}}o de Similaridade Sem{\^{a}}ntica e
Reconhecimento de Infer{\^{e}}ncia Textual (eng. Measuring
Semantic Similarity and Recognizing Textual Entailment).}
INESC-ID@ASSIN, 2016.

21. Faruqui M., Tsvetkov Y., Rastogi P., Dyer C. {\it Problems
with evaluation of word embeddings using word similarity tasks.}
arXiv preprint arXiv:1605.02276, 2016.

22. Paiva V., Rademaker A., Melo G. Openwordnet-pt: An open
brazilian wordnet for reasoning. {\it COLING 2012}. Mumbai, 2012.

23. Miller G. A. WordNet: a lexical database for English. {\it
Communications of the ACM}, 1995, vol.~38, no.~11, pp.~39--41.

24.  Lozkins A., Bure V. M. The probabilistic method of finding
the local-optimum of clustering. {\it Vestnik of Saint Petersburg
University. Series 10. Applied Mathematics. Computer science.
Control Processes}, 2016, iss.~1, pp.~28--37.


\vskip 1.5mm

%\noindent Recommendation: prof. L. A. Petrosyan.
%
%\vskip 1.5mm

Received:  November 18, 2018.

Accepted: March 15, 2019.

\vskip4.5mm%6mm
A\,u\,t\,h\,o\,r's \ i\,n\,f\,o\,r\,m\,a\,t\,i\,o\,n:

\vskip2mm\textit{Allan Silva} --- Master; allanbs@unisinos.br

\vskip2mm\textit{Aleksejs Lozkins} --- Postgraduate Student;
aleksejs.lozkin@gmail.com

\vskip2mm\textit{Luiz Ricardo Bertoldi} --- Master;
luizbertoldi@unisinos.br

\vskip2mm\textit{Sandro Rigo} --- Dr. Sci. in Computers;
rigo@unisinos.br

\vskip2mm\textit{Vladimir M. Bure} --- Dr. Sci. in Technics,
Professor; vlb310154@gmail.com

}
