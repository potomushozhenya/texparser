
{\normalsize

\vskip 6mm

\noindent{\bf Constrained optimality conditions in terms of
proper\\ and adjoint coexhausters$\,^*$}

}

\vskip 2mm

{\small

\noindent{\it M.~E.~Abbasov%
%, I.~�. Famylia%$\,^2$%

 }

\vskip 2mm

%%%%%%%%%%%%%%%%%%%%%%%%%%%%%%%%%%%%%%%%%%%%%%%%%%%%%%%%%%%%%%%%%%

\efootnote{
%%
\vspace{-3mm}\parindent=7mm
%%
\vskip 0.1mm $^{*}$ This work is supported by Russian Science
Foundation (project N 18-71-00006).%\par
%%
%%\vskip 2.0mm
%%
%\indent{\copyright} �����-������������� ���������������
%�����������, \issueyear%
%
}

%%%%%%%%%%%%%%%%%%%%%%%%%%%%%%%%%%%%%%%%%%%%%%%%%%%%%%%%%%%%%%%%%%

{\footnotesize

\noindent%
%$^2$~%
St.\,Petersburg State University, 7--9, Universitetskaya nab.,
St.\,Petersburg,

\noindent%
%\hskip2.45mm%
199034, Russian Federation

%\noindent%
%$^2$~%
%St.\,Petersburg State University, 7--9, Universitetskaya nab.,
%St.\,Petersburg,

%\noindent%
%\hskip2.45mm%
%199034, Russian Federation

}

%%%%%%%%%%%%%%%%%%%%%%%%%%%%%%%%%%%%%%%%%%%%%%%%%%%%%%%%%%%%%%%%%%

\vskip3mm

\noindent \textbf{For citation:}   Abbasov M. E. Constrained
optimality conditions in terms of proper and ad\-joint
coexhausters. {\it Vestnik of Saint~Petersburg University. Applied
Mathematics. Computer Science. Control Processes}, \issueyear,
vol.~15, iss.~\issuenum, pp.~\pageref{p1}--\pageref{p1e}.\\
\doivyp/\enskip%
\!\!\!spbu10.\issueyear.\issuenum01  (In Russian)

\vskip3mm

{\leftskip=7mm\noindent Coexhasuters are families of convex
compact sets allowing one to represent the approximation of the
increment of the studied function in the neighborhood of a
considered point in the form of MaxMin or MinMax of affine
functions. Researchers developed a calculus of these objects,
which makes it possible to build thesefamilies for a wide class of
nonsmooth functions. They also described unconstrained optimality
conditions in terms of coexhausters, which paved the way for the
construction of new optimization algorithms. In this paper we
derive constrained optimality conditions in terms of coexhausters.\\[1mm]
\textit{Keywords}: nonsmooth analysis, nondifferentiable
optimization, coexhausters, optimality con\-ditions.
\par}

\vskip 5mm

\noindent \textbf{References} }

\vskip 2mm

{\footnotesize

1. {Abankin~�.~�.} {Bez\-us\-lov\-naia minimizatsiia
$H$-giperdifferentsiruemykh funktsii [Unconstrained minimization
of $H$-hyperdifferentiable functions].\textit{Computational
Mathematics and Mathe\-matical Physics}, 1998, vol.~38(9),
pp.~1500--1508. (In Russian)}

2. {Demyanov~V.~F.} Exhausters and convexificators --- new tools
in nonsmooth analysis. \textit{Quasi\-differentiability and
related topics}. Eds by V.~Demyanov, A.~Rubinov. Dordrecht, Kluwer
Academic Publ., 2000, pp.~85--137.

3. {Abbasov~M.~E., Demyanov~V.~F.} {Extremum conditions for a
nonsmooth function in terms of exhausters and coexhausters.
\textit{Proceedings of the Steklov Institute of Mathematics},
2010, vol.~269(1),\linebreak  pp.~6--15.}

4. {Abbasov~M.~E., Demyanov~V.~F.} Adjoint coexhausters in
nonsmooth analysis and extremality conditions. \textit{Journal of
Optimization Theory and Applications}, 2013, vol.~156(3),
pp.~535--553.


5. {Demyanov~V.~F.} Proper exhausters and coexhausters in
nonsmooth analysis. \textit{Opti\-mi\-za\-tion}, 2012,
vol.~61(11),  pp.~1347--1368.

6. {Fominyh~A.~V., Karelin~V.~V., Polyakova~L.~N.} Application of
the hypodifferential descent method to the problem of constructing
an optimal control. \textit{Optimization Letters}, 2018,
vol.~12(8), pp.~1825--1839.

7. {Demyanov~V.~F., Vasilev~L.~V.} \textit{Nedifferentsiruemaya
opti\-mi\-zat\-si\-ya $[$Nondifferentiable opti\-mi\-za\-tion$]$},
1981, 384~p. (In Russian)

8.  {Polovinkin~E.~S., Balashov~M.~V.} {\textit{Elementy vypuklogo
i silno vypuklogo analiza $[$Elements of convex and strongly
convex analysis$]$}. Moscow, Fizmatlit Publ., 2007, 440~p. (In
Russian)}

9. {Demyanov~V.~F., Rubinov~A.~M.} {\textit{Osnovy negladkogo
analiza i kvazidifferencialnoe ischislenie $[$Foundations of
Nonsmooth Analysis. Quasidifferential Calculus$]$}. Moscow, Nauka
Publ., 1990, 431~p. (In Russian)}

\vskip 1.5mm

Received:  September 18, 2018.

Accepted: March 15, 2019.


\vskip4mm%5mm
A\,u\,t\,h\,o\,r's \ i\,n\,f\,o\,r\,m\,a\,t\,i\,o\,n:

\vskip1.5mm \textit{Majid E. Abbasov}~--- PhD in Physics and
Mathematics, Associate Professor; m.abbasov@spbu.ru\par
%
}
