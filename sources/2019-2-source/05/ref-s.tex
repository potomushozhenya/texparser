
{\normalsize

\vskip 6mm

\noindent{\bf Post-quantum electronic digital signature scheme
based on the enhanced form\\ of the hidden discrete logarithm
problem}

}

\vskip 2mm

{\small

\noindent{\it N.~A.~Moldovyan, I.~K.~Abrosimov%$\,^1$%
%, I.~�. Famylia%$\,^2$%
%, I.~�. Famylia%$\,^2$%

 }

\vskip 2mm

%%%%%%%%%%%%%%%%%%%%%%%%%%%%%%%%%%%%%%%%%%%%%%%%%%%%%%%%%%%%%%%%%%

%\efootnote{
%%
%\vspace{-3mm}\parindent=7mm
%%
%\vskip 0.1mm $^{*}$ This work is supported by Russian Science
%Foundation (project N 18-71-00006 ).\par
%%
%%\vskip 2.0mm
%%
%%\indent{\copyright} �����-������������� ���������������
%%�����������, \issueyear%
%%
%}

%%%%%%%%%%%%%%%%%%%%%%%%%%%%%%%%%%%%%%%%%%%%%%%%%%%%%%%%%%%%%%%%%%

{\footnotesize

\noindent%
%$^1$~%
St. Petersburg Institute for Informatics and Automation of the
Russian Academy of Sciences,

\noindent%
%\hskip2.45mm%
39, 14 Line V. I., St. Petersburg, 199178, Russian Federation

%\noindent%
%$^2$~%
%St.\,Petersburg State University, 7--9, Universitetskaya nab.,
%St.\,Petersburg,

%\noindent%
%\hskip2.45mm%
%199034, Russian Federation

}

%%%%%%%%%%%%%%%%%%%%%%%%%%%%%%%%%%%%%%%%%%%%%%%%%%%%%%%%%%%%%%%%%%

\vskip3mm


\noindent \textbf{For citation:}  Moldovyan~N.~A., Abrosimov~I.~K.
Post-quantum electronic digital signature scheme based on the
enhanced form of the hidden discrete logarithm problem. {\it
Vestnik of Saint~Petersburg University. Applied Mathematics.
Computer Science. Control Processes},\,\issueyear,
vol.~15,~iss.~\issuenum,~pp.~\pageref{p5}--\pageref{p5e}.
\doivyp/\enskip%
\!\!\!spbu10.\issueyear.\issuenum05  (In Russian)

\vskip3mm

{\leftskip=7mm\noindent A digital signature scheme based on the
computational difficulty of the hidden discrete logarithm problem
defined in finite non-commutative associative algebras is
proposed. The modified quaternion algebra and its properties are
considered as the algebraic carrier of the introduced post-quantum
digital signature scheme. Formulas describing the set of local
units associated with a given non-invertible vector of a modified
quaternion algebra are derived.\linebreak\newpage\noindent A new
form of the hidden discrete logarithm
problem has been formulated and a digital signature scheme has been developed on  its base.\\[1mm]
\textit{Keywords}: post-quantum cryptography, cryptographic
primitive, electronic signature, finite algebra, non-commutative
associative algebra.
\par}

\vskip5mm

\noindent \textbf{References} }

\vskip 2mm

{\footnotesize

1. Menezes A. J., Van Oorschot P. C., Vanstone S. A.
\textit{Handbook of applied cryptography}. Boca Raton, CRC Press,
1997, 780~p.

2. Shor P. W. Polynomial-time algorithms for prime factorization
and discrete logarithms on quantum computer. \textit{SIAM Journal
of Computing}, 1997, vol.~26, pp.~1484--1509.

3. Buchmann J., Dahmen E. \textit{Post-quantum cryptography}.
Berlin, Heidelberg, Springer Press, 2009, 245~p.

4. Merkle R. Ch. \textit{Secrecy, authentication, and public key
systems.} Technical report no.~1979-1. Stanford, 1979, 193~p.

5. McEliece R. J. A public-key cryptosystem based on algebraic
coding theory. \textit{DSN Progress Report 42--44}, 1978,
pp.~114--116.

6. Hoffstein J., Pipher J.,  Silverman J. H. NTRU: A ring based
public key cryptosystem. {\it Algorithmic Number Theory (ANTS
III)}. Portland, OR, June 1998.  Berlin, Springer-Verlag Press,
1998, pp.~267--288. (Lecture Notes in Computer Science,
vol.~1423.)

7. Courtois N. The security of Hidden Field Equations (HFE). {\it
Topics in Cryptology --- CT-RSA 2001}. Berlin, Heidelberg,
Springer Press, 2001, pp.~266--281.  (Lecture Notes in Computer
Science, vol.~2020.)

8. Post-quantum cryptography. \textit{9th Intern. conference,
PQCrypto 2018}. Fort Lauderdale, FL, USA, April 9--11, 2018,
Proceedings. Berlin, Heidelberg, Springer Press, 2018. (Lecture
Notes in Computer Science, vol.~10786.)

9. Federal Register :: Announcing Request for Nominations for
Public-Key Post-Quantum Cryp\-tographic Algorithms. {\it Federal
Register. The Daily Journal of the United States Goverment}.
Available at:
https://www.gpo.gov/fdsys/pkg/FR-2016-12-20/pdf/2016-30615.pdf
(accessed: 15.01.2019).

10. Verma G. K. A proxy blind signature scheme over braid groups.
\textit{Intern. Journal of Network Security}, 2009, vol.~9, no.~3,
pp.~214--217.

11. Myasnikov A., Shpilrain V., Ushakov A. A practical attack on a
braid group based cryptographic protocol. {\it Advances in
Cryptology
--- CRYPTO'05}. Berlin, Springer-Verlag Press, 2005, pp.~86--96.  (Lecture
Notes in Computer Science, vol.~3621.)

12. Moldovyan N. A., Moldovyan A. A. Finite non-commutative
associative algebras as carriers of hidden discrete logarithm
problem. \textit{Bulletin of the South Ural State University.
Series Mathematical Modelling, Programming $\&$ Computer Software
(Bulletin SUSU MMCS)}, 2019, vol.~12, no.~1, pp.~66--81.

13. Moldovyan D. N. Cryptoschemes over hidden conjugacy search
problem and attacks using homomorphisms. \textit{Quasigroups and
Related Systems}, 2010, vol.~18, no.~2, pp.~165--176.

14. Moldovyan D. N. Post-quantum public key-agreement scheme based
on a new form of the hidden logarithm problem. \textit{Computer
Science Journal of Moldova}, 2019, vol.~27, no.~1~(78),
pp.~56--72.

15. Moldovyan D. N., Moldovyan N. A. Osobennosti stroeniya grupp
vektorov i sintez kripto\-graficheskih shem na ih osnove [Features
of the structure of groups of vectors and the synthesis of
crypto\-graphic schemes based on them]. \textit{Vestnik of Saint
Petersburg University. Series~10. Applied Mathematics. Computer
Science. Control Processes}, 2011, iss.~4, pp.~84--93. (In
Russian)

16. Moldovyan N. A., Abrosimov I. K., Kovaleva I. V.
Postkvantovy`j protokol  besklyuchevogo shifrovaniya [Post-quantum
no-key encryption protocol]. \textit{Voprosi zaschity informatsii
$[$Information securi\-ty issues$]$}, 2017, no.~3~(118),
pp.~3--13. (In Russian)

17. Schnorr C. P. Efficient signature generation by smart cards.
\textit{Journal of Cryptology}, 1991, vol.~4, no.~3, pp.~161--174.


\vskip1.5mm Received:  February 14, 2019.

Accepted: March 15, 2019.


\vskip6mm A\,u\,t\,h\,o\,r's \ i\,n\,f\,o\,r\,m\,a\,t\,i\,o\,n:%

\vskip2mm \textit{Nikolai A. Moldovyan} ---  Dr. Sci. in Technics,
Professor, Chief Scientific Collaborate; nmold@mail.ru

\vskip2mm \textit{Ivan K. Abrosimov} --- Junior Scientific
Collaborate; ivnabr@yandex.ru

}
