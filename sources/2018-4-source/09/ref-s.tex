
{\normalsize

\vskip 6mm

\noindent{\bf Methodology for disseminating information channels
analysis\\ in social networks$^{*}$ }

}

\vskip 2mm

{\small

\noindent{\it A.~A. Pronoza$\,^1$%
, L.~A. Vitkova$\,^1$%
, A.~A. Chechulin$\,^1$%
, I.~V. Kotenko$\,^1$%
, D.~V. Sakharov$\,^2$%
%, I.~�. Famylia%$\,^2$%
%, I.~�. Famylia%$\,^2$%
}

\vskip 2mm

\efootnote{
%%
%\vspace{-3mm}
%\parindent=7mm
%%
\indent$^{*}$ The work is at partial supported by Russian Found of
Fundamental Research (project N 18-71-10094 mol-a).%\par
%%
%%\vskip 2.0mm
%%
%
}

%%%%%%%%%%%%%%%%%%%%%%%%%%%%%%%%%%%%%%%%%%%%%%%%%%%%%%%%%%%%%%%%%%

{\footnotesize

\noindent%
$^1$~%
St. Petersburg Institute for Informatics and Automation of the
Russian Academy of Sciences,

\noindent%
\hskip2.45mm%
39, 14 Line V. O., Saint Petersburg, 199178, Russian Federation

\noindent%
$^2$~%
The Bonch-Bruevich St. Petersburg State University of
Telecommunication, 22-1, Bolshevikov pr.,

\noindent%
\hskip2.45mm%
Saint Petersburg, 193232, Russian Federation

%\noindent%
%$^2$~%
%St.\,Petersburg State University, 7--9, Universitetskaya nab.,
%St.\,Petersburg,

%\noindent%
%\hskip2.45mm%
%199034, Russian Federation

}

%%%%%%%%%%%%%%%%%%%%%%%%%%%%%%%%%%%%%%%%%%%%%%%%%%%%%%%%%%%%%%%%%%

\vskip3mm

\noindent \textbf{For citation:}  Pronoza A.~A., Vitkova L.~A.,
Chechulin A.~A., Kotenko I.~V., Sakharov D.~V. Methodology for
disseminating information channels analysis in social networks.
{\it Vestnik of Saint~Petersburg University. Applied Mathematics.
Computer Science. Control Processes},\,\,\issueyear,
vol.~14,~iss.~\issuenum,~pp.~\pageref{p9}--\pageref{p9e}.
\doivyp/\enskip%
\!\!\!spbu10.\issueyear.\issuenum09  (In Russian)

\vskip3mm

{\leftskip=7mm\noindent The aim of the investigation was to
develop a methodology for disseminating information channels
analysis in social networks. The proposed methodology is based on
the three steps. The first one is the formation of a knowledge
base with information about the relationships between users and
groups. The second one is the interactive mapping of information
dissemination paths, and the third one is related to the visual
analysis of the results obtained in previous steps. The
interconnections of users and groups in the social network allow
one to form the connectivity graphs, and the facts of information
transfer through these channels make it possible to identify the
ways of content distribution. The results of experiments that
confirm the applicability of the proposed methodology are also
presented in the paper.\\[1mm]
\textit{Keywords}: social network, visualization, protection
against information, inappropriate information, connection graph,
visual analytics.
\par}

\vskip5mm

\noindent \textbf{References} }

\vskip 2mm

{\footnotesize

1. {\it Sotsial'nie seti v Rossii, leto 2017: tsifry i trendy}
[{\it Social networks in Russia, summer 2017: figures and
trends}]. URL: https://www.cossa.ru/289/ 166387/
https://www.slideshare.net/Taylli01/2017-77172443 (accessed:
20.08.2018). (In Russian)

2. Kotenko I.~V., Chechulin A.~A., Shorov A.~V., Komashinsky D.~V.
Analysis and evaluation of web pages classification techniques for
inappropriate content blocking. {\it 14th Industrial Conference on
Data Mining}, LNAI. New York e. a.,  Springer-Verlag Publ., 2014,
vol.~8557, pp.~39--54.

3. Novozhilov D.~A., Kotenko I.~V., Chechulin A.~A. Improving the
categorization of web sites by analysis of html-tags statistics to
block inappropriate content. {\it 9th Intern. Symposium on
Intelligent Distributed Computing}. New York e. a.,
Springer-Verlag Publ., 2016, pp.~257--263.

4. Kotenko I.~V., Chechulin A.~A., Komashinsky D.~V.
Categorisation of web pages for protection against inappropriate
content in the Internet. {\it Intern. Journal of Internet Protocol
Technology}, 2017, vol.~10(1), pp.~61--71.

5. Zadeh L., Abbasov A., Shahbazova S. Fuzzy based techniques in
human like processing of social network data. {\it Intern. Journal
of Uncertainty, Fuzziness and Knowledge-Based Systems}. Singapore,
World Scientific Publ., 2015, vol.~23 (Suppl.~1), pp.~1--14.

6. Gomzin A., Kuznetsov S. Methods of construction of
socio-demographic profiles of Internet users. {\it Programming and
Computer Software Journal}. Moscow, ISP RAS Publ., 2015,
vol.~27(4), pp.~129�144.

7. Drobyshevskiy M., Korshunov A., Turdakov D. Parallel modularity
computation for directed weighted graphs with overlapping
communities.  {\it Programming and Computer Software Journal}.
Moscow, ISP RAS Publ., 2016, vol.~28(6), pp.~153�170.

8. Barabasi A., Bonabeau E. Scale-free networks. {\it Scientific
American Journal}, 2003, vol.~288(5), pp.~50--59.

9. Zhang E., Wang G., Gao K., Zhao X., Zhang Y. Generalized
structural holes finding algorithm by bisection in social
communities. {\it 6th Intern. Conference on Genetic and
Evolutionary Computing}, 2013, pp.~276�279.

10. Liu Q., Zhang L. Information cascades in online reading: an
empirical investigation of panel data. {\it Library Hi Tech
Journal}, 2016, vol.~32(4), pp.~687--705.

11. Opsahl T., Agneessens F., Skvoretz J. Node centrality in
weighted networks: Generalizing degree and shortest paths. {\it
Social Networks Journal}, 2010, vol.~32(3), pp.~245�251.

12. Hickethier G., Tommelein I. D., Lostuvali B. Social network
analysis of information flow in an IPD-project design
organization. {\it 21st Annual Conference of the Intern. Group for
Lean Construction}, 2013, pp.~315--324.

13. Sudhahar S., Veltri G., Cristianini N. Automated analysis of
the US presidential elections using Big Data and network analysis.
{\it Big Data $\&$ Society Journal}, 2015, vol.~2(1), pp.~1�28.

14. Martinez A., Dimitriadis Y., Rubia B., Gomez E., de la Fuente
P. Combining qualitative evaluation and social network analysis
for the study of classroom social interactions. {\it Computers
$\&$ Education Journal}, 2003, vol.~41(4), pp.~353�368.

15. Lazer D., Pentland A. S., Adamic L., Aral S., Barabasi A. L.
Life in the network: the coming age of computational social
science. {\it Science Journal}. New York, NIH Public Access, 2009,
vol.~323(5915), p.~721.

16. Melman S., Bobkov V., Cherkashin A. Technology and system
visualization of large amounts of synoptic data. {\it Journal
Information Science and Control Systems}, 2015, vol.~3(45),
pp.~63--71.

17. Hornostal O. System of cluster analysis and visualization of
big data. {\it Intern. Scientific Journal Internauka}, 2016,
vol.~1(6), pp.~22--24.

18. Averbukh V., Manakov D. Analysis and visualization of �big
data�. {\it Intern. Scientific Conference Parallel Computational
Technologies}, 2015, pp.~332--340.

19. {\it ParaPhraser.ru: Perefrazirovanie i sinonimizatsia teksta}
[{\it Paraphrasing and synonymy of text
--- ParaPhraser.ru}]. URL: www.paraphraser.ru (accessed:
20.08.2018). (In Russian)

20. Kolomeec M.~V., Gonzalez-Granadillo G., Doynikova E.~V.,
Chechulin A.~A., Kotenko I.~V., Debar H. Choosing models for
security metrics visualization. {\it Computer Network Security.
Lecture Notes in Computer Science}. New York e. a.,
Springer-Verlag Publ., 2017, vol.~10446, pp.~75--87.

21. Kolomeec M.~V., Chechulin A.~A., Pronoza A.~A., Kotenko I.~V.
Technique of data visualization example of network topology
display for security monitoring. {\it Journal of Wireless Mobile
Networks, Ubiquitous Computing, and Dependable Applications},
2016, vol.~7(1), pp.~41--57.

\vskip1.5mm Received:  June 26, 2018.

Accepted: September 25, 2018.

\vskip6mm A\,u\,t\,h\,o\,r's \ i\,n\,f\,o\,r\,m\,a\,t\,i\,o\,n:%

\vskip2mm \textit{Anton A. Pronoza}~--- Postgraduate Student;
pronoza@gmail.com

\vskip2mm \textit{Lidia A. Vitkova}~--- Research Fellow;
iskinlidia@gmail.com

\vskip2mm \textit{Andrey A. Chechulin}~--- PhD in Technics,
Leading Research Fellow; chechulin@comsec.spb.com

\vskip2mm \textit{Igor V. Kotenko}~--- Dr. Sci. in Technics,
Professor; ivkote@comsec.spb.com

\vskip2mm \textit{Dmitry V. Sakharov}~--- PhD in Technics,
Associate Professor; d.sakharov@rkn.gov.ru

}
