
{\normalsize

\vskip 6mm

\noindent{\bf Calculus of second order coexhausters$^{*}$ }

}

\vskip 1.5mm

{\small

\noindent{\it M.~E.~Abbasov%$\,^1$%
%, I.~�. Famylia%$\,^2$%
%, I.~�. Famylia%$\,^2$%
}

\vskip 2mm

\efootnote{
%%
%\vspace{-3mm}\parindent=7mm
%%
\indent$^{*}$The work is supported by Russian Found of Fundamental
Research (project N 18-31-00014 mol-a). }

%%%%%%%%%%%%%%%%%%%%%%%%%%%%%%%%%%%%%%%%%%%%%%%%%%%%%%%%%%%%%%%%%%

{\footnotesize

\noindent%
%$^1$~%
St.\,Petersburg State University, 7--9, Universitetskaya nab.,
St.\,Petersburg,

\noindent%
%\hskip2.45mm%
199034, Russian Federation

%\noindent%
%$^2$~%
%St.\,Petersburg State University, 7--9, Universitetskaya nab.,
%St.\,Petersburg,

%\noindent%
%\hskip2.45mm%
%199034, Russian Federation

}

%%%%%%%%%%%%%%%%%%%%%%%%%%%%%%%%%%%%%%%%%%%%%%%%%%%%%%%%%%%%%%%%%%

\vskip3mm

\noindent \textbf{For citation:}   Abbasov M. E. Calculus of
second order coexhausters. {\it Vestnik of Saint~Petersburg
University. Applied Mathematics. Computer Science. Control
Processes}, \issueyear, vol.~14, iss.~\issuenum,
pp.~\pageref{p1}--\pageref{p1e}.
\doivyp/\enskip%
\!\!\!spbu10.\issueyear.\issuenum01 (In Russian)

\vskip3mm

{\leftskip=7mm\noindent Coexhasuter is a new notion in the
nonsmooth analysis that allows one to study extremal properties of
a wide class of functions. This class is introduced in a
constructive manner analogous to the ``classical'' smooth case.
Formulas of calculus were developed. Coexhausters are families of
convex compact sets allowing one to approximate the increment of
the studied function in the neighbourhood of the considered point
in the form of MaxMin or MiniMax of affine functions. For a more
detailed study of nonsmooth functions, a notion of second-order
coexhausters was introduced. These are also families of convex
compact sets which are used to represent the approximation of the
increment of the studied function in the form of MaxMin or MiniMax
of quadratic functions. These objects are used to build
second-order optimization algorithms. However, an important
problem of constructing calculus arises again. The solution to
this
problem is the subject of this paper.\\[1mm]
\textit{Keywords}: nonsmooth analysis, nondifferentiable
optimization, second order co\-ex\-haus\-ters.
\par}

\vskip5mm

\noindent \textbf{References} }

\vskip 2mm

{\footnotesize

1. Abankin~�.~�. Bezuslovnaia minimizatsiia
$H$-giperdifferentsiruemykh funktsii [Unconstrained minimization
of $H$-hyperdifferentiable functions]. {\em Computational
Mathematics and Mathematical Physics}, 1998, vol.~38(9),
pp.~1500--1508. (In Russian)

2. Demyanov~V.~F. Exhausters and convexificators --- New tools in
nonsmooth analysis. {\em  Quasi-dif\-fe\-ren\-tia\-bi\-li\-ty and
related topics}. Eds by V.~Demyanov, A.~Rubinov. Dordrecht, Kluwer
Academic Publ., 2000, pp.~85--137.

3. Abbasov~M.~E., Demyanov~V.~F. Extremum conditions for a
nonsmooth function in terms of exhausters and coexhausters. {\em
Proceedings of the Steklov Institute of Mathematics}, 2010,
vol.~269(1),\linebreak  pp.~6--15.

4. Abbasov~M.~E., Demyanov~V.~F. Adjoint coexhausters in nonsmooth
analysis and extremality conditions. {\em Journal of Optimization
Theory and Applications}, 2013, vol.~156(3), pp.~535--553.

5. Dolgopolik~M.~V. Inhomogeneous convex approximations of
nonsmooth functions. {\em Russian Mathematics}, 2012, vol.~56(12),
pp.~28--42.

6. Abbasov~M.~E. Second-order minimization method for nonsmooth
functions allowing convex quadratic approximations of the augment.
{\em Journal of Optimization Theory and Applications}, 2016,
vol.~171(2), pp.~666--674.


\vskip1.5mm Received:  March 23, 2018.

Accepted: September 25, 2018.


\vskip5mm A\,u\,t\,h\,o\,r's \ i\,n\,f\,o\,r\,m\,a\,t\,i\,o\,n:

\vskip2mm \textit{Majid E. Abbasov}~--- PhD in Physics and
Mathematics, Associate Professor; m.abbasov@spbu.ru

}
