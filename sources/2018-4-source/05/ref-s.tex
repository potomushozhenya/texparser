
{\normalsize

\vskip 4mm%6mm

\noindent{\bf Applied statistics to evaluate the quality of
education}

}

\vskip 1.5mm

{\small

\noindent{\it N. A. Bure$\,^1$%
, N. L. Grebennikova$\,^2$%
, K. Yu. Staroverova$\,^1$%
%, I.~�. Famylia%$\,^2$%
}

\vskip 1.5mm%2mm

%%%%%%%%%%%%%%%%%%%%%%%%%%%%%%%%%%%%%%%%%%%%%%%%%%%%%%%%%%%%%%%%%%

{\footnotesize

\noindent%
$^1$~%
St.\,Petersburg State University, 7--9, Universitetskaya nab.,
St.\,Petersburg,

\noindent%
\hskip2.45mm%
199034, Russian Federation

\noindent%
$^2$~%
Bashkir State University, 49, Lenin ave, Sterlitamak,

\noindent%
\hskip2.45mm%
453103, Bashkortostan Republic, Russian Federation

%\noindent%
%$^2$~%
%St.\,Petersburg State University, 7--9, Universitetskaya nab.,
%St.\,Petersburg,

%\noindent%
%\hskip2.45mm%
%199034, Russian Federation

}

%%%%%%%%%%%%%%%%%%%%%%%%%%%%%%%%%%%%%%%%%%%%%%%%%%%%%%%%%%%%%%%%%%

\vskip2mm%3mm

\noindent \textbf{For citation:}   Bure N. A., Grebennikova N. L.,
Staroverova K. Yu. Applied statistics to evaluate the quality of
education. {\it Vestnik of Saint~Petersburg University. Applied
Mathematics. Computer Science. Control Processes}, \issueyear,
vol.~14, iss.~\issuenum, pp.~\pageref{p5}--\pageref{p5e}.
\doivyp/\enskip%
\!\!\!spbu10.\issueyear.\issuenum05  (In Russian)

\vskip2mm%3mm

{\leftskip=7mm\noindent The application of statistical methods and
machine learning to analyze the data describing the education
process are considered. The solution of two problems typical of
the educational process but different in the organization is
shown.  The first problem is to analyze the results of students'
tests who study Russian as a foreign language to enter the
university in Russia. The purpose of the analysis is to evaluate
the adequacy of the teaching methods, in particular, the
consistency of results gained for the elementary and intermediate
tests with the result obtained for the advanced test.  Data is
transformed firstly, then the analysis of variance is conducted,
finally, the clustering is built. Found structure shows that
students successfully coping with elementary and intermediate
tests are likely to pass the advances level test. In the second
problem, the results of studying mathematics by junior pupils are
analyzed. Classification of pupils is made based on their marks
gained for the answer in the lesson. The classifier determines the
pupil mark for the final control work. The predictive model is
built as the ensemble of random forests trained on four samples:
the first is a sparse matrix of estimates, the others are the
transformation of
the first obtained by principal component analysis within a nuclear structure.\\[1mm]
\textit{Keywords}: statistics, random forest, clustering, the
methodics of studying Russian language and mathematics, the
analysis of education progress.
\par}

\vskip 4mm%5mm

\noindent \textbf{References} }

\vskip 1.5mm%2mm

{\footnotesize

1. Gurova G., Piattoeva N., Takala T. Quality of education and its
evaluation: An analysis of the Russian Academic Discussion.  {\it
European Education,} 2015, vol.~47(4), pp.~346--364.

2. \textit{Prikaz ministerstva obrazovaniya i nauki RF ot
18.06.2017 no. 667   ``O formah provedeniya gosudarstvennogo
testirovaniya po RKI'' $[$Order of RF ministry of education dated
18.06.2017 N~667 ``About forms of state testing on russian
lunguage as foreign language''$]$}. Moscow, 2017 (In Russian)

3. Andryushina N. P., Bitehtina G. �., Ivanona �. S. et al. {\it
Trebovaniya k urovnyu podgotovki po russkomu yazyku kak
inostrannomu. I sertifikacionnyj uroven. Obshchee vladenie
$[$Requirements on level of Russian  as foreign language
proficiency. I certification level. General skills$]$}. 2nd ed.
Moscow, Saint Petersburg, Zlatoust Publ., 2009, 32~p. (In Russian)

4.  Vladimirova T. E., Nahabina �. �., Soboleva N. I. et al. {\it
Gosudarstvennyj obrazovatelnyj standart po russkomu yazyku kak
inostrannomu. Ehlementarnyj uroven $[$State educational standart
on Russian as foreign language. Elementary level$]$}. 2nd ed.
Moscow, Saint Petersburg, Zlatoust Publ., 2001, 28~p. (In Russian)

5.  Nahabina M. M., Soboleva N. I., Starodub V. V. et al.{\it
Gosudarstvennyj obrazovatelnyj standart po russkomu yazyku kak
inostrannomu. Bazovyj uroven $[$State educational standart on
Russian as foreign language. Basic level$]$}. 2nd ed. Moscow,
Saint Petersburg, Zlatoust Publ., 2001, 32~p. (In Russian)

6.  Andryushina N. P., Bitehtina G. �., Vladimirova T. E. et al.
{\it Programma po russkomu yazyku dlya innostrannyh grazhdan.
Pervyj sertifikacionnyj uroven. Obshchee vladenie $[$Programm on
Russian language for foreigners. The first certification level.
General skills$]$}. Moscow, Saint Petersburg, Zlatoust Publ.,
2001, 176~p. (In Russian)

7. Kapitonova T. I., Moskovkin L. V. {\it Metodika obucheniya
russkomu yazyku kak inostrannomu na ehtape predvuzovskoj
podgotovki $[$Methodology of teaching Russian as foreign language
pre-university education$]$}. Saint Petersburg, Zlatoust Publ.,
2006, 360~p. (In Russian)

8. Esina Z. I., Ivanova A. S., Soboleva N. I. et al. {\it
Obrazovatelnaya programma po russkomu yazyku kak inostrannomu.
Predvuzovskoe obuchenie. Ehlementarnyj uroven.  Bazovyj uroven.
Pervyj sertifikacionnyj uroven $[$Educational programm  on Russian
as foreign language. Elementary level. Basic level. The first
certification level$]$}. Moscow, [without Publ.], 2001, 134~p. (In
Russian)

9. Bartlett M. S. Properties of sufficiency and statistical tests.
{\it Proceedings of the Royal Statistical Society}. {\it
Series~A}, 1937, no.~60,  pp.~268--282.

10. Raykov Y. P., Boukouvalas A., Baig F., Little M. A. What to do
when $K$-means clustering fails: A simple yet principled
alternative algorithm. {\it  PLoS ONE}, 2016, vol.~11, no.~9,
p.~28.

11. Grebennikova N. L. {\it Metody i priemy resheniya
nestandartnyh zadach $[$Methods and approaches to deal with
nonstandard problems$]$}. Sterlitamak, Sterlitamakskaya gos. ped.
akademia im. Zajnab Biishevoj, 2010, 310~p. (In Russian)

12. Grebennikova N. L., Koscova S. A. {\it Podgotovka budushchih
uchitelej nachalnyh klassov k realizacii standartov vtorogo
pokoleniya $[$Training of future teachers of primary school to
implement the standards of the second generation: Educational
monograph$]$}. Sterlitamak, Sterlitamakskaya gos. ped. akademia
im.~Zajnab Biishevoj, 2012, 245~p. (In Russian)

13. Grebennikova N. L., Koscova S. A. Sovremennye tekhnologii
nachalnogo matematicheskogo obrazovaniya $[$Modern technologies of
primary mathematics education.$]$  {\it Proceedings of V
Interna-tional extramural scientific and practical conference
``Education in spaces of school and adult school: experience,
problems and perspectives''}. Bashkortostan, Sterlitamak, 21 April
2016. Sterlitamak, Bashkir State University, 2016, pp.~82--83. (In
Russian)

14. Hastie T., Tibshirani R., Friedman J. {\it The elements of
statistical learning: data mining, inference, and prediction.} 2nd
ed. New York, Springer-Verlag Publ., 2009, 746~p.

15. Cohen J. A coefficient of agreement for nominal scales. {\it
Educational and Psychological Measurement,} 1960, vol.~20, no.~1,
pp.~37--46.

\vskip 1.5mm

Received:  August 28, 2018.

Accepted: September 25, 2018.


\vskip4mm%5mm
A\,u\,t\,h\,o\,r's \ i\,n\,f\,o\,r\,m\,a\,t\,i\,o\,n:

\vskip1.5mm \textit{Natalia A. Bure} --�  PhD in  Philology,
Associate Professor; nataly.bure@gmail.com

\vskip1.5mm \textit{Nadezhda L. Grebennikova} --� PhD in
Pedagogic, Associate Professor; nadezhda151049@mail.ru

\vskip1.5mm \textit{Kseniya Yu. Staroverova} ---  Postgraduate
Student;  st016234@student.spbu.ru\par
%
}
