
{\normalsize

\vskip 4.5mm%6mm

\noindent{\bf Mathematical model of the integrated supply chain}

}

\vskip 1.8mm%2mm

{\small

\noindent{\it V.~M.~ Bure$\,^1$%
, V.~V.~Karelin$^1$%
, L.~N.~Polyakova$^1$%
, A.~V.~Flegontov$\,^{1,2}$%%, I.~�. Famylia%$\,^2$%

 }

\vskip 1.8mm%2mm

%%%%%%%%%%%%%%%%%%%%%%%%%%%%%%%%%%%%%%%%%%%%%%%%%%%%%%%%%%%%%%%%%%

%\efootnote{
%%
%\vspace{-3mm}\parindent=7mm
%%
%\vskip 0.1mm $^{*}$ This work is supported by Russian Science
%Foundation (project N 18-71-00006 ).\par
%%
%%\vskip 2.0mm
%%
%%\indent{\copyright} �����-������������� ���������������
%%�����������, \issueyear%
%%
%}

%%%%%%%%%%%%%%%%%%%%%%%%%%%%%%%%%%%%%%%%%%%%%%%%%%%%%%%%%%%%%%%%%%

{\footnotesize

\noindent%
$^1$~%
St.\,Petersburg State University, 7--9, Universitetskaya nab.,
St.\,Petersburg, 199034, Russian Federation

%\noindent%
%\hskip2.45mm%



\noindent%
$^2$~%
Herzen State Pedagogical University of Russia,  48, nab. r. Moika,
St.\,Petersburg,

\noindent%
\hskip2.45mm%
191186, Russian Federation

}

%%%%%%%%%%%%%%%%%%%%%%%%%%%%%%%%%%%%%%%%%%%%%%%%%%%%%%%%%%%%%%%%%%

\vskip1.8mm%3mm


\noindent \textbf{For citation:}  Bure V.~M.,~Karelin
V.~V.,~Polyakova L.~N.,~Flegontov A.~V. Mathematical model of the
integrated supply chain. {\it Vestnik of Saint~Petersburg
University. Applied Mathematics. Computer Science. Control
Processes},\,\issueyear,
vol.~15,~iss.~\issuenum,~pp.~\pageref{p5}--\pageref{p5e}.\\
\doivyp/\enskip%
\!\!\!spbu10.\issueyear.\issuenum05  (In Russian)

\vskip3mm

{\leftskip=7mm\noindent Supply chain management occupies a very
important place in the activities of any company in a globalized
economy and increasing competition in the market. The main goal of
supply chain management is to coordinate the work of firms ---
suppliers of raw materials, firms-manufacturers and trading
companies selling goods on the market. The article studies a
continuous mathematical model describing the interaction of the
listed firms under conditions of a non-constant rate of supply of
some kind of raw materials. It is assumed that the speed of supply
of these raw materials can take two possible values, the choice of
which is determined by the manufacturer, the manufacturer, the
higher rate of supply of raw materials corresponds to the
intensive production variant of the product, the slower speed
corresponds to the usual production variant. Mathematical modeling
is carried out using differential equations. An optimization
problem is formulated, which consists in choosing the time point
for switching the mode of supply of raw materials from the
intensive version to the normal version in order to maximize the
income of the manufacturer-manufacturer.\\[1mm]
\textit{Keywords}: stock level of the goods, multi-echelon supply
chain, production.
\par}

\vskip5mm

\noindent \textbf{References} }

\vskip 2mm

{\footnotesize

1. {Pal B., Sana Sh. S., Chaudhuri K.} A three layer multi-item
production-inventory model for multiple suppliers and retailers.
{\it Economic Modelling}, 2012, vol.~29, pp.~2704�-2710.

2. {Ben-Daya M., Al-Nassar A.} An integrated inventory production
system in a three-layer supply chain. {\it Production Planning and
Control}, 2008, vol.~19~(2), pp.~97--104.

3. {Bhattacharya D. K.} On multi-item inventory. {\it European
Journal of Operational Research}, 2005, vol.~162, pp.~786--791.

4. {Brandimarte P.} Multi-item capacitated lot-sizing with demand
uncertainty. {\it Intern. J. of Pro\-duction Research}, 2006,
vol.~44~(15), pp.~2997--3022.

5. {Kamali A., Fatemi Ghomia S.~M.~T., Jolai F.} A multi-objective
quantity discount and joint optimization model for coordination of
a single-buyer multi-vendor supply chain. {\it Computers and
Mathematics with Applications}, 2011, vol.~62, pp.~3251--3269.

6. {Polyakova L. N., Bure V. M., Karelin V. V.} Maksiminniy podhod
k otsenke ob'ioma zakaza tovara v usloviah padenia sprosa [Maximin
approach in estimating of the goods order volume under condition
of falling demand]. {\it Vestnik of Saint Petersburg University.
Applied Mathematics. Computer Science. Control Processes}, 2018,
vol.~14, iss.~4, pp.~352--361.
htpps://doi.org/10.21638/11702/spbu10.2018.408 (In~Russian)

7. {Bure V. M., Karelin V. V., Bure A. V.} Otsenka ob'ioma zakaza
tovara pri vozmozhnom padenii sprosa [Evaluation of the volume of
ordering of goods while possible demand drop]. {\it Vestnik of
Saint Petersburg University. Applied Mathematics. Computer
Science. Control Processes}, 2018, vol.~14, iss.~3, pp.~252--260.
htpps://doi.org/10.21638/11702/spbu10.2018.306 (In Russian)

8. {Bure V. M., Karelin V. V., Polyakova L. N., Yagol'nik I. V.}
Modelirovanie protsessa zakaza dlia kusochno-lineinogo sprosa s
nasyscheniem [Modeling of the ordering process for
piecewise-linear demand with saturation]. {\it Vestnik of Saint
Petersburg University. Applied Mathematics. Computer Science.
Control Processes}, 2017, vol.~13, iss.~2, pp.~138--146. (In
Russian)

9. {Bure V. M., Karelin V. V., Myshkov S. K., Polyakova L. N.}
Determination of order quantity with piecewise-linear demand
function with saturation. {\it Intern. J. of Applied Engineering
Research}, 2017, vol.~12, no.~18, pp.~7857--7862.

10. {Bure V. M., Karelin V. V., Polyakova L. N.} Probabilistic
model of terminal services. {\it Applied Mathematical Sciences},
2016, vol.~10, no.~39, pp.~1945--1952.


\vskip1.5mm Received:  May 15, 2019.

Accepted: June 06, 2019.


\vskip6mm A\,u\,t\,h\,o\,r's \ i\,n\,f\,o\,r\,m\,a\,t\,i\,o\,n:%

\vskip1.5mm%2mm
\textit{Vladimir M. Bure} ---  Dr. Sci. in
Technics, Professor; vlb310154@gmail.com

\vskip1.5mm%2mm
\textit{Vladimir V. Karelin} ---  PhD in Physics and Mathematics,
Associated Professor; vlkarelin@mail.ru

\vskip1.5mm%2mm
\textit{Lyudmila N. Polyakova} ---  Dr. Sci. in Physics and
Mathematics, Professor; lnpol07@mail.ru

\vskip1.5mm%2mm
\textit{Aleksander V. Flegontov} --- Dr. Sci. in Physics and
Mathematics, Professor; flegontoff@yandex.ru

}
