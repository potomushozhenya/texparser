
{\normalsize

\vskip 6mm

\noindent{\bf Traffic queue management algorithm for availability
control\\ in telecommunication networks\!$\,^*$%
}

}

\vskip 2mm

{\small

\noindent{\it Yu. M. Monakhov, A. P. Kuznetsova, M. R. Ismailova%$\,^1$%
%, A. L. Chistov$\,^2$%
%, D. M. Kuchinsky$\,^1$%

 }

\vskip 2mm

%%%%%%%%%%%%%%%%%%%%%%%%%%%%%%%%%%%%%%%%%%%%%%%%%%%%%%%%%%%%%%%%%%

\efootnote{
%%
\vspace{-3mm}\parindent=7mm
%%
\vskip 0.1mm $^{*}$ This work was supported by Russian Foundation
for Basic Reaserch (grants N~16-47-330055, 18-07-01109).\par
%%
}

%%%%%%%%%%%%%%%%%%%%%%%%%%%%%%%%%%%%%%%%%%%%%%%%%%%%%%%%%%%%%%%%%%

{\footnotesize

\noindent%
%$^2$~%
Vladimir State University,  87, Gorky ul., Vladimir,

\noindent%
%\hskip2.45mm%
600026, Russian Federation

}

%%%%%%%%%%%%%%%%%%%%%%%%%%%%%%%%%%%%%%%%%%%%%%%%%%%%%%%%%%%%%%%%%%

\vskip3mm


\noindent \textbf{For citation:}  Monakhov Yu. M., Kuznetsova A.
P., Ismailova M. R. Traffic queue mana\-ge\-ment algorithm for
availability control in telecommunication networks. {\it Vestnik
of Saint Petersburg University. Applied Mathematics. Computer
Science. Control Processes},\,\issueyear, vol.~15, iss. \issuenum,
pp.~\pageref{p8}--\pageref{p8e}.
\doivyp/\enskip%
\!\!\!spbu10.\issueyear.\issuenum08  (In Russian)

\vskip3mm

{\leftskip=7mm\noindent The current state of telecommunications is
characterized by more and more increasing scale of networks,
increase in the speeds of data transmission and continuous
emergence of the new services and applications which use various
protocols and peruse resources of network diverently. However,
with development of sensor networks there is an increasing share
of the traffic sensitive to environmental changes. Therefore, for
more effective use of network resources the problems of priority
management and traffic control aimed at increasing the
availability of both the telecommunication system in general and
the prioritized services in particular, are relevant. In this
article authors offer the priority-based traffic queue management
algorithm which allows to optimize channel capacity utilization
and to provide minimum possible delay for prioritized classes.
This algorithm is based on the known hierarchical token-bucket
method and serves for minimization of packet delay times on the
routing device. Testing and comparison of the proposed algorithm
with the existing decisions has allowed to draw a conclusion that
the offered algorithm shows lower total delays for various classes
of traffic.\\[1mm]
\textit{Keywords}: network availability, flow control, traffic
shaping, scheduling algorithm, technology ``Quality of Service'',
algorithm Hierarchical Token Bucket.
\par}

\vskip5mm

\noindent \textbf{References} }

\vskip 2mm

{\footnotesize

1. Vegesna S. {\it IP Quality of Service}. Innopolis, Cisco Press,
2001, 232~p. (Rus. ed.: Vegesna~S. {\it Kachestvo obsluzhivaniya v
setyakh IP. Osnovopolagayushchie principy realizacii funkcij
kachestva ob\-slu\-zhi\-vaniya v setyakh Cisco}. Moscow, House
Williams Publ., 2003, 368~p.)

2. Roughan M., Sen S., Spatscheck O., Duffield N. Class-of-service
mapping for QoS: a statistical signature-based approach to IP
traffic classification, {\it Proceedings of the 4-th ACM SIGCOMM
conference on Internet measurement}, 2004, pp.~35--148.

3. {\it Quality of Service Overview}. The Cisco Company. Available
at: https://www.cisco.com/c/en/us/td  \\
/docs/ios/qos/configuration/guide/12\_2sr/qos\_12\_2sr\_book/qos\_overview.html
(accessed: 13.01.2019).

4. Chapurin E. N. Predlozheniya po povysheniyu effektivnosti
funkcionirovaniya infokommunika\-cionnoj seti svyazi [Suggestions
for improving the efficiency of the information and communication
network]. {\it Sovremennye problemy nauki i obrazovaniya $[$Modern
for problems of science and education$]$}, 2015, no.~1-1,
pp.~425--430. (In Russian)

5. Kobrin A. V. Adaptivnyj bufer kompensacii dzhittera zaderzhki
pribytiya paketov na osnove robastnogo fil'tra Kalmana [Adaptive
jitter compensation buffer for packet arrival delay based on
robust Kalman filter]. {\it Problemy telekommunikacij $[$Problems
of telecommunications$]$}. Khar'kov, Khar'kovskiy national
radioelectronic University Publ., 2013, 10~p. Available at:\\
http://openarchive.nure.ua/bitstream/document/459/1/131\_kobrin\_jitter.pdf
(accessed: 12.01.2019).~(In Russian)

6. Prasad R., Dovrolis C., Murray M., Claffy K. C. Bandwidth
estimation: metrics, measurement techniques, and tools. {\it IEEE
network}, 2003, vol.~17, no.~6, pp.~27--35.

7. Olifer V. G., Olifer N. A. {\it Komp'yuternye seti. Principy,
tekhnologii, protokoly $[$Computer networks. Principles,
technologies, protocols$]$}. Saint Pe\-ters\-burg, Peter Publ.,
2006, 958~p. (In Russian)

8. Kudzinovskaya I. P. {\it Analiz metodov obespecheniya kachestva
obsluzhivaniya v vysokoskorostnykh komp'yuternykh setyakh
$[$Analysis of methods for ensuring quality of service in
high-speed computer networks$]$}. Kiev, Nauka Publ., 2008.
Available at:\\
http://jrnl.nau.edu.ua/index.php/PIU/article/download/9276/11507
(accessed: 02.12.2018). (In Russian)

\vskip1.5mm Received:  December 03, 2018.

Accepted: June 06, 2019.


\vskip6mm A\,u\,t\,h\,o\,r's \ i\,n\,f\,o\,r\,m\,a\,t\,i\,o\,n:%

\vskip2mm \textit{Yury M. Monakhov}~--- PhD in Technics, Associate
Professor;  unklefck@gmail.com

\vskip2mm \textit{Anna P. Kuznetsova}~--- Postgraduate Student;
akuznecova@vlsu.ru

\vskip2mm \textit{Maria R. Ismailova}~--- Student;
maryfelin@gmail.com

}
