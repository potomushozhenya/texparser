
{\normalsize

\vskip 6mm

\noindent{\bf Digital control design based on predictive models
to keep\\ the controlled variables in a given range$\,^*$%
 }

}

\vskip 2mm

{\small

\noindent{\it M.~V.~Sotnikova%$\,^1$%
%, N. L. Grebennikova$\,^2$%
%, K. Yu. Staroverova$\,^1$%
%, I.~�. Famylia%$\,^2$%
} %

\vskip 2mm

%%%%%%%%%%%%%%%%%%%%%%%%%%%%%%%%%%%%%%%%%%%%%%%%%%%%%%%%%%%%%%%%%%

\efootnote{
%%
\vspace{-3mm}\parindent=7mm
%%
\vskip 0.1mm $^{*}$ This work is supported by Russian Found of
Fundamental Research (project N 17-07-00361a).\par
%%
%%\vskip 2.0mm
%%
%%\indent{\copyright} �����-������������� ���������������
%%�����������, \issueyear%
%%
}

%%%%%%%%%%%%%%%%%%%%%%%%%%%%%%%%%%%%%%%%%%%%%%%%%%%%%%%%%%%%%%%%%%

{\footnotesize

\noindent%
%$^1$~%
St.\,Petersburg State University, 7--9, Universitetskaya nab.,
St.\,Petersburg,

\noindent%
%\hskip2.45mm%
199034, Russian Federation

%\noindent%
%$^2$~%
%St.\,Petersburg State University, 7--9, Universitetskaya nab.,
%St.\,Petersburg,

%\noindent%
%\hskip2.45mm%
%199034, Russian Federation

}

%%%%%%%%%%%%%%%%%%%%%%%%%%%%%%%%%%%%%%%%%%%%%%%%%%%%%%%%%%%%%%%%%%

\vskip3mm

\noindent \textbf{For citation:}  Sotnikova M.~V. Digital control
design based on predictive models to keep the controlled variables
in a given range. {\it Vestnik of Saint~Petersburg University.
Applied Mathematics. Computer Science. Control Processes},
\issueyear, vol.~15, iss.~\issuenum,
pp.~\pageref{p9}--\pageref{p9e}.\\
\doivyp/\enskip%
\!\!\!spbu10.\issueyear.\issuenum09  (In Russian)

\vskip3mm

{\leftskip=7mm\noindent The problem of digital control design to
keep the controlled variables of a dynamic object in a given range
is considered, taking into account the constraints imposed on the
manipulated variables. The essence of the problem is to ensure
that the process variables are enter and then kept within the
required range. The change of variables within the range can be
arbitrary. Such problem necessitates the development of special
methods for the design of control laws, which are different from
the classical approaches, where the control objective is given by
a reference signal. An approach to the synthesis of digital
control law, based on the use of predictive models, is proposed.
In the framework of this approach, the control objective is
achieved by introducing a special quadratic cost functional, which
includes the penalty term for the output of controlled variables
from the required range. Minimization of this functional on the
prediction horizon, taking into account the existing constraints
on the manipulated variables, ensures that the controlled
variables fall in the given range. It is shown that the real-time
implementation of the control law imply solving the quadratic
programming problem at each instant of discrete time. The
effectiveness of the developed control algorithm is illustrated by
examples of modeling of oil refining processes in a distillation
column.\\[1mm]
\textit{Keywords}: digital control, predictive
model, optimization, control in a range, distillation column.
\par}

\vskip 5mm
%\newpage

\noindent \textbf{References} }

\vskip 2mm

{\footnotesize


{1}. {Veremey E. I.} {\it Lineynye sistemy s obratnoi svyaz'yu.}
Ucheb. posobie $[${\it Linear systems with opposite
communications}. Tutorial$]$. Saint Petersburg, ``Lan'' Publ.,
2013, 448~p. (In Russian)

{2}. {Burdick D. L., Leffler W. L.} {\it Petrochemicals in
nontechnical language}. Oklahoma, USA,  PennWell Publ. Company,
1990, 347~p.

{3}. {Veremei E. I., Korchanov V. M.} Multiobjective stabilization
of a certain class of dynamic systems. {\it Automation and Remote
Control}, 1989, vol.~49, no.~9, pp.~1210--1219.

{4}. {Veremey E. I.} Spektral'nyy podkhod k optimizatsii sistem
upravleniya po normam prostranstv H2 i Hinf [Spectrum method of
approach to optimization of system management on standards spaced
H2 and Hinf]. {\it Vestnik of Saint Petersburg State University.
Applied Mathematics. Computer Sciences. Control Processes}, 2004,
iss.~1--2, pp.~48--59. (In Russian)

{5}. {Veremey E. I., Sotnikova M. V.} {\it Upravlenie s
prognoziruyushchimi modelyami}. Ucheb. posobie $[${\it Management
with prognozing models}. Tutorial\,$]$. Voronezh, Nauchnaya kniga
Publ., 2016, 214~p. (In Russian)

{6}. {Lahiri S. K.} {\it Multivariable predictive control:
Applications in industry}. Hoboken, NJ, USA, John Wiley \& Sons
Publ., 2017, 304~p.

{7}. {Nocedal J., Wright S. J.} {\it Numerical optimization}. 2nd
ed. Berlin, New York, Springer-Verlag Publ., 2006, 449~p.


\vskip 1.5mm

Received:  May 02, 2019.

Accepted: June 06, 2019.


\vskip5mm A\,u\,t\,h\,o\,r's \ i\,n\,f\,o\,r\,m\,a\,t\,i\,o\,n:

\vskip1.5mm \textit{Margarita V. Sotnikova} --- Dr. Sci. in
Physics and Mathematics, Associate Professor;\\
m.sotnikova@spbu.ru
\par
%

}
