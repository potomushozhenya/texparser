
{\small

\vskip6mm

\noindent \textbf{References} }

\vskip 2mm

{\footnotesize

1. Godina O. V., Maksimenko L. S., Savtsova A. V. Dynamic approach
to strategic analysis of the innovative environment of the
regional socio-economic system. \emph{Indian Journal of Science
and Technology}, 2016, vol.~9~(16), pp.~1--6.
https://doi.org.10.17485/ijst/2016/v9i16/90048

2. Koltsaklis N. E., Dagoumas A. S., Georgiadis M. C., Papaioannou
G., Dikaiakos C. A mid-term, market-based power systems planning
model. \emph{Applied Energy}, 2016, vol.~179, pp.~17--35.
https://doi.org.10.1016/j.apenergy.2016.06.070

3. Antamoshkin N. A., Antamoshkina O. I., Hodos D. V.
Modelirovanie osnovnyh ehtapov for\-mi\-ro\-va\-niya programmy
innovacionnogo razvitiya [Basic steps of innovative
de\-ve\-lop\-ment program formation modeling]. \emph{Sibirskij
zhurnal nauki i tekhnologij} [\emph{Siberian Journal of Science
and Technology}], 2010, vol.~4~(30), pp.~204--206. (In Russian)

4. Nikolskij M. S. Uproshchennaya igrovaya model vzaimodejstviya
dvuh gosudarstv [A simplified game model of the interaction
between two countries]. \emph{Vestnik Moskovskogo universiteta.
Seriya 15. Vychislitel'naya matematika i kibernetika}
[\emph{Vestnik of Moscow University. Series 15. Computational
Mathematics and Cybernetics}], 2009, vol.~2, pp.~14--20. (In
Russian)

5. Buyanov B. B., Lubkov N. V., Polyak G. L. Sistema podderzhki
prinyatiya upravlencheskih reshenij s primeneniem imitacionnogo
modelirovaniya [Management decision support system using
simulation]. \emph{Problemy upravleniya} [\emph{Control
Sciences}], 2006, vol.~6, pp.~43--49. (In Russian)

6. Aitova J. S., Oreshnikov V. V. Podhody k modelirovaniyu
vzaimovliyaniya demograficheskogo potenciala i ehkonomicheskogo
razvitiya regionov Rossii [Approaches to modelling power relations
of demographic potential and economic development of Russian
regions].\emph{ Vestnik NGIEHI} [\emph{Bulletin NGII}], 2018,
vol.~12~(91), pp.~69--80. (In Russian)

7. Zhang X., Hua Q., Zhang L. Development and application of a
planning support system for regional spatial functional zoning
based on GIS.  \emph{Sustainability}, 2016,  vol.~8, pp.~1--17.\\
https://doi.org.10.3390/su8090909

8. Nizamutdinov M. M., Oreshnikov V. V. Instrumentarij
prognozirovaniya izmeneniya parametrov regionalnogo razvitiya na
osnove adaptivno-imitacionnogo podhoda [Tools predict changes in
the parameters of regional development on the basis of the
adaptive simulation approach]. \emph{The 3rd In\-ter\-na\-tional
conference ``Information technologies for intelligent
decision-making support ITIDS'2015''}. Ufa, USATU Publ., 2015,
pp.~211--215. (In Russian)

9. Zakharova A. A. Integralnaya ocenka innovacionnogo razvitiya
regiona na osnove nechetkih mnozhestv [Integrated assessment of
innovative development of the region on the basis of fuzzy sets].
\emph{Nauchnoe obozrenie. Tekhnicheskie nauki} [\emph{Scientific
review. Technical Science}], 2014, vol.~1, pp.~161--168. (In
Russian)

10. Palyukh B. V., Kakatunova T. V. Nechetkaya kognitivnaya karta
kak instrument modelirovaniya innovacionnoj deyatelnosti na
regionalnom urovne [A fuzzy cognitive map as a tool to model
innovation\linebreak at the regional level]. \emph {Programmnye
produkty i sistemy} [\emph{Software and Systems}], 2012, vol.~4,
pp.~128--131. (In Russian)

11. Gorelova G. V., Verba V. A., Zaharova E. N. Process prinyatiya
reshenij i ego podderzhka na osnove kognitivnogo modelirovaniya
[Decision-making process and its support on the basis of
cog\-ni\-ti\-ve modeling]. \emph{Izvestiya JuFU. Ingenerniye
nauki} [\emph{Proceedings of SFedU. Engineering Sciences}], 2005,
vol.~10~(54), pp.~13--20. (In Russian)

12. Tereljansky P. V. Matematicheskie i instrumental'nye sredstva
podderzhki prinyatiya reshenij v ehkonomike [Mathematical and
instrumental supporting of decision making in the economics].
\emph{Audit i finansovyj analiz} [\emph{Audit and Financial
Analysis}], 2008, vol.~6, pp.~461--471. (In Russian)

13. Tyushnyakov V. N., Zhertovskaya E. V., Yakimenko M. V.
Informacionno-analiticheskoe obespechenie situacionnogo centra kak
osnova razrabotki strategij innovacionnogo razvitiya regiona
[Information and analytical support of situational centers as the
basis for the strategy choice of region innovative development].
\emph{Fundamentalnye issledovaniya} [\emph{Fundamental Research}],
2015, vol.~11--6, pp.~1253--1257. (In Russian)

14. Chernyahovskaya L. R., Fedorova N. I., Nizamutdinova R. I.
Intellektualnaya podderzhka prinyatiya reshenij v operativnom
upravlenii delovymi processami predpriyatiya [Intellectual
decision support in the operational management of business
processes]. \emph{Vestnik Ufimskogo gosudarstvennogo aviacionnogo
tekhnicheskogo universiteta} [\emph{Ufa State Aviation Technical
University Journal}], 2011, vol.~2~(42), pp.~172--176. (In
Russian)

15. Pechatkin V. V. Konkurentosposobnost, konkurentoustojchivost i
ehkonomicheskaya bezopasnost kak osnova sistemy monitoringa
razvitiya regiona [Competitiveness, competitiveness and economic
security as a basis for monitoring the development of the region].
\emph{Konkurentosposobnost v globalnom mire: ehkonomika, nauka,
tekhnologii} [\emph{Competitiveness in the Global World:
Economics, Science, Technology}], 2017, vol.~12~(59),
pp.~984--987. (In Russian)

16. Nizamutdinov M. M. Konceptualnye i metodicheskie aspekty
zadachi modelirovaniya razvitiya territorial'nyh sistem
municipalnogo urovnya [Modeling municipal level territorial
systems� development: conceptual and methodical aspects].
\emph{Upravlencheskie nauki $[$Management Science$]$}, 2017,
vol.~7~(2), pp.~23--31. (In Russian)

17. Veremej E. I. Optimizacionnyj podhod k modelirovaniyu i
razrabotke informacionno-up\-rav\-lyayushchih system [Optimization
approach to modeling and development of information
ma\-na\-ge\-ment systems]. \emph{Prikladnaya informatika}
[\emph{Journal of Applied Informatics}], 2012, vol. 6(42), pp.
34--41. (In Russian)



\vskip 1.5mm

%\noindent Recommendation: prof. L. A. Petrosyan.
%
%\vskip 1.5mm

%\noindent
Received:  April 02, 2019.

Accepted: June 06, 2019.

\vskip6mm A\,u\,t\,h\,o\,r's \ i\,n\,f\,o\,r\,m\,a\,t\,i\,o\,n:

\vskip1.5mm\textit{Albert R. Bakhtizin} --- Dr. Sci. in Economics,
Corr. Member of RAS; albert.bakhtizin@gmail.com

\vskip1.5mm\textit{Marsel M. Nizamutdinov} --- PhD in Engineering;
marsel\_n@mail.ru

\vskip1.5mm\textit{Vladimir V. Oreshnikov} --- PhD in Economics;
voresh@mail.ru

}
