
{\normalsize

\vskip 6mm

\noindent{\bf The formula for the subdifferential of the distance
function\\
to a convex set in an nonsymmetrical space%$\,^*$%
}

}

\vskip 2mm

{\small

\noindent{\it V. V. Abramova%$\,^1$%
, S. I. Dudov%$\,^2$%
, A. V. Zharkova%$\,^2$%
%, I.~�. Famylia%$\,^2$%

 }

\vskip 2mm

%%%%%%%%%%%%%%%%%%%%%%%%%%%%%%%%%%%%%%%%%%%%%%%%%%%%%%%%%%%%%%%%%%

%\efootnote{
%%
%\vspace{-3mm}\parindent=7mm
%%
%\vskip 0.1mm $^{*}$ This work is supported by Russian Science
%Foundation (project N 18-71-00006).%\par
%%
%%\vskip 2.0mm
%%
%\indent{\copyright} �����-������������� ���������������
%�����������, \issueyear%
%%
%}

%%%%%%%%%%%%%%%%%%%%%%%%%%%%%%%%%%%%%%%%%%%%%%%%%%%%%%%%%%%%%%%%%%

{\footnotesize

\noindent%
%$^2$~%
Saratov National Research State University, 83, Astrakhanskaya
ul., Saratov,

\noindent%
%\hskip2.45mm%
410012, Russian Federation

%\noindent%
%$^2$~%
%St.\,Petersburg State University, 7--9, Universitetskaya nab.,
%St.\,Petersburg,

%\noindent%
%\hskip2.45mm%
%199034, Russian Federation

}

%%%%%%%%%%%%%%%%%%%%%%%%%%%%%%%%%%%%%%%%%%%%%%%%%%%%%%%%%%%%%%%%%%

\vskip3mm

\noindent \textbf{For citation:}   Abramova V. V., Dudov S. I.,
Zharkova A. V. The formula for the subdifferen\-tial of the
distance function to a convex set in an nonsymmetrical space. {\it
Vestnik of Saint~Peters\-burg University. Applied Mathematics.
Computer Science. Control Processes}, \issueyear, vol.~15,
iss.~\issuenum, pp.~\pageref{p1}--\pageref{p1e}.
\doivyp/\enskip%
\!\!\!spbu10.\issueyear.\issuenum01  (In Russian)

\vskip3mm

{\leftskip=7mm\noindent The distance function, defined by the
gauge (the Minkowsky gauge function) of a  convex body compact,
from a point to a convex closed set is considered in a
finite-dimensional space. It is known that this function is convex
in the whole space. The formula of its the subdifferential is
obtained. It includes the subdifferential of gauge function and
the cone of feasible directions of set to which the distance is
measured, taken  in one of the projection points on this set. This
circumstans makes it different from the subdifferentional formula
received earlier by B.~N.~Pshenichny in which another
characteristics of the objects, defined the distance function, are
used. Examples of applications of the obtained formula are given.
In particular, a specific form of the subdifferential formula is
given for the case when the set, the gauge of which specifies the
distance function, and the set to which the distance is measured
are lower Lebesgue sets of convex functions.\\[1mm]
\textit{Keywords}: distance function, gauge of set,
subdifferential, support function, cone of feasible directions.
\par}

\vskip 5mm

\noindent \textbf{References} }

\vskip 2mm

{\footnotesize

1. Rockafellar~R. {\it Convex analysis}. Princeton, New Jersey
Princeton University Press, 1970, 450~p.

2. Pschenichnyi B. N. {\it Vypukliy analiz i extremalnye zadachi}
[{\it Convex analysis and extremal problems}]. Moscow, Nauka
Publ., 1980, 319~p. (In Russian)

3. Dunham~Ch.~B. Asymmetric norms and linear approximation. {\it
Congr. Numer.}, 1989, vol.~69, pp.~113--120.

4. Romaguera~S., Schellekens~M. Quasi-metric properties of
complexity spaces. {\it Topology Appl.}, 1999, vol.~98, no.~1--3,
pp.~311--322.

5. De Blasi~F.~S., Myjak~J. On generalized best approximation
problem. {\it  J. Approx.Theory}, 1998, vol.~94, no.~1,
pp.~54--72.

6. Alegre~C. Continuous operators on asymmetric normed spaces.
{\it Acta Math.~Hungar.}, 2009, vol.~122, no.~4, pp.~357--372.

7. Cobzas~S. {\it Functional analysis in asymmetric normed
spaces}. Basel, Birkhauser Publ., 2013, 219~p.

8. Alimov A. R. {\it Approksimativno-geometricheskie svoistva
mnojestv v normirovannyh i ne\-sim\-met\-ricno normirovannyh
prostranstvah} [{\it Approximate-geometric properties of sets in
normed and asym\-met\-ri\-cal\-ly normed spaces}]. Moscow,
Lomonosov Moscow State University Press, 2014,  207~p. (In
Russian)\newpage

9. Alimov A. R. Vypuklost' ogranichennyh chebyshevskih mnozhestv v
konechnomernyh pro\-stran\-st\-vah s nesimmetrichnoi normoi
[Convexity of bounded Chebyshev sets in finite-dimensional
asym\-met\-ri\-cal\-ly normed spaces]. {\it Izv. Sarat. University
(N.~S.), Ser. Mathematics. Mechanics. Informatics. Journal
Profile}, 2014, vol.~14, no.~4, pp.~489--497. (In Russian)

10. Ivanov~G.~E., Lopushanski~M.~S. {Approksimativnye svoistva
slabo vypuklyh mnozhestv v prostranstvah s nesimmetrichnoy
polunormoy} [{Approximate properties of weakly convex sets in
spaces with asymmetric seminorm}]. {\it  Trudy  ��sk.
fiz.-tekhnich. in-t�} [{\it Works of Moscow Institute of Physics
and Technology}], 2012, vol.~4, no.~4, pp.~94--104. (In Russian)

11. Ivanov~G.~E., Lopushanski~M.~S. Separation theorems for
nonconvex sets in space with non\-sym\-met\-ric seminorm. {\it J.
Mathematical Inequalities and Applications}, 2017, vol.~20, no.~3,
pp.~737--754.

12. Demyanov V. F., Vasiliev L. V. {\it Nedifferentsiruemaia
optimizatsia} [{\it Nondifferentiable op\-ti\-mi\-za\-tion}].
Moscow, Nauka Publ., 1981, 384~p. (In Russian)

13. Demyanov V. F., Rubinov A. V. {\it Osnovy negladkogo analiza i
kvazidifferencial�noe ischislenie} [{\it Elements of nonsmooth
analysis and quasidifferential calculus}]. Moscow, Nauka Publ.,
1990, 431~p. (Series Optimization and investigation of operation,
iss.~23.) (In Russian)

14. Dudov S. I. Subdifferentsiruemost' i superdifferentsiruemost'
funktsii rasstoyania  [Subdifferentiabi\-lity and
superdifferentiability of distance functions]. {\it Math. Notes},
1997, vol.~61, no.~4, pp.~440--450. (In Russian)

15. Dudov S. I. Differentsiruemost' po napravleniyam funktsii
rasstoyania [Directional differentiability of the distance
function]. {\it Mat. Sb.}, 1995, vol.~186, no.~3, pp.~29--52. (In
Russian)

16. Gorokhovik V. V. {\it Konechnomerniye zadachi optimizatsii
$[$Finite-dimensional optimization problems$]$}. Minsk, Belarus.
State University Press, 2007, 240~p. (In Russian)

17. Demyanov~V.~F. Uslovniye proizvodniye i ekzostery v negladkom
analize [Conditional derivatives and exhausters in nonsmooth
analysis]. {\it Dokl. of Russian Academy Sciences}, 1999,
vol.~338, no.~6, pp.~730--733. (In Russian)

18. Demyanov~V.~F. Exhausters of positively homogeneous function.
{\it Optimization}, 1999, vol.~45, pp.~13--29.

\vskip 1.5mm

Received:  February 22, 2019.

Accepted: June 06, 2019.


\vskip5mm A\,u\,t\,h\,o\,r's \ i\,n\,f\,o\,r\,m\,a\,t\,i\,o\,n:


\vskip2mm \textit{Veronika V. Abramova}~--- Postgraduate Student;
Veronika0322@rambler.ru

\vskip2mm \textit{Sergei I. Dudov}~--- Dr. Sci. in Physics and
Mathematics; DudovSI@info.sgu.ru

\vskip2mm \textit{Anastasia V. Zharkova}~--- PhD in Physics and
Mathematics; ZharkovaAV@gmail.com\par
%
}
