


{\footnotesize

\vskip 4mm
%\newpage

\noindent {\small\textbf{References} }

\vskip 4mm


1. Morozova N. S. Virtual'nye formacii i virtual'nye lidery v
zadache o dvizhenii stroem gruppy robotov [Virtual formations and
virtual leaders in formation control problem for group of robots].
{\it Vestnik of Saint Petersburg University. Series 10. Applied
mathematics. Computer science. Control processes}, 2015, issue 1,
pp. 135--148. (In Russian)

2. Sotnikova M. V. Algoritm avtomaticheskogo uderzhaniya kolesnogo
robota na vizual'no zadannoj linii [Algorithm for visual path
following by wheeled fully actuated mobile robot]. {\it Vestnik of
Saint Petersburg University. Series 10. Applied mathematics.
Computer science. Control processes}, 2016, issue~1, pp. 99--108.
(In Russian)

3. Chandrasekaran G., Ergin M., Yang J., Lui S., Chen Y., Gruteser
M., Martin R. Empirical evaluation of the limits of localization
using signal strength: Beyond Cramer---Rao bounds. {\it Processing
of IEEE SECON}, 2004, pp. 406--414.

4. Elnahrawy E., Li X., Martin R. The limits of localization using
signal strength: A comparative study. {\it Processing of IEEE
SECON}, 2004, pp. 406--414.

5. Galov A., Moschevikin A. Bayesian filters for ToF and RSS
measurements for indoor positioning of a mobile object. {\it
Processing of the Intern. Conference on Indoor Positioning and
Indoor Navigation} (IPIN-2013). Montbeliard, France, October
28--31, 2013, pp. 310--317.

6. Galov A. S., Moschevikin A. P., Voronov R. V. Combination of
RSS localization and ToF ranging for increasing positioning
accuracy indoors. {\it Processing of the 11th Intern. Conference
on ITS Telecommunications} (ITST), 2011, pp. 299--304.

7. Ata O. W., Ala'Eddin M. S., Jawadeh M. I., Amro A. I. An indoor
propagation model based on a novel multi wall attenuation loss
formula at frequencies 900 MHz and 2.4~GHz. {\it Wireless Personal
Communications}, 2013, vol. 69, no. 1, pp. 23--36.

8. Chrysikos T., Georgopoulos G., Kotsopoulos S. Site-specific
validation of ITU indoor path loss model at 2.4 GHz. {\it IEEE
Intern. Symposium on a World of Wireless, Mobile and Multimedia
Networks $\&$ Workshops}, 2009, pp. 1--6.

9. Molisch A. F., Balakrishnan K., Chong C. C., Emami S., Fort A.,
Karedal J., Schantz H., Schuster~U. IEEE 802.15.4a {\it channel
model-final report}. Technical Report, Document IEEE
802.1504-0062-02-004a, 2005.

10. Battiti R., Brunato M., Delai A. Optimal wireless access point
placement for location-dependent services. Technical Report, 2003.
 Available at:
http://eprints.\-biblio.\-unitn.it/489/1/DIT-03-052-withCover.pdf
(accessed: 30.11.2016).

11. Chen Y., Francisco J. A., Trappe W., Martin R. P. A practical
approach to landmark deployment for indoor localization. {\it
Sensor and Ad Hoc Communications and Networks}, 2006. {\it
SECON'06. 3rd Annual IEEE Communications Society on IEEE}, 2006,
vol. 1, pp. 365--373.

12. Gondran A., Caminada A., Fondrevelle J., Baala O. Wireless LAN
planning: a didactical model to optimise the cost and effective
payback. {\it Intern. Journal of Mobile Network Design and
Innovation}, 2007, vol. 2, no. 1, pp. 13--25.

13. Kim T., Shin J., Tak S. Cell planning for indoor object
tracking based on RFID. {\it Mobile Data Management: Systems,
Services and Middleware, MDM'09. Tenth Intern. Conference on
IEEE}, 2009, pp. 709--713.

14. Farkas K., Husz{\'a}k {\'A}., G{\'o}dor G. Optimization of
Wi-Fi access point placement for indoor localization. {\it Journal
IIT $($Informatics $\&$ IT Today$)$}, 2013, vol. 1,  no. 1, pp.
28--33.

15. Vilovi\'c I., Burum N. Location optimization of WLAN access
points based on a neural network model and evolutionary
algorithms. {\it Automatika$:$ \v{c}asopis za automatiku,
mjerenje, elektroniku, ra\v{c}unarstvo i komunikacije
$[$Automatika$:$ Journal for control, measurement, electronics,
computing and communications$]$}, 2015, vol. 55, no. 3, pp.
317--329.

16. Voronov R. V., Moschevikin A. P. Primenenie uslovnoj jentropii
pri formirovanii rekomendacij po razmeshheniju bazovyh stancij v
lokal'nyh sistemah pozicionirovanija [Use of conditional entropy
for optimal disposition of base stations in local positioning
systems]. {\it Information Technology}, 2014, no. 10, pp. 11--16.
(In Russian)



\vskip 2mm

{\bf For citation:}  Voronov R. V. The problem of  optimal
placement of access points for the indoor positioning system. {\it
Vestnik of Saint Petersburg University. Applied mathematics.
Computer science. Control processes}, \issueyear, volume~13,
issue~\issuenum, pp.~\pageref{p6}--\pageref{p6e}.
\doivyp/spbu10.\issueyear.\issuenum06


}
