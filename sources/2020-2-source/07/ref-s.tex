
{\normalsize

\vskip 4.5mm%6mm

\noindent{\bf Model study of the influence of multi-joint muscles\\ on the frequency characteristics of the human body %$^{*}$
}

}

\vskip 2mm

{\small

\noindent{\it V. P. Tregubov, N. K. Egorova %$\,^1$%
%, I.~�. Famylia%$\,^2$%

 }

\vskip 2mm

%%%%%%%%%%%%%%%%%%%%%%%%%%%%%%%%%%%%%%%%%%%%%%%%%%%%%%%%%%%%%%%%%%

%\efootnote{
%%
%\vspace{-3mm}\parindent=7mm
%%
%\vskip 0.1mm $^{*}$ This work is supported by Russian Science
%Foundation (project N 18-71-00006 ).\par
%%
%%\vskip 2.0mm
%%
%%\indent{\copyright} �����-������������� ���������������
%%�����������, \issueyear%
%%
%}

%%%%%%%%%%%%%%%%%%%%%%%%%%%%%%%%%%%%%%%%%%%%%%%%%%%%%%%%%%%%%%%%%%

{\footnotesize


\noindent%
%$^1$~%
St.\,Petersburg State University, 7--9, Universitetskaya nab., St.\,Petersburg,

\noindent%
%\hskip2.45mm%
199034, Russian Federation

}

%%%%%%%%%%%%%%%%%%%%%%%%%%%%%%%%%%%%%%%%%%%%%%%%%%%%%%%%%%%%%%%%%%

\vskip1.8mm%3mm


\noindent \textbf{For citation:}  Tregubov V. P., Egorova N. K. Model study of the influence of multi-joint muscles on the frequency characteristics of the human body. {\it Vest\-nik of Saint~Petersburg
University. Applied Mathematics. Computer
Science. Control Pro\-cesses}, %\,
\issueyear,
vol.~16,~iss.~\issuenum,~pp.~\pageref{p7}--\pageref{p7e}. \\
\doivyp/\enskip%
\!\!\!spbu10.\issueyear.\issuenum07  (In Russian)

\vskip3mm

{\leftskip=7mm\noindent The analysis of previous works devoted to
experimental studies of the frequency cha\-rac\-teris\-tics of the
body of a sitting human body subjected to vibration and the
construction of its mechanical models is conducted. It is noted
that in most cases measurements of vibration are made on the seat
and the head of the person that allows us to build a transmission
function from the seat to the head. In doing so, the supposed
internal structure of the human mechanical model remained
unconfirmed. Under these conditions, it was emphasized that
vibration measurements should be performed on all modeled parts of
a human body. In addition, from an anatomical point of view, the
main contribution to the mechanical properties of the body of a
sitting person is made by the multi-jointed muscles of the spine.
However, this important fact has not been taken into account yet.
In this regard, the task was set to study the influence of
multi-articular muscles on the frequency properties of the body of
a sitting person. For this purpose, a number of mechanical models
were constructed in which multi-jointed muscles were modeled by
multi-link viscoelastic connections. In particular, on the
simplest model with two degrees of freedom it was shown how the
imposition of two-link connections in addition to a single-link
one leads to the appearance of an anti-resonant frequency on the
upper mass, that is impossible in their absence. A mechanical
model with an arbitrary number of degrees of freedom with
multi-link connections is given, for which the formulas of the
transfer function and the amplitude-frequency response for
the upper mass are obtained. In addition, we consider a mechanical
model with an arbitrary set of multi-link connections. As an
example, the results of numerical calculation of the frequency
response for a mechanical model with eight degrees of freedom in
the presence and absence of multi-link connections are given.\\[1mm]
\textit{Keywords}: mechanical model, human body, vibration, transfer function, input mechanical impedance, amplitude-frequency response, multi-joint muscles.\par}

\vskip5mm

\noindent \textbf{References} }

\vskip 2mm

{\footnotesize

1. {Tregubov V. P.} Problems of mechanical model identification for human body under vibration. {\it Mechanism and Machine Theory}, 2000, vol.~35, pp.~491--504.

2. {Coermann R.~R.} The mechanical impedance of the human body in sitting and standing position at low frequencies. {\it Human factors}, 1962, vol.~4, pp.~227--253.

3. {Griffin M.~J.}  Vertical vibration of seated subjects: effects of posture, vibration level, and frequency. {\it Aviation Space and Environmental Medicine}, 1975, vol. 46, pp.~269--276.

4. {Fairley T.~E., Griffin M.~J.} The apparent mass of the seated human body: Vertical vibration. { \it  Journal of Biomechanics}, 1989, vol. 22, N 2, pp.~81--94.

5. {Potemkin B. A., Frolov K. V.} Postroenie dinamitheskoy modeli tela theloveka-operatora, podverzhennogo deistviju shirokopolosnih sluthainih vibratsiy [Construction of dynamic model of human
body for man-operator exposed to broadband random vibra\-tions]. {\it Vibration-isolation of machine and vibration-protection of man-operator}. By otv. red. K. V. Frolov. Moscow, Nauka Publ., 1973, pp.~17--30. (In Russian)

6. {Bai Xian-Xu, Xu Shi-Xu, Cheng Wei, Qian Li-Jun}. On 4-degree-of-freedom bio\-dynamic models of seated occupants: Lumped-parameter modeling. {\it Journal of Sound and Vibration}, 2017, vol.~402, N~18, pp.~122--141.

7.  {Wan~Y., Schimmels~J.\:M.}  A simple model that captures the essential dynamics of a seated human exposed to whole body vibration. {\it Advances in Bioengineering}, 1995, vol. 31, ASME-publ. BED (Bioengineering Division), pp.~333--334.

8.  {Abbas~W., Abouelatta~O.\:B., El-Azab~M., Elsaidy~M., Megahed~A.\:A.}  Optimi\-zation of biodynamic seated human models using genetic algorithms. {\it  Engineering}, 2010, vol. 2, pp.~710--719.

9.  Boileau P.~E., Rakheja S. Whole-body vertical biodynamic response charac\-teristics of the seated vehicle driver: measurement and model development. {\it International Journal of Ind. Ergonomics}, 1998, vol.~22, pp.~449--472.

10. Zhang E., Xu L. A., Liu Z. H., Li X. L. Dynamic modeling and vibration characte\-ristics of multi-DOF upper part system of seated human body. {\it Chine Journal of Enginee\-ring Design}, 2008, vol.~15, pp.~244--249.

11. {Tregubov V.\: P., Egorova N.\: K.}\, O edinstvennosti\,
resheniya \,zadachi opredele\-niya parametrov mekhanicheskih
modelej tela cheloveka, podverzhennogo vibracionnomu
voz\-dejst\-viyu [On the uniqueness of the solution to the problem
of determining the parameters of mechanical models of the human
bo\-dy exposed to vibration]. {\it Vestnik of Saint Petersburg
University. Applied Mathematics. Computer Scien\-ce. Control
Processes}, 2019, vol.~15, iss.~4, pp.~565--577.
https://doi.org/10.21638/11701/spbu10.2019.412 (In Russian)

12. {Bae~J.-J., Kang N.} Development of a five-degree-of-freedom seated human model and parametric studies for its vibrational characteristics. {\it Shock and Vibration}, 2018, Article ID 1649180, 15~p.\\ https://doi.org/10.1155/2018/1649180

13. {Yoshimura ~T., Nakai~ K., Tamaoki~ G.} Multi-body dynamics modelling of seated human body under exposure to whole-body vibration. {\it Industrial Health}, 2005, vol.~43, pp.~441--447.

14. {Panjabi    M.\: M.,  Andersson G.\:B., Jorneus L., Hult E., Mattsson ~L.} In vivo measurements of spinal column vibrations. {\it The Journal of Bone and Joint Surgery}, 1986, vol. 68-A, pp.~695--702.

15. { Kitazaki S., Griffin M.} A modal analysis of whole body vertical vibration using a final element model of the human body. {\it Journal of Sound Vibration}, 1997, vol.~200, pp.~83--105.

16. {Tregubov V.~P., Selezneva N.~A.} Matematicheskoe modelirovanie dinamiki shej\-no\-go otdela pozvonochnika pri impul'snyh vozdejstviyah [Mathematical modeling of the dyna\-mics of the cervical spine under impulse effects]. {\it Vestnik of Saint Petersburg Universi\-ty.
Series 10. Applied Mathematics. Computer Science. Control Processes}, 2016, iss.~1,  pp.~53--66. (In Russian)

17. { Hill A.~V.} {\it First and last experiments in muscle mechanics}. Cambridge, Cambridge University Press, 1970, 180~p. (Russ. ed.: Hill A.~V. {\it Mehanika mishechnogo sokraschenia. Starie i novie opyty}.
Moscow, Mir Publ., 1972, 183~p.)

18. {Huxley A.~F.} Muscle structure and theories of contraction. {\it Progr. Biophys. and Biophys. Chem.}, 1957, vol.~7, pp.~255--318.

19. { Descherevsky V.~I.} {\it Matematicheskie  modeli   myshechnogo sokra\-shche\-niya $[$Mathe\-matical models of muscle contraction$]$}. Moscow, Nauka Publ., 1977, 160~p. (In Russian)

20. {Tregoubov V. P.} Development of the muscle kinetic theory and cyclic move\-ments. {\it Acta of Bioengineering and Biomechanics}, 2003, vol.~5, suppl.~1, pp.~512--519.

21. {Klikunova K. A., Tregubov V. P.} Matematicheskoe modelirovanie dinamiki shejno\-go otdela pozvonochnika pri impul'snyh vozdejstviyah [Mathematical modeling of tran\-sient modes of muscle contraction]. {\it Vestnik of Saint Petersburg University. Series 10. Applied Mathematics. Computer Science. Control Processes}, 2008,  iss.~3, pp.~56--62. (In Russian)


22. {Tregubov V.~ P., Azanchevsky V.~V., Zarin A., Klikunova K.}
Mechanical models of the in\-ter\-ver\-teb\-ral disk and cervical
spine. {\it Lecture Notes of the ICB Seminars}. Warsaw, 2007,
vol.~79, pp.~41--54.

23. {Glukharev K.~K., Potemkin B.~A., Frolov K.~V.} Osobennosti biodinami\-ki te\-la cheloveka pri vibraciyah [Features of human body biodyna\-mics under vibrations]. {\it Vibration-protection of man-operator and the issues of modeling}. By otv. red. K.~V. Frolov. Moscow, Nauka Publ., 1973, pp.~22--28. (In Russian)


24. {\it Sanitarnye\, normy\, 2.2.4/2.1.8.566-96} [{\it Sanitary
\,standards\, 2.2.4/2.1.8.566-96}]. Moscow, Mi\-nist\-ry of Health
of Russia, 1997, 35~p. (In Russian)

\vskip1.5mm Received:  March 07, 2020.

Accepted: May 28, 2020.

\vskip6mm A\,u\,t\,h\,o\,r\,s' \ i\,n\,f\,o\,r\,m\,a\,t\,i\,o\,n:%

\vskip2mm \textit{Vladimir P. Tregubov} --- Dr. Sci. in Physics and Mathematics, Professor; v.tregubov@spbu.ru

\vskip2mm \textit{Nadezhda K. Egorova} --- Postgraduate Student; nadezhda\_ego@mail.ru


}
