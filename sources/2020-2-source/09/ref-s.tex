
{\normalsize

\vskip 6mm

\noindent{\bf On the issue of semivariograms constructing automation \\ for precision agriculture problems$^{*}$
}

}

\vskip 3mm

{\small

\noindent{\it V.~P.~Iakushev$\,^1$%
, V.~M.~Bure$\,^{1,2}$%
, O.~A.~Mitrofanova$\,^{1,2}$%
, E.~P.~Mitrofanov$\,^{1,2}$%
%, I.~�. Famylia%$\,^2$%

 }

\vskip 3mm%2mm

%%%%%%%%%%%%%%%%%%%%%%%%%%%%%%%%%%%%%%%%%%%%%%%%%%%%%%%%%%%%%%%%%%

\efootnote{
%%
\vspace{-3mm}\parindent=7mm
%%
\vskip 0.1mm $^{*}$ This work was supported by the Russian Foundation for Basic Research (grant N 19-29-05184 mk).\par
%%
%%\vskip 2.0mm
%%
%%\indent{\copyright} �����-������������� ���������������
%%�����������, \issueyear%
%%
}

%%%%%%%%%%%%%%%%%%%%%%%%%%%%%%%%%%%%%%%%%%%%%%%%%%%%%%%%%%%%%%%%%%

{\footnotesize

\noindent%
$^1$~%
Agrophysical Research Institute, 14, Grazhdanskiy pr., St. Petersburg,

\noindent%
\hskip2.45mm%
195220, Russian Federation

\noindent%
$^2$~%
St.\,Petersburg State University, 7--9, Universitetskaya nab., St.\,Petersburg,

\noindent%
\hskip2.45mm%
199034, Russian Federation

}

%%%%%%%%%%%%%%%%%%%%%%%%%%%%%%%%%%%%%%%%%%%%%%%%%%%%%%%%%%%%%%%%%%

\vskip3mm


\noindent \textbf{For citation:}  Iakushev V.~P., Bure V.~M., Mitrofanova O.~A., Mitrofanov E.~P. On the issue of semivariograms constructing automation for precision agriculture problems. {\it Vest\-nik of Saint~Petersburg
University. Applied Mathematics. Computer
Science. Control Pro\-cesses}, %\,
\issueyear,
vol.~16,~iss.~\issuenum,~pp.~\pageref{p9}--\pageref{p9e}. %\\
\doivyp/\enskip%
\!\!\!spbu10.\issueyear.\issuenum09  (In Russian)

%\vskip3mm
\newpage

{\leftskip=7mm\noindent Most precision agriculture (PA) problems
are based on assessing the variability of ag\-ro\-eco\-lo\-gi\-cal
parameters (differentiated fertilizer application, allocation of
heterogeneous zones of the agricultural field, etc.). In this
connection, the study of the spatial structure of such data using
geostatistical methods seems to be a relevant and promising
direction. To validate the semivariogram model, determine its
parameters, analyze anisotropy, and carry out other stages of the
experiment, a large number of numerical calculations are required.
Manually all these calculations are extremely difficult to
perform, therefore, automation of these processes is necessary.
Most existing programs exclude certain actions that may be
important in solving PA problems (there is no possibility of
studying anisotropy or spatial trend, the number of theoretical
variogram models, etc., is limited), and the use of programming
languages (R, Python, etc.) requires deep expertise. Therefore,
there was a need to automate the solution of a certain range of
problems by geostatistical methods. For the implementation of the
module under consideration, it seems optimal to use the R
programming language, which has a number of significant
advantages: open source code and free access, a large number of
supported and regularly updated packages, wide graphical
capabilities, cross-platform, etc. General suggestions for
automating the construction of semivariograms are presented and
further use in solving a certain range of PA tasks.\\[1mm]
\textit{Keywords}: variogram analysis, precision agriculture, geostatistics, automation, programming language R.
\par}

\vskip6mm%5mm

\noindent \textbf{References} }

\vskip 3mm%2mm

{\footnotesize

1. {Iakushev~V.~V.}
{\it Tochnoe zemledelie: teoriia i praktika} [{\it Precision agriculture: theory and practice}]. Saint Petersburg, Agrophysical Research Institute Publ., 2016, 364 p. (In Russian)

2. {Chen N., Zhang X., Wang C.}
Integrated open geospatial web service enabled cyber-physical information infrastructure for precision agriculture monitoring. {\it Comput. Electron. Agric.,} 2015, vol. 111, pp. 78--91.

3. {Hohn M. E.}
{\it Geostatistics and petroleum geology}. 2nd ed. Dordrecht, The Netherlands, Springer, Science+Business Media Press, 1999, 235 p.

4. {Gomez J. L., Pastoriza F. T., Alvarez E. G., Oller P. E.}
Comparison between geostatistical interpolation and numerical weather model predictions for meteorological conditions mapping. {\it Infrastructures,} 2020, vol. 5, N 15. DOI:10.3390/infrastructures5020015

5. {Paz-Ferreiro J., Vazquez E. V., Vieira S. R.}
Geostatistical analysis of a geochemical dataset. {\it Bragantia,} 2009, vol. 69, pp. 121--129.

6. {Olthoff A. E., Gomez C., Alday J. G., Martinez-Ruiz C.}
Mapping forest vegetation patterns in an Atlantic---Mediterranean transitional area by integration of ordination and geostatistical techniques. {\it Journal of Plant Ecology,} 2018, vol. 11, iss. 1, pp. 114--122.

7. {Iakushev V. P., Zhukovskii E. E., Petrushin A. F., Iakushev V. V.}
{\it Variogrammnyi analiz prostranstvennoi neodnorodnosti sel'skokhoziaistvennykh polei dlia tselei tochnogo zemledeliia.} Metodicheskoe posobie [{\it A variogram analysis of the spatial heterogeneity of agricultural fields for precision agriculture.} Toolkit]. Saint Petersburg, Agrophysical Research Institute Publ., 2010, 52 p. (In Russian)

8. {Cambardella C. A., Moorman T. B., Novak J. M., Parkin T. B., Karlen D. L., Turko R. F. et al.}
 Field-scale variability of soil properties in central Iowa soils. {\it Soil Science Society of America Journal,} 1994, vol. 58, pp. 1501--1511.

9. {Jiang Q. X., Fu Q., Wang Z. L.}
Research on precision irrigation in Western Semiarid Area of Heilonngjiang province in China based on GIS. {\it Computer and Computing Technologies in Agriculture,} 2008, vol. 1, pp. 359--370.

10. {Moustafa M. M., Yomota A.} Use of a covariance variogram to
investigate influence of subsurface drainage on spatial
variability of soil-water properties. {\it Agricultural Water
Management,} 1998, vol. 37, pp.~1--19.

11. {Iakushev V. P., Bure V. M., Mitrofanova O. A., Mitrofanov E. P.}
Primenenie metodov geostatistiki dlia analiza tselesoobraznosti perekhoda k tekhnologiiam differentsirovannogo vneseniia agrokhimikatov [The use of geostatistical methods to analyze the transition feasibility to the differential application of agrochemicals technologies]. {\it Vestnik of Saint Peterburg University. Applied Mathematics. Computer Science. Control Processes,} 2020, vol. 16, iss. 1, pp. 31--40. (In Russian)

12.  {Dem'ianov~V.~V., Savel'eva~E.~A.} {\it Geostatistika:
teoriia i praktika} [{\it Geostatistics: theory
and\linebreak\newpage\noindent practice}]. Moscow, Nuclear safety
institute of the Russian Academy of Sciences, Nauka Publ., 2010,
327~p. (In Russian)

13. {Li Z., Zhang X., Clarke K. C., Liu G., Zhu R.}
An automatic variogram modeling method with high reliability fitness and estimates. {\it Computers and Geosciences,} 2018, vol. 120, pp. 48--59.

\vskip1.5mm Received:  March 03, 2020.

Accepted: May 28, 2020.

\vskip6mm A\,u\,t\,h\,o\,r\,s' \ i\,n\,f\,o\,r\,m\,a\,t\,i\,o\,n:%

\vskip2mm \textit{Viktor P. Iakushev}  --- RAS Academician, Dr. Sci. in Agriculture; vyakushev@agrophys.com

\vskip2mm \textit{Vladimir M. Bure}  --- Dr. Sci. in Technics, Professor; vlb310154@gmail.com


\vskip2mm \textit{Olga A. Mitrofanova}  --- Junior Researcher; omitrofa@gmail.com

\vskip2mm \textit{Evgenii P. Mitrofanov}  --- Junior Researcher; mjeka@bk.ru

}
