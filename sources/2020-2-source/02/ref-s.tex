
{\normalsize

\vskip 6mm%6mm

\noindent{\bf Multi-start method with deterministic restart mechanism$^{*}$%
}

}

\vskip 2mm%2mm

{\small

\noindent{\it G. A. Amirkhanova$\,^1$%
, A. Y. Gorchakov$\,^{2,3,4}$%
, A. J. Duysenbaeva$\,^{1}$%
, M. A. Posypkin$\,^{2,3}$%
%, I.~�. Famylia%$\,^2$%

}

\vskip 2mm%2mm

%%%%%%%%%%%%%%%%%%%%%%%%%%%%%%%%%%%%%%%%%%%%%%%%%%%%%%%%%%%%%%%%%%

\efootnote{
%%
\vspace{-3mm}\parindent=7mm
%%
\vskip 0.1mm $^{*}$ This work was financially supported by the Committee of Science of the Ministry of Education and Science of the Republic of Kazakhstan (grant no.~AP05133366).%\par
%%
%%\vskip 2.0mm
%%
%%\indent{\copyright} �����-������������� ���������������
%%�����������, \issueyear%
%%
}

%%%%%%%%%%%%%%%%%%%%%%%%%%%%%%%%%%%%%%%%%%%%%%%%%%%%%%%%%%%%%%%%%%

{\footnotesize

\noindent%
$^1$~%
Institute for Information and Computing Technologies, 125, Pushkina ul. (ug. Kurmangazy), Almaty,

\noindent%
\hskip2.45mm%
050000, Republic of Kazakhstan

\noindent%
$^2$~%
Federal Research Center  ``Computer Science and Control'' Russian
Academy of Sciences, 44,

\noindent%
\hskip2.45mm%
Vavilova ul., Moscow, 119333, Russian Federation


\noindent%
$^3$~%
Moscow Institute of Physics and Technology (State University), 9,
Institutsky per., Dolgoprudny,

\noindent%
\hskip2.45mm%
141701, Russian Federation

\noindent%
$^4$~%
National Research University Higher School of Economics, 20, Myasnitskaya ul., Moscow,

\noindent%
\hskip2.45mm%
101000, Russian Federation


%\noindent%
%%$^1$~%
%St.\,Petersburg State University, 7--9, Universitetskaya nab.,
%St.\,Petersburg,
%
%\noindent%
%%\hskip2.45mm%
%199034, Russian Federation

}

%%%%%%%%%%%%%%%%%%%%%%%%%%%%%%%%%%%%%%%%%%%%%%%%%%%%%%%%%%%%%%%%%%

\vskip3mm%3mm


\noindent \textbf{For citation:}  Amirkhanova G.\,A., Gorchakov
A.\,Y., Duysenbaeva A.\,J., Posypkin M.\,A. Multi- \-start method
with deterministic restart mechanism. {\it Vestnik of
Saint~Petersburg Uni\-ver\-si\-ty. Ap\-plied Mat\-he\-matics.
Computer Science. Control Processes},\,\issueyear,
vol.~16,~iss.~\issuenum,~pp.~\pageref{p2}--\pageref{p2e}. \\
\doivyp/\enskip%
\!\!\!spbu10.\issueyear.\issuenum02  (In Russian)

\vskip3mm

{\leftskip=7mm\noindent The work is devoted to the development and
study of a method for solving global optimization problems with
interval constraints. The paper proposes a global optimization
algorithm based on a deterministic method of selecting starting
points for local search methods. The starting points are the
extremum points of functions of one variable, obtained by
restricting the objective function to straight, collinear
coordinate vectors. The effectiveness of the proposed algorithm is
demonstrated by the example of the problem of minimizing the
energy of a fragment of a flat crystal lattice. The energy of
interatomic interaction is calculated using the Tersoff potential.
An experimental comparison is made of the developed algorithm with
the classical version of the multi-start method, in which
pseudo-random points uniformly distributed in the parallelepiped
are used to select starting points. As a local search method, in
both cases, one of the modifications of the coordinate wise
descent method is used. The developed method can be applied to
problems with an unknown analytical expression for an objective
function that is often encountered in practice.\\[1mm]
\textit{Keywords}: global optimization, multi-start method, fill sequences.
\par}

\vskip6mm

\noindent \textbf{References} }

\vskip 2mm

{\footnotesize

1. Evtushenko Y. G., Posypkin M. A. Varianty metoda neravnomernyh
pokrytij dlja global�noj  optimizacii  chastichno-celochislennyh
nelinejnyh  zadach [Variants of the method of non-uniform
coverages for global optimization of partially integer nonlinear
problems]. \textit{Doc. Academy of Sciences}, 2011, vol. 437, no.~2, pp.~168--172. (In Russian)

2. Karnopp D. C. Random search techniques for optimization
problems. \textit{Automatica}, 1963, vol.~1,  no.~2--3,
pp.~111--121.

3. Solis F. J., Wets R. J.-B. Minimization by random search
techniques. \textit{Mathematics of operations research}, 1981,
vol.~6, no.~1, pp.~19--30.

4. Pag\`{e}s G. The Quasi-Monte Carlo Method. \textit{Numerical
Probability}. Cham, Springer  Publ., 2018, pp.~95--132.

5. Owen A. B., Glynn P. W. \textit{Monte Carlo and Quasi-Monte
Carlo methods}. California, Springer  Publ., 2016, pp.~1--478.

6. Bousquet O., Gelly S., Kurach K. et al. \textit{Critical
hyper-parameters: No random, no cry}. arXiv Preprint arXiv:
1706.03200, 2017. Available at:
\url{https://arxiv.org/abs/1706.03200} (accessed: 15.05.2019).

7. Halton J. H. Algorithm 247: Radical-inverse quasi-random point
sequence. \textit{Communications of the  ACM}, 1964, vol.~7,
no.~12, pp.~701--702.

8. Muniraju G., Kailkhura B., Thiagarajan J., Bremer P. T.
\textit{Controlled random search improves sample mining and
hyper-parameter optimization}. arXiv Preprint arXiv: 1809.01712v1,
2018. Available at: \url{https://arxiv.org/abs/1706.03200}
(accessed: 15.05.2019).

9. Gerstner T., Griebel M. Sparse grids. \textit{Encyclopedia of
Quantitative Finance}, 2010, pp. 1--6. Avaliable at:
\url{https://doi.org/10.1002/9780470061602.eqf12011} (accessed:
21.05.2019).

10. Sobol I. M., Asotsky D., Kreinin A., Kucherenko K.
Construction and comparison of high-dimensional Sobol' generators.
\textit{Wilmott}, 2011, vol. 2011, no.~56, pp.~64--79.

11. Sobol I. M. \textit{Tochki, ravnomerno zapolnjajushhie
mnogomernyj kub} [\textit{Points uniformly filling a
multidimensional cube}]. Moscow, Znanie  Publ., 1985, 125~p. (In
Russian)

12. Hickernell F. J., Yuan Y. A simple multistart algorithm for
global optimization. \textit{CiteSeerX Preprint}, 1997. Available
at:
\url{http://citeseerx.ist.psu.edu/viewdoc/summary?doi=10.1.1.46.1346}
(accessed: 10.05.2019).

13.  Mart\'{i} R., Lozano J.A., Mendiburu A., Hernando L.
Multi-start methods. \textit{Handbook of Heuristics}.  Cham,
Springer Publ., 2016, pp.~1--21.

14. Mart\'{i} R., Moreno-Vega J. M., Duarte A. Advanced
multi-start methods. \textit{Handbook of Me\-ta\-heuristics}.
Boston, Springer Publ., 2010, pp.~265--281.

15. Mart\'{i} R., Aceves R., Leonet M. T.  et al. Intelligent
multi-start methods. \textit{Handbook of Me\-ta\-heuristics}.
Cham, Springer Publ., 2019, pp.~221--243.

16. Gorchakov A. Y., Posypkin M. A. Jeffektivnost�
metodovlokal�nogo poiska v zadache minimizacii jenergii ploskogo
kristalla [The effectiveness of local search methods in the
problem of finding the minimum energy of a 2-D crystal].
\textit{Modern Information Technology and IT-education}, 2017,
vol. 13, no.~2, pp.~97--102. (In Russian)

17. McKay M. D., Beckman R. J., Conover W. J. Comparison of three
methods for selecting values of input variables in the analysis of
output from a computer code. \textit{Technometrics}, 1979, vol.
21, no.~2, pp.~239--245.

18. Hofer R., Ktitzer P., Larcher G., Pillichshammer F.
Distribution properties of generalized van der Corput---Halton
sequences and their subsequences. \textit{International Journal of
Number Theory}, 2009, vol.~5, N~4, pp.~719--746.

19. Lambert J. P. Quasi-Monte Carlo, low discrepancy sequences,
and ergodic transformations. \textit{ Journal of Computational and
Applied Mathematics}, 1985, vol.~12, pp.~419--423.

20. Faure H. Discr\'{e}pance de suites associ\'{e}es \`{a} un
syst\`{e}me de num\'{e}ration (en dimensions). \textit{Acta
arithmetica}, 1982, vol. 41, no.~4, pp. 337--351.

21. Niederreiter H. Random number generation and quasi-Monte Carlo
methods. Philadelphia, Siam Publ., 1992, vol.~63, 242~p.

22. Lurie S. A., Posypkin M. A., Solyaev Yu. O. Metod
identifikacii masshtabnyh para\-met\-rov gradientnoj  teorii  uprugosti
na  osnove  chislennyh  jeksperimentov  dlja  ploskihkompozitnyh
struk\-tur [Method of identification of scale parameters of gradient
theory of elasticity on the basis of nu\-me\-ri\-cal experiments in flat
composite structures]. \textit{International Journal of Open
Information Tech\-no\-lo\-gies}, 2015, vol.~3, no.~6, pp.~1--5. (In
Russian)

23. Evtushenko Y. G., Lurie S. A., Posypkin M. A.  Primenenie
metodov optimizacii dlja poiska  ravnovesnyh  sostojanij
dvumernyhkristallov [Application of optimization methods for
finding equilibrium states of two-dimensional crystals].
\textit{Computational Mathematics and Mathematical Physics}, 2016,
vol.~56, no.~12, pp.~2032--2041. (In Russian)

24. Amirhanova G. A., Gorchakov A. Y., Dujsenbaeva A. J., Posypkin
M. A. Primeneniemetoda tochnyh shtrafnyh funkcij k zadache
minimizacii jenergii ploskogo kristalla [Application of the exact
penalty function method to the problem of minimizing the energy of
a plane crystal]. \textit{XIV Intern. Asian School-Seminar
``Problems of Optimizing Complex Systems''}, 20--31 Jule 2018,
Kyrgyz Republic, Lake Issyk-Kul / Inst. inform. and calcul.
technologies of the Ministry of Education and Science of the
Republic of Kazakhstan. Almaty, IICT Publ., 2018, pp.~107--113.
(In Russian)

25. Amirhanova G. A., Gorchakov A. Y., Dujsenbaeva A. J.
Gibridizacija metodov Monte-Karlo,  imitacii  otzhiga  i
lokal�nogo  poiska [Hybridization of Monte Carlo methods,
simulated annealing and local search]. \textit{III Intern.
Scientific Conference ``Informatics and Applied Mathematics''},
26--29 September 2018. Almaty, Kazakhstan, 2018, pp.~266--274. (In
Russian)

26. Amirkhanova G. A., Gorchakov A. Y., Duysenbaeva A. J. The
application of the methodology of fast automatic differentiation
to calculate the gradient of the potential REBO (LAMMPS).
\textit{DEStech Transactions on Computer Science and Engineering},
2018, N. optim, pp.~103--113.

27. Plimpton S. Fast parallel algorithms for short-range molecular
dynamics. \textit{Journal of Computational Physics}, 1995,
vol.~117, no.~1, pp.~1--19.

28. Plimpton S., Crozier P., Thompson A. Lammps-large-scale
atomic/molecular massively parallel simulator. \textit{Sandia
National Laboratories}, 2007, vol.~18, p.~43.

29. Wright S. J. Coordinate descent algorithms.
\textit{Mathematical Programming}, 2015, vol.~151, no.~1,
pp.~3--34.

30. Khamisov O., Posypkin M. Univariate global optimization with
point-dependent Lipschitz constants. \textit{AIP Conference
Proceedings / AIP Publishing}, 2019, vol.~2070, pp.~1--4.

31. Khamisov O., Posypkin M., Usov A. Piecewise linear bounding
functions for univariate global optimization.
\textit{International Conference on Optimization and
Applications}. Cham, Springer Publ., 2018, pp.~170--185.

32. Piyavskii S. A. An algorithm for finding the absolute extremum
of a function. \textit{USSR Computational Mathematics and
Mathematical Physics}, 1972, vol.~12, no.~4, pp.~57--67.

33. Hansen P., Jaumard B., Lu Shi-Hui. Global optimization of
univariate lipschitz functions: I. Survey and properties.
\textit{Mathematical Programming}, 1992, vol.~55, no.~1--3,
pp.~251--272.

34. Tersoff J. Modeling solid-state chemistry: Interatomic
potentials for multicomponent systems. \textit{ Physical Review,
B}, 1989, vol. 39, no.~8, pp. 55--66.

35. Evtushenko Y. G., Posypkin M. A., Sigal I. A framework for
parallel large-scale global optimization. \textit{Computer
Science-Research and Development}, 2009, vol. 23, no.~3--4,
pp.~211--215.

36. Bychkov I. V., Manzyuk M. O., Semenov A. A. et al. Technology
for integrating idle computing cluster resources into volunteer
computing projects. \textit{2015 5th International Workshop on
Computer Science and Engineering: Information Processing and
Control Engineering}, WCSE 2015--IPCE, 2015, pp.~109--114.

37. Posypkin M., Usov A. Implementation and verification of global
optimization benchmark problems. \textit{Open Engineering}, 2017,
vol.~7, no.~1, pp.~470--478.



\vskip1.5mm Received:  May 31, 2019.

Accepted: May 28, 2020.


\vskip6mm A\,u\,t\,h\,o\,r\,s' \ i\,n\,f\,o\,r\,m\,a\,t\,i\,o\,n:%

\vskip2mm \textit{Gulshat A. Amirkhanova}~--- PhD in  Physics and
Mathematics, Senior Researcher;\\ gulshat.aa@gmail.com

\vskip2mm \textit{Andrei Y.  Gorchakov}~--- PhD in  Physics and Mathematics, Senior Researcher; andrgor12@gmail.com

\vskip2mm \textit{Aigerim Z. Duysenbaeva}~--- aigerim.95.05@mail.ru

\vskip2mm \textit{Michael A.  Posypkin}~--- Dr. Sci. in Physics and Mathematics, Chief Researcher; mposypkin@gmail.com

}
