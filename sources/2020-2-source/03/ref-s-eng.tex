

{\small



\vskip6mm

\noindent \textbf{References} }

\vskip 2mm

{\footnotesize


1. Gaishun I.~V. \textit{Vpolne razreshimiye mnogomerniye
differentsialniye uravneniya} [\textit{Completely solvable
multidimensional differential equations}]. Minsk, Belarus, Nauka i
Technika Publ., 1983, 272~p. (In Russian)


2. Babadzanjanz L. K. Metod dopolnitelnih peremennih [The
additional variables method].  \textit{Vestnik of Saint Petersburg
University. Series 10. Applied Mathematics. Computer Science.
Control Processes}, 2010, iss. 4, pp. 3--11. (In Russian)


3. Babadzanjanz L.~K., Bregman K.~M. Algorithm metoda
dopolnitelnih peremennih [Algorithm of the additional variables
method]. \textit{Vestnik of Saint Petersburg University. Series
10. Applied Mathematics. Computer Science. Control Processes},
2012, iss. 2, pp. 3--12. (In Russian)


4. Abad A., Barrio R., Blesa F., Rodriguez M. Breaking the limits:
the Taylor series method. \textit{Appl. Math. and Computation},
2011, vol. 217, iss. 20, pp. 7940--7954.


5. Alesova I.~M., Babadzanjanz L.~K., Pototskaya I.~Yu., Pupysheva
Yu.~Yu., Saakyan A.~T. High-precision numerical integration of
equations in dynamics. \textit{International scientific conference
on mechanics. The Eighth Polyakhov's Reading, Saint Petersburg,}
\textit{AIP Conference Proceedings}. Saint Petersburg, 2018, vol.
1959, pp.~1--4, 080005. https://doi.org/10.1063/1.5034722


6. Babadzanjanz L.~K. Metod ryadov Taylora [The Taylor series
method]. \textit{Vestnik of Saint Petersburg University. Series
10. Applied Mathematics. Computer Science. Control Processes},
2010, iss. 3, pp. 13--29. (In~Russian)


7. Berz M. Cosy infinity version 8 reference manual.
\textit{Technical Report MSUCL-1088}. Michigan, Michigan National
Superconducting Cyclotron Lab., Michigan State University Press,
2003, 695 p.


8. Berz M., Bischof C., Corliss G. F., Griewank A.
\textit{Computational differentiation: techniques, applications,
and tools}. Philadelphia, Society for Industrial and Applied
Mathematics Publ., 1997, 424~p.


9. Babadzanjanz L.~K., Pototskaya I.~Yu., Pupysheva Yu.~Yu. Error
estimates for numerical integration of ODEs in the minimax
formulation. \textit{2017 Constructive Nonsmooth Analysis and
Related Topics (Dedicated to the memory of V. F. Demyanov), CNSA
2017. Proceedings}. Saint Petersburg, Russia, 2017, pp.~1--4.
https://doi.org/10.1109/CNSA.2017.7973932


10. Babadzanjanz L.~K., Bol'shakov A.~I. Primenenie metoda ryadov
Taylora dlya resheniya obiknovennih differentsialnih uravneniy
[Implementation of the Taylor series method for solving ordinary
differential equations]. \textit{Vychisl. Metody
Programm. $[$Computational programming methods$]$}, 2012, vol. 13,
iss. 4, pp. 497--510. (In~Russian)


11. Babadzanjanz L. K., Pupychev  Yu.~A., Pupycheva Yu.~Yu.
\textit{Classicheskaya mechanica} [\textit{Classical Mechanics}].
Saint Petersburg, Saint Petersburg State University Press, 2007,
240 p. Available at:
http://www.apmath.spbu.ru/ru/staff/babadzhanyants/publ/publ36.pdf) (accessed: 14.07.2019). (In Russian)


12. Bellman R. \textit{Introduction to matrix analysis}. 2nd ed.
Philadelphia, Society for Industrial and Applied Mathematics
Publ., 1997, 403 p.


13. Guntmacher F.~R. \textit{The theory of matrices}. In 2 vol.
New York, Chelsea, 1960,  374 p.


14. Wilkinson J.~N. \textit{The algebraic eigenvalue problem.}
Oxford, Clarendon Press,  1965, 564 p.


15. Ostrowski A.~M. \textit{Solution of equations and systems of
equations.}  2nd ed. Pure and Applied Mathematics: A Series of
Monographs and Textbooks. Elsevier, Acad. Press, vol. 9, 1966, 352
p.


16. Bregman K.~M. \textit{Matematicheskie modeli vozmuschennogo
dvighenia v tsentral�nyh poliah} [\textit{Mathematical models to
perturbed motion in central fields}]. PhD thesis, Saint
Petersburg, Saint Petersburg State University Press, 2014, 145 p.
(In Russian)


17. Babadzanjanz L.~K., Bregman A.~M., Bregman K.~M., Kasikova
P.~V., Petrosyan L.~A.  Polniye systemy uravneniy v zadache dvuh
tel [Total systems of equations to the two-body problem].
\textit{Engineering\linebreak\newpage\noindent sciences --- from
theory to applications. Coll. papers on LXI Intern. conference},
no.~8(56). Novosibirsk, SIBAK Publ., 2016, pp. 13--21. (In
Russian)

%
%18. Broucke R. Solution of the N-body problem with recurrent power
%series. \textit{Celestial Mechanics}, 1971, vol. 4, iss. 28, pp.
%110--115.
%

\vskip 1.5mm

%\noindent Recommendation: prof. L. A. Petrosyan.
%
%\vskip 1.5mm

%\noindent

Received:  July 27, 2019.

Accepted: May 28, 2020.

\vskip 4.5mm%6mm
A\,u\,t\,h\,o\,r\,s' \ i\,n\,f\,o\,r\,m\,a\,t\,i\,o\,n:

\vskip 2mm\textit{Levon K. Babadzanjanz} --- Dr. Sci. in Physics and Mathematics, Professor; levon-lkb@yandex.ru

\vskip 2mm\textit{Irina Yu. Pototskaya} --- PhD in Physics and
Mathematics, Associate Professor;\\ irinapototskaya@yandex.ru

\vskip 2mm\textit{Yulia Yu. Pupysheva} --- PhD in Physics and
Mathematics, Associate Professor;\\ j\_poupycheva@mail.ru

}
