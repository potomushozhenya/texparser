
{\normalsize

\vskip 6mm

\noindent{\bf Mathematical modeling of cancer treatment%$^{*}$%
}

}

\vskip 2mm

{\small

\noindent{\it A. B. Goncharova$^1$, E. P. Kolpak$^1$,
M. M. Rasulova$^1$, A. V. Abramova$^2$%$\,^1$%
%, A. Y. Gorchakov$\,^{2,3,4}$%
%, A. J. Duysenbaeva$\,^{1}$%
%, M. A. Posypkin$\,^{2,3}$%
%, I.~�. Famylia%$\,^2$%

}

\vskip 2mm

%%%%%%%%%%%%%%%%%%%%%%%%%%%%%%%%%%%%%%%%%%%%%%%%%%%%%%%%%%%%%%%%%%

%\efootnote{
%%
%\vspace{-3mm}\parindent=7mm
%%
%\vskip 0.1mm $^{*}$ This work was supported by the Russian Foundation for Basic Research (grant N~20-07-01086).%\par
%%
%%\vskip 2.0mm
%%
%%\indent{\copyright} �����-������������� ���������������
%%�����������, \issueyear%
%%
%}

%%%%%%%%%%%%%%%%%%%%%%%%%%%%%%%%%%%%%%%%%%%%%%%%%%%%%%%%%%%%%%%%%%

{\footnotesize

\noindent%
$^1$~%
St.\,Petersburg State University, 7--9, Universitetskaya nab.,
St.\,Petersburg,

\noindent%
\hskip2.45mm%
199034, Russian Federation

\noindent%
$^2$~%
N. N. Petrov National Medical Research Center of Oncology of the
Ministry of Healthcare

\noindent%
\hskip2.45mm%
of the Russian Federation, 68, ul. Leningradskaya, pos. Pesochny,
St.\,Petersburg,

\noindent%
\hskip2.45mm%
197758, Russian Federation

}

%%%%%%%%%%%%%%%%%%%%%%%%%%%%%%%%%%%%%%%%%%%%%%%%%%%%%%%%%%%%%%%%%%

\vskip3mm%3mm


\noindent \textbf{For citation:}  Goncharova A. B., Kolpak E. P.,
Rasulova M. M., Abramova A. V. Mathematical modeling of cancer
treatment. {\it Vestnik of Saint~Petersburg Uni\-ver\-si\-ty.
Ap\-plied Mat\-he\-matics. Computer Science. Control
Processes},\,\issueyear,
vol.~16,~iss.~\issuenum,~pp.~\pageref{p8}--\pageref{p8e}. \\
\doivyp/\enskip%
\!\!\!spbu10.\issueyear.\issuenum08  (In Russian)

\vskip3mm

{\leftskip=7mm\noindent The paper proposes mathematical models of
ovarian neoplasms. The models are based on a mathematical model of
interference competition. Two types of cells are involved in the
competition for functional space: normal and tumor cells. The
mathematical interpretation of the models is the Cauchy problem
for a system of ordinary differential equations. The dynamics of
tumor growth is determined on the basis of the model. A model for
the distribution of conditional patients according to four stages
of the disease, a model for assessing survival times for groups of
conditional patients, and a chemotherapy model are also proposed.\\[1mm]
\textit{Keywords}: differential equations, mathematical modeling,
neoplasms, ovarian cancer, morbidity, treatment, statistics.
\par}

\vskip6mm

\noindent \textbf{References} }

\vskip 2mm

{\footnotesize

1. Ryazhenov~V., Gorokhova~S.\:G., Zhordania~K.\:I.,
Payanidi~Y.\:G., Bunyatyan~N.\:D. Clinical and statistical
analysis of ovarian cancer incidence rate and identification of a
subgroup of patients with BRCA1/2 mutations in the Russian
Federation. \textit{Journal of Pharmaceutical Sciences and
Research}, 2018, no.~10,  pp.~2500--2502.

2. Chaplain~M.\:A., Sherratt~J.\:A. A new mathematical model for
avascular tumor growth. \textit{Journal of Mathematical Biology},
2000, vol.~43, no.~4, pp.~291--312.

3. Kolpak~E.\:P., Abuzyarova~R.\:T., Kabrits~S.\:A. Leukosis
mathematical model. \textit{Asian Journal of Pharmaceutics}, 2018,
no.~1, pp.~S340--S345.

4.  Kuznetsov~M.\:B., Kolobov~A.\:V. Influence of chemotherapy on
progression of biclonal tumor --- Analysis by means of
mathematical modeling. \textit{Biophysics (Russian Federation)},
2019, vol.~64(2),  pp.~279--292.
https://doi.org/10.1134/S0006350919020118

5.      Kolpak~E.\:P., Frantsuzova~I.\:S., Kuvshinova~K.\:V.,
Senkov~R.\:E. Neoplasm morbidity among the population of Russia.
\textit{ International Journal of Advanced Biotechnology and
Research}, 2017, vol.~8(3), pp.~2315--2322.

6.      Merabishvili~V.\:M. \textit{Onkologicheskaia statistika
(traditsionnye metody, novye informatsionnye tehnologii)}.
Rukovodstvo dlia vrachey  [\textit{Oncological statistics
(traditional methods, new information technologies)}]. Guidelines
for physicians. Saint\,Pe\-ters\-burg, Printing company KOSTA
Publ., 2015, 223~p. (In Russian)

7.  Kroeger~P.\:T., Drapkin~R. Pathogenesis and heterogeneity of
ovarian cancer. \textit{Curr Opin Obstet Gynecol},  2017,
vol.~29(1), pp.~26--34.
https://doi.org/10.1097/GCO.0000000000000340

8. Karst~A.\:M., Drapkin ~R. Ovarian cancer pathogenesis: a model
in evolution.  \textit{Molecular genetic markers in female
reproductive cancers}, 2010, no.~932371, p.~13.
https://doi.org/10.1155/2010/932371

9.  Urmancheeva~A.\:F. Chemotherapy for recurrent ovarian cancer
(literature review). \textit{Scientific and practical reviewed
journal. Siberian Journal of Oncology}, 2010, vol.~3(39),
pp.~28--33.

10. Beishembaev~A.\:M., Zhordania~K.\:I. Clinical and histological
features of purely stromal cell ovarian tumors.
\textit{Obstetrics, Gynecology and Reproduction}, 2019,  vol.~13,
iss.~4, pp.~289--296.\\
https://doi.org/10.17749/2313-7347.2019.13.4.289-296

11. Bamberger~E.\:S., Perrett~C.\:W. Angiogenesis in epithelian
ovarian cancer. \textit{Mol. Pathol}, 2002, vol.~55(6),
pp.~348--359. https://doi.org/10.1136/mp.55.6.348

12. Devouassoux-Shisheboran~M., Genestie~C. Pathobiology of
ovarian carcinomas. \textit{Chinese Journal of Cancer}, 2015,
vol.~34(1), pp.~50--55. https://doi.org/10.5732/cjc.014.10273

13. Limei~Wang, Xiaoyan~Liu, Hong~Wang, Shuhe~Wang. Correlation of
the expression of vascular endothelial growth factor and its
receptors with microvessel density in ovarian cancer.
\textit{Oncology Letters}, 2013, vol.~15, pp.~175--180.
https://doi.org/10.3892/ol.2013.1349

14. Gershtein~E.\:S., Kushlinsky~D.\:N., Levkina~N.\:V.,
Tereshkina~I.\:V., Nosov~V.\:B., Laktionov~K.\:P., Adamyan~L.\:V.
Relationship between the expression of VEGF signal components and
matrix me\-tal\-lo\-pro\-tei\-na\-ses in ovarian tumors.
\textit{Bulletin of Experimental Biology and Medicine}, 2011,
vol.~151, iss.~4, pp.~449--453.
https://doi.org/10.1007/s10517-011-1353-5.22448363

15. Bogush~T.\:A., Stenina~M.\:B., Bogush~E.\:A., Zarkua~V.\:T.,
Kalyuzhny~S.\:A., Mamichev~I.\:A., Tyulyandina~A.\:S.,
Tyulyandin~S.\:A., Polotsky~B.\:E., Davydov~M.\:M. The
quantitative indices of ERCC1 expression in serous ovarian cancer
tissue and the efficacy of first-line platinum-based chemotherapy.
\textit{ Antibiotics and Chemotherapy}, 2018, vol.~63,  iss.~1--2,
pp.~24--31.

16. Khokhlova~S.\:V., Cherkasova~M.\:V., Orel~N.\:F.,
Limareva~S.\:V., Bazaeva~I.\:A., Gorbunova~V.\:A. Kakim bol'nym
rakom yaichnika pokazana kombinatsia trabektedina s pegilirovannym
liposomal'nym doksorubitsinom [Which patients with ovarian cancer
shows the combination of trabectedin with pegylated liposomal
doxorubicin].  \textit{Annals of the Russian Academy of Medical
Sciences}, 2013, vol.~68, no.~11, pp.~115--121. https://doi.org/
10.15690/vramn.v68i11.852 (In Russian)

17. Gatti~L., Beretta~G.\:L. Ovarian cancer: pathogenesis,
diagnosis, and treatment. \textit{International Journal of
Molecular Sciences. Version 1},  2017, vol.~6, no.~84,
pp.~1--12.\\
https://doi.org/10.12688/f1000research.9977.1

18.  Das~P.\:M., Bast~R.\:C. Early detection of ovarian cancer.
\textit{Biomark Med.}, 2008, vol.~2(3), pp.~291--303.
https://doi.org/10.2217/17520363.2.3.291

19. Chu~E., DeVita~V.\:T. \textit{Physicians cancer chemotherapy
drug manual}. New York, Jones and Bartlett Publ., 2007, 455~p.

20. Belyaev~A.\:�., Gafton~G.\:I., Levchenko~E.\:V.,
Karachun~A.\:�., Gulyaev~A.\:V., Senchik~K.\:Yu., Bespalov~
V.\:G., Berlev~I.\:V.,  Urmancheeva~ A.\:F., Guseinov~K.\:D.,
Semiletova~Yu.\: V., Mamontov~O.\:N., Kalinin~P.\:V.,
Kireeva~G.\:S., Belyaeva~O.\:A., Alexeev~V.\:V. ChemoPerfuSion
technologies in treatment for malignant tumors. \textit{Problems
in Oncology}, 2015, vol.~61, no.~3, pp.~477--485.

21. \textit{Cortez~A.\:J., Tudrej~P., Kujawa~K.\;A.,
Lisowska~K.\:M.} Advances in ovarian cancer therapy. Cancer
Chemother Pharmacol, 2018, vol.~81(1), pp.~17�38.
https://doi.org/10.1007/s00280-017-3501-8

22. \textit{The State of Cancer Care Provided to the Population of
Russia in 2018}. Moscow, FSBI P.~A.~Her\-zen Moscow Cancer
Research Institute, Branch Office of FSBI NMRRC, Ministry of
Health of Russia,  2018, p.~236. Available at:
http://www.oncology.
ru/service/statistics  (accessed: May 22, 2020).\\
https://nnood.ru/wp-content/uploads/2019/04/Statichticheskijj-ezhegodnik-Gercena-2018.pdf


23. Merabishvili~V.\:M., Bogdanova~E.\:M., Urmancheeva~A.\:F.,
Chepik~O.\:F., Safronnikova~N.\:R., Laliantsi~E.\:I. Age-related
features of morbidity, mortality and morphological verification of
ovarian carcinoma. \textit{Problems in Onkology}, 2010,
vol.~56(2), pp.~144--51.

24. Aksel~E.\:M., Vinogradova~N.\:N. Statistics of malignant
neoplasms of female reproductive organs. \textit{Gynecologic
Oncology}, 2018, no.~3(27), pp.~64--78.

25. Kolpak~E.\:P., Frantsuzova~I.\:S., Evmenova~E.\:O. Oncological
diseases in St. Petersburg, Russia. \textit{Drug Invention Today},
2019, vol.~11(3), pp.~510--516.

26. Siegel~R.\:L., Miller~K.\:D., Jemal~A. Cancer statistics 2020.
\textit{CA: A Cancer Journal Clinicians}, 2020,  vol.~70, no.~1,
pp.~1--30. https://doi.org/10.3322/caac.21590


\vskip1.5mm Received:  October 02, 2020.

Accepted: October 23, 2020.


\vskip6mm A~u~t~h~o~r~s' \ i~n~f~o~r~m~a~t~i~o~n:%

\vskip2mm \textit{Anastaciya B. Goncharova} --- PhD in Physics and
Mathematics, Senior Teacher; a.goncharova@spbu.ru

\vskip2mm \textit{Eugeny P. Kolpak} --- Dr. Sci. in Physics and
Mathematics, Professor; e.kolpak@spbu.ru

\vskip2mm \textit{Madina M. Rasulova} --- Master Student;
st054684@student.spbu.ru

\vskip2mm \textit{Alina V. Abramova} --- Oncologist;
alinochkamv1991@gmail.com

}
