

{\footnotesize

\vskip 2mm
%\newpage

\noindent {\small\textbf{References} }

\vskip 1.5mm



1.  Gubanov D. A.,  Novikov D. A.,  Chhartishvili A. G. {\it
Sotsial'nye seti: modeli informatsionnogo vliianiia, upravleniia i
protivoborstva}  [{\it Social networks: models of information
influence, control and confrontation}].   Moscow, Fizmatlit Publ.,
2010, 228~p. (In Russian)

2. Almeida Costa L., Amaro de Matos J. Towards an organizational
model of attitude change.  {\it Comput. Math. Organ. Theory},
2002, vol.~8,  pp.~315--335.

3. Almeida Costa L., Amaro de Matos J. Attitude change in
arbitrarily large organizations. {\it Comput. Math. Organ.
Theory}. doi: 0.1007/s10588-013-9160-3. Publ. online 20 July,
2013.

4. Bure V. M., Ekimov A. V., Svirkin M. V. Imitacionnaja model'
formirovanija profilja mnenij vnutri kollektiva. [A simulation
model of forming profile opinions  within the collective]. {\it
Vestnik of Saint Petersburg State University. Series~10. Applied
mathematics. Computer science. Control processes}, 2014, issue~3,
pp.~93--98. (In Russian)

5. Bure V. M., Parilina E. M., Sedakov A. A. Konsensus v
social'noj seti s neodnorodnym mnozhestvom agentov i dvumja
centrami vlijanija  [Consensus in social network with
heterogeneous agents and two centers of  influence].  {\it Proc.
of the III Intern. Conference in memory of professor, member and
correspondent of RAS V.~I.~Zubov}. Saint Petersburg, Saint
Petersburg State University Publ., 2015, pp.~229--230. (In
Russian)





}
