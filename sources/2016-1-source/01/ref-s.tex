{\footnotesize

\vskip 3mm
%\newpage

\noindent{\small \textbf{References} }

\vskip 2mm


1.    Shary~S.~P. Solvability of Interval Linear Equations and
Data Analysis under Un\-cer\-tain\-ty. \textit{Automation and
Remote Control}, 2012, vol.~73, no.~2, pp.~310--322.

2. Shary~S.~P.  New characterizations for the solution set to
interval linear systems of equations.  \textit{Applied Mathematics
and Computation}, 2015, vol.~265, pp.~570--573.

3. Vasin~V.~V., Korotkii~M.~A. Tikhonov regularization with
nondifferentiable stabilizing functionals. \textit{Journal of
Inverse and Ill-posed Problems},  2007, vol.~15, no.~8,
pp.~853--865.

4. Velichko~A.~S. Dvojstvennyj algoritm dlja zadach
reguljarizascii s nedifferensci\-rue\-my\-mi stabilizatorami [Dual
algorithm for regularization problems with nondifferentiable
stabilizing functionals]. \textit{Vychislitel'nye tekhnologii}
[\textit{Computational Technologies}], 2014, vol.~19, no.~2,
pp.~14--19. (in Russian)

5.  Tamasyan~G.~Sh., Chumakov~A.~A.    Finding the distance
between the ellipsoids.     \textit{J. Appl. Ind. Math.},
2014,  vol.~8, no.~3, pp.~400--410.

6. Demyanov~V.~F., Giannessi~F., Karelin~V.~V. Optimal Control
Problems via Exact Penalty Functions. \textit{J. Global Optim.},
 1998, vol.~12, no.~3, pp.~215--223.

7. Filimonov~N.~B. Metody poliedral'nogo programmirovanija v
diskretnyh zadachah upravlenija i nabljudenija [Polyhedral
programming methods in discrete problems of control and
observation]. \textit{Metody klassicheskoj i sovremennoj teorii
avtomaticheskogo upravlenija: v 5 t.} [\textit{Methods of
classical and modern theory of automatic control. In 5 vol.}].
 Moscow, Bauman Moscow State Technical University
Press, 2004, vol.~5, ch.~7, pp.~647--720. (In Russian)

8.  Demenkov~M.~N.,  Filimonov~N.~B.  Variable Horizon Robust
Predictive Control via Adjustable Controllability Sets.
\textit{European J. Control},  2001, vol.~7, issue~6,
pp.~596--604.

9. Demyanov~V.~F. \textit{Uslovija ekstremuma i variascionnoe
ischislenie} [\textit{Conditions of Extremum and Calculus of
Variations}]. Moscow, Vysshaja shkola Publ., 2005, 335~p. (In
Russian)

10.    Demyanov~V.~F., Tamasyan~G.~Sh.  O prjamyh metodah
reshenija variascionnyh zadach [On direct methods for solving
variational problems]. \textit{Trudy In-ta matematiki i~mekhaniki
Ural. otd. RAN} [\textit{Works of Institute of mathe\-mat\-ics and
mechanics UrB RAS}],      2010, vol.~16, no.~5, pp.~36--47. (In
Russian)

11. Tamasyan~G.~Sh. O metodah naiskorejshego i
gipodifferenscial'nogo spuska v odnoj zadache variascionnogo
ischislenija [On Methods of Steepest and Hypodiffernetial Descent
in one Problem of Calculus of Variations]. \textit{Vychislitel'nye
metody i~programmirovanie: novye vychislitel'nye tekhnologii}
[\textit{Comp. Methods and Comp. Sci.}], 2012, vol.~13,
pp.~197--217. (In Russian)

12. Dolgopolik~M.~V., Tamasyan~G.~Sh. Ob ekvivalentnosti metodov
naiskorejshego i gipodifferenscial'nogo spuskov v nekotoryh
zadachah uslovnoj optimizascii   [On Equivalence of the Method of
Steepest Descent and the Method of Hypodifferential Descent in
Some Constrained Optimization Problems]. \textit{Izv. Saratovsk.
un-ta. Nov. ser. Ser. Matematika. Mekhanika. Informatika}
[\textit{Izv. of Saratovsk. State University. New Series. Series
Mathematics. Mechanics. Computer science}], 2014, vol.~14,
issue~4, pp.~532--542. (In Russian)

14. Clarke~F.~H. Generalized gradients and applications.
\textit{Trans. Amer. Math. Soc.}, 1975, vol.~205, pp.~247--262.

15. Clarke~F.~H. \textit{Optimization and nonsmooth analysis}. New
York, Wiley, 1983, 308~p.

16. Shor~N.~Z. \textit{Minimization Methods for Non-differentiable
Functions}. Berlin, Springer, 1985, 162~p.

17. Tuy~H. D.c. optimization: theory, methods and algorithms.
\textit{Handbook of Global Optimization}. Eds R.~Horst,
P.~M.~Pardalos. Dordrecht, Kluwer, 1995, pp.~149--216.

18. Hiriart-Urruty J. B. From convex minimization to nonconvex
minimization: nec\-es\-sary and sufficient conditions for global
optimality. \textit{Nonsmooth optimization and related topics.}
Eds F.~N.~Clarke, V.~F.~Demyanov, F.~Gianessi. New York, Plenum,
1989, pp.~219--240.

19. Strekalovsky~A.~S. \textit{Elementy nevypukloj optimizascii}
[\textit{Elements of nonconvex opti\-mi\-zation}]. Novosibirsk,
Nauka Publ., 2003, 356~p. (In Russian)

20. Strekalovsky~A.~S. Global optimality conditions for nonconvex
optimization. \textit{J. Global Optim.}, 1998, vol.~12, issue~4,
pp.~415--434.

21. Polyakova~L.~N. On global unconstrained minimization of the
difference of polyhedral functions. \textit{J. Global Optim.},
2011, vol.~50, pp.~179--195.

22. Alexandrov~A.~D. O poverhnostjah, predstavimyh v vide raznosti
vypuklyh funkscij [On surfaces which may be represented by a
difference of convex functions]. \textit{Izv. Akad. Nauk Kazakh.
SSR. Ser. Fizika i~matematika} [\textit{Izv. of Acad. of Sciences
of Kazakh. SSR. Series Physics and mathematics}], 1949, vol.~3,
pp.~3--20. (In Russian)

23. Alexandrov~A.~D. Poverhnosti, predstavimye v vide raznosti
vypuklyh funscij [On surfaces which may be represented by
difference of convex functions]. \textit{Dokl. Akad. Nauk SSSR}
[\textit{Proc. of Acad. of Sciences of USSR}], 1950, vol.~72,
no.~4, pp.~613--616. (In Russian)

24. Zalgaller~V.~A. O predstavimosti funkscij dvuh peremennyh v
vide raznosti vypuklyh funkscij [On the representation of
functions of two variables as a difference of convex functions].
\textit{Vestn. Leningr. un-ta}  [\textit{Vestnik of Leningrad
University}], 1963, no.~1, pp.~44--45. (In Russian)

25. Melzer~D. On the Expressibility of Piecewise-Linear Continuous
Functions as the Difference of two Piecewise-Linear Convex
Functions. \textit{Math.  Programming Study}, 1986, vol.~29,
pp.~118--134.

26.    Plotnikov~S.~V.     \textit{Metody proektirovaniia
v~zadachakh nelineinogo programmirovaniia. Dis. na soiskanie
uchen. stepeni kand. fiz.-mat. nauk}     [\textit{Projection
methods in nonlinear programming problems.     Unpublished
doctoral thesis}].     Sverdlovsk,     Ural State University,
1983, 126~p. (In Russian)

27. Volokitin~E.~P. O predstavlenii nepreryvnyh kusochno-linejnyh
funkscij [On the representation of continuous piecewise affine
functions]. \textit{Upravljaemye sistemy} [\textit{Management
systems}]. Novosibirsk, Nauka Publ., 1979, no.~19, pp.~14--21. (In
Russian)

28. Kripfganz~A., Schulze~R. Piecewise affine functions as a
difference of two convex functions. \textit{Optimization}, 1987,
vol.~18,  pp.~23--29.

29. Gorokhovik~V.~V., Zorko~I.~O. Piecewise affine functions and
polyhedral sets. \textit{Optimization}, 1994, vol.~31,
pp.~209--221.

30. Bartels~S., Knutz~L., Scholtes~S. Continous Selection of
Linear Functions and Nonsmooth Critical Point Theory.
\textit{Nonlinear Analysis, Theory, Meth. and Appl.}, 1995,
vol.~24, pp.~385--407.

31.  Eremin~I.~I.  Nekotorye voprosy kusochno-linejnogo
programmirovanija   [Some ques\-tions of piecewise linear
programming]. \textit{Izv. vuzov. Matematika} [\textit{Izv. Higher
educational institutuions. Mathematics}],  1997, no.~12,
pp.~49--61. (In Russian)

32. Gratzer~G. \textit{Lattice Theory: First Concepts and
Distributive Lattices}. San Francisco, Freeman, 1971, 212~p.

33. Demyanov~V.~F., Rubinov~A.~M. \textit{ Osnovy negladkogo
analiza i kvazidifferenscial'nogo ischislenija}
[\textit{Foundations of Nonsmooth Analysis and Quasidifferential
Calculus}].  Moscow, Nauka Publ., 1990, 432~p. (In Russian)

34. Preparata~F.~P., Shamos~M.~I. \textit{Computational Geometry:
An Introduction}. New York, Springer, 1985, 398~p.

35. Malozemov~V.~N. Nekotorye svojstva diskretnogo maksimuma [Some
properties of the discrete maximum]. \textit{Seminar ``CNSA $\&$
NDO''. Izbr. dokl.} [\textit{Seminar ``CNSA $\&$ NDO''. Selected
reports}].  14th~May 2015. Available at:
\text{http://www.apmath.spbu.ru/cnsa/reps15.shtml\#0514a}
(accessed: 23.06.2015). (In Russian)

36. Leichtveiss~K. \textit{Convex Mengen}. Berlin, Springer, 1980,
330~P. (Russ. ed.: Leichtveiss~K. \textit{Vypuklye mnozhestva}.
Moscow, Nauka Publ., 1985, 336~P. (In Russian)

37.  Andramonov~M.~Ju., Tamasyan~G.~Sh.  Realizascija
analiticheskogo kodifferensci\-ro\-va\-nija v pakete MATLAB
[Implementation of analytical codifferentiation in MATLAB].
\emph{Vychislitel'nye metody i~programmirovanie}
[\emph{Calculation Methods and Programming}], 2007, vol.~8,
issue~1, pp.~1--5. (In Russian)

38.  Angelov~T.~A.   O vychislenii kodifferenscialov  [On
evaluation of codifferentials]. \emph{Vychislitel'nye metody
i~programmirovanie} [\emph{Calculation Methods and Programming}],
2013, vol.~14, issue~1, pp.~113--122. (In Russian)

39. Pogorzelski~H.~A. Reviewed work(s): Remarks on Nicod's Axiom
and on ``Gen\-er\-al\-iz\-ing Deduction'' by Jan Lukasiewicz;
Jerzy Slupecki; Panstwowe Wydawnictwo Naukowe. \textit{The Journal
of Symbolic Logic}, 1965, vol.~30, no.~3, pp.~376--377.

}
