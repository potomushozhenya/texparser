

{\footnotesize

\vskip 2mm
%\newpage

\noindent {\small\textbf{References} }

\vskip 1.5mm




{1.} Gorbachev V. P., Erohin S. V., Stepanchuk V. P., Shlyapin V.
V. The Simulator of the Microtron. {\it Proc. of  Forth   Intern.
Workshop Beam Dynamics $\&$ Optimization}.  Dubna, 1998,
pp.~95--97.

{2.} Kapica S. P., Melehin V. N. {\it Mikrotron} [{\it
Microtron}]. Moscow, Nauka Publ., 1969, 210~p. (In Russian)

{3.} Berten'ev O. V. {\it Sovremennyj Fortran} [{\it Modern
Fortran}]. Moscow, Dialog MIFI, 2000, 448~p. (In Russian)

{4.} Nagel D. E., Knapp E. A., Knapp B. C. Coupled Resonant Model
for Standing Wave Accelerator Tanks. {\it Review of  Scientific
Instruments}, 1967, vol.~38, no.~11, pp.~1583--1587.

{5.} Bondus' A. A., Gorbachev V. P., Stepanchuk V. P. Perehodnye
processy v monobloke magnetron--uskoriaiushchij rezonator
mikrotrona   [Transients in monoblock magnetron--accelerating
cavity microtron] {\it Vestnik of Saint Petersburg State
University. Series~10. Applied mathematics. Computer science.
Control processes}, 2008, issue~3, pp.~34--41. (In Russian)

{6.} Bondus' A. A., Gorbachev V. P.,   Stepanchuk V. P., Maksimov
R. V., Mutasov D. V. Perspektivy uluchsheniia parametrov
malogabaritnogo mikrotrona trehsantimetrovogo diapazona [Prospects
for improving the parameters of small-sized $X$-bands microtron].
{\it Izvestiia Saratovskogo universiteta. Ser.~Phizika} [{\it Izv.
of Saratov State University. Series Physics}], 2010, vol.~10,
issue~2, pp.~47--55. (In Russian)

{7.} Gorbachev V. P.,  Stepanchuk V. P., Mutasov D. V. Mikrotrony
trehsantimetrovogo diapazona. (Issledovaniia, vypolnennye v SGU za
poslednie 15 let) [$X$-band microtrons. (Studies performed at
Saratov State University for the past 15 years)]. Saarbrucken,
Germany,  Lambert Academic Publ., 2013, 68~p. (In Russian)

{8.} Zavorotylo V. N., Milovanov O. S. Model' magnetronnogo
generatora dlia rascheta perehodnyh processov [Model magnetron
oscillator for the calculation of transients]. {\it Uskoriteli}
[{\it Guichenions}]. Moscow, Atomizdat, 1977, no.~16, pp.~34--37.
(In Russian)

{9.} Bondus' A. A, Gorbachev V. P., Stepanchuk V. P., Maksimov R.
V., Mutasov D. V., Marchen-\linebreak ko V. K. Elektronnaia
provodimost' magnetrona MI-505 [Electronic conductivity magnetron
MI-505]. {\it Izvestiia VUZOV. Prikladnaia nelinejnaia dinamika}
[{\it Proc. of the Higher educational institutions. Applied
non-linear dynamic}], 2008, no.~6, pp.~70--75. (In Russian)

{10.} Kosapev E. L. Processy ustanovleniia i predel'nyj tok v
mikrotrone [The process of establishing and limiting current in a
microtron]. {\it Zhurn. teor. fiziki} [{\it J. of Theor.
Physics}], 1972, no.~10, pp.~2239--2247. (In Russian)

{11.} Gorbachev V. P.  Vysokochastotnye sistemy mikrotrona. Dis.
kand. fiz.-mat. nauk. [Microwave system's of microtron. Cand.
phys. and math. sciences dis.]. Moscow, NIIIaF of Moscow State
University, 2007, 149~p. (In Russian)

{12.}  Arhangel'skij A. Ia., Tagin M. A. {\it Programmirovanie v}
C$++$ Builder {\it 6 i 2006} [{\it Programming in} C$++$ Builder
{\it 6 and 2006}]. Moscow, Binom Publ., 2007,  1184~p. (In
Russian)




}
