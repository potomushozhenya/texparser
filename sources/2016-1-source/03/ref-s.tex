

{\footnotesize

\vskip 3mm
%\newpage

\noindent{\small \textbf{References} }

\vskip 2mm

1. Quackenbush~J. Computational analysis of microarray data. {\it
Nature reviews genetics}, 2001, vol.~2, no.~6, pp.~418--427.

2. Shamir~R., Sharan\:R. Algorithmic approaches to clustering gene
expression data. {\it Current Topics in Computational Biology},
2001. Available at: http://citeseerx.ist.psu.edu/ (assessed:
15.09.2015).

3. Gordon~A.\:D. {\it Classification}. London, Chapman \& Hall/CRC
Monographs on Statistics \& Applied Probability, 1999. Available
at: http://www.citeulike.org/ (assessed: 20.08.2015).

4. Jain~A.\:K., Dubes~R.\:C. {\it Algorithms for clustering data}.
Prentice-Hall, Prentice-Hall, Inc., 1988. Available at:
http://dl.acm.org/ (assessed: 14.06.2015).

5. Buhmann~J. Data clustering and learning. {\it The Handbook of
Brain Theory and Neural Networks}, 1995, pp.~278--281.

6. Chakravarthy~S.\:V., Ghosh~J. Scale-based clustering using the
radial basis function network. {\it Neural Networks, IEEE
Transactions}, 1996, vol.~7, no.~5, pp.~1250--1261.

7. Gordon~A.\:D. Identifying genuine clusters in a classification.
{\it Computational Statistics $\&$ Data Analysis}, 1994, vol.~18,
no.~5, pp.~561--581.

8. Hartigan~J.\:A. Statistical theory in clustering. {\it Journal
of classification}, 1985, vol.~2, no.~1, pp.~63--76.

9. Milligan~G.\:W., Cooper~M.\:C. An examination of procedures for
determining the number of clusters in a data set. {\it
Psychometrika}, 1985, vol.~50, no.~2, pp.~159--179.

10. Sugar~C.\:A., James~G.\:M. Finding the number of clusters in a
dataset. {\it Journal of the American Statistical Association},
2003, vol.~98, no.~463, pp.~750--763.

11. Tibshirani~R., Walther~G. Cluster validation by prediction
strength. {\it Journal of Computational and Graphical Statistics},
2005, vol.~14, no.~3, pp.~511--528.

12. Hubert~L., Schultz~J. Quadratic assignment as a general data
analysis strategy. {\it British journal of mathematical and
statistical psychology}, 1976, vol.~29, no.~2, pp.~190--241.

13. Tibshirani~R., Walther~G., Hastie~T. Estimating the number of
clusters in a data set via the gap statistic. {\it Journal of the
Royal Statistical Society. Series B $($Statistical
Methodology$)$}, 2001, vol.~63, no.~2, pp.~411--423.

14. Wishart~D. Mode analysis: A generalization of nearest neighbor
which reduces chaining effects. {\it Numerical taxonomy}, 1969,
vol.~76, no.~17, pp.~282--311.

15. Hartigan~J.\:A. {\it Clustering algorithms}. New York, John
Wiley \& Sons, Inc., 1975. Available at: http://dl.acm.org/
(assessed: 28.08.2015).

16. Hartigan~J.\:A. Consistency of single linkage for high-density
clusters. {\it Journal of the American Statistical Association},
1981, vol.~76, no.~374, pp.~388--394.

17. Cuevas~A.,~Febrero~M., Fraiman~R. Estimating the number of
clusters. {\it Canadian Journal of Statistics}, 2000, vol.~28,
no.~2, pp.~367--382.

18. Cuevas~A., Febrero~M., Fraiman~R. Cluster analysis: a further
approach based on density estimation. {\it Computational
Statistics $\&$ Data Analysis}, 2001, vol.~36, no.~4,
pp.~441--459.

19. Stuetzle~W. Estimating the cluster tree of a density by
analyzing the minimal spanning tree of a sample. {\it Journal of
classification}, 2003, vol.~20, no.~1, pp.~25--47.

20. Pelleg~D., Moore~A.\:W. $X$-means: Extending $K$-means with
Efficient Estimation of the Number of Clusters. {\it ICML}, 2000,
pp.~727--734.

21. Volkovich~Z., Brazly~Z., Toledano-Kitai~D., Avros~R. The
Hotelling's metric as a cluster stability measure. {\it Computer
modelling and new technologies}, 2010, vol.~14, pp.~65--72.

22. Barzily~Z., Volkovich~Z., Akteke-Ozturk~B. On a minimal
spanning tree approach in the cluster validation problem. {\it
Informatica, Lith. Acad. Sci.}, 2009, vol.~20, no.~2,
pp.~187--202.

23. Hamerly~Y.\:F.\:G. PG-means: learning the number of clusters
in data. {\it Advances in neural information processing systems},
2007, vol.~19, pp.~393--400.

24. Breckenridge~J.\:N. Replicating cluster analysis: Method,
consistency, and validity. {\it Multivariate Behavioral Research},
1989, vol.~24, no.~2, pp.~147--161.

25. Dudoit~S., Fridlyand~J. A prediction-based resampling method
for estimating the number of clusters in a dataset. {\it Genome
biology}, 2002, vol.~3, no.~7, research0036. Available at: http://
www.genomebiology.com/ (assessed: 14.06.2015).

26. Lange~T., Roth~V., Braun~M.\:L., Buhmann~J.\:M.
Stability-based validation of clustering solutions. {\it Neural
computation}, 2004, vol.~16, no.~6, pp.~1299--1323.

27. Milligan~G.\:W., Cheng~R. Measuring the influence of
individual data points in a cluster analysis. {\it Journal of
classification}, 1996, vol.~13, no.~2, pp.~315--335.

28. Ben-Hur~A., Elisseeff~A., Guyon~I. A stability based method
for discovering structure in clustered data. {\it Pacific
symposium on biocomputing}, 2002, vol.~7, no.~6, pp.~6--17.

29. Levine~E., Domany~E. Resampling method for unsupervised
estimation of cluster validity. {\it Neural computation}, 2001,
vol.~13, no.~11, pp.~2573--2593.

30. Fowlkes~E.\:B., Mallows~C.\:L. A method for comparing two
hierarchical clusterings. {\it Journal of the American statistical
association}, 1983, vol.~78, no.~383, pp.~553--569.





}
