


{\footnotesize

\vskip 4mm
%\newpage

\noindent {\small\textbf{References} }

\vskip 3mm

1. Severy D. M., Mathewson L. H., Bechtol C. O. Controlled
automobile rear-end collisions --- an investigation of related
engineering and medical phenomena. {\it Proc. of the Montreal
conference on medical aspects of traffic accidents}, 1955,
pp.~152--184.

2.   Davidson  J., Deutscher C., Hell W., Linder A., Lovsund P.,
Svensson M. Y. Human volunteer kinematics in rear-end sled
collisions. {\it Proc. of Intern. Council on biomechanics of
impact}. Goteborg, Sweden, 1998, pp.~289--301.

3.    Yamazaki K., Ono K., Kaneoka K. A simulation analysis of
human cervical motion during low speed rear-end impacts. {\it
Proc. of Society of Automotive Engineers World Congress}. Detroit,
USA, 2000, pp.~186--192.

4.    Tencer A. F., Mirza S. K., Bensel K. L. The response of
human volunteers to rear-end impacts: the effect of head restraint
properties. {\it Spine}, 2001, no.~26, pp.~2432--2440.

5.    Martinez J. L., Garcia D. J. A model for whiplash. {\it
Journal of Biomechanics}, 1965, no.~1, pp.~23--32.

6.    McHenry R. R., Naab K. N. {\it Computer simulation of the
automobile crash victim in a frontal collision}. A validation
study report no.~YB-2126-V-1. New York, Cornell Aeronautical
Laboratory, 1966, 86~p.

7.   Tregubov V. P. Analis povedenia tela cheloveka pri
gorizontalnom impulsnom vozdejstvii [Analysis of the human body
behavior under horizontal pulse action]. {\it Fizicheskaja
mehanika} [{\it Physical mechanics}], Leningrad, Leningrad
University Publ., 1976, issue~2, pp. 140--154. (In Russian)

8.   Tregubov V. P., Klikunova K. A., Asanchevsky V. V. Some
peculiarities of the skeletal muscle modeling. {\it Acta of
Bioengineering and Biomechanics}, 2008, vol.~10, suppl.~1,
pp.~30--36.

9.   Orne D., King Y. Liu. A mathematical model of spinal response
to impact. {\it Journal of Biomechanics}, 1971, vol.~4, no.~1,
pp.~49--71.

10.  Mc\,Kenzie J. A., Williams J. F. Dynamics of cervical spine
during whiplash. {\it Journal of Biomechanics},  1971, vol.~4,
no.~6, pp. 477--491.

11. Panjubi M., Brand R., White A. Tree dimensional flexibility
and stiffness properties of human thoracic spine. {\it Journal of
Biomechanics}, 1976, vol.~9, pp.~185--192.

12.  Maroney S., Schultz A., Miller J., Anderson G.
Load-displacement properties of lower cervical spine motion
segments. {\it Journal of Biomechanics}, 1988, vol.~21,
pp.~769--779.

13. Shea M., Edwards W. T., White A. A., Hayes W. Variations of
stiffness and strength along the human cervical spine. {\it
Journal of Biomechanics}, 1991, no.~2, pp.~95--107.

14. Sobczak P., Golinski W. Z., Gentle R. C. Biomechanical
evaluation of anterior cervical spine stabilization using the
finite element approach. {\it Acta of Bioengineering and
Biomechanics}, 1999, vol.~1, no.~1, pp.~95--100.

15. Pintar F. A., Miclebust J., Sances A., Jr., Yoganandan N.
Biomechanical properties of human intervertebral disk in tension.
{\it Proc. ASMEF Advances in Bioengineering} (New York, USA),
1986, pp.~38--39.

16. Ciach M., Awrejcewicz J., Maciejczak A., Radek M. Experimental
and numerical investigations of C5--C6 cervical spinal segment
before and after discectomy using the Cloward operation technique.
{\it Acta of Bioengineering and Biomechanics}, 1999, vol.~1,
no.~1, pp.~101--105.

17. Yoganandan N., Kumaresan S., Pintar F. A. Geometrical and
mechanical properties of human cervical spine ligaments. {\it
Journal of Biomech. Engineering}, 2000, vol.~22, pp.~623--629.

18. Deng D. C., Goldsmith W. Response of a human
head/neck/upper-torso replica to dynamic loading. II.
Analytical-numerical model. {\it Journal of Biomechanics}, 1987,
vol.~20, no.~5, pp.~487--97.

19. Skrzupiec D. M., Pollintine P., Przybyla A., Dolan P., Adams
M. A. The internal mechanical properties of cervical
intervertebral discs as revealed by stress profilometry. {\it
European Spine Journal}, 2007, vol.~6, no.~10, pp.~1701--1709.

20. Stupakov G. P., Kozlovsy A. P., Kazejkin V S. {\it Biomehanika
pozvonochnika pri udarnih peregruzkah v praktike aviacionnih i
kosmicheskih poletov} [{\it Biomechanics of vertebral column under
the shock overload in the practice of aviation and space
flights}]. Leningrad, Nauka Publ., 1987, 240~p. (In Russian)

21. Stemper B. D., Yoganandan N., Pinter F. A. Validation of a
head--neck computer model for whiplash simulation. {\it Medical
$\&$ Biological Engineering $\&$ Computing}, 2004, no.~42,
pp.~333--338.

22. Takeshima T., Omokawa S., Takaoka T., Araki M., Ueda Y.,
Takakura Y. Sagittal alignment of cervical flexion and expansion.
{\it Spine}, 2002, vol.~22, pp.~2432--2442.

23. Hill A. V.  {\it  First and last experiments in muscle
mechanics}. Cambridge, Cambridge University Press, 1970, 180~p.
(Russ. ed.: Hill A. V. {\it Mehanika mishechnogo sokraschenia.
Starie i novie opyty}. Moscow, Mir Publ., 1972, 184~p. (in
Russian).




}
