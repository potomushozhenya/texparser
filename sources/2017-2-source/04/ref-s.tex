{\footnotesize

\vskip 4mm
%\newpage

\noindent{\small \textbf{References} }

\vskip 3mm


1. Elizavetin I. V., Shuvalov R. I., Bush V. A. Principy i metody
radiolokacionnoj s'emki dlja celej formirovanija cifrovoj modeli
mestnosti [Principles and methods of SAR Interferometry for the
purpose of forming a digital elevation model]. \textit{Geodesy and
cartography}, 2009, no.~1, pp.~39--45. (In Russian)

2. Ferretti A., Monti-Guarnieri A., Prati C. et al. \textit{InSAR
Principles: Guidelines for SAR Interferometry Processing and
Interpretation.} Available at:
http://www.esa.int/esapub/tm/tm19/TM-19\_ptA.pdf  (accessed:
02.08.2016).

3. Zhengxiao Li, Bethel J. Image coregistration in SAR
interferometry. \textit{The International Archives of the
Photogrammetry, Remote Sensing and Spatial Information Sciences.}
Beijing, 2008, vol.~XXXVII, pt~B1, pp.~433--438.

4. Massonnet D., Feigl K. L. Radar interferometry and its
application to changes in the earth�s surface. \textit{Reviews of
Geophysics}, 1998, vol.~36, issue~4, pp.~441--500.

5. Costantini M., Farina A., Zirilli F. A fast phase unwrapping
algorithm for SAR interferometry. \textit{IEEE Trans. GARS}, 1999,
vol.~37, no.~1, pp.~452--460.

6. Mistry P., Braganza S., Kaeli D., Leeser M. Accelerating phase
unwrapping and affine transformations for optical quadrature
microscopy using CUDA. \textit{Proceedings of 2nd Workshop on
General Purpose Processing on Graphics Processing Units. GPGPU.
Conference}. Washington, D.C., USA, ACM, 2009, pp.~28--37.

7. Karasev P. A., Campbell D. P., Richards M. A. Obtaining a 35x
speedup in 2d phase unwrapping using commodity graphics
processors. \textit{Radar Conference.} IEEE, 2007, pp.~574--578.

8. Wu Z., Ma W., Long G., Li Y., Tang Q., Wang Z. High performance
two-dimensional phase unwrapping on GPUs. \textit{Proceedings of
the 11th ACM Conference on Computing Frontiers~--- CF'14.} New
York, NY, USA, ACM, 2014, pp. 35:1--35:10.

9. Xin-Liang S., Xiao-Chun X. GPU acceleration of range alignment
based on minimum entropy criterion. \textit{Radar Conference. IET
International}, 14--16 April 2013, pp.~1--4.

10. Guerriero A., Anelli V. W., Pagliara A., Nutricato R., Nitti
D. O. High performance GPU implementation of InSAR time-consuming
algorithm kernels. \textit{Proceedings of the 1st WORKSHOP
on\linebreak the State of the art and Challenges of Research
Efforts at POLIBA.} Bari, Italy, Politecnico di Bari, 2014,
pp.~383.

11. Zhang F., Wang B., Xiang M. Accelerating InSAR raw data
simulation on GPU using CUDA. \textit{Geoscience and Remote
Sensing Symposium $($IGARSS$)$. IEEE International}. Bari, Italy,
Politecnico di Bari, 25--30 July 2010, pp.~2932--2935.

12. Marinkovic P. S., Hanssen R. F., Kampes B. M. Utilization of
parallelization algorithms in InSAR/PS-InSAR processing.
\textit{Proceedings of the 2004 Envisat ERS Symposium $($ESA
SP-572}). Salzburg, Austria, ESA, 6--10 September 2004, pp.~1--7.

13. Sheng G., Qi-Ming Z., Jian J., Cun-Ren L.  Qing-xi T. Parallel
processing of InSAR interferogram filtering with CUDA programming.
\textit{Zhongguo Cehui Kexue Yanjiuyan,} China, vol.~40, no.~1,
pp.~67--88.

14. Verba V.~S., Neronskij L.~B.,~Osipov I.~G.,~Turuk V.~Je. {\it
Radiolokacionnye sistemy zemleobzora kosmicheskogo bazirovaniya
$[$Space-based radio location systems of Earth Observation}].
Moscow, Radiotechnica Publ., 2010, 675~p. (In Russian)

15. Gabriel E., Fagg G. E., Bosilca G. et al. \textit{Open MPI:
Goals, Concept, and Design of a Next Generation MPI
Implementation.} Available at:
https://www.open-mpi.org/papers/euro-pvmmpi-2004-overview/euro-pvmmpi-2004-overview.pdf
(accessed: 30.06.2016).

16. Kampes B., Hanssen R., Perski Z. \textit{Radar Interferometry
with Public Domain Tools presentation.} Available at:
http://doris.tudelft.nl/Literature/kampes\_fringe03.pdf (accessed:
30.06.2016).

17. Frigo M., Johnson S. G., FFTW: An Adaptive software
architecture for the FFT. \textit{ICASSP conference proceedings.}
Seattle, Washington, USA, IEEE, 15 May 1998, vol.~3,
pp.~1381--1384.

18. Larkin J. \textit{Fast GPU Development with CUDA Libraries.}
Available at: https://www.olcf.ornl.
gov/wp-content/uploads/2013/02/GPU\_libraries-JL.pdf (accessed:
30.06.2016).

19. Demmel J., Dongarra J. \textit{ST-HEC$:$ Reliable and Scalable
software for linear algebra compu-\linebreak tations on High End
Computers.} Available at: https://people.eecs.berkeley.edu/
demmel/Sca-LAPACK-Proposal.pdf (accessed: 30.06.2016).

20. Feoktistov A. A., Zaharov A. I., Gusev M. A., Denisov P. V.
Issledovanie vozmozhnostej me- toda malyx bazovyx linij na primere
modulya SBaS programmnogo paketa SARScape i dannyx RSA
ASAR/ENVISat i PALSAR/ALOS. Ch.~1. Klyuchevye momenty metoda
[Investigation of the possibilities of the method of small
baselines technique on the example of the SBaS module of the
software package SARScape and the data of the RSA ASAR/ENVISat and
PALSAR/ALOS. Pt~1. Key points of the method]. \textit{Journal of
Radioelectronics}, 2015, no.~9, pp.~1--26. (In Russian)

21. Reyes-Ortiz J. L., Oneto L., Anguita D. Big Data analytics in
the cloud: Spark on Hadoop vs MPI/OpenMP on Beowulf. \textit{INNS
Conference on Big Data} Program. San Francisco, USA, 8--10 August
2015, pp.~121--130.

22. Kannan P. \textit{Beyond Hadoop MapReduce Apache Tez and
Apache Spark.} Available at:
http://www.sjsu.edu/people/robert.chun/courses/CS259Fall2013/s3/F.pdf
(accessed: 02.08.2016).

23. Nathan P. \textit{Real-Time Analytics with Spark Streaming.}
Available at: http://viva-lab.ece.
virginia.edu/foswiki/pub/InSAR/RitaEducation/InSAR Technology
Literature Search.pdf (accessed: 02.08.2016).

24. Nagler E. {\it Introduction to Oozie. Apache Oozie
Documentation}. Available at: http://www.cse.
buffalo.edu/bina/cse487/fall2011/Oozie.pdf (accessed: 02.08.2016).

25. Jhajj R. \textit{Apache Hadoop Hue Tutorial.} Available at:
https://examples.javacodegeeks.com/
enterprise-java/apache-hadoop/apache-hadoop-hue-tutorial/
(accessed: 02.08.2016).

26. Potapov V. P., Popov S. E. Vysokoproizvoditel'nyj algoritm
rosta regionov dlya razvertki interferometricheskoj fazy na baze
texnologii CUDA [High-Performance Region-Growing Algorithm for
InSAR Phase Unwrapping Based on CUDA]. \textit{Software
engineering,} 2016, no.~2, pp.~61--74. (In Russian).



\vskip 2mm

{\bf For citation:}  Potapov V. P., Kostylev M. A., Popov S. E.
The streaming processing of sar data in distributed environment
with Apache Spark. {\it Vestnik of Saint Petersburg University.
Applied Mathematics. Computer Science. Control Processes},
\issueyear, vol.~13, iss.~\issuenum,
pp.~\pageref{p4}--\pageref{p4e}.
\doivyp/spbu10.\issueyear.\issuenum04


}
