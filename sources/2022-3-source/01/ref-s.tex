
{\normalsize

\vskip 4.5%6
mm

\noindent{\bf Analysis and synthesis of communication network structures by state\\ enumeration method%$^{*}$%
}

}

\vskip 2mm

{\small

\noindent{\it K. A. Batenkov %

}

\vskip 2mm

%%%%%%%%%%%%%%%%%%%%%%%%%%%%%%%%%%%%%%%%%%%%%%%%%%%%%%%%%%%%%%%%%%

%\efootnote{
%%
%\vspace{-3mm}\parindent=7mm
%%
%\vskip 0.1mm $^{*}$ This work was supported by Russian %
%Foundation for Basic Research (project N 20-07-00531 A).%\par
%%
%%\vskip 2.0mm
%%
%%\indent{\copyright} �����-������������� ���������������
%%�����������, \issueyear%
%%
%}

%%%%%%%%%%%%%%%%%%%%%%%%%%%%%%%%%%%%%%%%%%%%%%%%%%%%%%%%%%%%%%%%%%

{\footnotesize


%\noindent%
%$^3$~%
%St\,Petersburg State University, 7--9, Universitetskaya nab.,
%St\,Petersburg,
%
%\noindent%
%\hskip2.45mm%
%199034, Russian Federation

\noindent%
%$^3$~%
Academy of Federal Guard Service of Russian Federation, 35, Priborostoitelnaya ul., Oryol,

\noindent%
%\hskip2.45mm%
302015, Russian Federation



}

%%%%%%%%%%%%%%%%%%%%%%%%%%%%%%%%%%%%%%%%%%%%%%%%%%%%%%%%%%%%%%%%%%

\vskip2mm%3mm


\noindent \textbf{For citation:} Batenkov K. A. Analysis and synthesis of communication network structures by state enumeration method. {\it Vestnik of Saint~Petersburg Uni\-ver\-si\-ty.
Ap\-plied Mat\-he\-matics. Computer Science. Control
Processes},\,\issueyear,
vol.~18,~iss.~\issuenum,~pp.~\pageref{p1}--\pageref{p1e}. \\
\doivyp/\enskip%
\!\!\!spbu10.\issueyear.\issuenum01 (In Russian)

\begin{hyphenrules}{english}

\vskip2mm

{\leftskip=7mm\noindent One of the methods of analysis and synthesis of communication network structures is consi\-dered, based on the simplest approach to calculating the probability of connectivity --- the method of iterating over the states of the network edges. Despite its significant drawback, which consists in the considerable complexity of the calculations carried out, it turns out to be quite in demand both at the stage of debugging new analysis methods and when performing the procedure of sequential synthesis of network structures. The proposed method of sequential synthesis can be presented in the form of stages, at each of which one or more edges (network elements) are added. An increase in the number of edges used leads to an increase in the number of variations of the connectivity functions of a graph with an added edge, and hence to an increase in the complexity of operations for calculating conditional probabilities. At the same time, such a complication makes it possible to more accurately solve the synthesis problem, since not in all situations the sequential addition of edges is equivalent to sorting through all possible alternatives. Both the described method of analyzing the structures of communication networks based on the enumeration of states and the synthesis method differ in the essential simplicity of the implementation of the processes of the calculations carried out. It is this circumstance that allows us to use these methods as reference. The accuracy of the calculations depends solely on the capabilities of hardware and software systems and is in no way limited directly by the method of sorting states. As a result, the calculation of the probability of connectivity with precision accuracy, which is typical for situations of comparative analysis of communication networks with the availability and survivability coefficients of individual network elements close to the threshold values, also turns out to be feasible on the basis of the methods considered.\\[1mm]
\textit{Keywords}: network, graph structure, connectivity probability, coefficient of readiness, coeffi\-cient of operational readiness, complete state enumeration method. \par}

%\vskip6mm%5

\end{hyphenrules}

\newpage

\noindent \textbf{References} }

\vskip 2mm

{\footnotesize

1. {Zuev K. M., Wu S., Beck J. L.} General network reliability problem and its efficient solution by Subset Simulation. \textit{Probabilistic Engineering Mechanics}, 2015, vol.~40, pp.~25--35.

2. {Mussel C., Hopfensitz M., Kestler H. A.} \textit{Boolnet package vignette}, 2019.\\ https://cran.r-project.org/web/packages/BoolNet/vignettes/BoolNet\_package\_vignette.Snw.pdf (acces\-sed: August 21, 2021).

3. {Teruggia R.} \textit{Reliability analysis of probabilistic networks}. PhD thesis. Turin, Univ. of Turin, School of Doctorate in Science and High Technology Publ., 2010, 214~p.

4. Dudnik B. Ya., Ovcharenko V. F. \textit{Nadezhnost' i zhivuchest' sistem svyazi} [\textit{The reliability and survivability of communication systems}]. Moscow, Radio i svyaz' Publ., 1984, 216~p. (In Russian)

5. GOST R 53111--2008. \textit{Ustojchivost' funkcionirovaniya seti svyazi obshchego pol'zovaniya. Tre\-bo\-va\-niya i metody proverki} [\textit{Stability of the public communication network. Requirements and ve\-ri\-fi\-ca\-tion methods}]. Moscow, Standardinform Publ., 2009, 16~p. (In Russian)

6. Oboskalov V. P. \textit{Strukturnaya nadezhnost' elektroenergeticheskih sistem} [\textit{Structural reliability of electric power systems}]. Ekaterinburg, Ural Federal University Press, 2012, 194 p. (In Russian)

7. Batenkov K. A. {Chislovye harakteristiki struktur setej svyazi} [Numerical characteristics of the structures of communication networks]. \textit{Trudy SPIIRAN} [\textit{Proceedings of SPIIRAS}], 2017, no.~4, pp.~5--28.\linebreak (In~Russian)

8. Filin B. P. \textit{Metody analiza strukturnoj nadezhnosti setej svyazi} [\textit{Methods of analysis of structural reliability of communication networks}]. Moscow, Radio i svyaz' Publ., 1988, 208~p. (In Russian)

9. Batenkov K. A. {Obshchie podhody k analizu i sintezu struktur setej svyazi} [General ap\-proa\-ches to the analysis and synthesis of structures of communication networks]. \textit{Sovremennye pro\-ble\-my te\-le\-kom\-mu\-ni\-kaciy. Materials of Russian scientific-technical conference} [\textit{Modern problems of tele\-com\-mu\-ni\-ca\-tions}]. Novosibirsk, Siberian State University of Telecommunications and Informatics Press, 2017, pp.~19--23. (In Russian)

10. Polovko A. M., Gurov S. V. \textit{Osnovy teorii nadezhnosti} [\textit{Fundamentals of reliability theory}]. St~Petersburg, BHV-Pe\-ters\-burg Publ., 2006, 704 p. (In Russian)

11. {Nozaki T., Nakano T., Wadayama T.} Analysis of breakdown probability of wireless sensor networks with unreliable relay nodes. \textit{2017 IEEE Intern. Symposium Inf. Theory.} Aachen, Germany, 2017, pp.~481--485.

12. {Takabe S., Nakano T., Wadayama T.} \textit{Fault tolerance of random graphs with respect to connectivity: phase transition in logarithmic average degree}. arXiv: 1712.07807, 2017.

13. {Tutte W. T.} \textit{Graph theory.} Addison, Addison-Wesley Publ., 1984, 423~p.

14. {Yagan O., Makowski A. M.} Zero-one laws for connectivity in random key graphs. \textit{IEEE Trans. Inf. Theory,} 2012, vol.~58, no.~5, pp.~2983--2999.

15. {Batenkov K. A.} K voprosu ocenki nadezhnosti dvuhpolyusnyh i mnogopolyusnyh setej svyazi [To the question of assessing the reliability of bipolar and multipolar networks]. \textit{Sovremennye problemy radioehlektroniki}. Sbornik nauch. trudov [\textit{Modern problems of radioelectronics}]. Krasnoyarsk, Siberian Federal University Press, 2017, pp.~604--608. (In Russian)

16. {Zhao J., Yagan O., Gligor V. }Connectivity in secure wireless sensor networks under transmission constraints. \textit{Allerton Conference on Communication, Control, and Computing}, 2014, pp.~1--18.

17. {Nu�ez A., Lacasa L., Valero E., G�mez J. P., Luque B.} Detecting series periodicity with horizontal visibility graphs. \textit{Intern. J. Bifurc. Chaos}, 2012, no.~22, pp.~1--10.

18. {Zhang H. C., Xu D. L., Lu C., Qi E. R., Tian C., Wu Y. S.} Connection effect on amplitude death stability of multi-module floating airport. \textit{Ocean Eng.}, 2017, pp.~46--56.

19. Batenkov K. A. \textit{Ustojchivost' setej svyazi} [\textit{Network stability}]. Oryol, Akademy of Federal Guard Service of Russian Federation Press, 2017, 277~p. (In Russian)

20. {Brown J. I., Tufts J.} On the roots of domination polynomials. \textit{Graphs Combin.}, 2014, no.~30, pp.~527--547.

21. {Cox D.} \textit{On network reliability}. PhD thesis. Halifax, Nova Scotia, Dalhousie University Press, 2013, 209~p.

22. Batenkov K. A. Ob analize zhivuchesti setej svyazi na osnove veroyatnostnogo podhoda [On the analysis of survivability of communication networks based on probabilistic approach]. \textit{Nedelya nauki SPbPU} [\textit{Science week of SPbSPU}]. Institut fisiki, nanotekhnologii i telekommunikatsii, St\,Petersburg, St\,Petersburg Polytechnical Institute Press, 2016, pp.~6--8. (In Russian)

23. {Huh J.} H-vectors of matroids and logarithmic concavity. \textit{Adv. Math.}, 2015, no.~270, pp.~49--59.

24. {Harris D. G., Srinivasan A.} Improved bounds and algorithms for graph cuts and network reliability. \textit{Proceedings of the 25$^{th}$ Annual ACM-SIAM Symposium on Discrete Algorithms}. ACM-SIAM, ACM Press, 2014, pp.~259--278.\pagebreak

25. {Karger D. R. }A fast and simple unbiased estimator for network (un)reliability. \textit{Proceedings of the 48$^{th}$ annual IEEE Symposium on Foundations of Computer Science.} New Brunswick, New Jersey, IEEE Computer Society Technical Committee on Mathematical Foundations of Computing Publ., 2016, pp.~635--644.

26. Batenkov K. A. Osobennosti ocenki kachestva funkcionirovaniya setej svyazi [Features of an estimation of quality of functioning of communication networks]. \textit{Resursoehffektivnye sistemy v upravlenii i kontrole: vzglyad v budushchee} [\textit{Resource-efficient system management and control: a look into the future}]. Sbornik nauch. trudov of V conference schools, students, postgradient students and young scientists. Tomsk, Tomsk Politekhnical University Press, 2016, pp.~30--31. (In Russian)

27. {Mishra K., Trivedi K., Some R.} Uncertainty analysis of the remote exploration and expe\-ri\-men\-ta\-tion system. \textit{Journal of Spacecraft and Rockets}, 2012, pp.~1032--1042.

28. {Ghosh R., Longo F., Frattini F., Russo S., Trivedi K.} Scalable analytics for IaaS cloud availability. \textit{IEEE Trans. on Cloud Computing,} 2014, pp.~57--70.

29. Karpov A. G., Klemeshev V. A., Kuranov D. Yu. Opredelenie rabotosposobnosti sistemy, struktura kotoroi zadana grafom [Determining the ability to work
of the system, the structure of which is given using graph]. \textit{Vestnik of Saint Petersburg University.
Applied Mathematics. Computer Science. Control Processes,} 2020, vol.~16, iss.~1, pp.~41--49. %\\
https://doi.org/\-10.21638/11702/spbu10.2020.104 (In Russian)

\vskip1.5mm

Received:  August 26, 2021.

Accepted: June 21, 2022.


\vskip6mm A~u~t~h~o~r's \ i~n~f~o~r~m~a~t~i~o~n:%

\vskip2mm \textit{Kirill A. Batenkov} --- Dr. Sci. in Engineering Sciences, Associate Professor; pustur@yandex.ru \par%
%
%
}
