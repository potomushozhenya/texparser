

{\small



\vskip4.5%6
mm

\noindent \textbf{References} }

\vskip 1.5%2
mm

{\footnotesize


1. Kovac �., Paranos M., Marcius D.	Hydrogen in energy transition: A review. \textit{Intern. Journal of Hyd\-ro\-gen Energy,} 2021, vol.~46, iss.~16, pp.~10016--10035.
	
	
	2. Kuznetsov V. Stereochemistry of simple molecules inside nanotubes and fullerenes: Unusual behavior of usual systems. \textit{ Molecules,} 2020, vol.~25, pp.~2437.
	
	3. Tarasov B. P., Goldshleger N. F., Moravsky A. P. Hydrogen-containing carbon nanostructures: synthesis and properties.  \textit{Russian Chemical Reviews,} 2001, vol.~70, iss.~2, pp.~131--146.
	
	4. Zuttel A., Sudan P., Mauron P., Kioyobayashi T., Emmenegger C., Schlapbach L. Hydrogen storage in carbon nanostructures.\textit{ Intern. Journal of Hydrogen Energy,} 2002, vol.~27, pp.~203--212.
	
	5. Cao D., Wang W.  Storage of hydrogen in single-walled carbon nanotube bundles with optimized parameters: effect of external surfaces. \textit{ Intern. Journal of Hydrogen Energy,} 2007, vol.~32, pp.~1939--1942.
	
	6. Seung M.L., Kay H.A., Young H.L., Seifert G., Frauenheim T. Novel mechanism of hydrogen storage in carbon nanotubes. \textit{Journal of the Korean Physical Society,} 2001, vol.~38, pp.~686--691.
	
	7. Suetin M. V. Numerical study of the processes of absorption, storage and release of hydrogen by fullerenes. \textit{Mathematical modeling and boundary value problems. Proceedings of the Second All-Russian Scientific Conference}, 2005, pp.~234--237.
	
	8. Vehvilainen T. T., Ganchenkova  M. G., Oikkonen L. E., Nieminen R. M. Hydrogen interaction with fullerenes: From C$_{20}$ to graphene. \textit{Physical Review B}, 2011, vol.~84, pp.~085447.
	
	9. Dolgonos G. A., Peslherbe G. H. Conventional and density-fitting local Moller\,---\,Plesset theory calculations of C-60 and its endohedral H-2@C-60 and 2H(2)@C-60 complexes. \textit{ Chemical Physics Letters,} 2011,  vol.~513, iss.~4--6, pp. 236--240.
	
	10. Cioslowski J. Endohedral chemistry: electronic structures of molecules trapped inside the C$_{60}$ cage. \textit{Journal of the American Chemical Society,} 1991, vol.~113, iss.~11, pp.~4139--4141.
	
	11. Koch W., Holthausen M. {\it A chemist's guide to density functional theory.} Ed.~2. Weinheim, Wiley-VCH Press, 2002, 293~p.
	
	12. Koch W., Becke A. D., Parr R. G. Density functional theory of electronic structure. {\it The Journal of Physical Chemistry,} 1996, vol.~100, iss.~31, pp.~12974--12980.
	
	13. Kohn W. Electronic structure of matter\,---\,wave functions and density functionals. {\it Reviews of~Mo\-dern Physics,} 1999, vol.~71, iss.~5, pp.~1253--1266.
	
	14.  Zhao Y., Truhlar  D. G. The M06 suite of density functional for main group thermochemistry, thermochemical kinetics, noncovalent interactions, excited states, and transition elements: two new functional and systematic testing of four M06-class functional and 12 other functional. {\it Theoretical Chemistry Accounts,} 2008, vol.~120, pp.~215--241.
	
	15. Frisch M. J., Trucks G. W., Schlegel H. B. et al.  GAUSSIAN-09, Rev. C.01. Wallingford, CT, Gaussian Inc. Press, 2010.
	
	16. Koch W., Sham L. J. Self-consistend equations including exchange and correlation effects. {\it Physical Review Journals,} 1965, vol.~140, iss.~4, pp.~1133--1138.
	

	17. Vasiliev A. A., Bedrina M. E., Andreeva T. A.  The dependence of calculation results on the density functional theory from the means of presenting the wave function.\textit{ Vestnik of Saint Petersburg University. Applied Mathematics. Computer Science. Control Processes}, 2018, vol.~14, iss.~1, pp.~51--58. \\
https://doi.org/10.21638/11702/spbu10.2018.106 (In Russian)
	
	18. Andreeva T. A., Bedrina M. E., Ovsyannikov D. A. Comparative analysis of calculation methods in electron spectroscopy.  \textit{Vestnik of Saint Petersburg University. Applied Mathematics. Computer Science. Control Processes,} 2019, vol.~15, iss.~4, pp.~518--528. \\
https://doi.org/10.21638/11702/spbu10.2019.406 (In Russian)
	
	19. Bedrina M. E., Egorov N. V., Kuranov D. Yu., Semenov S. G.  Raschet ftalocianinatov cinka na vysokoproizvoditelnom vychislitelnom komplekse [Calculation metalphthalocyaninatozinc on the high-efficiency computer complex]. {\it Vestnik of Saint Petersburg University. Series~10. Applied Mathematics. Computer Science. Control Processes}, 2011, iss.~3, pp.~13--21. (In Russian)
	
	20. Raik A. V., Bedrina M. E. Modeling of water adsorption process on crystal surfaces.  \textit{ Vestnik of Saint Petersburg University. Series~10. Applied Mathematics. Computer Science. Control Processes,} 2011, iss.~2, pp.~67--75. (In Russian)


\vskip 1.5mm

%\noindent Recommendation: prof. L. A. Petrosyan.
%
%\vskip 1.5mm

%\noindent

Received:  June 14, 2022.

Accepted: June 21, 2022.

\vskip 4.5%6
mm

%\pagebreak

A\,u\,t\,h\,o\,r\,s' \ i\,n\,f\,o\,r\,m\,a\,t\,i\,o\,n:


\vskip 1.5%2
mm
 \textit{Nikolay V. Egorov} --- Dr. Sci. in Physics and Mathematics,
				Professor; n.v.egorov@spbu.ru \par


\vskip 1.5%2
mm
 \textit{Alexander A. Vasilyev} ---  Postgraduate Student;
			 atlasnw@gmail.com \par


\vskip 1.5%2
mm
 \textit{Tatiana A. Andreeva} --- PhD in Physics and Mathematics,
					Associate Professor; t.a.andreeva@spbu.ru \par


\vskip 1.5%2
mm
 \textit{Marina E. Bedrina} --- Dr. Sci. in Physics and Mathematics,
				Professor; m.bedrina@spbu.ru
 \par
%
%
}
